\documentclass{article}
\usepackage{amsmath, amsthm, amssymb, booktabs, hyperref, graphicx, float, esint, xcolor}
\setlength{\abovedisplayskip}{0pt}
\setlength{\belowdisplayskip}{0pt}
\setlength{\abovedisplayshortskip}{0pt}
\setlength{\belowdisplayshortskip}{0pt}

\newcommand{\vr}{\vec{r}}
\newcommand{\vOmega}{\vec{\Omega}}
\newcommand{\vO}{\vec{\Omega}}
\newcommand{\bra}{\left\langle}
\newcommand{\ket}{\right\rangle}
\newcommand{\vdiv}{\vec{\nabla} \cdot}
\newcommand{\vgrad}{\vec{\nabla}}
\newcommand{\vbeta}{\vec{\beta} }
\newcommand{\pdx}{\frac{\partial}{\partial x}}
\newcommand{\pdy}{\frac{\partial}{\partial y}}
\newcommand{\pdz}{\frac{\partial}{\partial z}}

\begin{document}
\begin{center}
Ian Halvic \\
\end{center}

%%%%%%%%%%%%%%%%%%%%%%%%%%%%%%%%%%%%%%%%%%%%%%%%%%%%%%%%%%%%%%%%%%%%%%%%%%%%%%%%%%%%%%%%%%%%%%%%%%%%%%
%%%%%%%%%%%%%%%%%%%%%%%%%%%%%%%%%%%%%%%%%%%%%%%%%%%%%%%%%%%%%%%%%%%%%%%%%%%%%%%%%%%%%%%%%%%%%%%%%%%%%%
\section{1D-Sn Transport}
%%%%%%%%%%%%%%%%%%%%%%%%%%%%%%%%%%%%%%%%%%%%%%%%%%%%%%%%%%%%%%%%%%%%%%%%%%%%%%%%%%%%%%%%%%%%%%%%%%%%%%
%%%%%%%%%%%%%%%%%%%%%%%%%%%%%%%%%%%%%%%%%%%%%%%%%%%%%%%%%%%%%%%%%%%%%%%%%%%%%%%%%%%%%%%%%%%%%%%%%%%%%%

Define our forward Sn transport equation
\[
\vO \cdot \vgrad \psi + \sigma_t \psi = \frac{\sigma_s}{4 \pi} \phi + \frac{q}{2} \quad \vr \in V
\]
with BC 
\[
\psi(\vr,\vO) = \psi^{\text{inc}}(\vr,\vO) \quad \vr \in \partial V^-
\]
with $\partial V^{\mp}=\left\{ \partial V \text{ such that } \vr \cdot \vO \lessgtr 0 \right\}$.

The adjoint Sn transport equation
\[
- \vO \cdot \vgrad \psi^\dag + \sigma_t \psi^\dag = \frac{\sigma_s}{4 \pi} \phi^\dag + q^\dag
\]
Where the adjoint source is our response $q^\dag = \frac{r}{4 \pi}$
\textcolor{red}{always give bc as well}
Multiply the forward by $\psi^\dag$ and integrate over space and angle
\[
\int d^3 r \, \int d  \Omega \,  \psi^\dag \left[ \vO \cdot \vgrad \psi + \sigma_t \psi  \right] = 
\int d^3 r \, \int d  \Omega \,  \psi^\dag \left[ \frac{\sigma_s}{4 \pi} \phi + \frac{q}{2}  \right]
\]
Take a look at the RHS first
\begin{align*}
\int d^3 r \, \int d  \Omega \,  \psi^\dag \left[ \frac{\sigma_s}{4 \pi} \phi + \frac{q}{2}  \right]
&= 
\int d^3 r \, \int d  \Omega \,  \psi^\dag \frac{\sigma_s}{4 \pi} \phi + \int d^3 r \, \int d  \Omega \,  \psi^\dag \frac{q}{2}  \\
&= 
\int d^3 r \, \frac{\sigma_s}{4 \pi} \phi \int d  \Omega \,  \psi^\dag  + \int d^3 r \, \int d  \Omega \,  \psi^\dag \frac{q}{2}  \\
&= 
\int d^3 r \, \frac{\sigma_s}{4 \pi} \phi  \phi^\dag  + \int d^3 r \, \int d  \Omega \,  \psi^\dag \frac{q}{2}  \\
&= 
\int d^3 r \, \int d  \Omega \,  \psi^\dag \frac{\sigma_s}{4 \pi} \phi^\dag + \int d^3 r \, \int d  \Omega \,  \psi^\dag \frac{q}{2}  \\
\end{align*}
On the LHS, take a look at the first integral
\begin{align*}
\int d^3 r \, \int d  \Omega \,  \psi^\dag \left[  \vO \cdot \vgrad \psi \right]
&= \int d^3 r \, \int d  \Omega \,  \psi^\dag \left[  \vdiv (\vO \psi) \right] \\
&= - \int d^3 r \, \int d  \Omega \,  \vgrad \psi^\dag \cdot \left[(\vO \psi) \right] 
+ \int d^2 r \, \int d  \Omega \, \psi^\dag \vO \psi \cdot \vec{n} \\
&= - \int d^3 r \, \int d  \Omega \,  \psi \left[  \vO \cdot \vgrad \psi^\dag \right] 
+ \int d^2 r \, \int d  \Omega \, \psi^\dag \psi ( \vO \cdot \vec{n})
\end{align*}
Expressed as the in inner product notation of $\int d^3 r \, \int d  \Omega $ the above states
\begin{align*}
\bra \psi^\dag , q \ket &=  \bra \psi^\dag , \vO \cdot \vgrad \psi + \sigma_t \psi - \frac{\sigma_s}{4 \pi} \phi \ket \\
&=  \bra - \vO \cdot \vgrad \psi^\dag + \sigma_t \psi^\dag - \frac{\sigma_s}{4 \pi} \phi^\dag , \psi \ket + \int d^2 r \, \int d  \Omega \, \psi^\dag \psi ( \vO \cdot \vec{n} ) \\
&=  \bra q^\dag , \psi \ket + \int d^2 r \, \int d  \Omega \, \psi^\dag \psi ( \vO \cdot \vec{n} ) \\
\end{align*}
Our QoI is defined as 
\[
J=\bra q^\dag , \psi \ket
\]
So, using the adjoint manipulation from above
\[
J = \bra \psi^\dag , q \ket - \int d^2 r \, \int d  \Omega \, \psi^\dag \psi ( \vO \cdot \vec{n} )
\]
The surface interval can be split into incoming and outgoing 
\[
J = \bra \psi^\dag , q \ket - \int d^2 r \, \int_{\vO \cdot \vec{n} >0} d  \Omega \, \psi^\dag \psi ( \vO \cdot \vec{n} ) - \int d^2 r \, \int_{\vO \cdot \vec{n} <0} d  \Omega \, \psi^\dag \psi ( \vO \cdot \vec{n} )
\]
Where we would expect to specify the incoming flux for the forward and the outgoing flux for the adjoint.
\section{1D-Sn Transport Sensitivity}
Now assume we perturb our forward transport equation.
Define our forward Sn transport equation
\[
\vO \cdot \vgrad \left( \psi + \delta \psi \right) + \left( \sigma_t + \delta \sigma_t \right) \left( \psi + \delta \psi \right) = \frac{\left( \sigma_s + \delta \sigma_s \right)}{4 \pi} \left( \phi + \delta \phi \right) + \left( q + \delta q \right)
\]
Retain the adjoint equation
\[
- \vO \cdot \vgrad \psi^\dag + \sigma_t \psi^\dag = \frac{\sigma_s}{4 \pi} \phi^\dag + q^\dag
\]
Expand and drop the second order terms [What does this mean for BC?]. Define the quantity $\psi_p = \psi + \delta \psi$ and $\phi_p = \phi + \delta \phi$
\begin{align*}
\vO \cdot \vgrad  \psi_p + \ \sigma_t \psi_p + \delta \sigma_t \psi = \frac{ \sigma_s }{4 \pi}  \phi_p + \frac{ \delta \sigma_s }{4 \pi}  \phi  + q + \delta q 
\end{align*}
Rearrange to the suggestive form
\[
\vO \cdot \vgrad  \psi_p + \ \sigma_t \psi_p - \frac{ \sigma_s }{4 \pi}  \phi_p  = q - \delta \sigma_t \psi + \frac{ \delta \sigma_s }{4 \pi}  \phi  + \delta q 
\]
The LHS is the same essentially the same operator as the forward solution, except applied to $\phi_p$ instead of $\phi$, so it follows that 
\[
\bra \psi^\dag , \vO \cdot \vgrad  \psi_p + \ \sigma_t \psi_p - \frac{ \sigma_s }{4 \pi}  \phi_p  \ket = \bra - \vO \cdot \vgrad  \psi^\dag + \ \sigma_t \psi^\dag - \frac{ \sigma_s }{4 \pi}  \phi^\dag , \psi_p \ket +  \int d^2 r \, \int d  \Omega \, \psi^\dag \psi_p ( \vO \cdot \vec{n})
\]
Convert using the forward and adjoint equations
\[
\bra \psi^\dag ,  q + \delta q - \delta \sigma_t \psi + \frac{ \delta \sigma_s }{4 \pi}  \phi   \ket = \bra q^\dag , \psi_p \ket +  \int d^2 r \, \int d  \Omega \, \psi^\dag \psi_p ( \vO \cdot \vec{n})
\]
The first inner product on the RHS is our perturbed QoI, so
\[
J_p = \bra \psi^\dag ,  q + \delta q - \delta \sigma_t \psi + \frac{ \delta \sigma_s }{4 \pi}  \phi     \ket -  \int d^2 r \, \int d  \Omega \, \psi^\dag \psi_p ( \vO \cdot \vec{n})
\] 
Using $J_p = J + \delta J$, the sensitivity can be found. 
\[
\delta J = \bra \psi^\dag ,  \delta q - \delta \sigma_t \psi + \frac{ \delta \sigma_s }{4 \pi}  \phi    \ket -  \int d^2 r \, \int d  \Omega \, \psi^\dag \delta \psi ( \vO \cdot \vec{n})
\] 
And split the surface integral once again
\[
\delta J = \bra \psi^\dag ,  \delta q - \delta \sigma_t \psi + \frac{ \delta \sigma_s }{4 \pi}  \phi    \ket
- \int d^2 r \, \int_{\vO \cdot \vec{n} >0} d  \Omega \, \psi^\dag \delta \psi ( \vO \cdot \vec{n} ) 
- \int d^2 r \, \int_{\vO \cdot \vec{n} <0} d  \Omega \, \psi^\dag \delta \psi ( \vO \cdot \vec{n} )
\]
So we should know $\delta \psi$ on the incoming for boundary conditions, and we can specify $\psi^\dag$ on the outgoing.

%%%%%%%%%%%%%%%%%%%%%%%%%%%%%%%%%%%%%%%%%%%%%%%%%%%%%%%%%%%%%%%%%%%%%%%%%%%%%%%%%%%%%%%%%%%%%%%%%%%%%%
%%%%%%%%%%%%%%%%%%%%%%%%%%%%%%%%%%%%%%%%%%%%%%%%%%%%%%%%%%%%%%%%%%%%%%%%%%%%%%%%%%%%%%%%%%%%%%%%%%%%%%
\section{VEF Formulation}
%%%%%%%%%%%%%%%%%%%%%%%%%%%%%%%%%%%%%%%%%%%%%%%%%%%%%%%%%%%%%%%%%%%%%%%%%%%%%%%%%%%%%%%%%%%%%%%%%%%%%%
%%%%%%%%%%%%%%%%%%%%%%%%%%%%%%%%%%%%%%%%%%%%%%%%%%%%%%%%%%%%%%%%%%%%%%%%%%%%%%%%%%%%%%%%%%%%%%%%%%%%%%
\subsection{Equation}
Construct linear anisotropic transport equation
\[
\vOmega \cdot \vec{\nabla} \psi + \sigma \psi = \frac{\sigma_{s0} \phi_0 + 3 \sigma_{s1} \vOmega \cdot \vec{\phi_1}}{4 \pi} + q( \vO )
\]
Where
\[
\phi_0=\int d\Omega \, \vO \psi( \vO )
\]
\[
\phi_1=\int d\Omega \, \psi( \vO )
\]
Convert transport equation into coupled 0th and 1st moment equations by applying the $\int d\Omega \bullet $ and $\int d\Omega \vO \bullet$ operator respectively 
\[
\vec{\nabla} \cdot \vec{\phi_1} + \sigma \phi_0 = \sigma_{s0} \phi_0 + q_0
 \]
 \[
\vec{\nabla} \cdot \left(  \int d\Omega \vO \vO \psi \right) + \sigma \vec{\phi_1} =\sigma_{s1} \vec{\phi_1} + \vec{q}_1
 \]
 Of minor importance define $\sigma_r=\sigma-\sigma_{s0}$ and  $\sigma_{tr}=\sigma-\sigma_{s1}$. Then of particular importance define $E$ such that 
 \[
 E=\frac{\int d\Omega \vO \vO \psi}{\phi_0} = \frac{\phi_2}{\phi_0}
 \]
 At which point the coupled system reduces to
 \[
\vec{\nabla} \cdot \vec{\phi_1} + \sigma_r \phi_0 = q_0
 \]
 \[
\vec{\nabla} \cdot \left(E \phi_0 \right) + \sigma_{tr} \vec{\phi_1} =  \vec{q}_1
 \]
 Use 2nd equation to express $\vec{\phi_1}$ in terms of $\phi_0$
\[
\vec{\phi_1}=\frac{\vec{q}_1 - \vec{\nabla} \cdot \left(E \phi_0 \right)  }{\sigma_{tr} }
\]
Substitution into first equation
\[
-\vdiv \left( \frac{1}{\sigma_{tr}} \vdiv \left( E \phi_0 \right) \right) + \sigma_r \phi_0 = q_0 -  \vec{\nabla} \cdot \frac{1}{\sigma_{tr}} \vec{q}_1 
\]
\subsection{Boundary conditions}
Assume for the boundary condition to the system, the incident angular flux $\psi^{inc}$ is specified. To begin converting to a useful form for VEF, multiply by $| \Omega_i n_i |$ and integrate over incoming angles $\Omega_i n_i <0$
\[
\int_{\Omega_i n_i<0} d\Omega \, \left| \Omega_i n_i \right | \psi^{inc} 
\]
Convert the above into a full $4 \pi$ range interval, and split into two integrals
\begin{align*}
\int_{\Omega_i n_i<0} d\Omega \, \left| \Omega_i n_i \right | \psi^{inc} 
&= \frac{1}{2} \int_{4 \pi} d\Omega \, \left(\left| \Omega_i n_i \right | - \Omega_i n_i  \right)\psi^{inc} \\
&= \frac{1}{2} \int_{4 \pi} d\Omega \, \left| \Omega_i n_i \right | \psi^{inc} - \frac{1}{2} \int_{4 \pi} d\Omega \,  \Omega_i n_i \psi^{inc} \\ 
\end{align*}
We then recognize that the second term on the RHS is $n_i J_i$. We also multiply the first term by the identity factor.
\[
\frac{\phi}{\int_{4\pi} d\Omega \, \psi} 
\]
So the above becomes
\[
\int_{\Omega_i n_i<0} d\Omega \, \left| \Omega_i n_i \right | \psi^{inc} = \frac{1}{2} \left( \frac{\int_{4 \pi} d\Omega \, \left| \Omega_i n_i \right | \psi^{inc}}{\int_{4\pi} d\Omega \, \psi} \right) \phi - \frac{1}{2} n_i J_i
\]
Now define a "Boundary Eddington Factor" B such that
\[
B= \frac{\int_{4 \pi} d\Omega \, \left| \Omega_i n_i \right | \psi^{inc}}{\int_{4\pi} d\Omega \, \psi} \quad \vr \in \partial V
\]
We also know
\[
J = \vec{\phi_1}=\frac{\vec{q}_1}{\sigma_{tr} } - \frac{1}{\sigma_{tr} } \vec{\nabla} \cdot \left(E \phi_0 \right)  
\]
So the boundary term becomes 
\[
\int_{\Omega \cdot n<0} d\Omega \, \left| \Omega \cdot n \right | \psi^{inc} = J^- =  \frac{1}{2} B \phi - \frac{\vec{q}_1}{2 \sigma_{tr} } + \frac{1}{2 \sigma_{tr} } \vec{\nabla} \cdot \left(E \phi_0 \right)  \quad \vr \in \partial V
\]
Then rearrange to what will be the useful form
\[
\frac{1}{\sigma_{tr} } \vec{\nabla} \cdot \left(E \phi_0 \right)  = 2J^- - B \phi + \frac{\vec{q}_1}{ \sigma_{tr} } \quad \vr \in \partial V
\]

%%%%%%%%%%%%%%%%%%%%%%%%%%%%%%%%%%%%%%%%%%%%%%%%%%%%%%%%%%%%%%%%%%%%%%%%%%%%%%%%%%%%%%%%%%%%%%%%%%%%%%
%%%%%%%%%%%%%%%%%%%%%%%%%%%%%%%%%%%%%%%%%%%%%%%%%%%%%%%%%%%%%%%%%%%%%%%%%%%%%%%%%%%%%%%%%%%%%%%%%%%%%%
\section{VEF Adjoint Formulation}
%%%%%%%%%%%%%%%%%%%%%%%%%%%%%%%%%%%%%%%%%%%%%%%%%%%%%%%%%%%%%%%%%%%%%%%%%%%%%%%%%%%%%%%%%%%%%%%%%%%%%%
%%%%%%%%%%%%%%%%%%%%%%%%%%%%%%%%%%%%%%%%%%%%%%%%%%%%%%%%%%%%%%%%%%%%%%%%%%%%%%%%%%%%%%%%%%%%%%%%%%%%%%

\subsection{Test function applied to forward}
The forward VEF formulation, representing using the operator A
\[ -\vdiv \left( \frac{1}{\sigma_{tr}} \vdiv \left( E \phi \right) \right)
+ \sigma_r \phi
= q
\]
\textcolor{red}{always give bc as well}
The corresponding adjoint operator \textcolor{red}{careful, at some point, you should describe (1) where the VEF comes from, (2) state that the adjoint VEF can be defined in 2 manners: the adjoint of the forward VEF, which we will call the math adjoint, and the VEF of the adjoint Sn transport}


Multiply the forward by the test function $\phi^\dag$ (which will become our adjoint) and integrate over space. Focus on the first term. Two integration by parts are performed
\begin{align*}
\int d^3r \, \phi^\dag \left[  -\vdiv \left( \frac{1}{\sigma_{tr}} \vdiv \left( E \phi \right) \right) \right] 
=& \int d^3r \, \left[ \vgrad \phi^\dag \right] \cdot \left[ \left( \frac{1}{\sigma_{tr}} \vdiv \left( E \phi \right) \right) \right] 
- \int d^2 r \, \phi^\dag \left[ \left( \frac{1}{\sigma_{tr}} \vdiv \left( E \phi \right) \right) \cdot \vec{n} \right] \\
=& \int d^3r \, \left[ \frac{1}{\sigma_{tr}} \vgrad \phi^\dag \right] \cdot \left[ \left(  \vdiv \left( E \phi \right) \right) \right] 
- \int d^2 r \, \phi^\dag \left[ \left( \frac{1}{\sigma_{tr}} \vdiv \left( E \phi \right) \right) \cdot \vec{n} \right] \\
=& \int d^3r \, \left[- \vgrad \frac{1}{ \sigma_{tr}} \left(  \vgrad \phi^\dag \right) \right] : \left[ E \phi \right]^T 
+ \int d^2 r \, \left( E \phi \cdot \left( \frac{1}{ \sigma_{tr}} \left(  \vgrad \phi^\dag \right) \right)^T \right) \cdot \vec{n} \\
&- \int d^2 r \, \phi^\dag \left[ \left( \frac{1}{\sigma_{tr}} \vdiv \left( E \phi \right) \right) \cdot \vec{n} \right]\\
=& \int d^3r \, \left[- E^T \vgrad \frac{1}{ \sigma_{tr}} \left(  \vgrad \phi^\dag \right) \right] : \left[ \phi \right] 
+ \int d^2 r \, \left( E \phi \cdot \left( \frac{1}{ \sigma_{tr}} \left(  \vgrad \phi^\dag \right) \right)^T \right) \cdot \vec{n} \\
&- \int d^2 r \, \phi^\dag \left[ \left( \frac{1}{\sigma_{tr}} \vdiv \left( E \phi \right) \right) \cdot \vec{n} \right]
\end{align*}

\subsection{Adjoint equation}
From the previous result we obtain an expression for an adjoint equation
\[ 
E \cdot \left( - \vgrad \left( \frac{1}{\sigma_{tr}} \vgrad \phi^\dag \right) \right)
+ \sigma_r \phi^\dag
= q^\dag
\]
Note that this is NOT the adjoint equation we would expect from an "adjoint particle" transport treatment. We now need to build a boundary condition for the above adjoint. The boundary terms which appeared in the preceding test function formulation are 
\[
 \int d^2 r \, \left( E \phi \cdot \left( \frac{1}{ \sigma_{tr}} \left(  \vgrad \phi^\dag \right) \right)^T \right) \cdot \vec{n} 
- \int d^2 r \, \phi^\dag \left[ \left( \frac{1}{\sigma_{tr}} \vdiv \left( E \phi \right) \right) \cdot \vec{n} \right]
\]
Recalling that the boundary condition for the forward was
\[
\frac{1}{\sigma_{tr} } \vec{\nabla} \cdot \left(E \phi_0 \right)  = 2J^- - B \phi + \frac{\vec{q}_1}{\sigma_{tr} } \quad \vr \in \partial V
\]
Substitution into the boundary term leads to
\[
 \int d^2 r \, \left( E \phi \cdot \left( \frac{1}{ \sigma_{tr}} \left(  \vgrad \phi^\dag \right) \right) \right) \cdot \vec{n} 
- \int d^2 r \, \phi^\dag \left[ \left( 2J^- - B \phi + \frac{\vec{q}_1}{\sigma_{tr} }  \right) \cdot \vec{n} \right] \quad \vr \in \partial V
\]
From the above it appears the the most computationally "beneficial" boundary condition would be specifying 
\[
E \cdot \frac{1}{ \sigma_{tr}} \left(  \vgrad \phi^\dag \right) 
\]
Consider the proposed adjoint boundary condition
\[
E \cdot \frac{1}{ \sigma_{tr}} \left(  \vgrad \phi^\dag \right) =  2J^{ \dag + } - B \phi^\dag  \quad \vr \in \partial V
\]
Which is reminiscent of the forward boundary condition, but with specified outgoing partial current (which is frequently specified for transport derived adjoints. Using the above condition the boundary term becomes. 
\[
 \int d^2 r \, \phi \left[ \left( 2J^{ \dag + } - B \phi^\dag \right) \cdot \vec{n} \right]
- \int d^2 r \, \phi^\dag \left[ \left( 2J^- - B \phi + \frac{\vec{q}_1}{\sigma_{tr} } \right) \cdot \vec{n} \right]  \quad \vr \in \partial V
\]
Importantly, the boundary Eddington terms cancel, leaving a total boundary term of
\[
 \int d^2 r \, \left( 2\phi J^{ \dag + }  - 2\phi^\dag J^- -  \frac{\vec{q}_1}{\sigma_{tr}} \right) \cdot \vec{n}
\]
\subsection{QoI via adjoint}
Substitution of the adjoint equation with boundary terms into the test function earlier derivation yields
\[
\bra \phi^\dag , q \ket = \bra \phi, q^\dag \ket +  \int d^2 r \, \left( 2\phi J^{ \dag + }  - 2\phi^\dag J^- -  \frac{\vec{q}_1}{\sigma_{tr}} \right) \cdot \vec{n}
\]

Recognizing $\bra \phi, q^\dag \ket $ is the desired QoI
\[
J = \bra \phi, q^\dag \ket = \bra \phi^\dag , q \ket + \int d^2 r \, \left( 2\phi^\dag J^- - 2\phi J^{ \dag + }   +  \frac{\vec{q}_1}{\sigma_{tr}} \right) \cdot \vec{n}
\]
\section{VEF Adjoint Sensitivity}
Assume we perturb $\sigma_{tr}$, $\sigma_r$, and $q_0$; resulting in a perturbed forward solution $\phi_p = \phi + \delta \phi$. Assuming the Eddington tensor remains the same as the unperturbed scenario, we have.
\[
-\vdiv \left( \frac{1}{\sigma_{tr}+\delta \sigma_{tr}} \vdiv \left( E \left( \phi + \delta \phi \right) \right) \right)
+ \left( \sigma_r + \delta \sigma_r \right) \left( \phi + \delta \phi \right)
= q_0 + \delta q_0
\]
For notational compactness, let $c=\frac{1}{\sigma_{tr}}$ and $c + \delta c = \frac{1}{\sigma_{tr}+\delta \sigma_{tr}}$. Expanding the above leads to 
\[ 
-\vdiv \left( c \vdiv \left( E \phi_p \right) \right) 
- \vdiv \left( \delta c \vdiv \left( E \phi \right) \right) 
- \vdiv \left( \delta c \vdiv \left( E \delta \phi_p \right) \right) 
+ \sigma_r \phi_p
+ \delta \sigma_r \phi
+ \delta \sigma_r \delta \phi
= q_0 + \delta q_0 
\]
Drop 2nd order terms
\[ 
-\vdiv \left( c \vdiv \left( E \phi_p \right) \right) 
- \vdiv \left( \delta c \vdiv \left( E \phi \right) \right) 
+ \sigma_r \phi_p
+ \delta \sigma_r \phi
= q_0 + \delta q_0 
\]
Convert to the suggestive form
\[ 
-\vdiv \left( c \vdiv \left( E \phi_p \right) \right) + \sigma_r \phi_p
= q_0 + \delta q_0 - \delta \sigma_r \phi + \vdiv \left( \delta c \vdiv \left( E \phi \right) \right) 
\]
The LHS is simply the operator from the unperturbed scenario, so multiplication (of the first term) by $\phi^\dag$ and integration proceeds the same.
\begin{align*}
\int d^3r \, \phi^\dag \left[  -\vdiv \left( \frac{1}{\sigma_{tr}} \vdiv \left( E \phi_p \right) \right) \right] 
=& \int d^3r \, \left[- E^T \vgrad \frac{1}{ \sigma_{tr}} \left(  \vgrad \phi^\dag \right) \right] : \left[ \phi_p \right] 
+ \int d^2 r \, \left( E \phi_p \cdot \left( \frac{1}{ \sigma_{tr}} \left(  \vgrad \phi^\dag \right) \right)^T \right) \cdot \vec{n} \\
&- \int d^2 r \, \phi^\dag \left[ \left( \frac{1}{\sigma_{tr}} \vdiv \left( E \phi_p \right) \right) \cdot \vec{n} \right]
\end{align*}
Which is the same as the unperturbed case, except extra terms exist on the RHS "source" term in the forward.
\begin{align*}
\bra \phi^\dag , q_0 + \delta q_0 - \delta \sigma_r \phi + \vdiv \left( \delta c \vdiv \left( E \phi \right) \right) \ket =& \bra \phi_p, q^\dag \ket + \int d^2 r \, \left( E \phi_p \cdot \left( \frac{1}{ \sigma_{tr}} \left(  \vgrad \phi^\dag \right) \right)^T \right) \cdot \vec{n} \\
&- \int d^2 r \, \phi^\dag \left[ \left( \frac{1}{\sigma_{tr}} \vdiv \left( E \phi_p \right) \right) \cdot \vec{n} \right]
\end{align*}
Since $\bra \phi_p, q^\dag \ket $ is the desired QoI
\begin{align*}
J = \bra \phi_p, q^\dag \ket =& \bra \phi^\dag , q_0 + \delta q_0 - \delta \sigma_r \phi + \vdiv \left( \delta c \vdiv \left( E \phi \right) \right)  \ket - \int d^2 r \, \left( E \phi_p \cdot \left( \frac{1}{ \sigma_{tr}} \left(  \vgrad \phi^\dag \right) \right)^T \right) \cdot \vec{n} \\
&+ \int d^2 r \, \phi^\dag \left[ \left( \frac{1}{\sigma_{tr}} \vdiv \left( E \phi_p \right) \right) \cdot \vec{n} \right]
\end{align*}
In terms of the change in QoI
\begin{align*}
\delta J =& \bra \phi^\dag , \delta q_0 - \delta \sigma_r \phi + \vdiv \left( \delta c \vdiv \left( E \phi \right) \right)  \ket - \int d^2 r \, \left( E \delta \phi \cdot \left( \frac{1}{ \sigma_{tr}} \left(  \vgrad \phi^\dag \right) \right)^T \right) \cdot \vec{n} \\
&+ \int d^2 r \, \phi^\dag \left[ \left( \frac{1}{\sigma_{tr}} \vdiv \left( E \delta \phi \right) \right) \cdot \vec{n} \right]
\end{align*}

\newpage
\section{SOME SCRATH WORK ON Operator notation NEEDS REDONE}
The forward VEF formulation, representing using the operator A
\[
A \phi = -\vdiv \left( \frac{1}{\sigma_{tr}} \vdiv \left( E \phi \right) \right)
+ \sigma_r \phi
= q_0 
\]
The corresponding adjoint operator
\[
A^\dag \phi^\dag = E \cdot \left( - \vgrad \left( \frac{1}{\sigma_{tr}} \vgrad \phi^\dag \right) \right)
+ \sigma_r \phi^\dag
= q^\dag
\]
Assume we perturb $\sigma_{tr}$, $\sigma_r$, and $q_0$; resulting in a perturbed forward solution $\phi_p = \phi + \delta \phi$. Assuming the Eddington tensor remains the same as the unperturbed scenario, we have.
\[
-\vdiv \left( \frac{1}{\sigma_{tr}+\delta \sigma_{tr}} \vdiv \left( E \left( \phi + \delta \phi \right) \right) \right)
+ \left( \sigma_r + \delta \sigma_r \right) \left( \phi + \delta \phi \right)
= q_0 + \delta q_0
\]
For notational compactness, let $c=\frac{1}{\sigma_{tr}}$ and $c + \delta c = \frac{1}{\sigma_{tr}+\delta \sigma_{tr}}$. Expanding the above leads to 
\[ 
-\vdiv \left( c \vdiv \left( E \phi_p \right) \right) 
- \vdiv \left( \delta c \vdiv \left( E \phi \right) \right) 
- \vdiv \left( \delta c \vdiv \left( E \delta \phi_p \right) \right) 
+ \sigma_r \phi_p
+ \delta \sigma_r \phi
+ \delta \sigma_r \delta \phi
= q_0 + \delta q_0 
\]
Drop 2nd order terms
\[ 
-\vdiv \left( c \vdiv \left( E \phi_p \right) \right) 
- \vdiv \left( \delta c \vdiv \left( E \phi \right) \right) 
+ \sigma_r \phi_p
+ \delta \sigma_r \phi
= q_0 + \delta q_0 
\]
Convert to the suggestive form
\[ 
A \phi_p
= q_0 + \delta q_0 - \delta \sigma_r \phi + \vdiv \left( \delta c \vdiv \left( E \phi \right) \right) 
\]
Our new perturbed QoI is
\[
J_p = \bra \phi_p , q^\dag \ket = \bra \phi_p , A^\dag \phi^\dag\ket
\]
Move the $A^\dag$ operator to the perturbed forward, and sub in the suggestive form from above. (Note, this is the step where we pick up boundary terms)
\[
J_p = \bra \phi_p , A^\dag \phi^\dag\ket =  \bra A \phi_p , \phi^\dag\ket = \bra q_0 + \delta q_0 - \delta \sigma_r \phi + \vdiv \left( \delta c \vdiv \left( E \phi \right) \right), \phi^\dag \ket
\]

\subsection{1D with boundary terms}
Lets look at the operator notation step from above
\[
J_p = \bra \phi_p , q^\dag \ket = \bra \phi_p , A^\dag \phi^\dag\ket
\]
Expressing $\bra \phi_p , A^\dag \phi^\dag\ket$ in 1D geometry is
\[\int_a^b dx \, \phi_p \left[ - E  \pdx \left( \frac{1}{\sigma_{tr}} \pdx \phi^\dag \right) 
+ \sigma_r \phi^\dag \right]
\]
Since we are just looking for boundary terms, focus on the differential term,
\[\int_a^b dx \, \phi_p \left[ - E  \pdx \left( \frac{1}{\sigma_{tr}} \pdx \phi^\dag \right) \right]
\]
First integration by parts yields
\[\int_a^b dx \, \pdx \left( E \phi_p \right) \left[ \left( \frac{1}{\sigma_{tr}} \pdx \phi^\dag \right) \right] - \left. \left( E \phi_p   \left( \frac{1}{\sigma_{tr}} \pdx \phi^\dag \right) \right)  \right|_a^b
\]
Once more, int by parts
\[\int_a^b dx \,   - \pdx \left( \frac{1}{\sigma_{tr}} \pdx \left( E \phi_p \right) \right)  \left[ \phi^\dag \right] - \left. \left( E \phi_p   \left( \frac{1}{\sigma_{tr}} \pdx \phi^\dag \right) \right)  \right|_a^b + \left. \left( \frac{1}{\sigma_{tr}} \pdx \left( E \phi_p \right) \right) (\phi^\dag) \right|_a^b 
\]
So the two boundary terms we get from this are
\[
- \left. \left( E \phi_p   \left( \frac{1}{\sigma_{tr}} \pdx \phi^\dag \right) \right)  \right|_a^b + \left. \left( \frac{1}{\sigma_{tr}} \pdx \left( E \phi_p \right) \right) (\phi^\dag) \right|_a^b
\]
So
\[
J_p = \bra q_0 + \delta q_0 - \delta \sigma_r \phi + \vdiv \left( \delta c \vdiv \left( E \phi \right) \right), \phi^\dag \ket - \left. \left( E \phi_p   \left( \frac{1}{\sigma_{tr}} \pdx \phi^\dag \right) \right)  \right|_a^b + \left. \left( \frac{1}{\sigma_{tr}} \pdx \left( E \phi_p \right) \right) (\phi^\dag) \right|_a^b
\]

\end{document}