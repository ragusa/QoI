\documentclass{article}
\usepackage{amsmath, amsthm, amssymb, booktabs, hyperref, graphicx, float, esint, xcolor}
\setlength{\abovedisplayskip}{0pt}
\setlength{\belowdisplayskip}{0pt}
\setlength{\abovedisplayshortskip}{0pt}
\setlength{\belowdisplayshortskip}{0pt}

\newcommand{\vr}{\vec{r}}
\newcommand{\vOmega}{\vec{\Omega}}
\newcommand{\vO}{\vec{\Omega}}
\newcommand{\bra}{\left\langle}
\newcommand{\ket}{\right\rangle}
\newcommand{\vdiv}{\vec{\nabla} \cdot}
\newcommand{\vgrad}{\vec{\nabla}}
\newcommand{\vbeta}{\vec{\beta} }
\newcommand{\pdx}{\frac{\partial}{\partial x}}
\newcommand{\pdy}{\frac{\partial}{\partial y}}
\newcommand{\pdz}{\frac{\partial}{\partial z}}
\newcommand{\intrrr}{\int d^3 r \,}
\newcommand{\intrr}{\int d^2 r \,}
\newcommand{\dEdphi}{\partial_\phi E \,}
\newcommand{\dEdp}{\partial_p E \,}
\newcommand{\dBdphi}{\partial_\phi B \,}
\newcommand{\dBdp}{\partial_p B \,}
\newcommand{\adj}{\phi^\dag}
\newcommand{\surf}{\int_{\partial V}}
\newcommand{\vn}{\vec{n}}
\newcommand{\Edd}{\mathbf{E}}
\newcommand{\BEdd}{\mathbf{B}}
\newcommand{\sigt}{\sigma_t}
\newcommand{\sigs}{\sigma_s}
\newcommand{\siga}{\sigma_a}
\newcommand{\isigt}{c}
\newcommand{\angSource}{q_\Omega}
\newcommand{\scalSource}{q}
\newcommand{\angResp}{r_\Omega}
\newcommand{\scalResp}{r}

\begin{document}
\begin{center}
Ian Halvic \\
Proposal 
{\color{red}[DRAFT: Do not distribute]} \\
\end{center}


%%%%%%%%%%%%%%%%%%%%%%%%%%%%%%%%%%%%%%%%%%%%%%%%%%%%%%%%%%%%%%%%%%%%%%%%%%%%%%%%%%%%%%%%%%%%%%%%%%%%
\section{Introduction}
%%%%%%%%%%%%%%%%%%%%%%%%%%%%%%%%%%%%%%%%%%%%%%%%%%%%%%%%%%%%%%%%%%%%%%%%%%%%%%%%%%%%%%%%%%%%%%%%%%%%
{\color{red}[IWH: Notation Questions? At what point does it become acceptable to drop the independent variables ($(\vr,\vO,E)$ and such) on the flux and cross section terms? Should I be using uppercase $\Sigma$ instead of $\sigma$ for the cross-sections in this context? I know that the uppercase is more technically correct for neutrons, but we seemed to always just use lower cases when talking.]}
\subsection{Neutron transport equation}
The time dependent neutron transport equation without fission is given to be 
\begin{equation}
\frac{1}{v(E)} \frac{\partial}{\partial t}\psi + \vO \cdot \vgrad \psi + \sigt(\vr,E,t) \psi = \int_{4 \pi} d \Omega^\prime \, \int_0^\infty dE^\prime \sigs(\vr,E^\prime \to E, \vO^\prime \to \vO) \psi + S(\vr,E,\vO,t) 
\end{equation}
{\color{red}[IWH: Add boundary condition (numbering? should the BC have its own EQ number, or be lumped with the main). Introduction of terms in the TE here. Definition of relevant volume $V$ and boundary $\Gamma$]}
\\ \\
{\color{red}[IWH: Present why the TE in its complete form is unwieldy. Transition into the simplifying approximations below.]}
The first simplifying approximation to be made is that scattering is isotropic, allowing the use of an energy differential scattering cross section $\sigs(\vr,E^\prime \to E)$. This also allows for the use of the scalar flux $\phi$ in the scattering term.
\begin{equation}
\frac{1}{v(E)} \frac{\partial}{\partial t}\psi + \vO \cdot \vgrad \psi + \sigt(\vr,E,t) \psi = \frac{\phi}{4 \pi} \int_0^\infty dE^\prime \sigs(\vr,E^\prime \to E) + S(\vr,E,\vO,t) 
\end{equation}
 \\ \\
For practicality, the energy dependence is discritized into a finite number of energy groups. Let each each energy group be denoted by the subscript $g \in (1,G)$. 
\begin{equation}
\frac{1}{v(E)} \frac{\partial}{\partial t}\psi_g + \vO \cdot \vgrad \psi_g + \sigt(\vr,E,t) \psi_g = \frac{1}{4 \pi} \sum_{g^\prime=1}^G  \Sigma_{s,g^\prime}(\vr) \phi_{g^\prime} + S_g(\vr,\vO,t) 
\end{equation}
In the above equation, scattering terms for $g^\prime \neq q$ are inscattering terms, and act as a source term for a given energy group. Making the assumption that the fixed source term $S_g$ is isotropic, allows the generation of a total source term $q$.
\begin{equation}
\begin{split}
\frac{1}{v(E)} \frac{\partial}{\partial t}\psi_g &+ \vO \cdot \vgrad \psi_g + \sigt(\vr,E,t) \psi_g \\
&= \frac{1}{4 \pi} \sum_{g^\prime=1}^G  \Sigma_{s,g^\prime}(\vr) \phi_{g^\prime} + S_g(\vr,\vO,t) \\
&= \frac{1}{4 \pi} \Sigma_{s,g^\prime}(\vr) \phi_{g}
+ \frac{1}{4 \pi} \sum_{g^\prime=1}^G  \Sigma_{s,g^\prime}(\vr) \phi_{g^\prime} + S_g(\vr,\vO,t)  \quad :g^\prime \neq g\\
&= \frac{1}{4 \pi} \Sigma_{s,g^\prime}(\vr) \phi_{g}
+ \frac{q_g}{4 \pi}\
\end{split}
\end{equation}
{\color{red}[IWH: NEED TO SWITCH TO STEADY STATE AT SOME POINT! I think I need guidance on how to transition from the time dependent equation to the steady state, to transition into the steady state Sn and VET formulations we have primarily been working with]}

{\color{red}[IWH: I feel like I need to go back through the literature and cite more for the simplifying assumptions.]}


\subsection{Sensitivity}
{\color{red}[IWH: This I really just need to go back into the literature, (particularly National Research Council) and present why obtaining the sensitivity is of importance for verification/validation/UQ. What are sour of important things for me to look at here?]}

\subsection{Adjoint Method}
{\color{red}[IWH: Primarily distilling some of the basic ideas from Marchuk. Not sure how lengthy I need to go with this.]}

\subsubsection{Inner Products}
{\color{red}[IWH: Probably doesn't need it's own heading. Will introduce some QoI basics here. We have two inner products which should be unambiguous which we are talking about at any given time]}
\begin{equation}
\bra f(\vr, \vO) , g(\vr, \vO) \ket  = \int_V dV \int_{4 \pi} d \vO \,  f(\vr, \vO)g(\vr, \vO)
\end{equation}
\begin{equation}
\bra f(\vr) , g(\vr) \ket  = \int_V dV \,  f(\vr)g(\vr)
\end{equation}
\subsubsection{Generalized Adjoint Method}
\begin{equation}
\label{adjGeneral}
A^\dag \psi^\dag = \angResp
\end{equation}
\begin{equation}
\label{adjGeneral2}
QoI = \bra \psi, r \ket = \bra \psi  , A^\dag \psi^\dag \ket =  \bra A \psi  , \psi^\dag \ket =  \bra q  , \psi^\dag \ket 
\end{equation}
%%%%%%%%%%%%%%%%%%%%%%%%%%%%%%%%%%%%%%%%%%%%%%%%%%%%%%%%%%%%%%%%%%%%%%%%%%%%%%%%%%%%%%%%%%%%%%%%%%%%
\section{Discrete Ordinates (Sn) Transport}
%%%%%%%%%%%%%%%%%%%%%%%%%%%%%%%%%%%%%%%%%%%%%%%%%%%%%%%%%%%%%%%%%%%%%%%%%%%%%%%%%%%%%%%%%%%%%%%%%%%%
With the energy discritization performed, the transport equation system has become a system of $G$ equations of the form shown in equation \ref{1gTE}, coupled by the inscattering terms wrapped up in the source term $q$.
\begin{equation}
\label{1gTE}
\vO \cdot \vgrad \psi + \sigt \psi = \frac{\sigs}{4 \pi} \phi + \frac{q}{4 \pi} \quad \vr \in V 
\end{equation}
\begin{equation}
\psi(\vr,\vO) = \psi^{\text{inc}}(\vr,\vO) \quad \vr \in \Gamma, \vO \cdot \vec{n} < 0
\end{equation}
The unknowns in the above include the angular dependent flux $\psi=\psi(\vr,\vO)$ and the scalar flux $\phi=\phi(\vr)$. A discrete ordinates (Sn) method can be used to discritize the angular dependence. Using a total of $n$ angles with normalized weights $w$, the scalar flux can be represented as 
\[
\phi(\vr) \approx \sum_{j=1}^n w_j \psi_n(\vr)
\] 
where $\psi_n(\vr) = \psi(\vr, S_n)$.

{\color{red}[Put more stuff here, get to final SN form. Is the Sn form just discrete ordinates, or doe sit require full decomposition into the discretized FEM form?]}
\subsection{Adjoint Sn formulation}
In a fairly straight forward application of the adjoint method shown in equation \ref{adjGeneral2}, the adjoint equation which corresponds to the Sn transport formulation with response $\angSource$ is
\begin{equation}
\label{snAdj}
- \vO \cdot \vgrad \psi^\dag + \sigt \psi^\dag = \frac{\sigs}{4 \pi} \phi^\dag + \angSource
\end{equation}
\begin{equation}
\psi^\dag(\vr,\vO) = \psi^{\dag \text{out}}(\vr,\vO) \quad \vr \in \Gamma, \vO \cdot \vec{n} > 0
\end{equation}
where the definition of the adjoint scalar flux $\phi^\dag$ is analogous to the forward definition. It is worth noting that the SN adjoint equation is in the form of a transport equation, only with the direction of travel reversed ($\vO \to -\vO)$. This often allows for forward Sn transport solvers to be easily adapted to solving the Sn adjoint system. Once the adjoint solution is obtained, the corresponding QoI can be calculated with a simple inner product with the forward source term, as follows from equation \ref{adjGeneral2}. 
\begin{equation}
\label{snAdjQoI}
QoI = \bra \psi^\dag , \angSource \ket - \int d^2 r \, \int d  \Omega \, \psi^\dag \psi ( \vO \cdot \vec{n} )
\end{equation}
The surface interval in \ref{snAdjQoI} can be split into incoming and outgoing flux integrals, which are handled by the forward and adjoint boundary conditions respectively. 
\begin{equation}
QoI = \bra \psi^\dag , q \ket - \int d^2 r \, \int_{\vO \cdot \vec{n} >0} d  \Omega \, \psi^\dag \psi ( \vO \cdot \vec{n} ) - \int d^2 r \, \int_{\vO \cdot \vec{n} <0} d  \Omega \, \psi^\dag \psi ( \vO \cdot \vec{n} )
\end{equation}

\subsection{Sn Transport Adjoint Sensitivity}
Now consider perturbations to our system. Specifically perturbations of $\delta \sigt$, $\delta \sigs$, and $\delta q$ to the total cross section, scattering cross section and angular source term respectively. In addition, the incident angular flux is also perturbed by $delta \psi^{inc}$. These perturbations result is a perturbed solution to the Sn-transport equation $\psi_p$. Using this $\delta$ notation, the perturbed Sn-equation is shown in \ref{snPert}
\begin{equation}
\label{snPert}
\vO \cdot \vgrad \psi_p + \left( \sigt + \delta \sigt \right) \psi_p = \frac{\left( \sigs + \delta \sigs \right)}{4 \pi} \phi_p + \left( \angSource + \delta \angSource \right)
\end{equation}
\begin{equation}
\psi_p(\vr,\vO) = \psi^{\text{inc}}(\vr,\vO) + \delta \psi^{\text{inc}}(\vr,\vO)\quad \vr \in \Gamma
\end{equation}
This perturbation may result in a change to the QoI, now given by $QOI_p=\bra \psi_p , \angResp \ket$. Retaining the unperturbed adjoint equation given in \ref{snAdj} and using a first order approximation of the perturbed Sn transport equation \ref{snPert}, the perturbed QoI can be represented as an inner product not dependent on the perturbed forward solution.
\begin{equation}
\label{snSens}
\begin{split}
QoI_p &=\bra \psi_p , \angResp \ket \\
&=\bra \psi_p , - \vO \cdot \vgrad \psi^\dag + \sigt \psi^\dag - \frac{\sigs}{4 \pi} \phi^\dag  \ket \\
&= \bra  \vO \cdot \vgrad \psi_p + \sigt \psi_p - \frac{\sigs}{4 \pi} \phi_p , \psi^\dag  \ket - \int_{\Gamma} \int_{\Omega} \psi_p \psi^\dag \vO \cdot \vec{n}\\
&= \bra  \scalSource - \delta \sigt \psi + \frac{\delta\sigs}{4 \pi} \phi + \delta \scalSource , \psi^\dag  \ket - \int_{\Gamma} \int_{\Omega} \psi_p \psi^\dag \vO \cdot \vec{n}\\
\end{split}
\end{equation}
Subtraction of the unperturbed $QoI$ expression in equation \ref{snAdjQoI} supplies a final equation for computing the change in $QoI$ using only the system perturbations and the unperturbed forward and adjoint solutions, removing the need to solve the perturbed forward equation.
\begin{equation}
\delta QoI_p = \bra - \delta \sigt \psi + \frac{\delta\sigs}{4 \pi} \phi + \delta \scalSource , \psi^\dag  \ket - \int_{\Gamma} \int_{\Omega} \delta \psi \psi^\dag \vO \cdot \vec{n}\\
\end{equation}

%%%%%%%%%%%%%%%%%%%%%%%%%%%%%%%%%%%%%%%%%%%%%%%%%%%%%%%%%%%%%%%%%%%%%%%%%%%%%%%%%%%%%%%%%%%%%%%%%%%%
\section{VET formulation}
%%%%%%%%%%%%%%%%%%%%%%%%%%%%%%%%%%%%%%%%%%%%%%%%%%%%%%%%%%%%%%%%%%%%%%%%%%%%%%%%%%%%%%%%%%%%%%%%%%%%
\subsection{Motivation} 
While the Sn adjoint sensitivity formulation given by \ref{snSens} provides a first-order accurate method to determine the sensitivity to multiple perturbation scenarios using only a single forward and adjoint transport solve, it can quickly run into technical limitations. Specifically for even a one-group time independent system, all solutions to the forward and adjoint system in space and angle must be stored for retrieval later. For a spatially 3-dimensional system, this translates to storing descritized data across 5-dimensions.


\subsection{VET Formulation}
The "Variable Eddington Tensor" (VET) formulation shows promise of reducing the memory requirements when using the adjoint method for sensitivity. To begin the formulation, the steady state transport equation is expanded to the scalar and first angular moment by application of the $\int d \Omega$ and $\int d \Omega \, \vO$ operators to equation \ref{1gTE}, respectively. Using the notation
\begin{equation}
\phi=\int d\Omega \, \psi( \vO )
,\quad
\vec{J}= \int d\Omega \, \vO \psi( \vO )
\end{equation}
the scalar and first angular moment transport equations are
\begin{equation}
\label{0am}
\vdiv \vec{J} + \sigt \phi = \sigs \phi + \scalSource
\end{equation}
\begin{equation}
\label{1am}
\vdiv \left(  \int d\Omega \vO \vO \psi \right) + \sigt \vec{J} =0 
\end{equation}
The Eddington Tensor $\Edd$ is then introduced as a simplifying approximation relating the second angular moment term in equation \ref{1am} to the scalar flux. 
\begin{equation}
\label{EddDef}
\Edd=\frac{\int d\Omega \vO \vO \psi}{\phi}
\end{equation}
The inclusion of the Eddington tensor allow equation \ref{1am} to be expressed as $\sigt \vec{J} = - \vdiv \Edd \phi$. Using this as a definition of $\vec{J}$ allows us to convert \ref{0am} to the form shown in \ref{VEFForm}, which only has the scalar flux as an unknown. The substitution $\siga = \sigt-\sigs$ was used.
\begin{equation}
\label{VEFForm}
- \vdiv \left( \frac{1}{\sigt}\vdiv \Edd \phi \right) + \siga \phi = \scalSource
\end{equation}
The known incident flux can be used to generate a suitable boundary condition using a "Boundary Eddington Factor" $B$ \cite{Miften}.
\begin{equation}
\frac{1}{\sigma_{t} } \vec{\nabla} \cdot \left(\Edd \phi \right)  = - 2J^- - B \phi \quad \vr \in \Gamma
\end{equation}
\begin{equation}
B= \frac{\int_{4 \pi} d\Omega \, \left| \Omega_i n_i \right | \psi}{\int_{4\pi} d\Omega \, \psi} \quad \vr \in \Gamma
\end{equation}

\subsection{Adjoint VET formulation}
The VET formulation necessitates a reformulation of our adjoint. As shown in \ref{adjForm}, the double divergence term present in the forward equation contributes to a double gradient term in the adjoint equation.
\begin{equation}
\label{adjForm}
- \Edd : \left( \vgrad \left( \frac{1}{\sigt}\vgrad \phi^\dag \right) \right) + \siga \phi^\dag = \scalResp
\end{equation}
For reasons that will become apparent during sensitivity calculations, the boundary condition chosen for the adjoint equation is given in \ref{adjVETBC}. Unlike for Sn formulation, the VET adjoint equation does not take the form of a VET transport equation.
\begin{equation}
\label{adjVETBC}
\Edd \cdot \frac{1}{\sigma_{t} } \vec{\nabla} \phi^\dag  = - 2J^{\dag +} - B \phi^\dag \quad \vr \in \Gamma
\end{equation}
To obtain the QoI using this formulation, begin with the typical QoI definition, relocate operators to the forward to substitute the forward source, and apply boundary conditions.
\begin{equation}
\label{VETQoIAdjUnpDeriv}
\begin{split}
QoI=&\bra \phi , \scalResp \ket \\
=&\bra \phi , - \Edd : \left( \vgrad \isigt \vgrad \phi^\dag \right) + \siga \phi^\dag \ket \\
=& \bra - \vdiv \isigt \vdiv \left( \Edd \phi \right) + \siga \phi, \phi^\dag \ket 
- \int_\Gamma d^2 r \, \phi \left( \Edd \cdot \isigt \vgrad \phi^\dag \right) \cdot \vec{n}  \\ 
&+ \int_\Gamma d^2 r \, \phi^\dag \left(  \isigt \vgrad \Edd \phi \right) \cdot \vec{n} \\
=&\bra \scalSource , \phi^\dag \ket 
- \int_\Gamma d^2 r \, \phi \left( - 2J^{\dag +} - B \phi^\dag \right) \cdot \vec{n} + \int_\Gamma d^2 r \, \phi^\dag \left( - 2J^- - B \phi  \right) \cdot \vec{n}
\end{split}
\end{equation}
The boundary Eddington terms negate and yield a relatively compact form for the QoI
\begin{equation}
\label{VETQoIAdj}
QoI=\bra \scalSource , \phi^\dag \ket 
+ \int_\Gamma d^2 r \, 2  \left( \phi J^{\dag +}  - \phi^\dag J^- \right) \cdot \vec{n}
\end{equation}

\subsection{VET adjoint sensitivity}
As was done in the Sn transprt formulation, once again consider perturbations to the system parameters. However, in contrast to the Sn case, the assumption is also made that the Eddington factor remains unperturbed under these system perturbations. For brevity, the substitution of $\isigt = \sigt^{-1}$ was also made.
\begin{equation}
\label{VEFPert}
- \vdiv \left((\isigt + \delta \isigt)\vdiv \Edd \phi_p \right) + (\siga + \delta \siga)\phi_p = \scalSource + \delta \scalSource
\end{equation}
\begin{equation}
(\isigt + \delta \isigt) \vec{\nabla} \cdot \left(\Edd \phi_p \right)  = - 2J_p^- - B \phi_p \quad \vr \in \Gamma
\end{equation}
The usual adjoint process is performed, starting with the QoI definition using the perturbed solution and response. 
\begin{equation}
\label{VETSensDeriv}
\begin{split}
QoI=&\bra \phi_p , \scalResp \ket \\
=&\bra \phi_p , - \Edd : \left( \vgrad \isigt \vgrad \phi^\dag \right) + \siga \phi^\dag \ket \\
=& \bra - \vdiv \isigt \vdiv \left( \Edd \phi_p \right) + \siga \phi_p, \phi^\dag \ket 
- \int_\Gamma d^2 r \, \phi_p \left( \Edd \cdot \isigt \vgrad \phi^\dag \right) \cdot \vec{n}  \\ 
&+ \int_\Gamma d^2 r \, \phi^\dag \left(  \isigt \vgrad \Edd \phi_p \right) \cdot \vec{n} \\
\end{split}
\end{equation}
A first order perturbation approximation of equation \ref{VEFPert} can be used to substitute into the sensitivity equation \ref{VETSensDeriv}, yielding a form independ of the perturbed forward solution.
\begin{equation}
\label{QoIVETAdjNoBC}
\begin{split}
QoI =& \bra \scalSource + \delta \scalSource + \vdiv \delta \isigt \vdiv \left( \Edd \phi \right) - \delta \siga \phi, \phi^\dag \ket - \int_\Gamma d^2 r \, \phi_p \left( \Edd \cdot \isigt \vgrad \phi^\dag \right) \cdot \vec{n} 
\\ &+ \int_\Gamma d^2 r \, \phi^\dag \left(  \isigt \vdiv \Edd \phi_p \right) \cdot \vec{n} 
\end{split}
\end{equation}
The first surface term can be dealt with readily using the adjoint boundary condition. For the second surface term, a first order approximation of the perturbed forward boundary condition is used for substitution.
\begin{equation}
\label{QoIVETAdj}
\begin{split}
QoI =& \bra \scalSource + \delta \scalSource + \vdiv \delta \isigt \vdiv \left( \Edd \phi \right) - \delta \siga \phi, \phi^\dag \ket - \int_\Gamma d^2 r \, \phi_p \left( - 2J^{\dag +} - B \phi^\dag \right) \cdot \vec{n} 
\\ &+ \int_\Gamma d^2 r \, \phi^\dag \left( - 2J_p^- - B \phi_p - \delta \isigt \vdiv \Edd \phi \right) \cdot \vec{n} 
\end{split}
\end{equation}
Subtract the adjoint QoI formulation from equation \ref{VETQoIAdj} to obtain the sensitivity expression for the adjoint VET formulation.
\begin{equation}
\label{SensVETAdjNoBC}
\begin{split}
\delta QoI =& \bra \delta \scalSource + \vdiv \delta \isigt \vdiv \left( \Edd \phi \right) - \delta \siga \phi, \phi^\dag \ket + \int_\Gamma d^2 r \, 2  \left( \delta \phi J^{\dag +}  - \phi^\dag \delta J^- \right) \cdot \vec{n}
\\ &- \int_\Gamma d^2 r \,  \phi^\dag \left( \delta \isigt \vdiv \Edd \phi \right) \cdot \vec{n} 
\end{split}
\end{equation}


\subsection{Error from unperturbed Eddington assumption}
Beyond the first order approximation common to adjoint formulations, the VET sensitivity formulation also made the assumption that the Eddington tensor remained unperturbed under perturbations of the other parameters. To observe the terms that were dropped in this approximation, consider a reformulation of the perturbed forward equation, this time introducing $\delta \Edd$ and $\delta  B$ terms. 
\begin{equation}
\label{VEFPerEdd}
- \vdiv \left((\isigt + \delta \isigt)\vdiv (\Edd + \delta \Edd) \phi_p \right) + (\siga + \delta \siga)\phi_p = \scalSource + \delta \scalSource
\end{equation}
\begin{equation}
(\isigt + \delta \isigt) \vec{\nabla} \cdot \left((\Edd + \delta \Edd) \phi_p \right)  = - 2J_p^- - (\BEdd +\delta \BEdd) \phi_p \quad \vr \in \Gamma
\end{equation}
The above can be substituted into equation \ref{VETSensDeriv} to yield an expanded QoI equation, including the Eddington perturbation terms
\begin{equation}
\label{QoIVETAdjNoBCEdd}
\begin{split}
QoI =& \bra \scalSource + \delta \scalSource + \vdiv \delta \isigt \vdiv \left( \Edd \phi \right) + \vdiv \isigt \vdiv \left( \delta \Edd \phi \right) - \delta \siga \phi, \phi^\dag \ket \\
&- \int_\Gamma d^2 r \, \phi_p \left( \Edd \cdot \isigt \vgrad \phi^\dag \right) \cdot \vec{n} 
+ \int_\Gamma d^2 r \, \phi^\dag \left(  \isigt \vdiv \Edd \phi_p \right) \cdot \vec{n} 
\end{split}
\end{equation}
The boundary condition for the perturbed forward solution takes on a slightly more complex form, as the additional $\delta \Edd$ and $\delta \BEdd$ terms come into play, but the derivation of the sensitivity proceeds similarly to the case ignoring Eddington perturbations.
\begin{equation}
\label{QoIVETAdjEdd}
\begin{split}
\delta QoI =& \bra \delta \scalSource + \vdiv \delta \isigt \vdiv \left( \Edd \phi \right) + \vdiv \isigt \vdiv \left( \delta \Edd \phi \right) - \delta \siga \phi, \phi^\dag \ket \\
&+ \int_\Gamma d^2 r \, 2  \left( \delta \phi J^{\dag +}  - \phi^\dag \delta J^- \right) \cdot \vec{n}
- \int_\Gamma d^2 r \,  \phi^\dag \left( \delta \isigt \vdiv \Edd \phi \right) \cdot \vec{n}
\\
&- \int_\Gamma d^2 r \,  \phi^\dag \left( \isigt \vdiv \delta \Edd \phi \right) \cdot \vec{n}
- \int_\Gamma d^2 r \,  \phi^\dag \phi \delta \BEdd \cdot \vec{n}
\end{split}
\end{equation} 
Comparing the above formulation with the unperturbed Eddington case shows that the perms lost by the Unperturbed Eddington assumption are 
\begin{equation}
\label{EddErr}
 \bra \vdiv \isigt \vdiv \left( \delta \Edd \phi \right), \phi^\dag \ket
- \int_\Gamma d^2 r \,  \phi^\dag \left( \isigt \vdiv \delta \Edd \phi \right) \cdot \vec{n}
- \int_\Gamma d^2 r \,  \phi^\dag \phi \delta \BEdd \cdot \vec{n}
\end{equation} 

\section{Preliminary results}
{\color{red}[IWH: Explain what I am looking at (VET adjoint versus Sn adjoint, versus multiple forward solves). I need to touch up these graphs (do them in Matlab, not Excel). Introduce the FEM parameters used.]}
\subsection{Homogeneous initial system, Uniform Perturbations}
[Define system,  perturbations, and QoI. Vary MFP between cases.]

\subsection{Homogeneous initial system, Non-Uniform Perturbations}
[Define system,  perturbations, and QoI. Vary MFP between cases.]

\subsection{Streaming system}
[Define system,  perturbations, and QoI.]

\section{Goals}
$\bullet$ Introduce Variable Eddington Tensor (VET) formulation of the transport equation with boundary conditions presented in Miften and Larson.
\\ \\
$\bullet$ Formulate the adjoint equation corresponding to the VET transport formulation. Once formulated, working under the assumption of an unperturbed Eddington Tensor, define the inner products required for sensitivity calculations due to perturbations in cross sections, sources, and incident flux; using a first order approximation.
\\ \\
$\bullet$ Derive terms that were lost in the first order adjoint sensitivity formulation due to the assumption that the Eddington remains unperturbed. 
\\ \\
$\bullet$ Implement the Sn and VET adjoint methods in a 1D FEM solver. Using simple 1D test cases, compare the sensitivity values yielded by the VET adjoint method to those obtained by Sn adjoint, as well as sensitivity obtained directly by multiple forward system solves in both Sn and VET.
\\ \\
$\bullet$ Consider the alternate boundary conditions presented in Wieselquest. If promise is shown, derive the sensitivity inner products using this boundary condition. Repeat select test cases of the 1D implementation.


\newpage

{\color{red}[IWH: I need to get all my resources into a bibtex file+formatting.]}
\begin{thebibliography}{9}


\bibitem{Miften}
  M.M. Miften and Edward W. Larsen, \emph{A Symmetrized Quasidiffusion Method For Solving Transport Problems In Multidimensional Geometeries}, University of Michigan, Ann Arbor Michigan 1992.
  
  
\bibitem{Marchuk}
  Guri I. Marchuk, \emph{Adjoint Equations and Analysis of Complex Systems}, Institute of Numerical Mathematics, Russian Academy of Sciences, Moscow, Russia 1995. Springer.
  
  
\bibitem{Wiesel}
  William A. Wieselquest, \emph{The Quasidiffusion Method for Transport Problems on Unstructured Meshes}, North Carolina State University, 2009.

\bibitem{NRCVVUQ}
  National Research Council, \emph{Assessing the Reliability of Complex Models. Mathematical and Statistical Foundations of Verification, Validation, and Uncertainty Quantification.}, Washington, D.C. 2012. The National Academic Press.



\end{thebibliography}

\textbf{This document was created using \LaTeX}

\end{document}