%%%%%%%%%%%%%%%%%%%%%%%%%%%%%%%%%%%%%%%%%%%%%%%%%%%%%%%%%%%%%%%%%%%%%%
%%                           SECTION I
%%%%%%%%%%%%%%%%%%%%%%%%%%%%%%%%%%%%%%%%%%%%%%%%%%%%%%%%%%%%%%%%%%%%%


\pagestyle{plain} % No headers, just page numbers
\pagenumbering{arabic} % Arabic numerals
\setcounter{page}{1}


\chapter{\uppercase {Introduction and Literature Review}}

%%%%%%%%%%%%%%%%%%%%%%%%%%%%%%%%%%%%%%%%%%%%%%%%%%%%%%%%%%%%%%%%%%%%%%%%%%%%%%%%%%%%%%%%%%%%%%%%%%%%
\section{Introduction}
%%%%%%%%%%%%%%%%%%%%%%%%%%%%%%%%%%%%%%%%%%%%%%%%%%%%%%%%%%%%%%%%%%%%%%%%%%%%%%%%%%%%%%%%%%%%%%%%%%%%

Computational simulations have become important tools for engineers and scientists across a wide array of disciplines. These simulations allow for researchers to examine significantly complex and long life systems in a way that is frequently more economical in both time and money than construction of the real world system, if even feasible. An important step in creating one of these methods is confirmation that the results can be trusted to reasonably approximate the real life scenario. This can be accomplished using three process outlined by the National Research Council \cite{NRCVVUQ}


\begin{itemize}
\item Verification - How accurately does the computation solve the underlying equations of the model for the quantities of interest?
\item Validation - How accurately does the model represent reality for the quantities of interest?
\item Uncertainty Quantification (UQ) -  How do the various sources of error and uncertainty feed into uncertainty in the model-based prediction of the quantity of interest?
\end{itemize}


Adjoint methods are particularly useful for UQ. In general, adjoint methods provide a mechanism for computing quantities of interest (\qoi) and sensitivity coefficients and, more generally, for propagating uncertainty and error in the system variables to the error in the desired quantity of interest. They accomplish this in a particularly economical way, requiring only two full system solves (one forward and one adjoint solve) in order to determine sensitivity coefficients for any combination of sources of uncertainty, as opposed to performing multiple forward system solves to determine
the sensitivity due to each individual uncertain parameter.

Using operator notation to denote the forward and adjoint (linear) problems, $Au=q$ and $A^\dag u^\dag = q^\dag$, respectively, and the bracket notation for inner products over the phase space, it is easy to note that a \qoi can 
obtained using either the forward solution $u$ folded with the response function of interest (which is also the adjoint source, $q^\dag$)
\[
\qoi = \bra u, q^\dag \ket = \bra u^\dag, q \ket \,.
\]
A first-order sensitivity coefficient due to parameter $p$ can be obtained using two forward solves
\[
\delta_p \qoi = \frac{\qoi^\prime - \qoi}{\delta p} = \frac{ \bra (u^\prime - u), q^\dag \ket}{p^\prime - p}  \,,
\]
where the superscript $^\prime$ denotes a perturbed value. Equivalently, one can employ the unperturbed adjoint solution to obtain
an estimation of the sensitivity coefficient
\[
\delta_p \qoi = \bra u^\dag, \delta_p q - \delta_p A u \ket \,.
\]
The subsequent sections will provide additional details and will apply these formulations to neutron transport problems. However, it is already evident that if the forward source $q$ is uncertain and only a few \qoi are requested, it is preferable to employ the adjoint formulation to compute the quantity of interest. Furthermore, the sensitivity of the
\qoi to many uncertain parameters is more advantageously computed using the adjoint.


Adjoint methods are of particular interest for UQ. In general, adjoint methods provide a mechanism for propagating uncertainty and error in the system variables to the error in the desired quantity of interest. Adjoint methods accomplish this in a particularly economical way, sometimes requiring only two differential system solves which can then be used for any combination of sources of error, as opposed to performing an independent solve for each individual error scenario. These adjoint methods have been applied across various complex and time dependent systems. An example of adjoint methods applied to hydrodynamic systems with shocks can be found in Wildey et al. \cite{Wildey}. A more relevant adjoint example to neutron transport occurs in Stripling et al. in the form of reactor burn-up equations \cite{Stripling}.


Application of the adjoint method to time-dependent transport can pose a major technical limitation. In general, the adjoint method requires storing six-dimensional data (the forward angular flux) at each time step. When dealing with high resolution in these six dimensions and many time steps, this can potentially require an unreasonable amount of memory for data storage, rendering the method functionally unusable. For the previously mentioned depletion problem, this limitation was resolved by storing the converged scattering moments only at defined checkpoints then using interpolation to reconstruct the forward flux for use in the adjoint \cite{Stripling};
using converged scattering souce term, a single transport sweep is needed to recover the converged
angular flux. However this method does not extend to time-dependent transport in general because
the primary time-dependent variable, the angular flux, needs to be stored. 

A potential solution to the memory requirement for the time-dependent transport adjoint formulation is the use of a quasi-diffusion method to reduce the overall dimensionality of the transport problem, from 6D+time (space, direction, energy, time for transport) to 4D+time (space, energy, time for quasi-diffusion). The method examined in this work is termed  as a ``Variable Eddington Tensor'' formulation, and uses the unperturbed forward angular flux to compute the Eddington tensor needed in the quasi-diffusion approach.


%%%%%%%----------------------------------------------------------------------------------------------
\subsection{Steady-state one-group neutron transport equation}
%%%%%%%----------------------------------------------------------------------------------------------

This work will focus on a relatively simple transport equation form, the one-group steady-state transport equation. Examination of the quasi-diffusion's effectiveness in this setting will provide sufficient insight to advantages and shortcomings of the technique when applied to multigroup, time-dependent transport equation. The one-group steady-state transport equation with isotropic scattering for a volume $V$ bounded by its surface $\partial V$ is given below.

\begin{equation}
\label{SS1GTE}
\vO \cdot \grad \psi(\vr,\vO) + \sigt(\vr) \psi(\vr,\vO) = \frac{1}{4 \pi} \sigs(\vr) \phi(\vr) + q(\vr,\vO), \quad \forall \vr \in V
\end{equation}
\begin{equation}
\label{SS1GTE_bc}
\psi(\vr,\vO) = \psi^{\text{inc}}(\vr,\vO) \quad \vr \in \partial V^{-} = \{ \vr \in \partial V, \text{ s.t. }, \vO \cdot \vec{n}(\vr) < 0\}
\end{equation}
The possibly uncertain parameters in this system are: the total and scattering cross sections $\sigt$ and $\sigs$, the volumetric source term $q$, and the incident angular flux on the system given by $\psi^{\text{inc}}$. The unknowns (dependent variables) are the angular flux $\psi(\vr,\vO)$ and the scalar flux $\phi(\vr)$ given by
\[
\phi(\vr) = \int_{4\pi}d\Omega\,\psi(\vr,\vO) \,.
\]
%%%%%%%----------------------------------------------------------------------------------------------
%\subsection{Quantity of interest and their sensitivities}
%%%%%%%----------------------------------------------------------------------------------------------

%----------------------------------------------------------------------------------------------
\subsubsection{Quantity of interest, response function, and inner products}
%----------------------------------------------------------------------------------------------
Frequently, the solution to the transport equation is not the sought after value, but rather some Quantity of Interest (\qoi), a functional that depends on the transport solution. Given $\psi(\vr,\vO)$ the solution of the one-group steady-state transport (Eq.~\eqref{SS1GTE}), a \qoi
defined as
\begin{equation}
\qoi =  \int_V dV \int_{4 \pi} d \vO \,  R(\vr, \vO) \psi(\vr, \vO)
\end{equation}
where $R(\vr, \vO)$ is termed the ``response function'' which is used to characterize the meaning of the desired \qoi. The response function can take on physically defined forms, such as the cross section of a detector; or it may take a form of a mathematical construct, such as $R(\vr, \vO)=1/v$ to return the total neutrons of speed $v$ present in the system. To avoid confusion with the spatial location vector $\vr$, the response function will frequently be expressed as $q^\dag$, the adjoint source, as we have already noted that there is a relationship between the solution, the adjoint solution, and their respective source terms.

For the sake of brevity in this paper, particularly for expressing the \qoi, two volumetric inner products are defined both using $\bra \bullet , \bullet \ket$ notation. These two inner-products are for use with angular and scalar flux respectively. 
\begin{subequations}
\begin{equation}
\bra \psi , \angResp \ket_{V \times \mathcal{S}_2}  = \int_V dV \int_{4 \pi} d \Omega \,  \psi(\vr, \vO)\angResp(\vr, \vO)
\end{equation}
where $\mathcal{S}_2$ denotes the unit sphere.
\begin{equation}
\bra \phi(\vr) ,\angResp \ket_V  = \int_V dV \,  \phi(\vr)\angResp(\vr)
\end{equation}
\end{subequations}
For later use, two additional inner products are also defined as surface integrals over the region boundary $\partial V$. The latter splinting incoming and outgoing surface integrals.
\begin{subequations}
\begin{equation}
\sbra \psi , g \sket_{\bound \times \mathcal{S}_2}  = \int_{\bound} dS \int_{4 \pi} d \Omega \, \vO \cdot \vn(\vr) \, \psi(\vr, \vO)g(\vr, \vO)
\end{equation}
\begin{equation}
\sbra \psi , g \sket_{\pm}   = \int_{\bound} dS \int_{\vO \cdot \vn \gtrless 0} d\Omega \,  \vO \cdot \vn(\vr) \, \psi(\vr, \vO)g(\vr, \vO)
\end{equation}
\end{subequations}

The inner product subscripts will frequently be dropped when it is unambiguous which one being used, typically indicated by the presence of an angular-dependent variable, e.g., $\psi$ or an angular-independent variable, e.g., $\phi$. Therefore, with this notation, the quantity of interest can be compactly expressed as shown below.
\begin{equation}
\label{QoIDef}
\qoi = \bra \psi(\vr,\vO), \angResp(\vr,\vO) \ket  = \bra \phi(\vr) , \scalResp(\vr) \ket
\end{equation}
The boundary terms arise from the integration by parts of the streaming terms.

%----------------------------------------------------------------------------------------------
\subsubsection{Sensitivity Coefficients}
%----------------------------------------------------------------------------------------------
A hurdle in utilizing the transport equation numerically to make real world predictions is that none of the system's parameters ($\sigt$, $\sigs$, $q$, and $\psi^{inc}$) are not known exactly. This error in the system parameters is expected to translate to error in the \qoi value. Ideally, a reasonable error range would be determined for each system parameter and the system simulation would run over a finely discretized parameter space, using the resulting \qoi values to generate an error range for the \qoi. However, this straightforward method tends to be resource-intensive, requiring a complete forward solve of the transport equation for each input error scenario. Adjoint methods offer a way to drastically reduce the number of solves, while generally remaining fairly accurate for small perturbations around base or nominal values of the parameters.


%%%%%%%----------------------------------------------------------------------------------------------
\subsection{Adjoint Sensitivity}
%%%%%%%----------------------------------------------------------------------------------------------

Adjoint operators can provide a useful tool for sensitivity calculations. Using inner product notation $\bra \cdot , \cdot \ket$, consider the system of interest $A \psi = q$. Call this this the forward system, with forward operator $\mathbf{A}$. Consider a test function $\psi^\dag$, the adjoint operator $A^\dag$ is defined such that $\bra \mathbf{A} \psi, \psi^\dag \ket = \bra \psi, A^\dag \psi^\dag \ket $. For differential operators, derivation of $A^\dag$ generally relies on application of the divergence theorem (integration by parts), typically resulting in boundary terms ($BC$). Using the response function of the desired \qoi, the adjoint system can be constructed as $\mathbf{A^\dag} \psi^\dag = q^\dag$, leading to an alternate expression of the \qoi using the adjoint solution $\psi^\dag $.
\begin{equation}
\label{genAdjQoI}
\qoi = \bra \psi, R \ket = \bra q , \psi^\dag \ket + BC
\end{equation} 
From the above, it follows that a first order approximation to the change in the quantity of interest based on perturbations to the initial system, including perturbation to the forward operator $\delta \mathbf{A}$ and forward source $\delta q$, can be expressed in the form shown in Eq.~\eqref{genAdjSens} \cite{Marchuk}.
\begin{equation}
\label{genAdjSens}
\delta \qoi \approx \bra \delta q - \delta \mathbf{A} \psi , \psi^\dag \ket 
\end{equation}
The advantage of the above expression for $\delta \qoi$ is that two solves, one for the forward and another for the adjoint, can be used to approximate the sensitivity for a variety $\delta \mathbf{A}$ and $\delta q$.
\jcr{in the thesis, you need to go through the algebra for the transport and the VET equations. not here, this is only the proposal}
\iwh{Will do}
 

