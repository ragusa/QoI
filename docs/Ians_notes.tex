\documentclass[11pt]{article}
% PACKAGES
\usepackage{graphicx}
\usepackage{tabls}
\usepackage{afterpage}
\usepackage{amsmath}
\usepackage{amsfonts}
\usepackage{amssymb}
\usepackage{amstext}
\usepackage{amsbsy}
\usepackage{epsfig}
%\usepackage{cites}
\usepackage{epsf}
\usepackage{float} 
\usepackage{color} 

\usepackage{array}
\usepackage[section]{placeins} % force � mettre l'image o� on veut
\usepackage{lscape} %utilisation du mode paysage
\usepackage{xspace}

%\usepackage[pdftex,bookmarks=true]{hyperref}
%\usepackage{hyperref}
\usepackage{url}
\usepackage{verbatim}
%\usepackage[all]{hypcap}

\usepackage[labelsep=quad]{caption} % needed by the breakalgo environment

\usepackage{ifthen}
\usepackage{subfig}

\usepackage{algorithmic}
\usepackage{algorithm}
\usepackage{listings}
%\usepackage[noprefix]{nomencl}  % for nomenclature

%-----------------------------------------------------------
% NEW  DEFINITIONS
% margin par
\newcommand{\mt}[1]{\marginpar{ {\footnotesize #1} }}
% vector shortcuts
\newcommand{\vo}{\vec{\Omega}}
\newcommand{\vr}{\vec{r}}
\newcommand{\vn}{\vec{n}}
\newcommand{\vnk}{\vec{\mathbf{n}}}

% More Quick Commands
% 
\newcommand{\bi}{\begin{itemize}}
\newcommand{\ei}{\end{itemize}}
\newcommand{\ben}{\begin{enumerate}}
\newcommand{\een}{\end{enumerate}}

\renewcommand{\div}{\vec{\nabla}\! \cdot \!}
\newcommand{\grad}{\vec{\nabla}}
\newcommand{\oa}[1]{\fn{\frac{1}{3}\hat{\Omega}\!\cdot\!\overrightarrow{A_{#1}}}}

% common reference commands
\newcommand{\eqt}[1]{Eq.~(\ref{#1})}                     % equation
\newcommand{\fig}[1]{Fig.~\ref{#1}}                      % figure
\newcommand{\tbl}[1]{Table~\ref{#1}}                     % table

%
% Equation beginnings and endings
%
\newcommand{\bea}{\begin{eqnarray}}
\newcommand{\eea}{\end{eqnarray}}

\newcommand{\be}{\begin{equation}}
\newcommand{\ee}{\end{equation}}
\newcommand{\beas}{\begin{eqnarray*}}
\newcommand{\eeas}{\end{eqnarray*}}
\newcommand{\bdm}{\begin{displaymath}}
\newcommand{\edm}{\end{displaymath}}
%
\newcommand{\vj}{\vec{J}}
\newcommand{\sa}[1]{\sigma_{a #1}}
\newcommand{\vl}{\vec{\lambda}}
\newcommand{\vdj}{\delta \vec{J}}
\newcommand{\dphi}{\delta \Phi}
\newcommand{\lmax}{\ensuremath{L_{\textit{max}}}\xspace}
\newcommand{\pmax}{\ensuremath{p_{\textit{max}}}\xspace}
\newcommand{\sddx}{\frac{d}{dx}}
\newcommand{\sddt}{\frac{d}{dt}}
\newcommand{\der}[2]{\frac{\partial #1}{\partial #2}}
\newcommand{\vF}{\vec{F}}

\renewcommand{\O}{\mathcal{O}}
\newcommand{\mc}[1]{\mathcal{#1}}
\newcommand{\us}{{u^\ast}}

\newcommand{\ti}{ {t^{\text{init}}} }
\newcommand{\te}{ {t^{\text{end }}} }

\newcommand{\tcr}[1]{\textcolor{red}{#1}}

%-----------------------------------------------------------
\addtolength{\hoffset}{-2.0cm}
\addtolength{\textwidth}{4cm}
\addtolength{\textheight}{4.0cm}
\addtolength{\voffset}{-1.8cm}
\addtolength{\headsep}{-0.3cm}

\setlength{\parindent}{0pt}
\setlength{\parskip}{1.8ex plus 0.5ex minus 0.2ex}

\linespread{1.1}
%-----------------------------------------------------------
\begin{document}
%-----------------------------------------------------------
%%%%%%%%%%%%%%%%%%%%%%%%%%%%%%%%%%%%%%%%%%%%%%%%%%%%%%%%%%%%%%%%%%%%%%%%%%%%%%%%%%%%%%%%
%%%%%%%%%%%%%%%%%%%%%%%%%%%%%%%%%%%%%%%%%%%%%%%%%%%%%%%%%%%%%%%%%%%%%%%%%%%%%%%%%%%%%%%%

%%%%%%%%%%%%%%%%%%%%%%%%%%%%%%%%%%%%%%%%%%%%%%%%%%%%%%%%%%%%%%%%%%%%%%%%%%%%%%%%%%%%%%%%%%%%%%%%%%%%
%%%%%%%%%%%%%%%%%%%%%%%%%%%%%%%%%%%%%%%%%%%%%%%%%%%%%%%%%%%%%%%%%%%%%%%%%%%%%%%%%%%%%%%%%%%%%%%%%%%%
\begin{center}
{\huge Ians Notes}\\
{\Large Ian Halvic, Jean C. Ragusa, TAMU}\\
{\Large \textcolor{red}{No distribution please}}\\
\end{center}
%%%%%%%%%%%%%%%%%%%%%%%%%%%%%%%%%%%%%%%%%%%%%%%%%%%%%%%%%%%%%%%%%%%%%%%%%%%%%%%%%%%%%%%%%%%%%%%%%%%%

\tableofcontents
\pagebreak



%%%%%%%%%%%%%%%%%%%%%%%%%%%%%%%%%%%%%%%%%%%%%%%%%%%%%%%%%%%%%%%%%%%%%%%%%%%%%%%%%%%%%%%%
%%%%%%%%%%%%%%%%%%%%%%%%%%%%%%%%%%%%%%%%%%%%%%%%%%%%%%%%%%%%%%%%%%%%%%%%%%%%%%%%%%%%%%%%
\section{A nonlinear example: nonlinear heat conduction}
%%%%%%%%%%%%%%%%%%%%%%%%%%%%%%%%%%%%%%%%%%%%%%%%%%%%%%%%%%%%%%%%%%%%%%%%%%%%%%%%%%%%%%%%
%%%%%%%%%%%%%%%%%%%%%%%%%%%%%%%%%%%%%%%%%%%%%%%%%%%%%%%%%%%%%%%%%%%%%%%%%%%%%%%%%%%%%%%%

%%%%%%%%%%%%%%%%%%%%%%%%%%%%%%%%%%%%%%%%%%%%%%%%%%%%%%%%%%%%%%%%%%%%%%%%%%%%%%%%%%%%%%%%
\subsection{QoI: quantity of interest FROM RAGUSA}
%%%%%%%%%%%%%%%%%%%%%%%%%%%%%%%%%%%%%%%%%%%%%%%%%%%%%%%%%%%%%%%%%%%%%%%%%%%%%%%%%%%%%%%%

Consider the following \tcr{nonlinear} heat conduction problem ($A(\tcr{T})T=q$)
\be
\label{eq:ex_heat_conduction}
-\frac{d}{dx}k\tcr{(T)}\frac{dT}{dx}=q \qquad x\in [a,b]
\ee
with boundary conditions
\begin{subequations}
\label{eq:ex_heat_conduction_bc}
\be
T(a) = T_{\text{dir}} 
\ee
\be
-k(T) \left.\frac{dT}{dx}\right|_b = h(T(b)-T_{\infty})
\ee
\end{subequations}

Our QoI is a functional of the solution
\be
\label{eq:ex_heat_conduction_qoi}
J = \int_a^b r(x) T(x) dx
\ee
where $r(x)$ is the response function.

Consider the following adjoint problem ($A^\star(\tcr{T})T=r$)
\be
\label{eq:ex_heat_conduction_adjoint}
-\frac{d}{dx}k(T)\frac{d\phi}{dx}=r \qquad x\in [a,b]
\ee
with boundary conditions
\begin{subequations}
\label{eq:ex_heat_conduction_bc_adjoint}
\be
\phi(a) = \phi_{\text{dir}} = 0 
\ee
\be
-k(T) \left.\frac{d\phi}{dx}\right|_b = h(\phi(b)-\phi_{\infty}) = h\phi(b)
\ee
\end{subequations}
where we have chosen $\phi_{\text{dir}} = 0$ and $\phi_{\infty}=0$ for reasons that were explained previously. Do note that the adjoint problem is now \tcr{linear} but requires that the forward solution $T$ be known.

The rest of the derivation for the QoI, $J$, is identical to the one for the linear heat conduction problem. We do not reproduce here and just give the final result
\be
\boxed{
J := \int_a^b rT dx = \int_a^b q \phi dx + h T_{\infty} \phi(b) + \left.k\frac{d\phi}{dx}\right|_a T_{\text{dir}}
}
\ee


%%%%%%%%%%%%%%%%%%%%%%%%%%%%%%%%%%%%%%%%%%%%%%%%%%%%%%%%%%%%%%%%%%%%%%%%%%%%%%%%%%%%%%%%
\subsection{Sensitivity for a nonlinear problem FROM RAGUSA}
%%%%%%%%%%%%%%%%%%%%%%%%%%%%%%%%%%%%%%%%%%%%%%%%%%%%%%%%%%%%%%%%%%%%%%%%%%%%%%%%%%%%%%%%

Now things get really interesting. Notably the adjoint problem will be different than the previous ones seen so far.

Let us consider the unperturbed and perturbed forward problems ($p$ denotes a parameter):
\[
-\frac{d}{dx}k(T,p)\frac{dT}{dx} = q \qquad \text{with bc: } \left.\frac{dT}{dx}\right|_0=0 \ \text{  and } -k(T,p)\left.\frac{dT}{dx}\right|_L=h(T(L)-T_\infty)
\]
and
\[
-\frac{d}{dx}\left[\left(k(T,p)+\left.\frac{\partial k}{\partial T}\right|_{T,p}\delta T+\left.\frac{\partial k}{\partial p}\right|_{T,p}\delta p\right) \frac{d(T+\delta T)}{dx}\right] = q+\delta q \]
where we have used a first-order expansion for the perturbations in conductivity
\[
k(T+\delta T, p+\delta p) \simeq k(T,p)+\left.\frac{\partial k}{\partial T}\right|_{T,p}\delta T+\left.\frac{\partial k}{\partial p}\right|_{T,p}\delta p
\]
The perturbed boundary conditions are as follows: 
\[
 \left.\frac{d(T+\delta T)}{dx}\right|_0=0 \ \text{ and } -k(T+\delta T, p+\delta p)\left.\frac{d(T+\delta T)}{dx}\right|_L=(h+\delta h)(T(L)+\delta T(L)-T_\infty-\delta T_\infty)
\]

/We would like to subtract these two equations to obtain an equation for $\delta T$. However, in order to include the boundary conditions, we multiply both unperturbed and perturbed forward equations by $\phi$ and integrate by parts. This yields, for both equations,
\begin{equation}
\int q\phi = \int k(T,p)\frac{dT}{dx}\frac{d\phi}{dx} - \left.k(T,p)\frac{dT}{dx}\phi\right|_0^L  
= \int k(T,p)\frac{dT}{dx}\frac{d\phi}{dx} +h(T(L)-T_\infty) \phi(L) 
\end{equation}
and
\begin{equation}
\int (q+\delta q)\phi = \int k(T+\delta T,p+\delta p)\frac{d(T+\delta T)}{dx}\frac{d\phi}{dx} +(h+\delta h)(T(L)+\delta T(L)-T_\infty-\delta T_\infty) \phi(L) 
\end{equation}

Subtract the above two equations (now, we will obtain the weak form of the equation for $\delta T$ and the appropriate bc will naturally be evaluated in that process). We only keep first-order terms in the unknowns.
\begin{multline}
\int \delta q\phi = \int \left(\left.\frac{\partial k}{\partial T}\right|_{T,p}\delta T+\left.\frac{\partial k}{\partial p}\right|_{T,p}\delta p\right) \frac{dT}{dx}\frac{d\phi}{dx} + \int k(T,p)\frac{d\delta T}{dx}\frac{d\phi}{dx} \\
+h(\delta T(L)-\delta T_\infty)  \phi(L)
+\delta h(T(L)-T_\infty)  \phi(L)
\end{multline}

Let us place everything that does not depend on $\delta T$ on the LHS:
\begin{multline}
\label{eq:sensitivity_nonlinear_hc_1}
\int \delta q\phi 
-\int \left.\frac{\partial k}{\partial p}\right|_{T,p}\delta p \frac{dT}{dx}\frac{d\phi}{dx} 
-\delta h(T(L)-T_\infty)  \phi(L)
\\=
\int \left.\frac{\partial k}{\partial T}\right|_{T,p}\delta T \frac{dT}{dx}\frac{d\phi}{dx} 
+ \int k(T,p)\frac{d\delta T}{dx}\frac{d\phi}{dx} 
+h(\delta T(L)-\delta T_\infty)  \phi(L)
\end{multline}

Remember that we seek 
\[
\delta J = \langle \delta T, r \rangle = \langle \delta T, A^\star \phi \rangle 
\]
hence we need to remove all differential operators for $\delta T$ in \eqt{eq:sensitivity_nonlinear_hc_1}. Thus, we need to integrate by parts the second integral in the RHS of \eqt{eq:sensitivity_nonlinear_hc_1}:
\[
\int k(T,p)\frac{d\delta T}{dx}\frac{d\phi}{dx} 
=
-\int \frac{d}{dx}k(T,p)\frac{d\phi}{dx} \delta T + \left.k(T,p)\frac{d\phi}{dx}\delta T\right|_0^L
\]
!!IAN NOTE!! - The above doesn't look right to me, we are ignoring the $d \phi /dx$ when integrating by parts, unless k is not dependent on x but the doesn't seem to be the case. maybe paren. !!End!!
This tells us two things
\begin{enumerate}
\item the boundary conditions for $\phi$ should be
\[
\left.k(T,p)\frac{d\phi}{dx}\right|_L = -h\phi(L) \quad \text{and} \quad \left.\frac{d\phi}{dx}\right|_0 = 0
\]
\item The adjoint operator is (recall, the response function $r$ is the adjoint source)
\[
r=A^\star(\tcr{T}) \phi  = -\frac{d}{dx}k(T,p)\frac{d\phi}{dx} + \left.\frac{\partial k}{\partial T}\right|_{T,p} \frac{dT}{dx}\frac{d\phi}{dx} 
\]
$A^\star(\tcr{T})$ is linear in $\phi$ and depends on the unperturbed solution $T$.
\end{enumerate}

Now, \eqt{eq:sensitivity_nonlinear_hc_1} becomes
\begin{multline}
\int \delta q\phi 
-\int \left.\frac{\partial k}{\partial p}\right|_{T,p}\delta p \frac{dT}{dx}\frac{d\phi}{dx} 
-\delta h(T(L)-T_\infty)  \phi(L)
\\=
\underbrace{
\int \left(
- \frac{d}{dx}k(T,p)\frac{d\phi}{dx} 
+ \left.\frac{\partial k}{\partial T}\right|_{T,p} \frac{dT}{dx}\frac{d\phi}{dx} 
\right) 
\delta T 
}
_{=\langle r,\delta T \rangle = \delta J}
%-h\phi(L)\delta T(L)
%+h(\delta T(L)-\delta T_\infty)  \phi(L)
-h\delta T_\infty \phi(L)
\end{multline}

Finally, we have
\be
\boxed{
\delta J = 
\int \delta q\phi 
-\int \left.\frac{\partial k}{\partial p}\right|_{T,p}\delta p \frac{dT}{dx}\frac{d\phi}{dx} 
-\left(\delta h(T(L)-T_\infty)  - h\delta T_\infty \right)\phi(L)
}
\ee
where $\phi$ is the adjoint solution of the following problem:
\[
A^\star(T) \phi  = -\frac{d}{dx}k(T,p)\frac{d\phi}{dx} + \left.\frac{\partial k}{\partial T}\right|_{T,p} \frac{dT}{dx}\frac{d\phi}{dx} = r
\]
with boundary conditions:
\[
-\left.k(T,p)\frac{d\phi}{dx}\right|_L = h\phi(L) \quad \text{and} \quad \left.\frac{d\phi}{dx}\right|_0 = 0
\]

%%%%%%%%%%%%%%%%%%%%%%%%%%%%%%%%%%%%%%%%%%%%%%%%%%%%%%%%%%%%%%%%%%%%%%%%%%%%%%%%%%%%%%%%
\subsection{Nonlinear operator notation}
%%%%%%%%%%%%%%%%%%%%%%%%%%%%%%%%%%%%%%%%%%%%%%%%%%%%%%%%%%%%%%%%%%%%%%%%%%%%%%%%%%%%%%%%
As with the liner case Define forward, keeping in mind $A=A(T)$
\[
AT=q
\]
define adjoint
\[
A^* \phi =r
\]
We have the functional
\[
J=\langle T,r \rangle = \langle T, A^\star \phi \rangle = \langle (A^*)^t T,\phi \rangle
\]
We introduce our linear perturbation to the system, however, since the operator $A$ has a dependence on $T$, the perturbation to $A$ is nonlinear
\begin{equation}
(A+\der{A}{p} \delta p + \der{A}{T} \delta T)(T + \delta T)=(q+\delta q)
\end{equation}
Compute inner products
\begin{equation}
\langle \phi, q + \delta q \rangle = \langle \phi, (A+\der{A}{p} \delta p + \der{A}{T} \delta T)(T + \delta T) \rangle = \langle \phi , AT \rangle + \langle \phi , A \delta  T \rangle + \langle \phi , \der{A}{p} \delta p T \rangle + \langle \phi, \der{A}{T} \delta T T \rangle + O(\delta^2) 
\end{equation}
Use the forward problem definition $AT=q$ to reduce to 
\begin{equation}
\langle \phi, \delta q \rangle = \langle \phi , A \delta  T \rangle + \langle \phi , \der{A}{p} \delta p T \rangle + \langle \phi, \der{A}{T} \delta T T \rangle + O(\delta^2) 
\end{equation}

Drop second order terms and group all terms with a factor of $\delta T$
\[
\langle \phi , A \delta  T \rangle + \langle \phi, \der{A}{T} \delta T T \rangle = \langle \phi, \delta q \rangle- \langle \phi , \der{A}{p} \delta p T \rangle 
\]
Moving the lhs operator to act on $\phi$ instead of $\delta T$
\[
\langle \Big(A+\der{A}{T} T \Big)^T \phi ,  \delta T \rangle  = \langle \phi, \Big( \delta q - \der{A}{p} \delta p T \Big) \rangle.
\]
The lhs operator on $\phi$ becomes our adjoint operator $A^\star$
\[
\langle A^\star \phi ,  \delta T \rangle  = \langle \phi, \Big( \delta q - \der{A}{p} \delta p T \Big) \rangle.
\]
This is just our response function however 
\[
\langle r ,  \delta T \rangle  = \langle \phi, \Big( \delta q - \der{A}{p} \delta p T \Big) \rangle.
\]
the response acting on $\delta T$ yields $\delta J$, so we arrive at our desired point
\[
\delta J = \langle \phi, \Big( \delta q - \der{A}{p} \delta p T \Big) \rangle.
\]

%%%%%%%%%%%%%%%%%%%%%%%%%%%%%%%%%%%%%%%%%%%%%%%%%%%%%%%%%%%%%%%%%%%%%%%%%%%%%%%%%%%%%%%%
\subsection{General Case without BC}
%%%%%%%%%%%%%%%%%%%%%%%%%%%%%%%%%%%%%%%%%%%%%%%%%%%%%%%%%%%%%%%%%%%%%%%%%%%%%%%%%%%%%%%%

\[
-\frac{d}{dx}k(T,p)\frac{dT}{dx} = q 
\]
\[
-\frac{d}{dx}k(T+\delta T,p+\delta p)\frac{d(T+\delta T)}{dx} = q +\delta q 
\]
To the first order this is
\[
-\frac{d}{dx}\left[\left(k(T,p)+\left.\frac{\partial k}{\partial T}\right|_{T,p}\delta T+\left.\frac{\partial k}{\partial p}\right|_{T,p}\delta p\right) \frac{d(T+\delta T)}{dx}\right] = q+\delta q \] 
from the operator notation 
\[
\langle \phi , A \delta  T \rangle + \langle \phi, \der{A}{T} \delta T T \rangle = \langle \phi, \delta q \rangle- \langle \phi , \der{A}{p} \delta p T \rangle 
\]
or subtracting perturbed and unperturbed and dropping higher order terms.
\[
-\frac{d}{dx}\left( k(T,p) \frac{d(\delta T)}{dx} \right)
-\frac{d}{dx}\left(\left.\frac{\partial k}{\partial T}\right|_{T,p}\delta T \frac{d(T)}{dx}\right)
-\frac{d}{dx}\left(\left.\frac{\partial k}{\partial p}\right|_{T,p}\delta p \frac{d(T)}{dx} \right)
= \delta q \] 
Now inner product with $\phi$
\[
-\int \frac{d}{dx}\left( k(T,p) \frac{d(\delta T)}{dx} \right)\phi
-\int \frac{d}{dx}\left(\left.\frac{\partial k}{\partial T}\right|_{T,p}\delta T \frac{d(T)}{dx}\right)\phi
-\int \frac{d}{dx}\left(\left.\frac{\partial k}{\partial p}\right|_{T,p}\delta p \frac{d(T)}{dx} \right)\phi
= \int \delta q \phi
\] 
Collect all $\delta T$ terms
\[
-\int \frac{d}{dx}\left( k(T,p) \frac{d(\delta T)}{dx} \right)\phi
-\int \frac{d}{dx}\left(\left.\frac{\partial k}{\partial T}\right|_{T,p}\delta T \frac{d(T)}{dx}\right)\phi
= \int \delta q \phi 
+\int \frac{d}{dx}\left(\left.\frac{\partial k}{\partial p}\right|_{T,p}\delta p \frac{d(T)}{dx} \right)\phi
\] 
Go crazy with integration by parts, generating 3 boundary terms
\[
\int \left( k(T,p) \frac{d(\delta T)}{dx} \right)\frac{d \phi}{dx}
+\int \left(\left.\frac{\partial k}{\partial T}\right|_{T,p}\delta T \frac{d(T)}{dx}\right)\frac{d \phi}{dx} 
- \left. \left( k(T,p) \frac{d(\delta T)}{dx} \phi \right) \right|_0^L 
- \left. \left(\left.\frac{\partial k}{\partial T}\right|_{T,p}\delta T \frac{d(T)}{dx} \phi \right ) \right|_0^L
\]
\[
= \int \delta q \phi 
-\int \left(\left.\frac{\partial k}{\partial p}\right|_{T,p}\delta p \frac{d(T)}{dx} \right)\frac{d \phi}{dx}
+ \left. \left(\left.\frac{\partial k}{\partial p}\right|_{T,p}\delta p \frac{d(T)}{dx} \phi \right) \right|_0^L
\] 
There is still one term on the lhs that has an operator acting on $\delta T$ so we need to deal with that. lets focus in
\[
\int \left( k(T,p) \frac{d(\delta T)}{dx} \right)\frac{d \phi}{dx}
\] 
Product rule
\[
\int \left( \frac{d}{dx} (k(t,p) \delta T)- \frac{d k(T,p)}{dx}\delta T \right)\frac{d \phi}{dx}
\]
and int by parts the remaining term.
\[
\int \left( k(t,p) \delta T\right)\frac{d^2 \phi}{dx^2}
- \int \left( \frac{d k(T,p)}{dx}\delta T \right)\frac{d \phi}{dx}
+ \left.\left( k(t,p) \delta T \frac{d \phi}{dx}  \right) \right|_0^L
\]
So now we get the full expression with a bunch of boundary terms moved to the RHS 
\[
\int \left( k(t,p) \frac{d^2 \phi}{dx^2} \right) \delta T 
- \int \left( \frac{d k(T,p)}{dx} \frac{d \phi}{dx} \right)\delta T 
+\int \left(\left.\frac{\partial k}{\partial T}\right|_{T,p} \frac{d(T)}{dx} \frac{d \phi}{dx} \right)\delta T
\]
\[
= \int \delta q \phi 
-\int \left(\left.\frac{\partial k}{\partial p}\right|_{T,p}\delta p \frac{d(T)}{dx} \right)\frac{d \phi}{dx}
\]
\[
+ \left. \left(\left.\frac{\partial k}{\partial p}\right|_{T,p}\delta p \frac{d(T)}{dx} \phi \right) \right|_0^L
+ \left. \left( k(T,p) \frac{d(\delta T)}{dx} \phi \right) \right|_0^L 
+ \left. \left(\left.\frac{\partial k}{\partial T}\right|_{T,p}\delta T \frac{d(T)}{dx} \phi \right ) \right|_0^L
- \left.\left( k(t,p) \delta T \frac{d \phi}{dx}  \right) \right|_0^L
\] 
So the $A^\star$ operator is
\[
A^\star (T) = \left( k(t,p) \frac{d^2}{dx^2} \right) 
- \left( \frac{d k(T,p)}{dx} \frac{d}{dx} \right) 
+ \left(\left.\frac{\partial k}{\partial T}\right|_{T,p} \frac{d(T)}{dx} \frac{d}{dx} \right)
\]
Its probably worth noting that if we act on a linear mesh, the first term will go to 0 due to the second derivative.
%%%%%%%%%%%%%%%%%%%%%%%%%%%%%%%%%%%%%%%%%%%%%%%%%%%%%%%%%%%%%%%%%%%%%%%%%%%%%%%%%%%%%%%%
\subsection{Dirichlet}
%%%%%%%%%%%%%%%%%%%%%%%%%%%%%%%%%%%%%%%%%%%%%%%%%%%%%%%%%%%%%%%%%%%%%%%%%%%%%%%%%%%%%%%%
If we choose our $\phi$ to have homogeneous Dirichlet BC, the first 3 BC terms from above go to 0. Leaving the last term 
\[
- \left.\left( k(t,p) \delta T \frac{d \phi}{dx}  \right) \right|_0^L
\]
Now, if our $T$ and $\delta T$ were given Dirichlet BC, then $\left. \delta T \right|_0^L$ is known. If the BC were the same for both, this term disappears also.

\end{document}
