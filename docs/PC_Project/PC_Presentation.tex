\documentclass[xcolor={usenames,dvipsnames,svgnames,table}]{beamer}

\mode<presentation>
\usetheme{Madrid}

\usecolortheme[RGB={80,0,0}]{structure}
\useoutertheme[subsection=false]{miniframes}
\useinnertheme{default}

% hide navigation controlls
\setbeamertemplate{navigation symbols}{}

\setbeamercolor{normal text}{fg=black}
\setbeamercovered{dynamic}
\beamertemplatetransparentcovereddynamicmedium
%\usepackage{chronology}
\setbeamertemplate{caption}[numbered]

\definecolor{Maroon}{RGB}{80,0,0}
\definecolor{BurntOrange}{RGB}{204,85,0}


% load macros and prevent authblk from loading
\input{../Presentations/common/macros.tex}
\dontusepackage{authblk}

% load packages, settings and definitions
\input{../Presentations/common/packages.tex}
\input{../Presentations/common/settings.tex}
%\input{../common/definitions.tex}

\newcommand{\vr}{\vec{r}}
\newcommand{\vp}{\vec{p}}
\newcommand{\vOmega}{\vec{\Omega}}
\newcommand{\vJ}{\vec{J}}
\newcommand{\vO}{\vec{\Omega}}
\newcommand{\bra}{\left\langle}
\newcommand{\ket}{\right\rangle}
\newcommand{\sbra}{\left[}
\newcommand{\sket}{\right]}
\newcommand{\braSN}{\left\langle \! \left\langle}
\newcommand{\ketSN}{\right\rangle \! \right\rangle}
\newcommand{\sbraSN}{\left[ \! \left[}
\newcommand{\sketSN}{\right] \! \right]}
\renewcommand{\div}{\vec{\nabla} \cdot}
\newcommand{\grad}{\vec{\nabla}}
\newcommand{\vbeta}{\vec{\beta} }
\newcommand{\pdx}{\frac{\partial}{\partial x}}
\newcommand{\pdy}{\frac{\partial}{\partial y}}
\newcommand{\pdz}{\frac{\partial}{\partial z}}
\newcommand{\intrrr}{\int d^3 r \,}
\newcommand{\intrr}{\int d^2 r \,}
\newcommand{\dEdphi}{\partial_\phi E }
\newcommand{\dEdp}{\partial_p E }
\newcommand{\dBdphi}{\partial_\phi B }
\newcommand{\dBdp}{B }
\newcommand{\adj}{\phi^\dag}
\newcommand{\vefadj}{\varphi^\dag}
\newcommand{\surf}{\int_{\partial V}}
\newcommand{\domain}{V}
\newcommand{\bound}{\partial V}
\newcommand{\vn}{\vec{n}}
\newcommand{\Edd}{\mathbb{E}}
\newcommand{\BEdd}{B}
\newcommand{\sigt}{\sigma_t}
\newcommand{\sigs}{\sigma_s}
\newcommand{\siga}{\sigma_a}
\newcommand{\isigt}{\sigma_t^{-1}}
\newcommand{\isigtp}{\sigma_{t,p}^{-1}}
%\newcommand{\isigt}{\ell_t}
%\newcommand{\isigtp}{\ell_{t,p}}
\newcommand{\angSource}{\frac{q}{4 \pi}}
\newcommand{\angSourcep}{\frac{q_p}{4 \pi}}
\newcommand{\angSourcepd}{\frac{q+\delta q}{4 \pi}}
\newcommand{\angSourced}{\frac{\delta q}{4 \pi}}
\newcommand{\scalSource}{q}
\newcommand{\angResp}{q^\dag}
\newcommand{\scalResp}{q^\dag}
\newcommand{\qoi}{{\it QoI}\xspace}
\newcommand{\tcr}[1]{\textcolor{red}{#1}}
\newcommand{\tcb}[1]{\textcolor{blue}{#1}}
\newcommand{\tcm}[1]{\textcolor{magenta}{#1}}
\newcommand{\tcg}[1]{\textcolor{BlueGreen}{#1}}


% nicer item settings
\setlist[1]{nolistsep,label=\(\textcolor{Maroon}{\blacksquare}\)}
\setlist[2]{nolistsep,label=\(\textcolor{Maroon}{\bullet}\)}

\setenumerate[1]{
	label=\protect\usebeamerfont{enumerate item}%
	\protect\usebeamercolor[fg]{enumerate item}%
	\insertenumlabel.
}

%%%%%%%%%%%%%%%%%%%%%%%%%%%%%%%%%%%%%%%%%%%%%%%
%%% edit to fit your document

% set up pdf support and indexing
\hypersetup{
    pdftitle={Using Polynomial Chaos Expansion for Eddington Approximation for use in VET Adjoint Calculations},
    pdfauthor={Ian Halvic},
}

\title[]{Using Polynomial Chaos Expansion for Eddington Approximation for use in VET Adjoint Calculations}
\author[ ]{Ian Halvic}
\institute[Texas A\&M]{Department of Nuclear Engineering \\ Texas A\&M University}
\date[12/13/18]

\begin{document}

% title page, do not edit
{
\setbeamertemplate{headline}[default] 
\begin{frame}
\vspace{-1.1cm}
	\begin{figure}[t]
		\centering
			\includegraphics[width=.25\textwidth]{seal.png}
	\end{figure}
\vspace{-0.75cm}
\titlepage
\end{frame}
}

%%%%%%%%%%%%%%%%%%%%%%%%%%%%%%%%%%%%%%%%%%%%%%%%%%%%%%%%%%%%%%%%%%%%%%%%%%%%%%%%%%%%%%%%%%%%%%%%%%%%%%%%%%%
\section{Background}	% define sections here, it is possible to get section slides automatically, but this is not enabled
\subsection{}	% we have to keep these to get the navigation
%%%%%%%%%%%%%%%%%%%%%%%%%%%%%%%%%%%%%%%%%%%%%%%%%%%%%%%%%%%%%%%%%%%%%%%%%%%%%%%%%%%%%%%%%%%%%%%%%%%%%%%%%%%

\begin{frame}[t]\frametitle{Motivation}
\begin{flushleft}
\begin{block}{Primarily an expansion of some previous work (master's thesis) \cite{MyMasters}}
\begin{itemize}
	\item Previous work focused on reducing memory requirement of transport adjoint methods by storing only scalar flux $\phi$ instead of angular flux $\psi$
	\item Accomplished using a Vairable Eddington Tensor (VET) formulation of the problem and a corresponding adjoint
	\item However assumptions were made which were shown to fail in certain test cases
\end{itemize}
\end{block}
This work attempts to relax these assumptions using a selected polynomial chaos expansion to fill in unknown data.
\end{flushleft}
\end{frame}
 
 %%%%%%%%%%%%%%%%%%%%%%%%%%%%%%%%%%%%%%%%%%%%%%%%%%%%%%%%%%%%%%%%%%%%%%%%%%%%%%%%%%%%%%%%%%%%%%%%%%%%%%%%%%%
 
 
 %%%%%%%%%%%%%%%%%%%%%%%%%%%%%%%%%%%%%%%%%%%%%%%%%%%%%%%%%%%%%%%%%%%%%%%%%%%%%%%%%%%%%%%%%%%%%%%%%%%%%%%%%%%

\begin{frame}[t]\frametitle{Why Not Transport?}
\begin{flushleft}
Consider the one-group steady state \tcr{forward} transport system with isotropic source $q$ and scattering $\sigs$
\begin{subequations}
\begin{equation}
\label{SNfwd}
 \vO \cdot \grad \psi + \sigt \psi = \frac{\sigs}{4 \pi} \phi + \angSource
\end{equation}
\begin{equation}
\psi(\vr) = \psi^{ \text{inc}}(\vr) \quad \vr \in \partial V^{-} = \{  \vr \in \bound , \quad \vO \cdot \vec{n} < 0 \}
\end{equation}
\end{subequations}
The adjoint of the above nearly self adjoint ($\vO \to -\vO$). 
\begin{block}{First order sensitivity inner product}
\begin{equation}
\label{snSens}
\begin{split}
\delta \qoi 
 &\delta \approx \bra \angSourced, \phi^\dag \ket - \tcr{\braSN \delta\sigt\psi , \psi^\dag \ketSN} + \bra \frac{\delta \sigs}{4 \pi} \phi
 , \phi^\dag \ket - \sbraSN \delta \psi^{\text{inc}}, \psi^\dag \sketSN \\
\end{split}
\end{equation}
Storing $\psi$ and $\psi^\dag$ is \tcr{BAD}!
\end{block}
\end{flushleft}
\end{frame}
 
 %%%%%%%%%%%%%%%%%%%%%%%%%%%%%%%%%%%%%%%%%%%%%%%%%%%%%%%%%%%%%%%%%%%%%%%%%%%%%%%%%%%%%%%%%%%%%%%%%%%%%%%%%%%
 
 %%%%%%%%%%%%%%%%%%%%%%%%%%%%%%%%%%%%%%%%%%%%%%%%%%%%%%%%%%%%%%%%%%%%%%%%%%%%%%%%%%%%%%%%%%%%%%%%%%%%%%%%%%%

\begin{frame}[t]\frametitle{VET Formulation}
\begin{flushleft}
Formulate the Eddington Tensor $\Edd$ and a Boundary Eddington Factor $\BEdd$ 
\begin{equation}
\Edd(\vr)=\frac{\int d\Omega \vO \vO \psi(\vr,\vO)}{\phi(\vr)}
, \, \vr \in V \quad \quad 
\BEdd(\vr) = \frac{\int_{4 \pi} d\Omega \, | \vO \cdot \vn | \psi}{\phi(\vr)} , \, \vr \in \bound
\end{equation}
Use with P1 equations to form the quasi-diffusive VET formulation \cite{Miften}
\begin{subequations}
\begin{equation}
\label{VEFForm}
- \div \left( \frac{1}{\sigt}\div \Edd \phi \right) + \siga \phi = \scalSource \, \quad \vr \in V 
\end{equation}
\begin{equation}
2 J^{\text{inc}} = \BEdd \phi + \vn \cdot \frac{1}{\sigt} \div \Edd \phi \,\quad  \vr \in \bound
\end{equation}
\end{subequations}
If $\Edd$ and $\BEdd$ are known, then $\phi^{\text{VET}} = \phi^{\text{SN}}$
\end{flushleft}
\end{frame}
 
%%%%%%%%%%%%%%%%%%%%%%%%%%%%%%%%%%%%%%%%%%%%%%%%%%%%%%%%%%%%%%%%%%%%%%%%%%%%%%%%%%%%%%%%%%%%%%%%%%%%%%%%%%%

 %%%%%%%%%%%%%%%%%%%%%%%%%%%%%%%%%%%%%%%%%%%%%%%%%%%%%%%%%%%%%%%%%%%%%%%%%%%%%%%%%%%%%%%%%%%%%%%%%%%%%%%%%%%

\begin{frame}[t]\frametitle{VET Adjoint}
\begin{flushleft}
From the previous balance equation, a form for the VET adjoint emerges:
\begin{subequations}
\begin{equation}
\label{adjForm}
- \Edd : \left( \grad \left( \frac{1}{\sigt}\grad \vefadj \right) \right) + \siga \vefadj = \scalResp,    \quad \vr \in V
\end{equation}
\begin{equation}
\label{adjVETBC}
2J^{\dag,\text{out}} = B \vefadj+ \vn \cdot
\Edd \cdot \frac{1}{\sigma_{t} } \vec{\nabla} \vefadj,    \quad \vr \in \bound
\end{equation}
\end{subequations}
Solving the system yields a VET adjoint scalar flux $\vefadj$. This is not the same adjoint scalar flux one would find using the transport adjoint. $\vefadj \neq \phi^\dag$
\vspace{3mm}

We choose $J^{\dag,\text{out}} =0$
\end{flushleft}
\end{frame}
 
%%%%%%%%%%%%%%%%%%%%%%%%%%%%%%%%%%%%%%%%%%%%%%%%%%%%%%%%%%%%%%%%%%%%%%%%%%%%%%%%%%%%%%%%%%%%%%%%%%%%%%%%%%%
 
 
 %%%%%%%%%%%%%%%%%%%%%%%%%%%%%%%%%%%%%%%%%%%%%%%%%%%%%%%%%%%%%%%%%%%%%%%%%%%%%%%%%%%%%%%%%%%%%%%%%%%%%%%%%%%

\begin{frame}[t]\frametitle{VET Sensitivity}
\begin{flushleft}
\begin{block}{VET adjoint first-order sensitivity inner-product}
\begin{equation}
\label{EQ:VETsens}
\begin{split}
\delta \qoi &\approx \bra \delta \scalSource, \vefadj  \ket - \bra \delta \siga \phi, \vefadj \ket  - \bra \delta \isigt \div \left( \Edd^u \phi \right) , \grad \vefadj \ket + \sbra \vefadj, 2 \delta J^{\text{inc}} \sket \\
& \quad - \bra  \isigt \div \left( \tcr{\delta \Edd} \phi \right), \grad \vefadj \ket - \sbra \vefadj, \phi \tcr{\delta \BEdd} \sket \\
\end{split}
\end{equation}
\end{block}
\begin{itemize}
\item Contains system perturbations $\delta \scalSource$, $\delta J^{\text{inc}}$, $\delta \siga$, $\delta \sigt$ while also not requiring $\psi$ stored
\item Eddington perturbations $\delta \Edd$ and $\delta \BEdd$ are problematic, they cannot be determined from system perturbations without another transport solve
\item The previous work primarily made the assumption that $\delta \Edd$ and $\delta \BEdd=0$, which was adequate for a variety of scenarios, \tcr{but not all}
\end{itemize}
\end{flushleft}
\end{frame}
 
 %%%%%%%%%%%%%%%%%%%%%%%%%%%%%%%%%%%%%%%%%%%%%%%%%%%%%%%%%%%%%%%%%%%%%%%%%%%%%%%%%%%%%%%%%%%%%%%%%%%%%%%%%%%
 
 %%%%%%%%%%%%%%%%%%%%%%%%%%%%%%%%%%%%%%%%%%%%%%%%%%%%%%%%%%%%%%%%%%%%%%%%%%%%%%%%%%%%%%%%%%%%%%%%%%%%%%%%%%%

\begin{frame}[t]\frametitle{VET Sensitivity: Voids}
\begin{flushleft}
\begin{minipage}{.45\textwidth}
\begin{itemize}
\item Introduction of voids regions into the system caused the VET sensitivity to perform poorly
\item 5 region Reed problem: attempting to predict $\delta \qoi$ in response to $\delta \sigs$ failed spectacularly
\end{itemize}
\end{minipage}
\begin{minipage}{.45\textwidth}
\begin{figure}[H]
\centering
  \includegraphics[width=1\linewidth]{772sigsSens.png}
\label{fig:thesissens}
\end{figure}
\end{minipage}
\begin{itemize}
\item A linear approximation of $\delta \Edd$ was made using an additional transport solve, this appeared to drastically improve the results, but what if we could do better...
\end{itemize}
\end{flushleft}
\end{frame}
 
%%%%%%%%%%%%%%%%%%%%%%%%%%%%%%%%%%%%%%%%%%%%%%%%%%%%%%%%%%%%%%%%%%%%%%%%%%%%%%%%%%%%%%%%%%%%%%%%%%%%%%%%%%%
 
 
 %%%%%%%%%%%%%%%%%%%%%%%%%%%%%%%%%%%%%%%%%%%%%%%%%%%%%%%%%%%%%%%%%%%%%%%%%%%%%%%%%%%%%%%%%%%%%%%%%%%%%%%%%%%
\section{Method+Results}	% define sections here, it is possible to get section slides automatically, but this is not enabled
\subsection{}	% we have to keep these to get the navigation
%%%%%%%%%%%%%%%%%%%%%%%%%%%%%%%%%%%%%%%%%%%%%%%%%%%%%%%%%%%%%%%%%%%%%%%%%%%%%%%%%%%%%%%%%%%%%%%%%%%%%%%%%%%
 
%%%%%%%%%%%%%%%%%%%%%%%%%%%%%%%%%%%%%%%%%%%%%%%%%%%%%%%%%%%%%%%%%%%%%%%%%%%%%%%%%%%%%%%%%%%%%%%%%%%%%%%%%%%

\begin{frame}[t]\frametitle{Reed Problem}
\begin{flushleft}
\begin{block}{1D Reed Problem}
\begin{itemize}
\item Region 1: $x \in [0,2), \quad \tcb{\siga=50}, \, 			\sigs=0, \, \tcb{q=50}, \, q^\dag=0 $
\item Region 2: $x \in [2,3), \quad \tcb{\siga=5}, \, 			\sigs=0, \, q=0, \, q^\dag=0$ 
\item Region 3: $x \in [3,5), \quad \siga \approx 0, \,	\sigs=0, \, q=0, \,  \tcr{q^\dag=1}$
\item Region 4: $x \in [5,6), \quad \tcb{\siga=0.1}, \, 		\tcb{\sigs=0.9}, \, \tcb{q=1}, \, q^\dag=0$
\item Region 5: $x \in [6,8], \quad \tcb{\siga=0.1}, \, 		\tcb{\sigs=0.9}, \, q=0, \, q^\dag=0$
\end{itemize} 
\end{block}
\begin{minipage}{.45\textwidth}
$\qoi=$ total flux within the void region $x \in [3,5)$. 
\end{minipage}
\begin{minipage}{.45\textwidth}
\begin{figure}[H]
\centering
  \includegraphics[width=0.9\linewidth]{phi_reed.png}
\label{fig:phi_reed}
\end{figure}
\end{minipage}
\end{flushleft}
\end{frame}
 
%%%%%%%%%%%%%%%%%%%%%%%%%%%%%%%%%%%%%%%%%%%%%%%%%%%%%%%%%%%%%%%%%%%%%%%%%%%%%%%%%%%%%%%%%%%%%%%%%%%%%%%%%%%

%%%%%%%%%%%%%%%%%%%%%%%%%%%%%%%%%%%%%%%%%%%%%%%%%%%%%%%%%%%%%%%%%%%%%%%%%%%%%%%%%%%%%%%%%%%%%%%%%%%%%%%%%%%

\begin{frame}[t]\frametitle{Polynomial Chaos Expansion (PCE)}
\begin{flushleft}
We consider a PCE of the Eddington Tensor across our computational domain $x$, with uncertain parameters $\xi$
\begin{equation}
\Edd(x,\vec{\xi}) = \sum \Edd_i(x)P(\vec{\xi}) \quad
\end{equation}
For the Reed system  $\vec{\xi} \in \mathbb{R}^8$. With coefficients found we could approximate $\delta \Edd$
\vspace{3mm}

This PCE expansion was performed using UQ-Lab \cite{UQLab}
\begin{itemize}
 \item Standard Deviation 10\% of nominal
 \item Maximum Polynomial Degree=2
 \item Varied Number of Sample Points
\end{itemize}
\vspace{3mm}
\end{flushleft}
\end{frame}
 
%%%%%%%%%%%%%%%%%%%%%%%%%%%%%%%%%%%%%%%%%%%%%%%%%%%%%%%%%%%%%%%%%%%%%%%%%%%%%%%%%%%%%%%%%%%%%%%%%%%%%%%%%%%

%%%%%%%%%%%%%%%%%%%%%%%%%%%%%%%%%%%%%%%%%%%%%%%%%%%%%%%%%%%%%%%%%%%%%%%%%%%%%%%%%%%%%%%%%%%%%%%%%%%%%%%%%%%

\begin{frame}[t]\frametitle{PCE Results}
\begin{flushleft}
$\delta \Edd$ was estimated and $\delta \qoi$ predicted using VET adjoint. Number of sample points in PCE varied
\begin{figure}[H]
\centering
  \begin{subfigure}{.5\textwidth}
  \includegraphics[width=0.98 \linewidth]{E8_nsamples.png}
  \caption{$\delta E$}
  \label{fig:E8n}
  \end{subfigure}%
  \begin{subfigure}{.5\textwidth}
  \includegraphics[width=0.98 \linewidth]{E8_dqoi.png}
  \caption{$\delta \qoi$}
  \label{fig:E8dqoi}
  \end{subfigure}%
  %\label{fig:E2n}
\end{figure} 
\end{flushleft}
\end{frame}
 
%%%%%%%%%%%%%%%%%%%%%%%%%%%%%%%%%%%%%%%%%%%%%%%%%%%%%%%%%%%%%%%%%%%%%%%%%%%%%%%%%%%%%%%%%%%%%%%%%%%%%%%%%%%



%%%%%%%%%%%%%%%%%%%%%%%%%%%%%%%%%%%%%%%%%%%%%%%%%%%%%%%%%%%%%%%%%%%%%%%%%%%%%%%%%%%%%%%%%%%%%%%%%%%%%%%%%%%

\begin{frame}[t]\frametitle{WAIT!}
\begin{flushleft}
This is stupid!
\begin{itemize}
\item Each $\Edd$ sample in PCE takes a transport solve
\item We could just expand $\phi$ and avoid adjoints all together
\end{itemize}
But what if we could reduce the cost of our $\Edd$ expansion...
\end{flushleft}
\end{frame}
 
%%%%%%%%%%%%%%%%%%%%%%%%%%%%%%%%%%%%%%%%%%%%%%%%%%%%%%%%%%%%%%%%%%%%%%%%%%%%%%%%%%%%%%%%%%%%%%%%%%%%%%%%%%%

%%%%%%%%%%%%%%%%%%%%%%%%%%%%%%%%%%%%%%%%%%%%%%%%%%%%%%%%%%%%%%%%%%%%%%%%%%%%%%%%%%%%%%%%%%%%%%%%%%%%%%%%%%%

\begin{frame}[t]\frametitle{Sobol Indices}
\begin{flushleft}
Lets take a look at Sobol indices of the PCE expansion with a $\qoi$ of $\lVert \delta \Edd \rVert$
\vspace{3mm}

\begin{minipage}{.65\textwidth}
\begin{itemize}
\item Just $\xi_4$ and $\xi_7$ contribute to the vast majority of the variance
\item Corresponds to $\siga$ to the left of void and $\sigs$ to the right of void
\end{itemize}
\end{minipage}
\begin{minipage}{.30\textwidth}
\begin{table}[H]
\footnotesize
\centering
  \begin{tabular}{| l | r | }
    \hline
           & $S$\\ \hline
   $\xi_1$ & 0.00152535\\  \hline
   $\xi_2$ & 0\\  \hline
   $\xi_3$ & 0 \\  \hline
   $\xi_4$ & \tcb{0.659784} \\  \hline
   $\xi_5$ & 0.00162217\\  \hline
   $\xi_6$ & 0.00710227\\  \hline
   $\xi_7$ & \tcb{0.282522} \\  \hline
   $\xi_8$ & 0.00170134 \\  \hline
    \end{tabular}
\end{table}
\end{minipage}
Now we instead consider a PCE expansion of $\Edd$, but with $\vec{\xi} \in \mathbb{R}^2$. This hopefully reduces the total number of samples required.
\end{flushleft}
\end{frame}
 
%%%%%%%%%%%%%%%%%%%%%%%%%%%%%%%%%%%%%%%%%%%%%%%%%%%%%%%%%%%%%%%%%%%%%%%%%%%%%%%%%%%%%%%%%%%%%%%%%%%%%%%%%%%

%%%%%%%%%%%%%%%%%%%%%%%%%%%%%%%%%%%%%%%%%%%%%%%%%%%%%%%%%%%%%%%%%%%%%%%%%%%%%%%%%%%%%%%%%%%%%%%%%%%%%%%%%%%

\begin{frame}[t]\frametitle{PCE Results (reduced $\vec{\xi}$)}
\begin{flushleft}
$\delta \Edd$ and $\delta \qoi$ predictions using reduced $\vec{\xi} \in \mathbb{R}^2$ PCE model
\begin{figure}[H]
\centering
  \begin{subfigure}{.5\textwidth}
  \includegraphics[width=0.98 \linewidth]{E2_nsamples.png}
  \caption{$\delta E$}
  \label{fig:E8n}
  \end{subfigure}%
  \begin{subfigure}{.5\textwidth}
  \includegraphics[width=0.98 \linewidth]{E2_dqoi.png}
  \caption{$\delta \qoi$}
  \label{fig:E8dqoi}
  \end{subfigure}%
  %\label{fig:E2n}
\end{figure} 
\end{flushleft}
\end{frame}
 
%%%%%%%%%%%%%%%%%%%%%%%%%%%%%%%%%%%%%%%%%%%%%%%%%%%%%%%%%%%%%%%%%%%%%%%%%%%%%%%%%%%%%%%%%%%%%%%%%%%%%%%%%%%

 %%%%%%%%%%%%%%%%%%%%%%%%%%%%%%%%%%%%%%%%%%%%%%%%%%%%%%%%%%%%%%%%%%%%%%%%%%%%%%%%%%%%%%%%%%%%%%%%%%%%%%%%%%%
\section{Wrap-Up}	% define sections here, it is possible to get section slides automatically, but this is not enabled
\subsection{}	% we have to keep these to get the navigation
%%%%%%%%%%%%%%%%%%%%%%%%%%%%%%%%%%%%%%%%%%%%%%%%%%%%%%%%%%%%%%%%%%%%%%%%%%%%%%%%%%%%%%%%%%%%%%%%%%%%%%%%%%%

%%%%%%%%%%%%%%%%%%%%%%%%%%%%%%%%%%%%%%%%%%%%%%%%%%%%%%%%%%%%%%%%%%%%%%%%%%%%%%%%%%%%%%%%%%%%%%%%%%%%%%%%%%%

\begin{frame}[t]\frametitle{Takeaways}
\begin{flushleft}
\begin{block}{Observations}
\begin{itemize}
\item The PCE expansion showed stronger agreement than the previous $\delta \Edd=0$ approximation
\item Reducing the PCE to $\vec{\xi} \in 2$ helped reduce overall PCE computation time will still providing an improved $\delta \qoi$
\item Still need to compete with simple linear approximation
\end{itemize}
\end{block}

\begin{block}{Possible Further Work}
\begin{itemize}
\item Consider scenarios where $\xi$'s may more strongly interact with eachother
\item Test with different PCE settings in UQLab
\item Consider extending to higher spacial dimensions
\end{itemize}
\end{block}
\end{flushleft}
\end{frame}
 
%%%%%%%%%%%%%%%%%%%%%%%%%%%%%%%%%%%%%%%%%%%%%%%%%%%%%%%%%%%%%%%%%%%%%%%%%%%%%%%%%%%%%%%%%%%%%%%%%%%%%%%%%%%

%%%%%%%%%%%%%%%%%%%%%%%%%%%%%%%%%%%%%%%%%%%%%%%%%%%%%%%%%%%%%%%%%%%%%%%%%%%%%%%%%%%%%%%%%%%%%%%%%%%%%%%%%%%

\begin{frame}[t]\frametitle{References}
\begin{flushleft}
\bibliography{QoI_MS} 
\bibliographystyle{ieeetr}
\end{flushleft}
\end{frame}
 
%%%%%%%%%%%%%%%%%%%%%%%%%%%%%%%%%%%%%%%%%%%%%%%%%%%%%%%%%%%%%%%%%%%%%%%%%%%%%%%%%%%%%%%%%%%%%%%%%%%%%%%%%%%

%%%%%%%%%%%%%%%%%%%%%%%%%%%%%%%%%%%%%%%%%%%%%%%%%%%%%%%%%%%%%%%%%%%%%%%%%%%%%%%%%%%%%%%%%%%%%%%%%%%%%%%%%%%%
%
%\begin{frame}[t]\frametitle{BLANK SLIDE}
%\begin{flushleft}
%
%\end{flushleft}
%\end{frame}
% 
%%%%%%%%%%%%%%%%%%%%%%%%%%%%%%%%%%%%%%%%%%%%%%%%%%%%%%%%%%%%%%%%%%%%%%%%%%%%%%%%%%%%%%%%%%%%%%%%%%%%%%%%%%%%


%%%%%%%%%%%%%%%%%%%%%%%%%%%%%%%%%%%%%%%%%%%%%%%%%%%%%%%%%%%%%%%%%%%%%%%%%%%%%%%%%%%%%%%%%%%%%%%%%%%%%%%%%%%%
%
%\begin{frame}[t]\frametitle{BLANK SLIDE}
%\begin{flushleft}
%
%\end{flushleft}
%\end{frame}
% 
%%%%%%%%%%%%%%%%%%%%%%%%%%%%%%%%%%%%%%%%%%%%%%%%%%%%%%%%%%%%%%%%%%%%%%%%%%%%%%%%%%%%%%%%%%%%%%%%%%%%%%%%%%%%

\end{document}