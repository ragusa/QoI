\documentclass{article}
\usepackage{amsmath, amsthm, amssymb, amsfonts, booktabs, hyperref, graphicx, float, esint, xcolor, subcaption, xspace}
\usepackage[margin=1.0in]{geometry}
% mathbbol, causing errors
\setlength{\abovedisplayskip}{0pt}
\setlength{\belowdisplayskip}{0pt}
\setlength{\abovedisplayshortskip}{0pt}
\setlength{\belowdisplayshortskip}{0pt}

\newcommand{\vr}{\vec{r}}
\newcommand{\vx}{\vec{x}}
\newcommand{\vOmega}{\vec{\Omega}}
\newcommand{\vJ}{\vec{J}}
\newcommand{\vO}{\vec{\Omega}}
\newcommand{\bra}{\left\langle}
\newcommand{\ket}{\right\rangle}
\newcommand{\ketbd}{\right\rangle_{\delta \Omega}}
\newcommand{\sbra}{\left[}
\newcommand{\sket}{\right]}
\renewcommand{\div}{\vec{\nabla} \cdot}
\newcommand{\grad}{\vec{\nabla}}
\newcommand{\vbeta}{\vec{\beta} }
\newcommand{\pdx}{\frac{\partial}{\partial x}}
\newcommand{\pdy}{\frac{\partial}{\partial y}}
\newcommand{\pdz}{\frac{\partial}{\partial z}}
\newcommand{\intrrr}{\int d^3 r \,}
\newcommand{\intrr}{\int d^2 r \,}
\newcommand{\dEdphi}{\partial_\phi E }
\newcommand{\dEdp}{\partial_p E }
\newcommand{\dBdphi}{\partial_\phi B }
\newcommand{\dBdp}{B }
\newcommand{\adj}{\phi^\dag}
\newcommand{\surf}{\int_{\partial V}}
\newcommand{\domain}{V}
\newcommand{\bound}{\partial V}
\newcommand{\vn}{\vec{n}}
\newcommand{\Edd}{\mathbb{E}}
\newcommand{\BEdd}{B}
\newcommand{\sigt}{\sigma_t}
\newcommand{\sigs}{\sigma_s}
\newcommand{\siga}{\sigma_a}
\newcommand{\isigt}{c}
% why \newcommand{\angSource}{q_\Omega}
\newcommand{\angSource}{q}
\newcommand{\scalSource}{q}
\newcommand{\angResp}{q^\dag}
\newcommand{\scalResp}{q^\dag}
\newcommand{\qoi}{{\it QoI}\xspace}
\newcommand{\Ui}{U_i}
\newcommand{\Uipo}{U_{i+1}}
\newcommand{\Uimo}{U_{i-1}}
\newcommand{\Uo}{U_o}
\newcommand{\Uopo}{U_{o+1}}
\newcommand{\Uomo}{U_{o-1}}
\newcommand{\vT}{\vec{T}}




\begin{document}
\begin{center}
Ian Halvic \\
NUEN 618\\
NOTES (based on project)\\
\end{center}

\section{Forward problem}
To begin, we consider a typical nonlinear steady-state heat-conduction problem with Dirichlet boundary conditions
\begin{equation}
\label{forwardSS}
\begin{split}
& - \div ( k(T) \grad T(\vx) ) = q(\vx) \quad \vx \in \Omega \\
&T(\vx)=a \quad \vx \in \partial \Omega .\\
\end{split}
\end{equation}
Typically, the desired result of a simulation isn't necessarily the solution $T(\vr)$, but rather some some quantity of interest ($\qoi$), such as a detector's response to $T(\vr)$. We will consider QoIs which can be presented in the inner-product form
\begin{equation}
\qoi = \int_{\Omega} r(\vx) T(\vx)
\end{equation}
where our desired QoI is characterized by a response function $r(\vx)$. For example, to find the average temperature in a region $\Omega_r \subseteq \Omega$, set $r(\vx)=\frac{1}{|\Omega_r|}$ for $\vx \in \Omega_r$ and $r(\vx)$ otherwise. Therefor the QoI is the expected form for the average temperature.
\begin{equation}
\qoi = \int_{\Omega} r(\vx) T(\vx) = \frac{1}{|\Omega_r|} \int_{\Omega_r} T(\vx) 
\end{equation}
As a matter of notation, introduce the following notation for volume and surface inner-products.
\begin{equation}
\begin{split}
&\bra f, g \ket = \int_\Omega f g \\
&\bra f, g \ketbd = \oint_{\partial \Omega} f g \cdot \vec{n} \\
\end{split}
\end{equation}
As such, the QoI now takes the simple form
\begin{equation}
\qoi = \bra T , r \ket
\end{equation}

\section{The adjoint expression}
We introduce the adjoint expression of Eq.~(\ref{forwardSS}), with the adjoint source the response function. Homogeneous boundary conditions are chosen for the adjoint system.
\begin{equation}
\label{adjointSS}
\begin{split}
& - \div ( k(T) \grad \phi (\vx) ) = r(\vx) \quad \vx \in \Omega \\
&\phi(\vx)=0 \quad \vx \in \partial \Omega .\\
\end{split}
\end{equation}
The adjoint expression can be used to reformulate the $\qoi$ expression. This is accomplished primarily by integration by parts until the forward equation emerges, then substitute in the forward source.
\begin{equation}
\label{qoidirv}
\begin{split}
\qoi &= \bra T , r \ket \\
&= \bra T , - \div ( k(T) \grad \phi ) \ket \\
&= \bra  \grad T ,  k(T) \grad \phi   \ket  - \bra T, k(T) \grad \phi  \ketbd\\
&= \bra - \div ( k(T) \grad T ), \phi   \ket  - \bra T, k(T) \grad \phi  \ketbd + \bra k(T) \grad T , \phi (\vx) \ketbd\\
&= \bra q , \phi   \ket  - \bra a, k(T) \grad \phi  \ketbd \\
\end{split}
\end{equation}
There now exists a dual representation of the $\qoi$, however the adjoint form is far from helpful. In addition to requiring a forward solve to obtain $\phi$, due to the nonlinear $k(T)$ term we would need to solve for $T$ regardless.
\begin{equation}
\qoi  = \bra T , r \ket = \bra q , \phi   \ket  - \bra a, k(T) \grad \phi  \ketbd
\end{equation}
\section{Sensitivity}
The real power of the adjoint method for determining $\qoi$ shows itself when perturbations are considered. These perturbations can be intentional perturbations to the system, or represent uncertainty in system properties. Consider a perturbation of Eq.~(\ref{forwardSS}), with a potentially perturbed thermal conductivity $k_p$ and source $q_p$. The solution to this system similarly is a perturbed temperature function $T_p$. The perturbed $\qoi_p$ is as expected.
\begin{equation}
\label{forwardSSpert}
\begin{split}
& - \div ( k_p \grad T_p(\vx) ) = q_p(\vx) \quad \vx \in \Omega \\
&T_p(\vx)=a \quad \vx \in \partial \Omega \\
&\qoi_p = \bra T_p , r \ket.
\end{split}
\end{equation}
A $\delta$ notation is introduced for perturbations such that $k_p = k + \delta k$, $q_p = q + \delta q$, and $T_p =  T + \delta T$. We use this notation to express the perturbed forward system.
\begin{equation}
- \div ( (k+\delta k) \grad (T + \delta T) ) = q + \delta q 
\end{equation}
Now for the trick of nonlinear.
\begin{equation}
\delta k \approx \frac{\partial k}{\partial T} \delta T + \frac{\partial k}{\partial p} \delta p 
\end{equation}

\begin{equation}
- \div ( k \grad (T + \delta T) ) - \div ( \frac{\partial k}{\partial T} \delta T \grad (T + \delta T) ) - \div ( \frac{\partial k}{\partial p} \delta p \grad (T + \delta T) ) = q + \delta q 
\end{equation}
first order it
\begin{equation}
- \div ( k \grad (T + \delta T) ) - \div ( \frac{\partial k}{\partial T} \delta T \grad T  ) - \div ( \frac{\partial k}{\partial p} \delta p \grad T ) = q + \delta q 
\end{equation}
Small rearrange
\begin{equation}
- \div ( k \grad (T + \delta T) ) - \div ( \frac{\partial k}{\partial T} \delta T \grad T  )  = q + \delta q + \div ( \frac{\partial k}{\partial p} \delta p \grad T )
\end{equation}
Go to weak form
\begin{equation}
\bra k \grad (T + \delta T) , \grad \phi \ket + \bra \frac{\partial k}{\partial T} \delta T \grad T  , \grad \phi \ket  = \bra q + \delta q , \phi \ket - \bra \frac{\partial k}{\partial p} \delta p \grad T , \grad \phi \ket
\end{equation}
Subtract out the weak unperturbed
\begin{equation}
\bra k \grad \delta T , \grad \phi \ket + \bra \frac{\partial k}{\partial T} \delta T \grad T  , \grad \phi \ket  = \bra  \delta q , \phi \ket - \bra \frac{\partial k}{\partial p} \delta p \grad T , \grad \phi \ket
\end{equation}
isolate $\delta T$
\begin{equation}
\bra \delta T , - \div k \grad \phi \ket + \bra  \delta T , \frac{ \partial k}{\partial T} \grad T  \grad \phi \ket  = \bra  \delta q , \phi \ket - \bra \frac{\partial k}{\partial p} \delta p \grad T , \grad \phi \ket
\end{equation}
meh
\begin{equation}
\bra \delta T , - \div k \grad \phi + \frac{ \partial k}{\partial T} \grad T  \grad \phi \ket  = \bra  \delta q , \phi \ket - \bra \frac{\partial k}{\partial p} \delta p \grad T , \grad \phi \ket
\end{equation}
Now we have our new adjoint.
\begin{equation}
 - \div k \grad \phi + \frac{ \partial k}{\partial T} \grad T  \grad \phi = r
\end{equation}

\end{document}