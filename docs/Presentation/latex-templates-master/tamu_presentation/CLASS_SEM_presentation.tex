\documentclass[xcolor={usenames,dvipsnames,svgnames,table}]{beamer}

\mode<presentation>
\usetheme{Madrid}

\usecolortheme[RGB={80,0,0}]{structure}
\useoutertheme[subsection=false]{miniframes}
\useinnertheme{default}

% hide navigation controlls
\setbeamertemplate{navigation symbols}{}

\setbeamercolor{normal text}{fg=black}
\setbeamercovered{dynamic}
\beamertemplatetransparentcovereddynamicmedium
%\usepackage{chronology}
\setbeamertemplate{caption}[numbered]

\definecolor{Maroon}{RGB}{80,0,0}
\definecolor{BurntOrange}{RGB}{204,85,0}


% load macros and prevent authblk from loading
\input{../common/macros.tex}
\dontusepackage{authblk}

% load packages, settings and definitions
\input{../common/packages.tex}
\input{../common/settings.tex}
%\input{../common/definitions.tex}

\newcommand{\vr}{\vec{r}}
\newcommand{\vp}{\vec{p}}
\newcommand{\vOmega}{\vec{\Omega}}
\newcommand{\vJ}{\vec{J}}
\newcommand{\vO}{\vec{\Omega}}
\newcommand{\bra}{\left\langle}
\newcommand{\ket}{\right\rangle}
\newcommand{\sbra}{\left[}
\newcommand{\sket}{\right]}
\newcommand{\braSN}{\left\langle \! \left\langle}
\newcommand{\ketSN}{\right\rangle \! \right\rangle}
\newcommand{\sbraSN}{\left[ \! \left[}
\newcommand{\sketSN}{\right] \! \right]}
\renewcommand{\div}{\vec{\nabla} \cdot}
\newcommand{\grad}{\vec{\nabla}}
\newcommand{\vbeta}{\vec{\beta} }
\newcommand{\pdx}{\frac{\partial}{\partial x}}
\newcommand{\pdy}{\frac{\partial}{\partial y}}
\newcommand{\pdz}{\frac{\partial}{\partial z}}
\newcommand{\intrrr}{\int d^3 r \,}
\newcommand{\intrr}{\int d^2 r \,}
\newcommand{\dEdphi}{\partial_\phi E }
\newcommand{\dEdp}{\partial_p E }
\newcommand{\dBdphi}{\partial_\phi B }
\newcommand{\dBdp}{B }
\newcommand{\adj}{\phi^\dag}
\newcommand{\vefadj}{\varphi^\dag}
\newcommand{\surf}{\int_{\partial V}}
\newcommand{\domain}{V}
\newcommand{\bound}{\partial V}
\newcommand{\vn}{\vec{n}}
\newcommand{\Edd}{\mathbb{E}}
\newcommand{\BEdd}{B}
\newcommand{\sigt}{\sigma_t}
\newcommand{\sigs}{\sigma_s}
\newcommand{\siga}{\sigma_a}
%\newcommand{\isigt}{\sigma_t^{-1}}
%\newcommand{\isigtp}{\sigma_{t,p}^{-1}}
\newcommand{\isigt}{\ell_t}
\newcommand{\isigtp}{\ell_{t,p}}
\newcommand{\angSource}{\frac{q}{4 \pi}}
\newcommand{\angSourcep}{\frac{q_p}{4 \pi}}
\newcommand{\angSourcepd}{\frac{q+\delta q}{4 \pi}}
\newcommand{\angSourced}{\frac{\delta q}{4 \pi}}
\newcommand{\scalSource}{q}
\newcommand{\angResp}{q^\dag}
\newcommand{\scalResp}{q^\dag}
\newcommand{\qoi}{{\it QoI}\xspace}


% nicer item settings
\setlist[1]{nolistsep,label=\(\textcolor{Maroon}{\blacksquare}\)}
\setlist[2]{nolistsep,label=\(\textcolor{Maroon}{\bullet}\)}

\setenumerate[1]{
	label=\protect\usebeamerfont{enumerate item}%
	\protect\usebeamercolor[fg]{enumerate item}%
	\insertenumlabel.
}

%%%%%%%%%%%%%%%%%%%%%%%%%%%%%%%%%%%%%%%%%%%%%%%
%%% edit to fit your document

% set up pdf support and indexing
\hypersetup{
    pdftitle={Adjoint-based sensitivity for radiation transport using an Eddington tensor formulation},
    pdfauthor={Ian Halvic},
}

\title[ ]{}
\author[ ]{Ian Halvic}
\institute[Texas A\&M]{Department of Nuclear Engineering \\ Texas A\&M University}
\date[2/24/17]

\begin{document}

% title page, do not edit
{
\setbeamertemplate{headline}[default] 
\begin{frame}
\vspace{-1.1cm}
	\begin{figure}[t]
		\centering
			\includegraphics[width=.25\textwidth]{images/seal.png}
	\end{figure}
\vspace{-0.75cm}
\titlepage
\end{frame}
}

%%%%%%%%%%%%%%%%%%%%%%%%%%%%%%%%%%%%%%%%%%%%%%%%%%%%%%%%%%%%%%%%%%%%%%%%%%%%%%%%%%%%%%%%%%%%%%%%%%%%%%%%%%%
\section{Introduction}	% define sections here, it is possible to get section slides automatically, but this is not enabled
\subsection{}	% we have to keep these to get the navigation
%%%%%%%%%%%%%%%%%%%%%%%%%%%%%%%%%%%%%%%%%%%%%%%%%%%%%%%%%%%%%%%%%%%%%%%%%%%%%%%%%%%%%%%%%%%%%%%%%%%%%%%%%%%

\begin{frame}[t]\frametitle{Motivation}
\begin{flushleft}
\begin{itemize}
\item (Layout why we want to use Adjoints. No additional solves to get $\delta \qoi$)
\item (Why we can't just use SN adjoint? memory issues.)
\end{itemize}
\end{flushleft}
\end{frame}
 
 %%%%%%%%%%%%%%%%%%%%%%%%%%%%%%%%%%%%%%%%%%%%%%%%%%%%%%%%%%%%%%%%%%%%%%%%%%%%%%%%%%%%%%%%%%%%%%%%%%%%%%%%%%%

\begin{frame}\frametitle{Approach}
\begin{flushleft}
\begin{itemize}
\item (Maybe Brief Operator overview of adjoint method using generic operator A.)
\item (Layout the problem to be considered: 1 group transport, scalar sources)
\end{itemize}
\end{flushleft}
\end{frame}


%%%%%%%%%%%%%%%%%%%%%%%%%%%%%%%%%%%%%%%%%%%%%%%%%%%%%%%%%%%%%%%%%%%%%%%%%%%%%%%%%%%%%%%%%%%%%%%%%%%%%%%%%%%
\section{Formulations}	% define sections here, it is possible to get section slides automatically, but this is not enabled
\subsection{}	% we have to keep these to get the navigation
%%%%%%%%%%%%%%%%%%%%%%%%%%%%%%%%%%%%%%%%%%%%%%%%%%%%%%%%%%%%%%%%%%%%%%%%%%%%%%%%%%%%%%%%%%%%%%%%%%%%%%%%%%%

\begin{frame}\frametitle{SN Transport}
 \begin{flushleft}
To begin, consider the one-group SN transport system with isotropic source $q$ and scattering $\sigs$
\begin{subequations}
\begin{equation}
\label{SNfwd}
- \vO \cdot \grad \psi + \sigt \psi = \frac{\sigs}{4 \pi} \phi + \angSource
\end{equation}
\begin{equation}
\psi(\vr) = \psi^{ \text{inc}}(\vr) \quad \vr \in \partial V^{+} = \{  \vr \in \bound , \quad \vO \cdot \vec{n} < 0 \}
\end{equation}
\end{subequations}
We are concerned with a volumetric and isotropic \qoi given by $\scalResp$
\begin{equation}
\begin{split}
\qoi 	&= \bra \phi , \scalResp \ket \equiv \int_V \phi \scalResp \, dV \\
		&= \braSN \psi , \scalResp \ketSN \equiv \int_V \int_\Omega \psi \scalResp \, d\Omega dV \\
\end{split}
\end{equation} 
\end{flushleft}
\end{frame}

 %%%%%%%%%%%%%%%%%%%%%%%%%%%%%%%%%%%%%%%%%%%%%%%%%%%%%%%%%%%%%%%%%%%%%%%%%%%%%%%%%%%%%%%%%%%%%%%%%%%%%%%%%%%
 %%%%%%%%%%%%%%%%%%%%%%%%%%%%%%%%%%%%%%%%%%%%%%%%%%%%%%%%%%%%%%%%%%%%%%%%%%%%%%%%%%%%%%%%%%%%%%%%%%%%%%%%%%%

\begin{frame}\frametitle{SN Transport}
 \begin{flushleft}
This presents the most straightforward (and most expensive) way to determine the change in the \qoi due to perturbations in the system parameters
\begin{subequations}
\begin{equation}
\label{SNfwdp}
- \vO \cdot \grad \psi_p + ( \sigt \delta \sigt) \psi_p = \frac{\sigs+\delta \sigs}{4 \pi} \phi_p + \angSourcepd
\end{equation}
\begin{equation}
\psi(\vr) = \psi_p^{ \text{inc}}(\vr) \quad \vr \in \partial V^{+} = \{  \vr \in \bound , \quad \vO \cdot \vec{n} < 0 \}
\end{equation}
\end{subequations}
Leading to the simple result
\begin{equation}
\delta \qoi = \bra \phi_p , \scalResp \ket - \bra \phi , \scalResp \ket = \bra \delta \phi , \scalResp \ket 
\end{equation} 
\end{flushleft}
\end{frame}

 %%%%%%%%%%%%%%%%%%%%%%%%%%%%%%%%%%%%%%%%%%%%%%%%%%%%%%%%%%%%%%%%%%%%%%%%%%%%%%%%%%%%%%%%%%%%%%%%%%%%%%%%%%%
 
 
 %%%%%%%%%%%%%%%%%%%%%%%%%%%%%%%%%%%%%%%%%%%%%%%%%%%%%%%%%%%%%%%%%%%%%%%%%%%%%%%%%%%%%%%%%%%%%%%%%%%%%%%%%%%

\begin{frame}\frametitle{SN Transport Adjoint}
 \begin{flushleft}
To avoid a forward solve for every perturbation case, an adjoint SN method may be employed. The adjoint system is defined
\begin{subequations}
\begin{equation}
- \vO_d \cdot \grad \psi^\dag_d + \sigt \psi^\dag_d = \frac{\sigs}{4 \pi} \phi^\dag + \scalResp
\end{equation}
\begin{equation}
\psi^\dag(\vr) = \psi^{\dag \text{out}}(\vr)=0 \quad \vr \in \partial V^{+} = \{  \vr \in \bound , \quad \vO \cdot \vec{n} > 0 \}
\end{equation}
\end{subequations}
Through the standard adjoint method a surface term is gained in the \qoi
\begin{equation}
\begin{split}
\qoi &= \bra \phi^\dag , \angSource \ket - \sbraSN \psi^\dag , \psi \sketSN \\
&\equiv \bra \phi^\dag , \angSource \ket - \oint_{\partial V} \int_\Omega (\vO \cdot \vn) \psi^\dag  \psi \, d\Omega dS \\
\end{split}
\end{equation} 
\end{flushleft}
\end{frame}

 %%%%%%%%%%%%%%%%%%%%%%%%%%%%%%%%%%%%%%%%%%%%%%%%%%%%%%%%%%%%%%%%%%%%%%%%%%%%%%%%%%%%%%%%%%%%%%%%%%%%%%%%%%%\\
  %%%%%%%%%%%%%%%%%%%%%%%%%%%%%%%%%%%%%%%%%%%%%%%%%%%%%%%%%%%%%%%%%%%%%%%%%%%%%%%%%%%%%%%%%%%%%%%%%%%%%%%%%%%

\begin{frame}\frametitle{SN Transport Adjoint Sensitivity}
 \begin{flushleft}
 Multiply by the perturbed $\psi_p$ by $\scalResp$ and integrate. First order approximation used on last step.
\begin{equation}
\label{snSensPart}
\begin{split}
\qoi_p &=\braSN \psi_p , \angResp \ketSN \\
&=\braSN \psi_p , - \vO \cdot \grad \psi^\dag + \sigt \psi^\dag - \frac{\sigs}{4 \pi} \phi^\dag  \ketSN \\
&=\braSN  \vO \cdot \grad \psi_p + \sigt \psi_p - \frac{\sigs}{4 \pi} \phi_p , \psi^\dag  \ketSN - \sbraSN \psi_p, \psi^\dag \sketSN \\
&= \braSN  \vO \cdot \grad \psi_p + \sigma_{t,p}\psi_p - \delta\sigt\psi_p - \frac{\sigma_{s,p}}{4 \pi} \phi_p
+\frac{\delta \sigs}{4 \pi} \phi_p
 , \psi^\dag  \ketSN - \sbraSN \psi_p, \psi^\dag \sketSN \\
&\approx \braSN  \angSourcepd - \delta\sigt\psi_p + \frac{\delta \sigs}{4 \pi} \phi_p
 , \psi^\dag  \ketSN - \sbraSN \psi_p, \psi^\dag \sketSN
\end{split}
\end{equation}

\end{flushleft}
\end{frame}

 %%%%%%%%%%%%%%%%%%%%%%%%%%%%%%%%%%%%%%%%%%%%%%%%%%%%%%%%%%%%%%%%%%%%%%%%%%%%%%%%%%%%%%%%%%%%%%%%%%%%%%%%%%%
 \begin{frame}\frametitle{SN Transport Adjoint Sensitivity}
 \begin{flushleft}
Perform $\qoi_p - \qoi$ to obtain the change in $\qoi$
\begin{equation}
\label{snSens}
\begin{split}
\delta \qoi 
&\approx \braSN  \angSourced - \delta\sigt\psi_p + \frac{\delta \sigs}{4 \pi} \phi_p
 , \psi^\dag  \ketSN - \sbraSN \delta \psi, \psi^\dag \sketSN
\end{split}
\end{equation}
\begin{itemize}
\item + Do not need SN solve for each perturbation scenario, just 2 SN solves to get $\psi$ and $\psi^\dag$
\item - Only first order perturbation accurate ($\delta \sigs \delta \phi=0$, $\delta \sigt \delta \phi=0$)
\item - Need to store angular dependent $\psi$ and $\psi^\dag$, which can be prohibitive
\end{itemize}
\end{flushleft}
\end{frame}

 %%%%%%%%%%%%%%%%%%%%%%%%%%%%%%%%%%%%%%%%%%%%%%%%%%%%%%%%%%%%%%%%%%%%%%%%%%%%%%%%%%%%%%%%%%%%%%%%%%%%%%%%%%%
 
 
 %%%%%%%%%%%%%%%%%%%%%%%%%%%%%%%%%%%%%%%%%%%%%%%%%%%%%%%%%%%%%%%%%%%%%%%%%%%%%%%%%%%%%%%%%%%%%%%%%%%%%%%%%%%

\begin{frame}\frametitle{VET Transport}
 \begin{flushleft}
Formulate the P1 equations and define an Eddington Tensor $\Edd$ and a Boundary Eddington Factor $\BEdd$ 
\begin{equation}
\div \vec{J} + (\sigt-\sigs) \phi = \scalSource
, \quad \quad 
\div \left(  \int d\Omega \vO \vO \psi \right) + \sigt \vec{J} = 0 
\end{equation}
\begin{equation}
\Edd(\vr)=\frac{\int d\Omega \vO \vO \psi(\vr,\vO)}{\phi(\vr)}
, \quad \quad 
\BEdd(\vr) = \frac{\int_{4 \pi} d\Omega \, | \vO \cdot \vn | \psi}{\int_{4\pi} d\Omega \, \psi} \, \vr \in \bound
\end{equation}
Combine into the VET transport formulation \cite{Miften}
\begin{subequations}
\begin{equation}
\label{VEFForm}
- \div \left( \frac{1}{\sigt}\div \Edd \phi \right) + \siga \phi = \scalSource 
\end{equation}
\begin{equation}
2 J^{\text{inc}} = \BEdd \phi + \vn \cdot \frac{1}{\sigt} \div \Edd \phi 
\end{equation}
\end{subequations}
\end{flushleft}
\end{frame}

%%%%%%%%%%%%%%%%%%%%%%%%%%%%%%%%%%%%%%%%%%%%%%%%%%%%%%%%%%%%%%%%%%%%%%%%%%%%%%%%%%%%%%%%%%%%%%%%%%%%%%%%%%%
 %%%%%%%%%%%%%%%%%%%%%%%%%%%%%%%%%%%%%%%%%%%%%%%%%%%%%%%%%%%%%%%%%%%%%%%%%%%%%%%%%%%%%%%%%%%%%%%%%%%%%%%%%%%

\begin{frame}\frametitle{VET Transport}
 \begin{flushleft}
Formulate the P1 equations and define an Eddington Tensor $\Edd$ and a Boundary Eddington Factor $\BEdd$ 
\begin{equation}
\div \vec{J} + (\sigt-\sigs) \phi = \scalSource
, \quad \quad 
\div \left(  \int d\Omega \vO \vO \psi \right) + \sigt \vec{J} = 0 
\end{equation}
\begin{equation}
\Edd(\vr)=\frac{\int d\Omega \vO \vO \psi(\vr,\vO)}{\phi(\vr)}
, \quad \quad 
\BEdd(\vr) = \frac{\int_{4 \pi} d\Omega \, | \vO \cdot \vn | \psi}{\int_{4\pi} d\Omega \, \psi} \, \vr \in \bound
\end{equation}
Combine into the VET transport formulation \cite{Miften}
\begin{subequations}
\begin{equation}
\label{VEFForm}
- \div \left( \frac{1}{\sigt}\div \Edd \phi \right) + \siga \phi = \scalSource 
\end{equation}
\begin{equation}
2 J^{\text{inc}} = \BEdd \phi + \vn \cdot \frac{1}{\sigt} \div \Edd \phi 
\end{equation}
\end{subequations}
\end{flushleft}
\end{frame}

%%%%%%%%%%%%%%%%%%%%%%%%%%%%%%%%%%%%%%%%%%%%%%%%%%%%%%%%%%%%%%%%%%%%%%%%%%%%%%%%%%%%%%%%%%%%%%%%%%%%%%%%%%%
 
%%%%%%%%%%%%%%%%%%%%%%%%%%%%%%%%%%%%%%%%%%%%%%%%%%%%%%%%%%%%%%%%%%%%%%%%%%%%%%%%%%%%%%%%%%%%%%%%%%%%%%%%%%%

\begin{frame}\frametitle{VET Transport Adjoint}
 \begin{flushleft}
To derive an expression for the VET adjoint equation $\vefadj$, start with the forward VET balance equation, Eq.~\eqref{VEFForm}
\begin{equation}
\label{VEFadjFormDeriv}
\begin{split}
\bra \scalSource , \vefadj \ket &= - \bra \div \left( \frac{1}{\sigt}\div \Edd \phi \right), \vefadj \ket +  \bra \siga \phi, \vefadj \ket   \\
&= \bra \frac{1}{\sigt}\div \Edd \phi, \grad \vefadj \ket  +  \bra  \phi, \siga \vefadj \ket - \sbra \vn \cdot \frac{1}{\sigt}\div \Edd \phi, \vefadj \sket   \\
 &=  - \bra \phi, \Edd : \left( \grad \left( \frac{1}{\sigt}\grad \vefadj \right) \right) \ket  +  \bra  \phi, \siga \vefadj \ket \\
 & - \sbra \vn \cdot  \frac{1}{\sigt}\div \Edd \phi, \vefadj \sket + \sbra \phi, \vn \cdot  \Edd \cdot \frac{1}{\sigt} \grad \vefadj \sket \\
\end{split}
\end{equation}
Where the inner product $\sbra f ,g \sket = \oint_{\partial V} fg \, dS$ has been defined
\end{flushleft}
\end{frame}

 %%%%%%%%%%%%%%%%%%%%%%%%%%%%%%%%%%%%%%%%%%%%%%%%%%%%%%%%%%%%%%%%%%%%%%%%%%%%%%%%%%%%%%%%%%%%%%%%%%%%%%%%%%%
 %%%%%%%%%%%%%%%%%%%%%%%%%%%%%%%%%%%%%%%%%%%%%%%%%%%%%%%%%%%%%%%%%%%%%%%%%%%%%%%%%%%%%%%%%%%%%%%%%%%%%%%%%%%

\begin{frame}\frametitle{VET Transport Adjoint}
 \begin{flushleft}
From the previous balance equation, an obvious form of the VET adjoint emerges
\begin{subequations}
\begin{equation}
\label{adjForm}
- \Edd : \left( \grad \left( \frac{1}{\sigt}\grad \vefadj \right) \right) + \siga \vefadj = \scalResp
\end{equation}
\begin{equation}
\label{adjVETBC}
2J^{\dag,\text{out}} = B \vefadj+ \vn \cdot
\Edd \cdot \frac{1}{\sigma_{t} } \vec{\nabla} \vefadj    \quad \vr \in \bound
\end{equation}
\end{subequations}
Choosing $2J^{\dag,\text{out}} =0$ results in the relatively simple \qoi expression using this new formulation
\begin{equation}
\label{adjVETqoi}
\qoi=\bra \scalSource , \vefadj \ket  + \sbra \vefadj, 2J^{\text{inc}} \sket
\end{equation}
Of particular note is that $ \phi^\dag \neq \vefadj$
\end{flushleft}
\end{frame}

 %%%%%%%%%%%%%%%%%%%%%%%%%%%%%%%%%%%%%%%%%%%%%%%%%%%%%%%%%%%%%%%%%%%%%%%%%%%%%%%%%%%%%%%%%%%%%%%%%%%%%%%%%%%
  %%%%%%%%%%%%%%%%%%%%%%%%%%%%%%%%%%%%%%%%%%%%%%%%%%%%%%%%%%%%%%%%%%%%%%%%%%%%%%%%%%%%%%%%%%%%%%%%%%%%%%%%%%%

\begin{frame}\frametitle{VET Transport Adjoint Response}
 \begin{flushleft}
Now an assumption is made that the $\Edd$ and $\BEdd$ remain unperturbed under system perturbation. Note $\ell_t = \frac{1}{\sigt}$
\begin{equation}
\label{VETSensDeriv}
\begin{split}
\qoi_p = &\bra \phi_p , \scalResp \ket \\
       = &\bra \phi_p , - \Edd : \left( \grad \isigt \grad \varphi^\dag \right) + \siga \vefadj \ket \\
=& \bra \scalSource + \delta \scalSource + \div \delta \isigt \div \left( \Edd \phi \right) - \delta \siga \phi, \vefadj \ket - \sbra \phi_p, \Edd \cdot \isigt \grad \vefadj \sket \\
&+ \sbra \vefadj, \isigt \div \Edd \phi_p \sket \\
=& \bra q, \vefadj \ket  + \bra \delta \scalSource + \div \delta \isigt \div \left( \Edd \phi \right)  - \delta \siga \phi, \vefadj \ket \\
& - \sbra \phi_p, \Edd \cdot \isigt \grad \vefadj \sket + \sbra \vefadj, \isigt \div \Edd \phi_p \sket 
\end{split}
\end{equation}

\end{flushleft}
\end{frame}

 %%%%%%%%%%%%%%%%%%%%%%%%%%%%%%%%%%%%%%%%%%%%%%%%%%%%%%%%%%%%%%%%%%%%%%%%%%%%%%%%%%%%%%%%%%%%%%%%%%%%%%%%%%%

  %%%%%%%%%%%%%%%%%%%%%%%%%%%%%%%%%%%%%%%%%%%%%%%%%%%%%%%%%%%%%%%%%%%%%%%%%%%%%%%%%%%%%%%%%%%%%%%%%%%%%%%%%%%

\begin{frame}\frametitle{VET Transport Adjoint Response}
 \begin{flushleft}
 Subtracting off the unperturbed \qoi in Eq.~\eqref{adjVETqoi} and cleaning up some boundary terms results in
 \begin{equation}
\label{VETsens}
\begin{split}
\delta \qoi =&  \bra \delta \scalSource - \delta \siga \phi, \vefadj \ket  - \bra \delta \isigt \div \left( \Edd \phi \right) , \grad \vefadj \ket + \sbra \vefadj, 2 \delta J^{\text{inc}} \sket \\
\end{split}
\end{equation}
\begin{itemize}
\item + Requires only 1 SN solve to get $\Edd$ and $\phi$, and one scalar VET solve for $\varphi^\dag$
\item + No angular fluxes stored, only $\Edd$
\item - Only first order perturbation accurate (as was SN adjoint)
\item - Major assumption that $\delta \Edd=0$, which is not always true. This means that the method in general is not exact for source perturbations $\delta q$ and $\delta \psi^{\text{inc}}$, which was the case for SN

\end{itemize}
\end{flushleft}
\end{frame}
 %%%%%%%%%%%%%%%%%%%%%%%%%%%%%%%%%%%%%%%%%%%%%%%%%%%%%%%%%%%%%%%%%%%%%%%%%%%%%%%%%%%%%%%%%%%%%%%%%%%%%%%%%%%

\begin{frame}\frametitle{``Blended'' Approach}
 \begin{flushleft}
Form a $\delta \qoi$ inner-product by picking the exact source terms from SN adjoint Eq.~\eqref{snSens} and cross-section terms from VET adjoint Eq.~\eqref{VETsens}
\begin{equation}
\label{Blendsens}
\begin{split}
\delta \qoi =&  \bra \angSourced , \phi^\dag \ket - \bra \delta \siga \phi, \vefadj \ket - \bra \delta \isigt \div \left( \Edd \phi \right) , \grad \vefadj \ket
- \sbraSN \delta \psi^\text{inc}, \psi^\dag \sketSN \\
\end{split}
\end{equation}
\begin{itemize}
\item + Requires 2 SN solves to get $\Edd$, $\phi$ and $\phi^\dag$ then one VET solve for $\varphi^\dag$
\item + No angular fluxes stored
\item + Still exact for source perturbations (equivalent to SN adjoint)
\item - Lacks rigorous derivation
\item - Could experience issues when both source and cross-section perturbation are present, compared to pure VET adjoint

\end{itemize}
\end{flushleft}
\end{frame}

 %%%%%%%%%%%%%%%%%%%%%%%%%%%%%%%%%%%%%%%%%%%%%%%%%%%%%%%%%%%%%%%%%%%%%%%%%%%%%%%%%%%%%%%%%%%%%%%%%%%%%%%%%%%
 
  %%%%%%%%%%%%%%%%%%%%%%%%%%%%%%%%%%%%%%%%%%%%%%%%%%%%%%%%%%%%%%%%%%%%%%%%%%%%%%%%%%%%%%%%%%%%%%%%%%%%%%%%%%%

\begin{frame}\frametitle{Estimate $\delta \Edd$}
 \begin{flushleft}
\begin{equation}
\delta \Edd \approx \frac{\partial \Edd}{\partial \vp} \cdot \delta \vp \approx \left( \frac{\Edd(\vp_1) - \Edd(\vp_0)}{\vp_1 - \vp_0} \right) \delta \vp
\end{equation}
\end{flushleft}
\end{frame}

 %%%%%%%%%%%%%%%%%%%%%%%%%%%%%%%%%%%%%%%%%%%%%%%%%%%%%%%%%%%%%%%%%%%%%%%%%%%%%%%%%%%%%%%%%%%%%%%%%%%%%%%%%%%
 
 %%%%%%%%%%%%%%%%%%%%%%%%%%%%%%%%%%%%%%%%%%%%%%%%%%%%%%%%%%%%%%%%%%%%%%%%%%%%%%%%%%%%%%%%%%%%%%%%%%%%%%%%%%%

\begin{frame}\frametitle{Adjoint-VET (Alternate VET)}
\begin{flushleft}
Reverse the order of the VET derivation, start with adjoint P1 and works towards an alternate forward $\varphi$
\begin{equation}
\label{0amAlt}
\div \vec{J}^\dag + (\sigt-\sigs) \phi^\dag  = 4\pi \scalResp 
, \quad \quad 
\div \left(  \int d\Omega \vO \vO \psi^\dag  \right) + \sigt \vec{J}^\dag  = 0
\end{equation}
\begin{equation}
\Edd^\dag(\vr)=\frac{\int d\Omega \vO \vO \psi^\dag(\vr,\vO)}{\phi^\dag(\vr)}
, \quad \quad 
\BEdd(\vr) = \frac{\int_{4 \pi} d\Omega \, | \vO \cdot \vn | \psi^\dag}{\int_{4\pi} d\Omega \, \psi^\dag} \, \vr \in \bound
\end{equation}
Gives the familiar form
\begin{subequations}
\begin{equation}
\label{TranAdjVEFForm}
- \div \left( \frac{1}{\sigt}\div \Edd^\dag \phi^\dag  \right) + \siga \phi^\dag  = 4\pi \scalResp  \,.
\end{equation}
\begin{equation}
2 J^{^\dag,\text{out}}(\vr) = \BEdd^\dag(\vr) \phi^\dag(\vr)  + \vn \cdot \frac{1}{\sigt} \div \Edd^\dag  \phi^\dag  \,.
\end{equation}
\end{subequations}
\end{flushleft}
\end{frame}

 %%%%%%%%%%%%%%%%%%%%%%%%%%%%%%%%%%%%%%%%%%%%%%%%%%%%%%%%%%%%%%%%%%%%%%%%%%%%%%%%%%%%%%%%%%%%%%%%%%%%%%%%%%%
 
 %%%%%%%%%%%%%%%%%%%%%%%%%%%%%%%%%%%%%%%%%%%%%%%%%%%%%%%%%%%%%%%%%%%%%%%%%%%%%%%%%%%%%%%%%%%%%%%%%%%%%%%%%%%

\begin{frame}\frametitle{Adjoint-VET (Alternate VET)}
 \begin{flushleft}
Adjoint of this Alternate VET is proposed
\begin{subequations}
\begin{equation}
\label{ForwardVEFAlt}
- \Edd^\dag : \grad \left( \frac{1}{\sigt}\grad \varphi \right) + \siga \varphi  = \angSource  \,.
\end{equation}
\begin{equation}
2 J^{\text{inc}}(\vr) = \BEdd^\dag(\vr) \varphi(\vr) + \Edd^\dag \cdot \frac{1}{\sigt} \grad \varphi  \,
\end{equation}
\end{subequations}
\qoi derived starting with basic SN definition
 \begin{equation}
\label{AdjQoIAltExpand}
\begin{split}
QoI = \bra \phi , \scalResp \ket &= \bra \phi^\dag , \angSource \ket - \sbraSN \psi^\dag,  \psi \sketSN \\
&= \bra \phi^\dag , - \Edd^\dag : \grad \left( \frac{1}{\sigt}\grad \varphi \right) + \siga \varphi \ket - \sbra \psi^\dag,  \psi \sket \\
&= \bra 4\pi \scalResp  ,\varphi \ket - \sbra \psi^\dag,  \psi \sket  
- \sbra \Edd^\dag \cdot \frac{1}{\sigt}\grad \varphi,  \phi^\dag \sket 
+ \sbra \frac{1}{\sigt} \div \Edd^\dag \phi^\dag,  \varphi \sket \\
&=  \bra 4\pi \scalResp  ,\varphi \ket - \sbra \psi^\dag,  \psi \sket - \sbra \phi^\dag, 2J^{\text{inc}} \sket + \sbra \varphi , 2 J^{\dag,\text{out}} \sket
\end{split}
\end{equation}
\end{flushleft}
\end{frame}

 %%%%%%%%%%%%%%%%%%%%%%%%%%%%%%%%%%%%%%%%%%%%%%%%%%%%%%%%%%%%%%%%%%%%%%%%%%%%%%%%%%%%%%%%%%%%%%%%%%%%%%%%%%%

 %%%%%%%%%%%%%%%%%%%%%%%%%%%%%%%%%%%%%%%%%%%%%%%%%%%%%%%%%%%%%%%%%%%%%%%%%%%%%%%%%%%%%%%%%%%%%%%%%%%%%%%%%%%

\begin{frame}\frametitle{Adjoint-VET (Alternate VET)}
\begin{flushleft}
Perturbations are introduced in the adjoint VET $\phi^\dag$ system, $\varphi$ system remains unperturbed 
\begin{equation}
\begin{split}
- \bra\div \left( \isigt \div \Edd^\dag \phi^\dag_p  \right) +  \siga  \phi^\dag_p , \varphi \ket  =& \bra \div \left( \delta \isigt \div \Edd^\dag \phi^\dag  \right) - \delta \siga , \varphi \ket \\ 
&+ \bra 4 \pi (\scalResp +\delta \scalResp) \phi^\dag , \varphi \ket \\
\end{split}
\end{equation}
The adjoint process is then applied to the LHS
\begin{equation}
\label{AltVetPertDeriv2}
\begin{split}
 \bra  \phi^\dag_p , \angSource \ket  =& \bra 4 \pi \scalResp + 4 \pi\delta \scalResp , \varphi \ket + \bra\div \left( \delta \isigt \div \Edd^\dag \phi^\dag  \right), \varphi \ket 
- \bra \delta \siga \phi^\dag , \varphi \ket \\
&- \sbra \phi_p^\dag, 2J^{\text{inc}} \sket + \sbra \varphi , 2 J^{\dag,\text{out}} - \delta \isigt \div \Edd \phi^\dag \sket  \\
\end{split}
\end{equation} 
\end{flushleft}
\end{frame}

 %%%%%%%%%%%%%%%%%%%%%%%%%%%%%%%%%%%%%%%%%%%%%%%%%%%%%%%%%%%%%%%%%%%%%%%%%%%%%%%%%%%%%%%%%%%%%%%%%%%%%%%%%%%
 
  %%%%%%%%%%%%%%%%%%%%%%%%%%%%%%%%%%%%%%%%%%%%%%%%%%%%%%%%%%%%%%%%%%%%%%%%%%%%%%%%%%%%%%%%%%%%%%%%%%%%%%%%%%%

\begin{frame}\frametitle{Adjoint-VET (Alternate VET)}
\begin{flushleft}
Some cleanup gives 
\begin{equation}
\delta \qoi = \bra \delta q^\dag, \varphi \ket - \bra\left( \delta \isigt \div \Edd^\dag \phi^\dag  \right), \grad \varphi \ket \
- \bra \delta \siga \phi^\dag , \varphi \ket + \bra \delta q , \phi \ket - \sbraSN \delta \psi^{\text{inc}}, \psi^\dag \sketSN
\end{equation}
\begin{itemize}
\item + Requires 1 SN solve to get $\Edd^\dag$ and $\phi^\dag$ then one VET solve for $\varphi$
\item + No angular fluxes stored
\item + Still exact for volumetric source perturbations $\delta q$
\item - Assumes an unperturbed $\Edd^\dag$
\item - Not exact for perturbed response $\delta q^\dag$, which was trivial for forward derived methods 
\end{itemize}
\end{flushleft}
\end{frame}

 %%%%%%%%%%%%%%%%%%%%%%%%%%%%%%%%%%%%%%%%%%%%%%%%%%%%%%%%%%%%%%%%%%%%%%%%%%%%%%%%%%%%%%%%%%%%%%%%%%%%%%%%%%%



%%%%%%%%%%%%%%%%%%%%%%%%%%%%%%%%%%%%%%%%%%%%%%%%%%%%%%%%%%%%%%%%%%%%%%%%%%%%%%%%%%%%%%%%%%%%%%%%%%%%%%%%%%%
\section{Test Cases}	% define sections here, it is possible to get section slides automatically, but this is not enabled
\subsection{}	% we have to keep these to get the navigation
%%%%%%%%%%%%%%%%%%%%%%%%%%%%%%%%%%%%%%%%%%%%%%%%%%%%%%%%%%%%%%%%%%%%%%%%%%%%%%%%%%%%%%%%%%%%%%%%%%%%%%%%%%%
 %%%%%%%%%%%%%%%%%%%%%%%%%%%%%%%%%%%%%%%%%%%%%%%%%%%%%%%%%%%%%%%%%%%%%%%%%%%%%%%%%%%%%%%%%%%%%%%%%%%%%%%%%%%

\begin{frame}\frametitle{Solver}
\begin{flushleft}
\begin{itemize}
\item (Describe solver approach: 1D, FEM, ect.)
\end{itemize}
\end{flushleft}
\end{frame}

\begin{frame}\frametitle{Homogeneous System, Homogeneous Perturbation}
 \begin{flushleft}

\begin{figure}[H]
\label{Trial1}
\centering
\begin{subfigure}{.5\textheight}
  \centering
  \includegraphics[width=.98\linewidth]{figures2/22qSens.png}
  \label{T1:sfig1}
\end{subfigure}%
\begin{subfigure}{.5\textheight}
  \centering
  \includegraphics[width=.98\linewidth]{figures2/22sigaSens.png}
  \label{T1:sfig2}
\end{subfigure}
%
\begin{subfigure}{.5\textheight}
  \centering
  \includegraphics[width=.98\linewidth]{figures2/22sigsSens.png}
  \label{T1:sfig3}
\end{subfigure}%
\begin{subfigure}{.5\textheight}
  \centering
  \includegraphics[width=.98\linewidth]{figures2/22qsigaSens.png}
  \label{T1:sfig4}
\end{subfigure}
\end{figure}
\end{flushleft}
\end{frame}

\begin{frame}\frametitle{Homogeneous System, Homogeneous Perturbation}
 \begin{flushleft}
\begin{table}[H]
\label{TableT1}
\centering
  \begin{tabular}{| l | r | r | r | r |}
    \hline
    Method  &  $+10\% q $  & $-10\% \siga $ & $+10\% \sigs $ & $+10\% q,-10\% \siga$ \\ \hline
     SN Fwd 			&0.39998 &0.44419 &5.7577e-05 & 0.88858\\ \hline
     VET Fwd			&0.39998 &0.44428 &2.7131e-05 &0.88868\\ \hline
     SN Adj			&0.39998 &0.39983 &6.6307e-05 &0.79980\\ \hline
     VET Adj 			&0.39998 &0.39988 &2.8534e-05 &0.79986\\ \hline
     Blended 			&0.39998 &0.39988 &2.8534e-05 &0.79986\\ \hline
     VET $\delta \Edd$ 	&0.39998 &0.39983 &5.9537e-05 &0.79981\\ \hline
     VET Alt			& --  & --  &	--		&  --\\ \hline
    \end{tabular}
  \caption{Table of selected results for the homogeneous system under homogeneous perturbations. Values given are absolute $\delta \Edd$. }
\end{table}
\end{flushleft}
\end{frame}


\begin{frame}\frametitle{Homogeneous System, Non-Homogeneous Perturbation}
 \begin{flushleft}

\begin{figure}[H]
\label{Trial1}
\centering
\begin{subfigure}{.5\textheight}
  \centering
  \includegraphics[width=.98\linewidth]{figures2/23qSens.png}
  \label{T1:sfig1}
\end{subfigure}%
\begin{subfigure}{.5\textheight}
  \centering
  \includegraphics[width=.98\linewidth]{figures2/23sigaSens.png}
  \label{T1:sfig2}
\end{subfigure}
%
\begin{subfigure}{.5\textheight}
  \centering
  \includegraphics[width=.98\linewidth]{figures2/23sigsSens.png}
  \label{T1:sfig3}
\end{subfigure}%
\begin{subfigure}{.5\textheight}
  \centering
  \includegraphics[width=.98\linewidth]{figures2/23qsigaSens.png}
  \label{T1:sfig4}
\end{subfigure}
\end{figure}
\end{flushleft}
\end{frame}

\begin{frame}\frametitle{Homogeneous System, Non-Homogeneous Perturbation}
 \begin{flushleft}
\begin{table}[H]
\centering
  \begin{tabular}{| l | r | r | r | r |}
    \hline
    Method  &  $+10\% q $  & $-10\% \siga $ & $+10\% \sigs $ & $+10\% q,-10\% \siga$ \\ \hline
     SN Fwd 			&0.36309 &0.39952 &2.9680e-05 & 0.79915\\ \hline
     VET Fwd			&0.35947 &0.39517 &1.5072e-05 &0.79040\\ \hline
     SN Adj  			&0.36309 &0.36301 &3.4051e-05 &0.72610\\ \hline
     VET Adj 			&0.35947 &0.35941 &1.5733e-05 &0.71888\\ \hline
     Blended 			&0.36309 &0.35941 &1.5733e-05 &0.72250\\ \hline
     VET $\delta \Edd$ 	&0.36287 &0.36290 &3.0640e-05 &0.72586\\ \hline
     VET Alt			& --  & --  &	--		&  --\\ \hline
    \end{tabular}
  \caption{Table of selected results for the homogeneous system under inhomogeneous perturbations. Values given are absolute $\delta \Edd$. }
\end{table}
\end{flushleft}
\end{frame}


\begin{frame}\frametitle{Shielded Incident Flux}
 \begin{flushleft}

\begin{figure}[H]
\label{Trial1}
\centering
\begin{subfigure}{.5\textheight}
  \centering
  \includegraphics[width=.98\linewidth]{figures2/24qSens.png}
  \label{T1:sfig1}
\end{subfigure}%
\begin{subfigure}{.5\textheight}
  \centering
  \includegraphics[width=.98\linewidth]{figures2/24sigaSens.png}
  \label{T1:sfig2}
\end{subfigure}
%
\begin{subfigure}{.5\textheight}
  \centering
  \includegraphics[width=.98\linewidth]{figures2/24sigsSens.png}
  \label{T1:sfig3}
\end{subfigure}%
\begin{subfigure}{.5\textheight}
  \centering
  \includegraphics[width=.98\linewidth]{figures2/24qsigaSens.png}
  \label{T1:sfig4}
\end{subfigure}
\end{figure}
\end{flushleft}
\end{frame}

\begin{frame}\frametitle{Homogeneous System, Non-Homogeneous Perturbation}
 \begin{flushleft}
\begin{table}[H]
\centering
  \begin{tabular}{| l | r | r | r | r |}
    \hline
    Method  &  $+10\% \psi^- $  & $-10\% \siga $ & $+10\% \sigs $ & $+10\% \psi^-,-10\% \siga$ \\ \hline
     SN Fwd 			&0.023401 &0.021079 &-0.0067476 & 0.046588\\ \hline
     VET Fwd			&0.023181 &0.019670 &-0.0066481 &0.044818\\ \hline
     SN Adj  			&0.023401 &0.019975 &-0.0068956 &0.043376\\ \hline
     VET Adj 			&0.023181 &0.018981 &-0.0067751 &0.042162\\ \hline
     Blended 			&0.023401 &0.018981 &-0.0067751 &0.042381\\ \hline
     VET $\delta \Edd$ 	&0.023181 &0.020070 &-0.0065156 &0.043251\\ \hline
     VET Alt			&0.023401 &0.035869 &	--		&  --\\ \hline
    \end{tabular}
  \caption{Table of selected results for the shielding system under perturbations. Values given are absolute $\delta \Edd$. }
\end{table}
\end{flushleft}
\end{frame}

%%%%%%%%%%%%%%%%%%%%%%%%%%%%%%%%%%%%%%%%%%%%%%%%%%%%%%%%%%%%%%%%%%%%%%%%%%%%%%%%%%%%%%%%%%%%%%%%%%%%%%%%%%%
%%%%%%%%%%%%%%%%%%%%%%%%%%%%%%%%%%%%%%%%%%%%%%%%%%%%%%%%%%%%%%%%%%%%%%%%%%%%%%%%%%%%%%%%%%%%%%%%%%%%%%%%%%%

%%%%%%%%%%%%%%%%%%%%%%%%%%%%%%%%%%%%%%%%%%%%%%%%%%%%%%%%%%%%%%%%%%%%%%%%%%%%%%%%%%%%%%%%%%%%%%%%%%%%%%%%%%%
\section{Wrap Up}	% define sections here, it is possible to get section slides automatically, but this is not enabled
\subsection{}	% we have to keep these to get the navigation
%%%%%%%%%%%%%%%%%%%%%%%%%%%%%%%%%%%%%%%%%%%%%%%%%%%%%%%%%%%%%%%%%%%%%%%%%%%%%%%%%%%%%%%%%%%%%%%%%%%%%%%%%%%
 %%%%%%%%%%%%%%%%%%%%%%%%%%%%%%%%%%%%%%%%%%%%%%%%%%%%%%%%%%%%%%%%%%%%%%%%%%%%%%%%%%%%%%%%%%%%%%%%%%%%%%%%%%%

\begin{frame}\frametitle{Solver}
\begin{flushleft}
\begin{itemize}
\item (Final remarks, next steps, issues with scattering in 1D, references,)
\end{itemize}
\end{flushleft}
\end{frame}

\end{document}