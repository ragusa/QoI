\documentclass[12pt]{report}

\usepackage{tamuconfig}

% Most of the packages that set the default settings
% for the document have moved to the style file
% tamuconfig.sty.

%These next lines change the font. Fixes for certain
%fonts will be implemented in a future release.

%Comment this line if you do not wish to use Times
%New Roman. The font used will then be the LaTeX
%default of Computer Modern.
\usepackage{times}
%\usepackage{cmbright}
\usepackage[T1]{fontenc}

%This package allows for the use of graphics in the
%document.
\usepackage{amsmath, amsthm, amssymb, amsfonts, booktabs, graphicx, float, esint, subcaption, xspace, xcolor}

%\usepackage{stix}
\usepackage{stmaryrd}
\usepackage{hyperref}

%If you have JPEG format images, add .jpg as an
%allowed file extension below. Same for Bitmaps (.bmp).
\DeclareGraphicsExtensions{.png}

%It is best practice to keep all your pictures in
%one folder inside the main directory in which your
%TeX file is kept. Here the folder is named "graphic."
%Replace the name here with your folder's name, if needed.
%The period is needed due to relative referencing.
\graphicspath{ {./graphic/} }

%%%%%%%%%%%%%%%%%%%%%%%%%%%%%%%%%%%%%%%%%%%%%%%%%%%%%%%%%
%Please place all your personal packages here. Check to
%see if the packages you wish to use are not already
%declared above. Placing all your personal packages
%here allows me to determine if there are any package
%issues in compilation, as well as any conflicts
%that may arise by the order of loading.
%--Sean Zachary Roberson
%%%%%%%%%%%%%%%%%%%%%%%%%%%%%%%%%%%%%%%%%%%%%%%%%%%%%%%%%
%%%%%%%%%%%%%%%%%%%%%%%%%%%%%%%%%%%%%%%%%%%%%%%%%%%%%%%%%
%Begin student defined packages.
%%%%%%%%%%%%%%%%%%%%%%%%%%%%%%%%%%%%%%%%%%%%%%%%%%%%%%%%%


\setlength{\abovedisplayskip}{0pt}
\setlength{\belowdisplayskip}{0pt}
\setlength{\abovedisplayshortskip}{0pt}
\setlength{\belowdisplayshortskip}{0pt}

\newcommand{\vr}{\vec{r}}
\newcommand{\vp}{\vec{p}}
\newcommand{\vOmega}{\vec{\Omega}}
\newcommand{\vJ}{\vec{J}}
\newcommand{\vO}{\vec{\Omega}}
\newcommand{\bra}{\left\langle}
\newcommand{\ket}{\right\rangle}
\newcommand{\braSN}{\left\langle \! \left\langle}
\newcommand{\ketSN}{\right\rangle \! \right\rangle}
\newcommand{\sbraSN}{\left[ \! \left[}
\newcommand{\sketSN}{\right] \! \right]}
\newcommand{\sbra}{\left[}
\newcommand{\sket}{\right]}
\renewcommand{\div}{\vec{\nabla} \cdot}
\newcommand{\grad}{\vec{\nabla}}
\newcommand{\vbeta}{\vec{\beta} }
\newcommand{\pdx}{\frac{\partial}{\partial x}}
\newcommand{\pdy}{\frac{\partial}{\partial y}}
\newcommand{\pdz}{\frac{\partial}{\partial z}}
\newcommand{\intrrr}{\int d^3 r \,}
\newcommand{\intrr}{\int d^2 r \,}
\newcommand{\dEdphi}{\partial_\phi E }
\newcommand{\dEdp}{\partial_p E }
\newcommand{\dBdphi}{\partial_\phi B }
\newcommand{\dBdp}{B }
\newcommand{\adj}{\phi^\dag}
\newcommand{\vefadj}{\varphi^\dag}
\newcommand{\surf}{\int_{\partial V}}
\newcommand{\domain}{V}
\newcommand{\bound}{\partial V}
\newcommand{\vn}{\vec{n}}
\newcommand{\Edd}{\mathbb{E}}
\newcommand{\BEdd}{B}
\newcommand{\sigt}{\sigma_t}
\newcommand{\sigs}{\sigma_s}
\newcommand{\siga}{\sigma_a}
%\newcommand{\isigt}{\sigma_t^{-1}}
%\newcommand{\isigtp}{\sigma_{t,p}^{-1}}
\newcommand{\isigt}{\ell_t}
\newcommand{\isigtp}{\ell_{t,p}}
\newcommand{\angSource}{\frac{q}{4 \pi}}
\newcommand{\angSourcep}{\frac{q_p}{4 \pi}}
\newcommand{\angSourcepd}{\frac{q_p+\delta q_p}{4 \pi}}
\newcommand{\angSourced}{\frac{\delta q}{4 \pi}}
\newcommand{\scalSource}{q}
\newcommand{\angResp}{q^\dag}
\newcommand{\scalResp}{q^\dag}
\newcommand{\qoi}{{\it QoI}\xspace}


\newcommand{\comment}[2]{\marginpar{\textcolor{#2}{$\star$}}\textcolor{#2}{#1}\newline}

%-----------------------------------------------------------
%-----------------------------------------------------------
\usepackage{ifthen}
\newboolean{draftversion}
\setboolean{draftversion}{false}
%-----------------------------------------------------------
%----------------------------------------------------------

\ifthenelse{\boolean{draftversion}}
{
\newcommand{\iwh}[1]{\comment{#1}{red}}
\newcommand{\jcr}[1]{\comment{#1}{blue}}
\newcommand{\todo}[1]{\comment{#1}{purple}}
}
{
\newcommand{\iwh}[1]{\phantom{a}}
\newcommand{\jcr}[1]{\phantom{a}}
\newcommand{\todo}[1]{\phantom{a}}
}
\newcommand{\tcr}[1]{\textcolor{red}{#1}}

%%%%%%%%%%%%%%%%%%%%%%%%%%%%%%%%%%%%%%%%%%%%%%%%%%%%%%%%%
%End student defined packages.
%%%%%%%%%%%%%%%%%%%%%%%%%%%%%%%%%%%%%%%%%%%%%%%%%%%%%%%%%

% End preamble. Document begins below.

\begin{document}

%The title of your document goes here.
%Spacing may need to be adjusted if your title is long
%and pushes the copyright off the page.
\renewcommand{\tamumanuscripttitle}{Adjoint-based Sensitivity for Radiation Transport Using an Eddington Tensor Formulation}

%Type only Thesis, Dissertation, or Record of Study.
\renewcommand{\tamupapertype}{Thesis DRAFT}

%Your full name goes here, as it is in university records. Check your student record on Howdy if there is any mismatch.
\renewcommand{\tamufullname}{Ian Halvic}

%The degree title goes here. See the OGAPS site for more info.
\renewcommand{\tamudegree}{Master of Science}
\renewcommand{\tamuchairone}{Professor Jean C. Ragusa}


% Uncomment out the next line if you have co-chairs.  You will also need to edit the titlepage.tex file.
%\newcommand{\tamuchairtwo}{Additional Chair Name}
\renewcommand{\tamumemberone}{Professor Marvin L. Adams}
\newcommand{\tamumembertwo}{Professor Bojan Popov}
\renewcommand{\tamudepthead}{Professor Yassin Hassan}

%Type only May, August, or December.
\renewcommand{\tamugradmonth}{March}
\renewcommand{\tamugradyear}{2018}
%Your department name goes here.
\renewcommand{\tamudepartment}{Nuclear Engineering}


\include{data/titlepage} % This is simply a file that formats and adds your titlepage, please do not edit this unless you have a specific need. .
%%%%%%%%%%%%%%%%%%%%%%%%%%%%%%%%%%%%%%%%%%%%%%%%%%%
%%%%%%%%%%%%%%%%%%%%%%%%%%%%%%%%%%%%%%%%%%%%%%%%%%%%%%%%%%%%%%%%%%%%%
%%                           ABSTRACT 
%%%%%%%%%%%%%%%%%%%%%%%%%%%%%%%%%%%%%%%%%%%%%%%%%%%%%%%%%%%%%%%%%%%%%

\chapter*{ABSTRACT}
\addcontentsline{toc}{chapter}{ABSTRACT} % Needs to be set to part, so the TOC doesnt add 'CHAPTER ' prefix in the TOC.

\pagestyle{plain} % No headers, just page numbers
\pagenumbering{roman} % Roman numerals
\setcounter{page}{2}

\todo{Add Abstract}


 

\pagebreak{}

%%%%%%%%%%%%%%%%%%%%%%%%%%%%%%%%%%%%%%%%%%%%%%%%%%%%%%%%%%%%%%%%%%%%%%%
%%                           DEDICATION
%%%%%%%%%%%%%%%%%%%%%%%%%%%%%%%%%%%%%%%%%%%%%%%%%%%%%%%%%%%%%%%%%%%%%
\chapter*{DEDICATION}
\addcontentsline{toc}{chapter}{DEDICATION}  % Needs to be set to part, so the TOC doesnt add 'CHAPTER ' prefix in the TOC.



\begin{center}
\vspace*{\fill}
\todo{Add Dedications Here}
\vspace*{\fill}
\end{center}

\pagebreak{}

%%%%%%%%%%%%%%%%%%%%%%%%%%%%%%%%%%%%%%%%%%%%%%%%%%%%%%%%%%%%%%%%%%%%%%
%%                           ACKNOWLEDGMENTS
%%%%%%%%%%%%%%%%%%%%%%%%%%%%%%%%%%%%%%%%%%%%%%%%%%%%%%%%%%%%%%%%%%%%%
\chapter*{ACKNOWLEDGMENTS}
\addcontentsline{toc}{chapter}{ACKNOWLEDGMENTS}  % Needs to be set to part, so the TOC doesnt add 'CHAPTER ' prefix in the TOC.
I would like to thank my advisor Professor Jean C. Ragusa for his invaluable help on this work and other academic endeavors. 
 
I would also like to extend my thanks to Professor Adams and Professor Popov for appearing on the committee for the defense of this thesis.



\pagebreak{}

\include{data/contributors}
%%%%%%%%%%%%%%%%%%%%%%%%%%%%%%%%%%%%%%%%%%%%%%%%%%%%%%%%%%%%%%%%%%%%%%
%%                           NOMENCLATURE
%%%%%%%%%%%%%%%%%%%%%%%%%%%%%%%%%%%%%%%%%%%%%%%%%%%%%%%%%%%%%%%%%%%%%

\chapter*{NOMENCLATURE}
\addcontentsline{toc}{chapter}{NOMENCLATURE}  % Needs to be set to part, so the TOC doesnt add 'CHAPTER ' prefix in the TOC.

%A note about aligning: These entries will align
%themselves according to the ampersand (&).
%No extra spaces are needed, as seen in some of
%the entries below.

%Example of the longtable environment.
\hspace*{-1.25in}
\vspace{12pt}
\begin{spacing}{1.0}
	\begin{longtable}[htbp]{@{}p{0.35\textwidth} p{0.62\textwidth}@{}}
	   % \begin{tabular}{@{}p{0.33\textwidth} p{0.62\textwidth}@{}}
		VET 		& Variable Eddington Tensor\\	[2ex]
		aVET 		& Adjoint Variable Eddington Tensor\\	[2ex]
		SN 			& Discrete Ordinate Method\\	[2ex]
		$\bra \bullet , \bullet \ket$ 				& Volumetric Inner Product $\int_V dV$ \\	[2ex]
		$\braSN \bullet , \bullet \ketSN$ 			& Volumetric-Angular Inner Product $\int_V \int_{4 \pi} d\Omega dV$ \\	[2ex]
		$\sbra \bullet , \bullet \sket$ 			& Boundary Inner Product $\oint_{\partial V} dS$ \\	[2ex]
		$\sbraSN \bullet , \bullet \sketSN$ 		& Boundary-Angular Inner Product $\oint_{\partial V} \int_{4 \pi} d\Omega dS (\vO \cdot \vn)$ \\	[2ex]
	   % \end{tabular}%
	\end{longtable}
\end{spacing}

\pagebreak{}

\include{data/lists}  % This is simply a file that formats and adds your toc, lof, and lot, please do not edit this unless you have a specific need.

%%%%%%%%%%%%%%%%%%%%%%%%%%%%%%%%%%%%%%%%%%%%%%%%%%%%%%%%%%%%%%%%%%%%%%
%%                           SECTION I
%%%%%%%%%%%%%%%%%%%%%%%%%%%%%%%%%%%%%%%%%%%%%%%%%%%%%%%%%%%%%%%%%%%%%


\pagestyle{plain} % No headers, just page numbers
\pagenumbering{arabic} % Arabic numerals
\setcounter{page}{1}


\chapter{\uppercase {Introduction}}

%%%%%%%%%%%%%%%%%%%%%%%%%%%%%%%%%%%%%%%%%%%%%%%%%%%%%%%%%%%%%%%%%%%%%%%%%%%%%%%%%%%%%%%%%%%%%%%%%%%%
\section{Introduction}
%%%%%%%%%%%%%%%%%%%%%%%%%%%%%%%%%%%%%%%%%%%%%%%%%%%%%%%%%%%%%%%%%%%%%%%%%%%%%%%%%%%%%%%%%%%%%%%%%%%%

Computational simulations have become important tools for engineers and scientists across a wide array of disciplines. These simulations allow for researchers to examine significantly complex and long life systems in a way that is frequently more economical in both time and money than construction of the real world system, if even feasible. An important step in creating one of these methods is confirmation that the results can be trusted to reasonably approximate the real life scenario. This can be accomplished using three processes outlined by the National Research Council \cite{NRCVVUQ}


\begin{itemize}
\item Verification - How accurately does the computation solve the underlying equations of the model for the quantities of interest?
\item Validation - How accurately does the model represent reality for the quantities of interest?
\item Uncertainty Quantification (UQ) -  How do the various sources of error and uncertainty feed into uncertainty in the model-based prediction of the quantity of interest?
\end{itemize}

Adjoint methods are of particular interest for UQ. In general, adjoint methods provide a mechanism for propagating uncertainty and error in the system variables to the uncertainty in the desired quantity of interest. Adjoint methods accomplish this in a particularly economical way, sometimes requiring only two differential system solves which can then be used for any combination of sources of error, as opposed to performing an independent solve for each individual error scenario. These adjoint methods have been applied across various complex and time dependent systems. An example of adjoint methods applied to hydrodynamic systems with shocks can be found in Wildey et al. \cite{Wildey}. A more relevant adjoint example to neutron transport occurs in Stripling et al. in the form of reactor burn-up equations \cite{Stripling}.


Application of the adjoint method to time-dependent transport can pose a major technical limitation. In general, the adjoint method applied to radiation transport requires storing six-dimensional data (the forward angular flux) at each time step. When dealing with high resolution in these six dimensions and many time steps, this can potentially require an unreasonable amount of memory for data storage, rendering the method functionally unusable. 

Typically, a checkpointing method can be employed to alleviate some of the memory limitations of a time dependent adjoint system \cite{Carey} \cite{Griewank}. In this scheme, the forward solution, i.e., the angular flux in the case of transport, is stored at checkpoint timesteps when performing a forward solve from $t=0$ to $T$. The adjoint method then initializes at the final time step, which sweeps in the reverse time direction. At any point the forward solution is needed in the reverse sweep for a quantity of interest calculation the forward solution is reconstructed by performing a forward solve starting with the nearest checkpoint. Ultimately, this becomes a balancing act of memory usage for storing checkpoints and additional computation time for recomputing the forward solution from checkpoints. An efficient checkpointing scheme which chooses the checkpoints in a binomial fashion is presented by Griewank and Walther \cite{Griewank}. However this method still requires the storing of the primary time-dependent variable, the angular flux, at the checkpoint times. For large phase-space solutions, this can severely limit the number of checkpoints that can be stored in memory, resulting in a increase in compute time for calculating the uncertainty propagation.

\iwh{In response to: The forward solution at the final time step $T$ is used to initialize the adjoint method [not true. where did you see this?]. Sorry, I think is was slipped out from reading the generalization of the checkpoint method, and not transport specifically. If my comprehension is right then in general this is possible, though not for transport as considered here. Would occur at least if we had some dependence on the forward solution in the adjoint formulation. Say temperature dependent cross sections, where the temperature is due to the forward flux $\sigma(\phi)$, so to start the adjoint transport solve we would need the $\sigma(\phi)$ value for TF, so the forward scalar flux may need to be retrieved. In this point checkpointing would not be just for qoi calculation, but for forming the system parameters for the entire adjoint solve. }

%For the previously mentioned depletion reactor problem of Stripling, this limitation was not existent because a quasi-static approximation was used for the neutron transport equation and only isotopic concentrations were %time-dependent. For time-independent radiation problems, only storing the converged scattering moments at defined checkpoints (in burnup) is sufficient to reconstruct the flux solution. If a flux solution is requested %outside of the checkpointed burnups, the forward flux can be reconstructed by interpolation \cite{Stripling} \cite{Carey}.
%Using converged scattering source term, a single transport sweep is needed to recover the converged
%angular flux. However this method does not extend to time-dependent transport in general because
%the primary time-dependent variable, the angular flux, needs to be stored as well. Furthermore, while not in scope for this writing, even with a checkpointing method applied to the transport equation in a straightforward %way, the adjoint solution which would solve in the reverse time direction as well as possibly the method to recompute the forward angular flux would typically require using an angular flux solver \cite{Carey}. Using some of %the methods laid out in this thesis could allow for the angular flux solves to be replaced with a quasi-diffusion solver, which could provide increased performance or provide another method for balancing between computation %and I/O.

Moreover, many quantities of interest in transport can be computed from the scalar flux and do not require the angular flux. Having to store and recompute the angular flux only for it to be integrated over angle and converted to a scalar flux for use can be a computational burden. A potential solution to the memory requirement for the time-dependent transport adjoint formulation is the use of a quasi-diffusion method to reduce the overall dimensionality of the transport problem, from 6D+time (space, direction, energy, time for transport) to 4D+time (space, energy, time for quasi-diffusion). The quasi-diffusion method examined in this work is termed  as a ``Variable Eddington Tensor'' (VET) formulation and uses the unperturbed forward angular flux to compute the Eddington tensor needed in the quasi-diffusion approach. While not the focus of this thesis, in the VET formulation the primary time-dependent variable is the scalar flux. Therefore, with the proper adjoint formulation a checkpointing method could be applied to the VET formulation in which scalar flux are checkpointed as a method to further reduce memory requirements.

This thesis is organized as follows. In Chapter \ref{chap:background} the typical transport adjoint method is demonstrated and the inner products required for sensitivity calculations are derived. Then in Chapter \ref{chap:VET} a quasi-diffusion formulation is taken of the transport equation. This quasi-diffusion approach then undergoes an adjoint treatment to derive sensitivity inner products that do not require storing any angular fluxes. An alternate quasi-diffusion method is proposed in Chapter \ref{chap:aVET}, a quasi-diffusion formulation is taken of the adjoint transport equation and a similar approach is taken in an attempt to derive another set of inner products to use for sensitivity. The three methods and the relation between them are shown in Figure~\ref{fig:roadmap}. In Chapter \ref{chap:results} the methods are tested against each other on simple system to verify expected behavior and examine their accuracy. Conclusions about the derived methods based on the test cases are presented in Chapter \ref{chap:conclusion}.

\begin{figure}[H]
\centering
  \includegraphics[scale=0.6]{figures2/RoadMap.png}
  \caption{Diagram showing relations between various formulations in this thesis.}
  \label{fig:roadmap}
\end{figure}


%%%%%%%%%%%%%%%%%%%%%%%%%%%%%%%%%%%%%%%%%%%%%%%%%%%%%%%%%%%%%%%%%%%%%%%%%%%%%%%%%%%%%%%%%%%
\chapter{\uppercase {Background (Transport)}} \label{chap:background}
%%%%%%%%%%%%%%%%%%%%%%%%%%%%%%%%%%%%%%%%%%%%%%%%%%%%%%%%%%%%%%%%%%%%%%%%%%%%%%%%%%%%%%%%%%%%%%%%%%%%

This work will focus on a relatively simple transport equation form, the one-group steady-state transport equation. Examination of effectiveness of an adjoint formalism using the quasi-diffusion approximation instead of the full transport solution in this setting will provide insight to the advantages and shortcomings of the technique when applied to multigroup, time-dependent transport equation. A cursory formulation of the one-group time-dependent quasi-diffusion equation can be found in Appendix~\ref{chap:appx1}, which shows some terms shared with the steady state formulation. In the following chapter, the forward steady-state one-group transport equation is presented, the corresponding adjoint equation is derived, then sensitivity inner products are derived using a first order approximation. 

%%%%%%%----------------------------------------------------------------------------------------------
\section{Steady-state One-group Neutron Transport Equation}
%%%%%%%----------------------------------------------------------------------------------------------

The one-group steady-state transport equation with isotropic sources and isotropic scattering for a volume $V$ bounded by its surface $\partial V$ is given below.
\begin{equation}
\label{SS1GTE}
\vO \cdot \grad \psi(\vr,\vO) + \sigt(\vr) \psi(\vr,\vO) = \frac{1}{4 \pi} \sigs(\vr) \phi(\vr) + \frac{1}{4 \pi} q(\vr), \quad \forall \vr \in V
\end{equation}
\begin{equation}
\label{SS1GTE_bc}
\psi(\vr,\vO) = \psi^{\text{inc}}(\vr,\vO) \quad \vr \in \partial V^{-} = \{ \vr \in \partial V, \text{ s.t. }, \vO \cdot \vec{n}(\vr) < 0\}
\end{equation}
The possibly uncertain parameters in this system are: the total and scattering cross sections $\sigt$ and $\sigs$, the volumetric source term $q$, and the incident angular flux on the system given by $\psi^{\text{inc}}$. The unknowns (dependent variables) are the angular flux $\psi(\vr,\vO)$ and the scalar flux $\phi(\vr)$ given by
\[
\phi(\vr) = \int_{4\pi}d\Omega\,\psi(\vr,\vO) \,.
\]
%%%%%%%----------------------------------------------------------------------------------------------
%\subsection{Quantity of interest and their sensitivities}
%%%%%%%----------------------------------------------------------------------------------------------

%----------------------------------------------------------------------------------------------
\subsection{Quantity of Interest, Response Functions, and Inner Products}
%----------------------------------------------------------------------------------------------
Frequently, the solution to the transport equation is not the sought after value, but rather some Quantity of Interest (\qoi), a functional that depends on the transport solution. Given $\psi(\vr,\vO)$, the solution of the one-group steady-state transport (Eq.~\eqref{SS1GTE}), and $R(\vr, \vO)$, a ``response function'' specific to the desired \qoi, the quantity of interest is defined as
\begin{equation}
\qoi :=  \int_V dV \int_{4 \pi} d \vO \,  R(\vr, \vO) \psi(\vr, \vO)
\end{equation}
The response function $R$ can take on physically defined forms, such as the cross section of a detector; or it may take a form of a mathematical construct, such as $R(\vr, \vO)=1/v$ to return the total number of neutrons present in the system. Another example is to let $R(\vr, \vO)=\sigma \chi(\vr)$ to obtain the total reaction rate in a portion of the domain ($\chi(\vr)=1$ if $\vr \in$ region of interest, and 0 otherwise). Note that the response function will frequently be expressed as $q^\dag$, the adjoint source, as we have already noted that there is a relationship between the solution, the adjoint solution, and their respective source terms. 

Two volumetric inner products are defined using $\braSN \bullet , \bullet \ketSN$ and $\bra \bullet , \bullet \ket$ notations. These two inner-products are for use with angular and scalar fluxes, respectively. 
\begin{subequations}
\begin{equation}
\braSN \psi , f \ketSN  = \int_V dV \int_{4 \pi} d \Omega \,  \psi(\vr, \vO)f(\vr, \vO) \,,
\end{equation}
\begin{equation}
\bra \phi(\vr) , f \ket  = \int_V dV \,  \phi(\vr) g(\vr) \,.
\end{equation}
\end{subequations}
For later use, two additional inner products are also defined as surface integrals over the domain 
boundary $\partial V$. The second definition is used to distinguish between incoming and outgoing surface integrals.
\begin{subequations}
\begin{equation}
\sbraSN \psi , f \sketSN  = \int_{\bound} dS \int_{4 \pi} d \Omega \, \vO \cdot \vn(\vr) \, \psi(\vr, \vO)f(\vr, \vO) \,,
\end{equation}
\begin{equation}
\sbraSN \psi , f \sketSN_{\pm}   = \int_{\bound} dS \int_{\vO \cdot \vn \gtrless 0} d\Omega \,  \vO \cdot \vn(\vr) \, \psi(\vr, \vO)f(\vr, \vO) \,.
\end{equation}
\end{subequations}
Therefore, with this notation, the quantity of interest can be compactly expressed as shown below.
\begin{equation}
\label{QoIDef}
\qoi := \braSN \psi(\vr,\vO), \scalResp(\vr) \ketSN  = \bra \phi(\vr) , \scalResp(\vr) \ket
\end{equation}

%----------------------------------------------------------------------------------------------
\subsection{Sensitivity Coefficients}
%----------------------------------------------------------------------------------------------
A hurdle in utilizing the transport equation numerically to make real world predictions is that 
the system's parameters ($\sigt$, $\sigs$, $q$, and $\psi^{inc}$) may not be known exactly. This uncertainty in 
the system parameters is expected to translate to an uncertainty in the \qoi value. Ideally, a reasonable error 
range would be determined for each system parameter and the system simulation would run over a finely 
discretized parameter space, using the resulting \qoi values to generate an error range for the \qoi. 
However, this straightforward method tends to be resource-intensive, requiring a complete forward solve 
of the transport equation for each input uncertainty scenario. Adjoint methods offer a way to drastically 
reduce the number of solves, while generally remaining fairly accurate for small perturbations around 
base or nominal values of the parameters.

%%%%%%%----------------------------------------------------------------------------------------------
\section{Adjoint Sensitivity}
%%%%%%%----------------------------------------------------------------------------------------------

Adjoint operators can provide a useful tool for sensitivity calculations. Using inner product 
notation consider the system of interest $\mathbf{A} \psi = q$. Call this this 
the forward system, with forward operator $\mathbf{A}$. Consider a test function $\psi^\dag$, the 
adjoint operator $\mathbf{A^\dag}$ is defined such that $\braSN \mathbf{A} \psi, \psi^\dag \ketSN = \braSN \psi, 
\mathbf{A^\dag} \psi^\dag \ketSN $. For differential operators, derivation of $\mathbf{A^\dag}$ generally relies on 
application of the divergence theorem (integration by parts), typically resulting in boundary 
terms ($BC$). Using the response function of the desired \qoi, the adjoint system can be constructed 
as $\mathbf{A^\dag} \psi^\dag = q^\dag$, leading to an alternate expression of the \qoi using the 
adjoint solution $\psi^\dag $.
\begin{equation}
\label{genAdjQoI}
\qoi = \braSN \psi, \scalResp \ketSN 
= \braSN \psi, \mathbf{A^\dag} \psi^\dag \ketSN 
= \braSN \mathbf{A} \psi, \psi^\dag \ketSN + BC
= \braSN q , \psi^\dag \ketSN + BC
\end{equation} 
Using the \qoi definition, we formulate an approximation to the change in the quantity of interest ($\delta \qoi$) using only perturbations in the initial system, i.e., using perturbation to the forward 
operator $\delta \mathbf{A}$ and forward source $\delta q$, but omitting changes in the forward solution itself \cite{Marchuk}.  Derivation of this approximation to the sensitivity coefficient begins with the perturbed system $\mathbf{A}_p \psi_p = q_p$ multiplied by the adjoint function $\psi^\dag$ defined above. After expressing the perturbations in a $\delta$ form a first order approximation of $\delta \mathbf{A} \delta \psi = 0$ is used. An integration by parts is used to transpose $\mathbf{A}$ to $\mathbf{A^\dag}$, resulting in boundary terms appearing. 
\begin{equation}
\begin{split}
\braSN \mathbf{A}_p \psi_p ,\psi^\dag\ketSN &= \braSN q_p ,\psi^\dag \ketSN \\
\braSN \left( \mathbf{A}+\delta \mathbf{A} \right) \left( \psi + \delta \psi \right),\psi^\dag\ketSN &= \braSN q+\delta q ,\psi^\dag \ketSN \\
\braSN \mathbf{A} \psi +\delta \mathbf{A} \psi + \mathbf{A} \delta \psi ,\psi^\dag\ketSN &\approx \braSN q+\delta q ,\psi^\dag \ketSN \\
\braSN \mathbf{A} \psi + \mathbf{A} \delta \psi ,\psi^\dag\ketSN &= \braSN q+\delta q - \delta \mathbf{A} \psi  ,\psi^\dag \ketSN \\
\braSN \psi +\delta \psi , \mathbf{A^\dag} \psi^\dag\ketSN &= \braSN q+\delta q - \delta \mathbf{A} \psi  ,\psi^\dag \ketSN + BC \\
\braSN \psi_p , q^\dag \ketSN &= \braSN q+\delta q - \delta \mathbf{A} \psi  ,\psi^\dag \ketSN + BC \\
\end{split}
\end{equation}
The left side of the final step of the above derivation is the perturbed \qoi. Subtracting Eq.~\eqref{genAdjQoI} from this yields the desired expression for the $\delta\qoi$.
\begin{equation}
\label{genAdjSens}
\delta \qoi \approx \braSN \delta q - \delta \mathbf{A} \psi , \psi^\dag \ketSN + \delta BC
\end{equation}
The advantage of the above expression for $\delta\qoi$ is that two solves, one for the forward and another for the adjoint, can be used to approximate the sensitivity for a variety of operator and source perturbations, $\delta \mathbf{A}$ and $\delta q$.

%%%%%%%----------------------------------------------------------------------------------------------
\subsection{Adjoint Formulation for Transport}
%%%%%%%----------------------------------------------------------------------------------------------
In a fairly straightforward application of the adjoint method previously shown, 
\begin{equation}
\label{SnAdjDeriv}
\begin{split}
\braSN \frac{q}{4 \pi} , \psi^\dag  \ketSN  
=& \braSN \vO \cdot \grad \psi + \sigt \psi - \frac{\sigs}{4 \pi} \phi, \psi^\dag \ketSN \\
=& \braSN \vO \cdot \grad \psi, \psi^\dag \ketSN  + \braSN \sigt \psi, \psi^\dag \ketSN  - \braSN \frac{\sigs}{4 \pi}  \phi, \psi^\dag \ketSN \\
=& - \braSN  \psi, \vO \cdot \grad \psi^\dag \ketSN  + \braSN \psi, \sigt \psi^\dag \ketSN  - \braSN   \psi, \frac{\sigs}{4 \pi} \phi^\dag \ketSN + \sbraSN \psi^\dag,  \psi \sketSN\\
=& \braSN  \psi,-  \vO \cdot \grad \psi^\dag  +  \sigt \psi^\dag -  \frac{\sigs}{4 \pi} \phi^\dag \ketSN + \sbraSN \psi^\dag,  \psi \sketSN\\
=& \braSN  \psi,q^\dag \ketSN + \sbraSN \psi^\dag,  \psi \sketSN\\
\end{split}
\end{equation}
the adjoint equation which corresponds to the transport formulation with adjoint source (response function)
$\angResp$ is

\begin{subequations}
\begin{equation}
\label{transAdj}
- \vO \cdot \grad \psi^\dag + \sigt \psi^\dag = \frac{\sigs}{4 \pi} \phi^\dag + \scalResp
\end{equation}
\begin{equation}
\psi^\dag(\vr) = \psi^{\dag, \text{out}}(\vr)=0 \quad \vr \in \partial V^{+} = \{  \vr \in \bound , \quad \vO \cdot \vec{n} > 0 \}
\end{equation}
\end{subequations}
where the definition of the adjoint scalar flux $\phi^\dag$ is analogous to that of 
the forward scalar flux. It is worth noting that the adjoint equation is in the form of the standard transport equation, only with the direction of travel reversed ($\vO \to -\vO)$. This often allows for forward transport solvers to be easily adapted to solving the adjoint transport system. Once the adjoint solution is obtained, the corresponding \qoi can be calculated with a simple inner product with the forward source term, as follows from equations Eq.~\eqref{genAdjQoI} and Eq.~\eqref{SnAdjDeriv}. 
\begin{equation}
\label{snAdjQoI}
\qoi := \braSN  \psi,q^\dag \ketSN = \braSN \psi^\dag , \angSource \ketSN - \sbraSN \psi^\dag,  \psi \sketSN
\end{equation}
%
The surface interval in \eqref{snAdjQoI} can be split into incoming and outgoing flux integrals, 
\[
\sbraSN \psi^\dag, \psi \sketSN_- 
= \sbraSN \psi^\dag,  \psi^{\text{inc}} \sketSN_- + \sbraSN \psi^{\dag,\text{out}},\psi \sketSN_+
\]
which are handled by the forward and adjoint boundary conditions respectively. Setting $\psi^{\dag, \text{out}}=0$ removes the outgoing flux integral:
%
\begin{equation}
\label{snAdjQoIBCsplit}
\begin{split}
\qoi &= \braSN \psi^\dag , \angSource \ketSN - \sbraSN \psi^\dag,  \psi^{\text{inc}} \sketSN_- \\
&= \bra \phi^\dag , \angSource \ket - \sbraSN \psi^\dag,  \psi^{\text{inc}} \sketSN_- \,.
\end{split}
\end{equation}

%%%%%%%----------------------------------------------------------------------------------------------
\subsection{Transport Adjoint Sensitivity}
%%%%%%%----------------------------------------------------------------------------------------------

Now consider perturbations to our system. Specifically perturbations of $\delta \sigt$, $\delta \sigs$, and $\delta q$ to the total cross section, scattering cross section and angular source term respectively. In addition, the incident angular flux is also perturbed by $\delta \psi^{inc}$. These perturbations result in a perturbed solution to the SN-transport equation $\psi_p$. 
\begin{equation}
\label{snFwdPert}
\vO \cdot \grad \psi_p + \sigma_{t,p} \psi_p = \frac{\sigma_{s,p}}{4 \pi} \phi_p + \frac{q_p}{4\pi},  \quad \forall \vr \in V
\end{equation}
\begin{equation}
\psi_p(\vr) = \psi_p^{\text{inc}}(\vr), \quad \forall \vr \in \partial V^{-}
\end{equation}
Any quantity with a subscript $p$ is to be understood as the perturbed value, that is, as 
the sum of the unperturbed value and the perturbation amount: $a_p = a + \delta a$.

This perturbation may result in a change to the \qoi, now given by $\qoi_p=\bra \psi_p , \angResp \ket$. Note that we assumed that the response function (adjoint source) was not affected by the perturbation. Using the {\it unperturbed} adjoint equation given in Eq.~\eqref{transAdj},
the perturbed \qoi can be expressed as:
\begin{equation}
\label{snSensPart}
\begin{split}
QoI_p &=\braSN \psi_p , \angResp \ketSN \\
&=\braSN \psi_p , - \vO \cdot \grad \psi^\dag + \sigt \psi^\dag - \frac{\sigs}{4 \pi} \phi^\dag  \ketSN \\
\end{split}
\end{equation}
Next, we perform an integration by parts and obtain:
\begin{equation}
\label{snSensPart2}
QoI_p = \braSN  \vO \cdot \grad \psi_p + \sigt \psi_p - \frac{\sigs}{4 \pi} \phi_p , \psi^\dag  \ketSN - \sbraSN \psi_p, \psi^\dag \sketSN
\end{equation}
Note that the cross sections are unperturbed in Eq.~\eqref{snSensPart2}.
Using a $\delta$ notation for the perturbed system variables ($\sigma_{s,p} = \sigs + \delta \sigs$ for example), we
can introduce the perturbed quantities:
\begin{equation}
\label{snSensPart3}
\begin{split}
QoI_p &= \braSN  \vO \cdot \grad \psi_p + (\sigma_{t,p}- \delta\sigt)\psi_p - \frac{\sigma_{s,p}-\delta \sigs}{4 \pi} \phi_p
 , \psi^\dag  \ketSN - \sbraSN \psi_p, \psi^\dag \sketSN \\
 &= \braSN  \angSourcep - \delta\sigt\psi_p + \frac{\delta \sigs}{4 \pi} \phi_p
 , \psi^\dag  \ketSN - \sbraSN \psi_p, \psi^\dag \sketSN
\end{split}
\end{equation}
Next, note that the cross section terms have double perturbations terms $\delta \sigt \delta \psi$ and $\delta \sigs \delta \psi$ once it is observed
that $\psi_p=\psi+\delta\psi$. In a first-order approximation, these doubly perturbed terms are ignored, yielding:
\begin{equation}
\label{snAdjQoIp}
QoI_p \approx \braSN  \frac{q+\delta q}{4\pi} - \delta\sigt\psi + \frac{\delta \sigs}{4 \pi} \phi
 , \psi^\dag  \ketSN - \sbraSN \psi_p, \psi^\dag \sketSN \,.
\end{equation}

Subtraction of the unperturbed \qoi expression in Eq.~\eqref{snAdjQoI} supplies a final equation for computing the change in \qoi using only the system perturbations and the unperturbed forward solution $\psi$ and the adjoint unperturbed
solution $\psi^\dag$, removing the need to solve the perturbed forward equation. 
Furthermore the boundary terms can be split into incoming and outgoing contributions.
Using a zero outgoing boundary condition for the adjoint flux
(thus $\sbra \delta \psi, \psi^{\dag,\text{out}} \sket_+=0$), one obtains the final 
form of the perturbation in the \qoi.
\begin{equation}
\label{snSens}
\begin{split}
\delta QoI &= \braSN \frac{\delta q}{4\pi} - \delta \sigt \psi + \frac{\delta\sigs}{4 \pi} \phi  , \psi^\dag  \ketSN - \sbraSN \delta \psi, \psi^\dag \sketSN \\
&= \braSN \angSourced - \delta \sigt \psi + \frac{\delta\sigs}{4 \pi} \phi , \psi^\dag  \ketSN - \sbraSN \delta \psi^{\text{inc}}, \psi^\dag \sketSN_- \\
&= \bra \angSourced  + \frac{\delta\sigs}{4 \pi} \phi , \phi^\dag  \ket - \braSN  \delta \sigt \psi , \psi^\dag \ketSN - \sbraSN \delta \psi^{\text{inc}}, \psi^\dag \sketSN_- \,.
\end{split}
\end{equation}
Note that in the final formula, only the forward and adjoint scalar fluxes
are required to evaluate the first-order sensitivity due to perturbations in either the isotropic external source or the isotropic scattering source term. However, the forward angular fluxes are needed to evaluate sensitivities due to perturbations in the total cross section. Additionally, the adjoint incident flux
is also needed to assess sensitivity due to the boundary source, but these inner
products concern only the domain boundary and the memory storage requirements for these are small compared to volumetric inner products.

%%%%%%%%%%%%%%%%%%%%%%%%%%%%%%%%%%%%%%%%%%%%%%%%%%%%%%%%%%%%%%%%%%%%%%%%%%%%%%%%%%%%%%%%%%%%%%%%%%%%
\chapter{\uppercase {Variable Eddington Tensor Formulations For Adjoint-based Uncertainty Quantification}} \label{chap:VET}
%%%%%%%%%%%%%%%%%%%%%%%%%%%%%%%%%%%%%%%%%%%%%%%%%%%%%%%%%%%%%%%%%%%%%%%%%%%%%%%%%%%%%%%%%%%%%%%%%%%%

%%%%%%%%%%%%%%%%%%%%%%%%%%%%%%%%%%%%%%%%%%%%%%%%%%%%%%%%%%%%%%%%%%%%%%%%%%%%%%%%%%%%%%%%%%%%%%%%%%%%
\section{Motivation for Sensitivity Analysis based on Variable Eddington Tensor Formulations}
%%%%%%%%%%%%%%%%%%%%%%%%%%%%%%%%%%%%%%%%%%%%%%%%%%%%%%%%%%%%%%%%%%%%%%%%%%%%%%%%%%%%%%%%%%%%%%%%%%%%

%%%%%%%----------------------------------------------------------------------------------------------
%\subsection{Motivation} 
%%%%%%%----------------------------------------------------------------------------------------------
While the adjoint transport sensitivity formulation given by Eq.~\eqref{snSens} provides a first-order accurate method to determine the sensitivity to multiple perturbation scenarios using only one forward transport solve and one adjoint transport solve, it can quickly run into limitations for time-dependent systems, where the forward and adjoint systems in space, energy, and angle must be stored at various time moments for subsequent retrieval to compute the sensitivities $\delta \qoi$. For a 3-d geometry, this translates to storing the angular
flux data across 6-dimensions (space/energy/angle). For time independent systems, Stripling \cite{Stripling} showed that a converged scattering source can be stored to reconstruct 
the forward steady-state transport solution on the fly. For time-dependent problems, one possibility to circumvent the storage issues is to employ a Variable Eddington Tensor (VET) approach (which only requires
storing a space/energy, hence 4-d, solution at various moments in time), provided that input parameter perturbations do not affect the values of the Eddington tensor. In this work,
we investigate this question in the simpler setting of a steady-state system.
\jcr{in the MS thesis, expand on how this would work for time-dependent problems. the entire
gist of the VET idea was to simplify sensitivity calculations for TD transport, so you need
to demonstrate/explain that the ideas you applied in SS were actually going to be useful as well for TD}

\todo{Add in more detail about time dependence. The ideal scenario would be to add in the Eddington equation with time derivatives, however that wouldn't make sense at this exact point in the thesis, as we haven't even derived the steady-state Eddington formulation yet. After a qualitative discussion here, consider looping back towards the end of thesis, presenting the time-dependent VET formulation.}

\iwh{Going to place the derivation in an appendix. This is requested in a few places, not sure if you want it both here and earlier.}

%%%%%%%----------------------------------------------------------------------------------------------
\subsection{Forward VET Formulation}
%%%%%%%----------------------------------------------------------------------------------------------

A VET formulation will reduce the memory requirements when using the adjoint method for sensitivity evaluations. To present the VET formulation, the zero-th and first angular moments of steady-state transport equation are computed by application of the $\int d \Omega$ and $\int d \Omega \, \vO$ operators to Eq.~\eqref{1gTE}, respectively. Using the notation

\begin{equation}
\label{VETFormStart}
\phi(\vr)=\int d\Omega \, \psi( \vr,\vO )
,\quad
\vec{J}(\vr)= \int d\Omega \, \vO \psi( \vr,\vO ) \,,
\end{equation}
the zero-th and first angular moment transport equations are~: 
%
\begin{subequations}
%
\begin{equation}
\label{0am}
\div \vec{J} + (\sigt-\sigs) \phi = \scalSource \,,
\end{equation}
and
\begin{equation}
\label{1am}
\div \left(  \int d\Omega \vO \vO \psi \right) + \sigt \vec{J} = 0 \,.
\end{equation}
%
\end{subequations}
The Eddington Tensor $\Edd$ is then introduced to relate the second angular moment term in equation \eqref{1am} to the scalar flux. The caveat to the Eddington Tensor is that it requires the angular flux solution be known.
\begin{equation}
\label{EddDef}
\Edd(\vr)=\frac{\int d\Omega \vO \vO \psi(\vr,\vO)}{\phi(\vr)} \,.
\end{equation}Note that the Eddington tensor is spatially dependent.
The inclusion of the Eddington tensor allows Eq.~\eqref{1am} to be expressed as 
\begin{equation}
\label{currentE}
\vec{J} = - \frac{1}{\sigt} \div \Edd \phi \,.
\end{equation}
If $\psi(\vr,\vO)$ is a linear function in angle, then $\Edd(\vr)=\tfrac{1}{3}\mathbb{I}$ and one recovers Fick's law for the neutron diffusion current, $\vec{J} = - \frac{1}{3\sigt} \grad \phi$ (note the change from $\div$ to $\grad$). Using Eq.~\eqref{currentE} for the definition of $\vec{J}$ allows us to convert Eq.~\eqref{0am} to the form shown in \eqref{VEFForm}, which only has the scalar flux as an unknown. The substitution $\siga = \sigt-\sigs$ was also used.
\begin{equation}
\label{VEFForm}
- \div \left( \frac{1}{\sigt}\div \Edd \phi \right) + \siga \phi = \scalSource \,.
\end{equation}
The known incident angular flux can be used to generate a suitable boundary conditions using a 
``Boundary Eddington Factor'' $\BEdd(\vr)$ \cite{Miften}. To derive the VET boundary condition for, begin by multiplying the transport boundary condition,
Eq.~\eqref{SS1GTE_bc}, by $2 | \vO \cdot \vn |$. We obtain, for $\vr \in \bound$,
\begin{equation}
2 J^{\text{inc}}(\vr) \equiv  2 \int_{\vO \cdot \vn <0 }  d \Omega\, | \vO \cdot \vec{n} | \psi^{\text{inc}}(\vr,\vO) 
= 2\int_{\vO \cdot \vn <0 } d \Omega\,  | \vO \cdot \vn |  \psi(\vr,\vO) \,.
\end{equation}
$J^{\text{inc}}$ is the partial incoming current. 
Manipulating the second half-range integral yields:
\begin{equation}
2 J^{\text{inc}}(\vr) = \int_{4\pi} d \Omega\,  \left( | \vO \cdot \vn |- \vO\cdot\vn\right)  \psi(\vr,\vO) 
= \BEdd(\vr) \phi(\vr) - \vn(\vr) \cdot \vJ(\vr)
\end{equation}
with $\vJ$ the net current. The boundary Eddington factor is defined as
\begin{equation}
\BEdd(\vr) = \frac{\int_{4 \pi} d\Omega \, | \vO \cdot \vn | \psi}{\phi} ,\ \ \vr \in \bound \,.
\end{equation}
$\vJ$ can substituted using \eqref{currentE} to get the final form of the VET forward boundary condition:
\begin{equation}
2 J^{\text{inc}} = \BEdd \phi + \vn \cdot \frac{1}{\sigt} \div \Edd \phi \,.
\end{equation}
Note that the Robin boundary conditions for diffusion is recovered when $\Edd = \tfrac{1}{3} \mathbb{I}$ and $B=\tfrac{1}{2}$:
\[
2 J^{\text{inc}} = \frac{\phi}{2} + \vn \cdot \frac{1}{3\sigt} \grad \phi \,.
\]


%%%%%%%----------------------------------------------------------------------------------------------
\subsection{Adjoint VET Formulation}
%%%%%%%----------------------------------------------------------------------------------------------


Since the VET formulation generates a new forward equation to describe the system, a new adjoint corresponding to Eq.~\eqref{VEFForm} must also be formulated. The typical adjoint process is followed in which the forward equation is multiplied by a test function and all operators are transferred to the test function using integration by parts. The following notation is used: $(\grad \grad u)_{ij} = \partial_{x_i} \partial_{x_j} u$
and a tensor dot product $\mathbb{A} : \mathbb{B} = \sum_i \sum_j A_{ij}B_{ij}$.
Next, we compute $\bra \scalSource , \vefadj \ket$ and use the forward VET balance equation, Eq.~\eqref{VEFForm}:
\begin{equation}
\label{VEFadjFormDeriv}
\begin{split}
\bra \scalSource , \vefadj \ket &= - \bra \div \left( \frac{1}{\sigt}\div \Edd \phi \right), \vefadj \ket +  \bra \siga \phi, \vefadj \ket   \\
&= \bra \frac{1}{\sigt}\div \Edd \phi, \grad \vefadj \ket  +  \bra  \phi, \siga \vefadj \ket - \sbra \vn \cdot \frac{1}{\sigt}\div \Edd \phi, \vefadj \sket   \\
 &=  - \bra \phi, \Edd : \left( \grad \left( \frac{1}{\sigt}\grad \vefadj \right) \right) \ket  +  \bra  \phi, \siga \vefadj \ket \\
 &\quad\quad  - \sbra \vn \cdot  \frac{1}{\sigt}\div \Edd \phi, \vefadj \sket + \sbra \phi, \vn \cdot  \Edd \cdot \frac{1}{\sigt} \grad \vefadj \sket  \,,
\end{split}
\end{equation}
where the new boundary inner product for use with VET is defined as
\begin{equation}
\sbra \phi(\vr) , g(\vr)  \sket = \int_{\partial V} dS \, \phi (\vr) g (\vr)  \,.
\end{equation}
A more detailed derivation of the boundary terms along with the double gradient term is given in appendix \ref{chap:appx2}. Note that we used the fact that the Eddington tensor is symmetric, $\Edd^T=\Edd$.
This leads to the adjoint equation of the VET formulation, given in Eq.~\eqref{adjForm}. Of particular note is that the double divergence term present in the forward equation contributes to a double gradient term in the adjoint equation below. The VET adjoint solution is represented by $\vefadj$ to avoid confusion with $\phi^\dag$, which is the adjoint scalar flux from the transport  method. 

\begin{equation}
\label{adjForm}
- \Edd : \left( \grad \left( \frac{1}{\sigt}\grad \vefadj \right) \right) + \siga \vefadj = \scalResp
\end{equation}
By inspection of the boundary terms in Eq.~\eqref{VEFadjFormDeriv} and for reasons that will become apparent during sensitivity calculations, the boundary condition chosen for the adjoint equation is given in Eq.~\eqref{adjVETBC}.
\begin{equation}
\label{adjVETBC}
2J^{\dag,\text{out}} = B \vefadj+ \vn \cdot
\Edd \cdot \frac{1}{\sigma_{t} } \vec{\nabla} \vefadj    \quad \vr \in \bound
\end{equation}
Note that the Robin boundary conditions are recovered for the adjoint diffusion problem, when $\Edd = \tfrac{1}{3} \mathbb{I}$ and $B=\tfrac{1}{2}$:
\[
2 J^{\dag,\text{out}} = \frac{\vefadj}{2} +\vn \cdot \frac{1}{3\sigt} \grad \vefadj \,.
\]
In contrast to the adjoint transport formulation, the VET adjoint equation does not take the form of the  
forward VET equation, therefore the forward VET solver 
cannot necessarily be re-used to solve the adjoint equation. 
To obtain the \qoi using this formulation, we substitute the adjoint equation definition 
into Eq.~\eqref{VEFadjFormDeriv} and use the forward and adjoint boundary conditions.
\begin{equation}
\label{VETQoIAdjUnpDeriv}
\begin{split}
 \bra \scalSource , \vefadj \ket 
&=   \bra \phi, \scalResp \ket - \sbra \vn \cdot \frac{1}{\sigt}\div \Edd \phi, \vefadj \sket + \sbra \phi, \vn \cdot \Edd \cdot \frac{1}{\sigt} \grad \vefadj \sket \\
&=   \qoi - \sbra 2J^{\text{inc}} - B \phi, \vefadj \sket + \sbra \phi, 2J^{\dag,\text{out}} - B \vefadj \sket \,.
\end{split}
\end{equation}
The $B$ terms negate and yield a relatively compact form for the \qoi
\begin{equation}
\label{VETQoIAdj}
\qoi=\bra \scalSource , \vefadj \ket 
- \sbra \phi, 2J^{\dag,\text{out}} \sket  + \sbra \varphi^\dag, 2J^{\text{inc}} \sket
\end{equation}
In order to have a \qoi that can be expressed entirely in terms of the adjoint solution, it is advantageous to 
set $J^{\dag,\text{out}}=0$ in the adjoint boundary condition, yielding the final expression
\begin{equation}
\qoi=\bra \scalSource , \vefadj \ket  + \sbra \vefadj, 2J^{\text{inc}} \sket \,.
\end{equation}

%%%%%%%----------------------------------------------------------------------------------------------
\subsection{VET Adjoint Sensitivity}
%%%%%%%----------------------------------------------------------------------------------------------

As was done for the transport formulation, we consider perturbations to the system parameters. However, in contrast to the transport case, the assumption is also made that the Eddington factor remains unperturbed under these system perturbations. This is an approximation because, in the general case, changing the system's parameters should change the angular flux solution, and hence
the Eddington tensor may be altered as well. Because the Eddington tensor is an integrated quantity, we conjecture that it is 
less prone to perturbations. The effects of perturbations on $\Edd$ are considered later.

A total mean free path notation $\isigt=\frac{1}{\sigt}$ is used for the total cross section to allow the $\delta$ notation to be used in a straightforward fashion. The relation between $\delta \isigt$ and $\delta \sigt$ is shown. 
\begin{equation}
\begin{split}
\delta \isigt  = \ell_{t,p} - \isigt 
 = \frac{1}{\sigt + \delta \sigt} - \frac{1}{\sigt} 
 %= \frac{\sigt - (\sigt + \delta \sigt)}{\sigt (\sigt + \delta \sigt) } 
 = \frac{ - \delta \sigt}{\sigt^2 + \sigt ( \delta \sigt) } 
 \approx  \frac{ - \delta \sigt}{\sigt^2} \left( 1 - \frac{\delta \sigt}{\sigt}\right)
\end{split}
\end{equation}
The perturbed VET forward problem is given below.
\begin{subequations}
\begin{equation}
\label{VEFPert}
- \div \left((\isigt + \delta \isigt)\div \Edd \phi_p \right) + (\siga + \delta \siga)\phi_p = \scalSource + \delta \scalSource  \quad \vr \in \domain
\end{equation}
\begin{equation}
 2J_p^\text{inc} = B \phi_p  + \vn \cdot  (\isigt + \delta \isigt) \vec{\nabla} \cdot \left(\Edd \phi_p \right)  \quad \vr \in \bound
\end{equation}
\end{subequations}
The usual adjoint process is performed, starting with the perturbed \qoi definition using the response function (adjoint source) and the perturbed forward solution. Then, the left-hand side of the adjoint equation is inserted and the operators are carried over to the perturbed forward solution (integration by parts):
\begin{equation}
\label{VETSensDeriv}
\begin{split}
\qoi_p = &\bra \phi_p , \scalResp \ket \\
       = &\bra \phi_p , - \Edd : \left( \grad \isigt \grad \varphi^\dag \right) + \siga \vefadj \ket \\
       = & \bra - \div \isigt \div \left( \Edd \phi_p \right) + \siga \phi_p, \vefadj \ket 
 - \sbra \phi_p , \vn \cdot \left( \Edd \cdot \isigt \grad \vefadj\right) \sket  \\ 
&+ \sbra \vefadj , \vn \cdot  \left(  \isigt \grad \Edd \phi_p \right) \sket \\
\end{split}
\end{equation}
A first-order perturbation approximation of Eq.~\eqref{VEFPert}
\footnote{Eq.~\eqref{VEFPert} can be viewed as $A_p \phi_p = q+\delta q = A \phi_p + \delta A \phi + O(\delta^2)$, hence $A \phi_p  \simeq  q+\delta q -\delta A \phi$} is then used to substitute into the sensitivity Eq.~\eqref{VETSensDeriv}, yielding a form independent of the perturbed forward solution in the volumetric inner products:
\begin{equation}
\label{QoIVETAdjNoBC}
\begin{split}
\qoi_p =& \bra \scalSource + \delta \scalSource + \div \delta \isigt \div \left( \Edd \phi_p \right) - \delta \siga \phi_p, \vefadj \ket - \sbra \phi_p, \vn \cdot \left(  \Edd \cdot \isigt \grad \vefadj \right) \sket \\ 
&+ \sbra \vefadj, \vn \cdot \left(  \isigt \div \Edd \phi_p \right) \sket \\
\approx & \bra q, \vefadj \ket  + \bra \delta \scalSource + \div \delta \isigt \div \left( \Edd \phi \right)  - \delta \siga \phi, \vefadj \ket \\
& - \sbra \phi_p, \vn \cdot \left(  \Edd \cdot \isigt \grad \vefadj \right) \sket + \sbra \vefadj, \vn \cdot \left(  \isigt \div \Edd \phi_p \right) \sket \\
\end{split}
\end{equation}

The first surface term in Eq.~\eqref{QoIVETAdjNoBC} 
can be dealt with readily using the adjoint boundary condition ($\phi_p$ will be dealt with shortly). For the second surface term, we use a first-order approximation of the perturbed forward boundary condition  
\[
2 J_p^{\text{inc}} = B \phi_p + \sigma_{t,p}^{-1}\div \Edd \phi_p
= B \phi_p + (\isigt + \delta \isigt) \div \Edd \phi_p
\approx B \phi_p + \isigt  \div \Edd \phi_p + \delta \isigt  \div \Edd \phi
\]
Reporting back in the surface terms, we obtain:
\begin{equation}
\begin{split}
 - \sbra \phi_p, \vn \cdot \left( \Edd \cdot \isigt \grad \vefadj \right) \sket  &+ \sbra \vefadj, \vn \cdot \left(  \isigt \div \Edd \phi_p \right) \sket \\
&=- \sbra \phi_p, 2J^{\dag,\text{out}} - B \vefadj \sket 
+ \sbra \vefadj, 2 J_p^{\text{inc}} - B \phi_p - \vn \cdot \left( \delta \isigt \div \Edd \phi_p \right) \sket \\
&\approx- \sbra \phi_p, 2J^{\dag,\text{out}} \sket + \sbra \vefadj, 2 J_p^{\text{inc}} - \vn \cdot \left( \delta \isigt \div \Edd \phi \right) \sket  
\end{split}
\end{equation}
Note that the first-otder approximation was used in the first step. However, the last step is a true equality and we have exact cancelation of $\sbra \phi_p,B \vefadj \sket - \sbra \vefadj , B\phi_p\sket=0$. Finally, a first-order expression for the sensitivity in the \qoi is as follows:
\begin{equation}
\label{QoIVETAdjlong}
\begin{split}
\delta \qoi =& \qoi_p - \qoi \\ 
=& \bra \delta \scalSource , \vefadj \ket - \bra \div \delta \isigt \div \left( \Edd \phi \right)  + \delta \siga \phi, \vefadj \ket 
- \sbra \delta \phi, 2J^{\dag,\text{out}} \sket + \sbra \vefadj, 2 \delta J^{\text{inc}} \sket \\
& - \sbra \vefadj, \vn \cdot \left( \delta \isigt \div \Edd \phi \right) \sket \,.
\end{split}
\end{equation}
Again, it is advantageous to use $J^{\dag,\text{out}}=0$ in the adjoint boundary condition to remove a dependency on the 
perturbed forward VET solution. Integration by parts can be applied to condense the $\delta \isigt$ terms. The result is the the final form for computing the \qoi response using the adjoint VET flux $\vefadj$. The accuracy of this method is one of the primary foci of this work.
\begin{equation}
\label{VETsens}
\begin{split}
\delta \qoi =&  \bra \delta \scalSource - \delta \siga \phi, \vefadj \ket  - \bra \delta \isigt \div \left( \Edd \phi \right) , \grad \vefadj \ket
 + \sbra \vefadj, 2 \delta J^{\text{inc}} \sket \,.
\end{split}
\end{equation}



\section{Refinements on the VET Approach for Sensitivity Estimation}

In this section, attempts are made to reduce the error between the VET adjoint method and the transport adjoint method. In general, these come at the cost of extra solves, but still do not require storing angular fluxes. At the heart of it, the main source of difference between transport and VET is the assumption that the Eddington remained unperturbed in the VET method, so we begin with examining Eddington perturbation ($\delta \Edd$) terms. 
%%%%%%%----------------------------------------------------------------------------------------------
\subsection{Error From Unperturbed Eddington Assumption}
%%%%%%%----------------------------------------------------------------------------------------------

To observe the terms that were dropped in the unperturbed Eddington approximation, consider a reformulation of the perturbed forward equation, this time introducing $\delta \Edd$ and $\delta  B$ terms. 
\begin{subequations}
\begin{equation}
\label{VEFPerEdd}
- \div \left((\isigt + \delta \isigt)\div (\Edd + \delta \Edd) \phi_p \right) + (\siga + \delta \siga)\phi_p = \scalSource + \delta \scalSource \ , \quad \vr \in V
\end{equation}
\begin{equation}
2J_p^{\text{inc}} =
(\isigt + \delta \isigt) \vec{\nabla} \cdot \left((\Edd + \delta \Edd) \phi_p \right)  + (\BEdd +\delta \BEdd) \phi_p \ ,\quad \vr \in \partial V
\end{equation}
\end{subequations}
The above can be substituted into Eq.~\eqref{VETSensDeriv} to yield an expanded $\qoi$ equation, including the Eddington perturbation terms. 
\begin{equation}
\label{QoIVETAdjNoBCEdd}
\begin{split}
\qoi_p = & \bra \scalSource + \delta \scalSource + \div \delta \isigt \div \left( \Edd_p \phi_p \right) + \div \ell_{t,p} \div \left( \delta \Edd \phi_p \right) - \delta \siga \phi, \vefadj \ket \\
&- \sbra \phi_p, \Edd \cdot \isigt \grad \vefadj \sket 
+ \sbra \vefadj , \isigt \div \Edd \phi_p \sket
\end{split}
\end{equation}
The boundary condition for the perturbed forward solution takes on a slightly more complex form, as the additional $\delta \Edd$ and $\delta \BEdd$ terms come into play, but the derivation of the sensitivity proceeds similarly to the case ignoring Eddington perturbations. Once again a first order approximation is taken, and the 2nd and 3rd order delta terms are ignored.
\begin{equation}
\label{QoIVETAdjEdd}
\begin{split}
\delta \qoi =& \bra \delta \scalSource + \div \delta \isigt \div \left( \Edd \phi \right) + \div \isigt \div \left( \delta \Edd \phi \right) - \delta \siga \phi, \vefadj \ket \\
&- \sbra 2\delta \phi, J^{\dag, \text{out}} \sket  + \sbra 2 \vefadj, \delta J^{\text{inc}} \sket
- \sbra \vefadj, \delta \isigt \div \Edd \phi \sket
\\
&- \sbra  \vefadj ,\isigt \div \delta \Edd \phi \sket
- \sbra \vefadj, \delta \BEdd \phi \sket
\end{split}
\end{equation} 
Comparing the above formulation with the unperturbed Eddington case shows that the terms lost by the unperturbed Eddington assumption are 
\begin{equation}
\label{EddErrLong}
 \bra \div \isigt \div \left( \delta \Edd \phi \right), \vefadj \ket
- \sbra  \vefadj ,\isigt \div \delta \Edd \phi \sket
- \sbra \vefadj, \delta \BEdd \phi \sket \,,
\end{equation} 
more compactly expressed in the form
\begin{equation}
\label{EddErr}
 - \bra  \isigt \div \left( \delta \Edd \phi \right), \grad \vefadj \ket
- \sbra \vefadj, \delta \BEdd \phi \sket \,.
\end{equation} 

%%%%%%%----------------------------------------------------------------------------------------------
\subsection{Blending $\phi^\dag$ and $\varphi^\dag$}
%%%%%%%----------------------------------------------------------------------------------------------
As a brief recap, two adjoint methods of determining sensitivity have been considered. The first is a transport approach with sensitivity approximated by Eq.~\eqref{snSens}, which utilizes the forward transport solutions $\psi$ and $\phi$ in conjunction with the transport adjoint solutions $\psi^\dag$ and $\phi^\dag$. This equates to two angular transport solves as well as storing two angular flux solutions for use. 

The second method is the VET formulation in which sensitivity is approximated by Eq.~\eqref{VETsens}. This method utilizes the forward solution $\phi$ as well as the VET adjoint solution $\vefadj$, in total this requires a forward SN solve to determine $\phi$ and the unperturbed Eddington $\Edd$ followed by a scalar VET solve for $\vefadj$, but importantly no angular flux solution $\psi$ is stored.

The former method holds a clear advantage when dealing with source perturbations, both volumetric $\delta q$ and boundary $\delta \psi^\text{inc}$. Neither of these perturbations required a first order approximation, therefore the adjoint method is exact. Additionally neither requires storing the angular flux; only the scalar flux. The transport derivation gives the exact $\delta \qoi$ value. Given that source perturbations in general do cause perturbations in $\Edd$ and $\BEdd$, the two VET formulations below are not necessarily equal. In addition, there is a $4 \pi$ factor separating the $\delta q$ term between the transport and VET formulation. From here it is clear to see that $\varphi^\dag \neq \phi^\dag$.
For instance, the sensitivity due to external (volumetric and/or boundary) source perturbations are as follows for each method:
\begin{equation}
\label{sourceDeltaQoI}
\delta \qoi = 
\begin{cases}
\bra \angSourced   , \phi^\dag  \ket - \sbraSN \delta \psi^{\text{inc}}, \psi^\dag \sketSN_- , & \text{Transport}  \\
\bra \delta \scalSource , \vefadj \ket + \sbra \vefadj, 2 \delta J^{\text{inc}} \sket, & \text{VET with $\delta \Edd, \delta B=0$} \\
\bra \delta \scalSource , \vefadj \ket + \sbra \vefadj, 2 \delta J^{\text{inc}} \sket  & \text{VET with no $\delta \Edd,\delta B$ assumption} \\
\quad - \bra  \isigt \div \left( \delta \Edd \phi \right), \grad \vefadj \ket - \sbra \vefadj, \delta \BEdd \phi \sket,  \\
\end{cases} 
\end{equation}

From this, a ``blended'' method is proposed which uses the relevant transport inner products for source perturbation, and VET inner products for cross-section perturbation terms. Contrasted to the VET adjoint, this blended method requires an additional transport solve to determine $\phi^\dag$. In a scenario where only sources are perturbed or only cross sections are perturbed, the blended method will match the results of the transport method or VET method. What is not clear is how the blended method fares when both sources an cross sections are perturbed.
\begin{equation}
\label{Blendsens}
\begin{split}
\delta \qoi =&  \bra \angSourced , \phi^\dag \ket - \bra \delta \siga \phi, \vefadj \ket - \bra \delta \isigt \div \left( \Edd \phi \right) , \grad \vefadj \ket
- \sbraSN \delta \psi^\text{inc}, \psi^\dag \sketSN \\
\end{split}
\end{equation}


%%%%%%%----------------------------------------------------------------------------------------------
\section{Estimating $\delta \Edd$}
%%%%%%%----------------------------------------------------------------------------------------------
Both the transport and VET adjoint methods require a first-order perturbation assumption to be made, but the VET method also introduces the additional assumption that the Eddington remains unperturbed. As such, it seems reasonable that methods to approximate the Eddington perturbation $\delta \Edd$ may allow for reduced error in Eddington derived sensitivity, ideally approaching that of transport adjoint. 

A linear approximation scheme is considered in this work. Let $\vp$ be a point the  parameter space, in this case $\lbrace \sigma_s , \sigma_t, q, \psi^{\text{inc}} \rbrace$ and $\delta p$ be the perturbation in that space.

\begin{equation}
\delta \Edd \approx \frac{\partial \Edd}{\partial \vp} \cdot \delta \vp
\end{equation}
Since no algebraic analytical form is known for $\Edd ( \vp )$, the derivative must be approximated using an additional value for $\Edd$ from an additional forward transport solve.
\begin{equation}
\label{Eddapprox}
\frac{\partial \Edd}{\partial \vp} \approx \frac{\Edd(\vp_1) - \Edd(\vp_0)}{\vp_1 - \vp_0}
\end{equation}
The need for an additional transport solve to find $\Edd(\vp_1)$ is the primary cost of performing this method, and a constraint in situations where this approximation is viable. An analogous process can be used to approximate $\delta \BEdd$. With approximations for $\delta \Edd$ and $\delta \BEdd$ in hand, Eq.~\eqref{EddErr} can be used to to refine the VET sensitivity calculation. For situations where many perturbation scenarios need to be considered, this approximation method could prove useful. 


%%%%%%%%%%%%%%%%%%%%%%%%%%%%%%%%%%%%%%%%%%%%%%%%%%%%%%%%%%%%%%%%%%%%%%%%%%%%%%%%%%%%%%%%%%%%%%%%%%%%
%%%%%%%%%%%%%%%%%%%%%%%%%%%%%%%%%%%%%%%%%%%%%%%%%%%%%%%%%%%%%%%%%%%%%%%%%%%%%%%%%%%%%%%%%%%%%%%%%%%%
\chapter{\uppercase {ALternate Variable Eddington Tensor Formulation Using Adjoint Transport Eddington}} \label{chap:aVET}
%%%%%%%%%%%%%%%%%%%%%%%%%%%%%%%%%%%%%%%%%%%%%%%%%%%%%%%%%%%%%%%%%%%%%%%%%%%%%%%%%%%%%%%%%%%%%%%%%%%%
%%%%%%%%%%%%%%%%%%%%%%%%%%%%%%%%%%%%%%%%%%%%%%%%%%%%%%%%%%%%%%%%%%%%%%%%%%%%%%%%%%%%%%%%%%%%%%%%%%%%
The process by which the VET adjoint equation was formulated involved first converting the angular dependent transport equation to the VET form, then taking the adjoint of the resulting VET quasi-diffusion equation. It is worth considering if performing the operations in a switched order would yield the same result, which is to say, first derive the adjoint  angular flux and then apply the VET treatment. This method of transport adjoint derived VET will be termed aVET.

%%%%%%%----------------------------------------------------------------------------------------------
%%%%%%%----------------------------------------------------------------------------------------------
\section{The aVET Formulation}
%%%%%%%----------------------------------------------------------------------------------------------
%%%%%%%----------------------------------------------------------------------------------------------
The aVET formulation of the adjoint flux is analogous to the forward formulation shown above starting at Eq.~\eqref{VETFormStart}, but starts with the adjoint transport formulation. 

\begin{equation}
\label{snAdjAlt}
- \vO \cdot \grad \psi^\dag + \sigt \psi^\dag = \frac{\sigs}{4 \pi} \phi^\dag + \scalResp
\end{equation}
%
\begin{equation}
\psi^{\dag, \text{out}}(\vr)=0 \quad \vr \in \partial V^{+} = \{  \vr \in \bound , \quad \vO \cdot \vec{n} > 0 \}
\end{equation}
The zero-th and first angular moments of the adjoint expression are then taken and combined into a single system, as was done with the forward-derived VET method.
\begin{subequations}
\begin{equation}
\label{0amAlt}
\div \vec{J}^\dag + (\sigt-\sigs) \phi^\dag  = 4\pi \scalResp \,,
\end{equation}
\begin{equation}
\label{1amAlt}
\div \left(  \int d\Omega \vO \vO \psi^\dag  \right) + \sigt \vec{J}^\dag  = 0 \,.
\end{equation}
\end{subequations}
%
Notable is that a factor of $4 \pi$ exists in the zero-th order equation due to the integration of the scalar response. This carries through to the corresponding VET formulation.
\begin{subequations}
\begin{equation}
\label{TranAdjVEFForm}
- \div \left( \frac{1}{\sigt}\div \Edd^\dag \phi^\dag  \right) + \siga \phi^\dag  = 4\pi \scalResp  \,.
\end{equation}
\begin{equation}
2 J^{^\dag,\text{out}}(\vr) = \BEdd^\dag(\vr) \phi^\dag(\vr)  + \vn \cdot \frac{1}{\sigt} \div \Edd^\dag  \phi^\dag  \,.
\end{equation}
\end{subequations}
In the above, an ``Adjoint Eddington Tensor'' and an ``Adjoint Boundary Eddington Factor'' terms are required, and defined as
\begin{equation}
\label{AdjEddDef}
\Edd^\dag(\vr)=\frac{\int d\Omega \vO \vO \psi^\dag(\vr,\vO)}{\phi^\dag(\vr)} \ , \quad \vr \in V
\end{equation} 
\begin{equation}
\BEdd^\dag(\vr) = \frac{\int_{4 \pi} d\Omega \, | \vO \cdot \vn | \psi^\dag (\vr,\vO)}{\phi^\dag(\vr)} \ ,\quad  \vr \in \bound \,.
\end{equation}
Since the above was directly derived from the transport adjoint equation, the \qoi definition still holds.
\begin{equation}
\label{AdjQoIAlt}
QoI = \braSN \psi , \scalResp\ketSN = \bra \phi^\dag , \angSource \ket - \sbraSN \psi^\dag,  \psi \sketSN 
\end{equation}
The above formulation also has a corresponding forward equation, with solution denoted by $\varphi$. Where as the forward-derived VET method retained the transport forward solution $\phi$ but generated an alternative adjoint $\varphi^\dag$, the aVET method retains the transport adjoint solution $\phi$ and generates an alternate forward solution $\varphi$. The result is an alternate forward system that is analogous to the alternate adjoint system for $\varphi^\dag$ from earlier.
\begin{subequations}
\begin{equation}
\label{ForwardVEFAlt}
- \Edd^\dag : \grad \left( \frac{1}{\sigt}\grad \varphi \right) + \siga \varphi  = \angSource  \,
\end{equation}
\begin{equation}
2 J^{\text{inc}}= \BEdd^\dag \varphi + \Edd^\dag \cdot \frac{1}{\sigt} \grad \varphi \quad \vr \in \bound \,.
\end{equation} 
\end{subequations}
A new expression for the \qoi can be found by using Eq.~\eqref{AdjQoIAlt} and Eq.~\eqref{ForwardVEFAlt}, this time expressed using the $\varphi$.
 \begin{equation}
\label{AdjQoIAltExpand}
\begin{split}
QoI = \bra \phi , \scalResp \ket &= \bra \phi^\dag , \angSource \ket - \sbraSN \psi^\dag,  \psi \sketSN \\
&= \bra \phi^\dag , - \Edd^\dag : \grad \left( \frac{1}{\sigt}\grad \varphi \right) + \siga \varphi \ket - \sbraSN \psi^\dag,  \psi \sketSN \\
&= \bra 4\pi \scalResp  ,\varphi \ket - \sbraSN \psi^\dag,  \psi \sketSN  
- \sbra \Edd^\dag \cdot \frac{1}{\sigt}\grad \varphi,  \phi^\dag \sket 
+ \sbra \frac{1}{\sigt} \div \Edd^\dag \phi^\dag,  \varphi \sket \\
&=  \bra 4\pi \scalResp  ,\varphi \ket - \sbraSN \psi^\dag,  \psi \sketSN - \sbra \phi^\dag, 2J^{\text{inc}} \sket + \sbra \varphi , 2 J^{\dag,\text{out}} \sket
\end{split}
\end{equation}

%%%%%%%----------------------------------------------------------------------------------------------
\section{First-order Sensitivity Estimation Using the aVET Form}
%%%%%%%----------------------------------------------------------------------------------------------
For sensitivity, a perturbed adjoint $\phi^\dag_p$ is now considered. This is expanded to the first order using the $\delta$ notation, and the typical adjoint method is applied. A perturbation to the response is also introduced. For the previous methods, this could be dealt with trivially by the product $\delta \qoi = \bra \phi , \delta q^\dag \ket$, however, since the value of $\phi$ isn't necessarily available in the aVET method, this type of perturbation requires additional attention. In addition $\delta q$ and $\delta \psi^{\text{inc}}$ perturbations may have occurred, but are not reflected in the perturbed adjoint equation below. As was done with the forward derived VET the assumption $\delta \Edd^\dag$ is made.
\begin{subequations}
\begin{equation}
- \div \left( (\isigt + \delta \isigt )\div \Edd^\dag \phi^\dag_p  \right)  +  (\siga + \delta \siga) \phi^\dag_p  =  4 \pi \scalResp + 4 \pi\delta \scalResp
\end{equation}
\begin{equation}
2 J^{^\dag,\text{out}} = \BEdd^\dag \phi^\dag_p  + \vn \cdot (\isigt + \delta \isigt) \div \Edd^\dag  \phi^\dag_p \quad \vr \in \bound \, .
\end{equation}
\end{subequations}
An inner-product with $\varphi$ is taken and the perturbation terms are collected on the RHS leaving the original operator on the LHS.
\begin{equation}
\begin{split}
- \bra\div \left( \isigt \div \Edd^\dag \phi^\dag_p  \right), \varphi \ket + \bra \siga  \phi^\dag_p , \varphi \ket  =& \bra 4 \pi \scalResp + 4 \pi\delta \scalResp , \varphi \ket + \bra\div \left( \delta \isigt \div \Edd^\dag \phi^\dag  \right), \varphi \ket \\ 
&- \bra \delta \siga \phi^\dag , \varphi \ket \\
\end{split}
\end{equation}
The adjoint process is then applied to the RHS, resulting in boundary terms
\begin{equation}
\label{AltVetPertDeriv}
\begin{split}
 \bra  \phi^\dag_p , -\Edd^\dag : \grad \left( \isigt \grad \varphi \right) + \siga  \varphi \ket  =& \bra 4 \pi \scalResp + 4 \pi\delta \scalResp , \varphi \ket + \bra\div \left( \delta \isigt \div \Edd^\dag \phi^\dag  \right), \varphi \ket \\ 
&- \bra \delta \siga \phi^\dag , \varphi \ket - \sbra \phi^\dag_p, \Edd \cdot \isigt \grad \varphi \sket + \sbra \varphi, \isigt \div \Edd \phi^\dag_p \sket  \\
\end{split}
\end{equation}
Take a look at the boundary conditions. The assumption is made that $2 J^{\dag,\text{out}}$ remains unperturbed.
\begin{equation}
\begin{split}
 - \sbra \phi^\dag_p, \Edd \cdot \isigt \grad \varphi \sket  + \sbra \varphi, \isigt \div \Edd \phi_p^\dag \sket 
=&- \sbra \phi_p^\dag, 2J^{\text{inc}} - B \varphi \sket \\ 
&+ \sbra \varphi, 2 J^{\dag,\text{out}} - B \phi_p^\dag - \delta \isigt \div \Edd \phi_p^\dag \sket \\
\approx&- \sbra \phi_p^\dag, 2J^{\text{inc}} \sket + \sbra \varphi , 2 J^{\dag,\text{out}} - \delta \isigt \div \Edd \phi^\dag \sket 
\end{split}
\end{equation}
Taking boundary conditions into account, Eq.~\eqref{AltVetPertDeriv} becomes
\begin{equation}
\label{AltVetPertDeriv2}
\begin{split}
 \bra  \phi^\dag_p , \angSource \ket  =& \bra 4 \pi \scalResp + 4 \pi\delta \scalResp , \varphi \ket + \bra\div \left( \delta \isigt \div \Edd^\dag \phi^\dag  \right), \varphi \ket 
- \bra \delta \siga \phi^\dag , \varphi \ket \\
&- \sbra \phi_p^\dag, 2J^{\text{inc}} \sket + \sbra \varphi , 2 J^{\dag,\text{out}} - \delta \isigt \div \Edd \phi^\dag \sket  \quad . \\
\end{split}
\end{equation} 
A important relation to keep in mind that the perturbed \qoi can be expressed using the perturbed adjoint. A first order approximation must be made, namely $\bra \phi^\dag_p ,  \angSourced \ket \approx \bra \phi^\dag , \angSourced \ket$, however, for $\delta q$ and $\delta \psi^{\text{inc}}$ perturbations the adjoint system remains unperturbed so $\phi=\phi^\dag$.
\begin{equation}
\begin{split}
\qoi_p  = \bra \psi_p , \angSource_p^\dag \ket &= \bra \phi^\dag_p , \angSourcep \ket - \sbraSN \psi^\dag,  \psi_p \sketSN \\
&\approx \bra \phi^\dag_p , \angSource \ket + \bra \phi^\dag , \angSourced \ket - \sbraSN \psi^\dag,  \psi_p \sketSN \\
\end{split} 
\end{equation}
The above relation can be combined with the unperturbed \qoi form derived in Eq.~\eqref{AdjQoIAltExpand} and the expression in  Eq.~\eqref{AltVetPertDeriv2} to give a sensitivity product. 
\begin{equation}
\label{AltVETsens}
\delta \qoi = \bra \delta q^\dag, \varphi \ket - \bra\left( \delta \isigt \div \Edd^\dag \phi^\dag  \right), \grad \varphi \ket \
- \bra \delta \siga \phi^\dag , \varphi \ket + \bra \delta q , \phi \ket - \sbraSN \delta \psi^{\text{inc}}, \psi^\dag \sketSN_-
\end{equation}

\iwh{There is actually something really strange going on with the aVET ``forward''  $\varphi$ equation boundary condition. The most natural way is to base it off of of the true forward incident $J$, but unless I am missing something there is actually some form of freedom there. The ramifications for this are most present in Test case 3 below, where the Alternate VET method is particularly bad at dealing with incident flux scenarios when cross sections are perturbed. Its maybe worth discussing in person at some point}.

%%%%%%%%%%%%%%%%%%%%%%%%%%%%%%%%%%%%%%%%%%%%%%%%%%%%%%%%%%%%%%%%%%%%%%%%%%%%%%%%%%%%%%%%%%%%%%%%%%%%
%%%%%%%%%%%%%%%%%%%%%%%%%%%%%%%%%%%%%%%%%%%%%%%%%%%%%%%%%%%%%%%%%%%%%%%%%%%%%%%%%%%%%%%%%%%%%%%%%%%%
\chapter{\uppercase {Results}} \label{chap:results}
%%%%%%%%%%%%%%%%%%%%%%%%%%%%%%%%%%%%%%%%%%%%%%%%%%%%%%%%%%%%%%%%%%%%%%%%%%%%%%%%%%%%%%%%%%%%%%%%%%%%
%%%%%%%%%%%%%%%%%%%%%%%%%%%%%%%%%%%%%%%%%%%%%%%%%%%%%%%%%%%%%%%%%%%%%%%%%%%%%%%%%%%%%%%%%%%%%%%%%%%%

%%%%%%%%%%%%%%%%%%%%%%%%%%%%%%%%%%%%%%%%%%%%%%%%%%%%%%%%%%%%%%%%%%%%%%%%%%%%%%%%%%%%%%%%%%%%%%%%%%%%
\section{Transport Solution Method}
%%%%%%%%%%%%%%%%%%%%%%%%%%%%%%%%%%%%%%%%%%%%%%%%%%%%%%%%%%%%%%%%%%%%%%%%%%%%%%%%%%%%%%%%%%%%%%%%%%%%
A discrete ordinates (SN) method can be used to solve the one-group steady-state transport 
equation (Eq.~\eqref{SS1GTE}). In the SN method, a discrete angular quadrature is used to represent
the angular variable and carry out the angular integration.
Using an angular quadrature with $D$ directions $\vO_d$, the transport 
equation is solved along each direction:
\begin{equation}
\label{1gSNTE}
\vO_d \cdot \grad \psi_d + \sigt \psi_d = \frac{\sigs}{4 \pi} \phi + \angSource \quad \vr \in V ,\  \forall d\in [1,D]
\end{equation}
%
The scalar flux can be computed from the angular flux as follows
\[
\phi(\vr) \approx \sum_{d=1}^D w_d \psi_d(\vr) \,,
\] 
where $\psi_d(\vr) = \psi(\vr, \vO_d)$ and $w_d$ is the angular quadrature weight. This leads to a coupled system of $D$ equations of the form shown in Eq.~\eqref{1gTE}, where the system is coupled through the scattering source term. This system of equations, where both 
the angular flux and scalar flux are unknowns, is solved iteratively using source iteration as follows:
\begin{subequations}
\begin{equation}
\label{1gTE}
\vO_d \cdot \grad \psi_d^{(\ell+1)} + \sigt \psi_d^{(\ell+1)} = \frac{\sigs}{4 \pi} \phi^{(\ell)} + \angSource \,,
\end{equation}
\begin{equation}
\phi^{(\ell+1)}(\vr) = \sum_{d=1}^D w_d \psi_d^{(\ell+1)}(\vr) \,.
\end{equation}
\end{subequations}
Iteration terminates once $\left|\phi^{(\ell+1)} - \phi^{(\ell)} \right| \leq \text{Tol}$, which was set to $10^{-8}$ for this work.

The test cases in this writing use a slab geometry however, so with that in mind the above can be slightly simplified. Namely, the angular flux no longer needs to be function of the full $\vO$ angular range, but can be represented using a cosine factor $\mu \in [-1,1]$. The main mathematical result of this is a factor of $2 \pi$ resulting from the relation 
\begin{equation}
\psi(\vr,\vO)=\frac{1}{2 \pi} \psi(\vr,\mu) \, .
\end{equation} 
The SN method then discretizes this $\mu$ range into ordinates, and the system takes the form
\begin{subequations}
\begin{equation}
\label{1gTEmu}
\vO_d \cdot \grad \psi_d^{(\ell+1)} + \sigt \psi_d^{(\ell+1)} = \frac{\sigs}{2} \phi^{(\ell)} + \frac{q}{2}\,,
\end{equation}
\begin{equation}
\phi(\vr)^{(\ell+1)} = \sum_{d=1}^D w_d \psi_d^{(\ell+1)}(\vr) \,.
\end{equation}
\end{subequations}
Within the VET formulations, this transform of $4 \pi \to 2$ holds, as the source terms $q$ and $q^\dag$ propagate this $2 \pi$ factor through the VET formulation.

With the angular dependence handeled by the SN method for transport, each of the $D$ spatially dependent SN equations in Eq.~\eqref{1gTE} is then solved using a discontinuous finite element method with up-winding \cite{ReedHill}. For this work and S$_8$ angular quadrature used is used and the spatial domain is discretized into 2,000 uniformly-spaced elements.



%%%%%%%%%%%%%%%%%%%%%%%%%%%%%%%%%%%%%%%%%%%%%%%%%%%%%%%%%%%%%%%%%%%%%%%%%%%%%%%%%%%%%%%%%%%%%%%%%%%%
\section{VET Solution Method}
%%%%%%%%%%%%%%%%%%%%%%%%%%%%%%%%%%%%%%%%%%%%%%%%%%%%%%%%%%%%%%%%%%%%%%%%%%%%%%%%%%%%%%%%%%%%%%%%%%%%
Solution of the quasi-diffusion VET formulations are performed using a Discontinuous Galerkin (DG) method with an interior-penalty approach; see Arnold \cite{Arnold}, for instance. Compared to a standard diffusion approach, slight modifications to the interior penalty terms are required to support quasi-diffusion. For standard diffusion the current terms used at the mesh interface would take the form $\frac{1}{3 \sigt} \grad \phi$ but for the Eddington approach the current terms take the form $\frac{1}{\sigt} \div ( \Edd \phi)$. Linear basis functions are used as the finite element basis. As was the case for SN transport, the spatial domain is discretized into 2,000 uniform elements for the results shown in the following sections.
\jcr{``The'' modifications are required. ? This makes people curious. What are they?}
\iwh{sufficient, or do I need to get more rigorous with this? I may have a question or two about using the this method on this form when we can get to a whiteboard and have a free moment. More about theory than anything}
%%%%%%%%%%%%%%%%%%%%%%%%%%%%%%%%%%%%%%%%%%%%%%%%%%%%%%%%%%%%%%%%%%%%%%%%%%%%%%%%%%%%%%%%%%%%%%%%%%%%
\section{Results}
%%%%%%%%%%%%%%%%%%%%%%%%%%%%%%%%%%%%%%%%%%%%%%%%%%%%%%%%%%%%%%%%%%%%%%%%%%%%%%%%%%%%%%%%%%%%%%%%%%%%

We have presented 5 separate methods for unperturbed \qoi calculation have been discussed: 
\begin{enumerate}
\item forward and adjoint inner products for both SN and VET formulations (yielding a total of 4 different approaches), and,
\item the alternate aVET method in Eq.~\eqref{AdjQoIAltExpand}.
\end{enumerate} 
as well as 7 separate methods for $\delta$\qoi calculation: 
\begin{enumerate}
\item forward and adjoint inner products for both SN and VET formulations (4 approaches),
\item the ``blended'' method presented in Eq.~\eqref{Blendsens},
\item the method of approximating $\delta \Edd$ shown in Eq.~\eqref{Eddapprox}, and,
\item the aVET inner product in  Eq.~\eqref{AltVETsens}.
\end{enumerate} 

When dealing with sensitivity calculations, techniques based on the forward solutions are expected to give the most exact answer simply because it involves an additional forward solve of the perturbed system for each perturbation case. The adjoint methods for sensitivity are relying on a first-order approximation (we dropped the double $\delta$ terms), however they only require a single forward solve and a single adjoint for use with all perturbation cases. 

The seven methods were implemented in a MATLAB finite element method (FEM) solver. One-dimensional test cases were run, varying which parameter experienced a perturbation and the magnitude of that perturbation. The results of the five different sensitivity methods were analyzed to identify cases where the efficient VET adjoint showed promise as a time and memory efficient method for computing sensitivity. As a representation of this, the \% \qoi response is plotted against the \% change in a given parameter $p$, which are defined as
\begin{subequations}
\begin{equation}
\text{\qoi \% response}=\frac{\delta \qoi}{\qoi} \,,
\end{equation}
\begin{equation}
\text{$p$ \% change}=\frac{\delta p}{p} \,.
\end{equation}    
\end{subequations}
Any references to an ``exact'' solution for a method means that it agrees with the $\delta \qoi$ value found by subtracting two forward (unperturbed and perturbed) SN transport answers.

%%%%%%%----------------------------------------------------------------------------------------------
\subsection{Homogeneous System, Homogeneous Perturbation}
To start, a test case consisting of a relatively simple system is chosen. The system is a homogeneous material with a volumetric source throughout and no incident flux. The response function is the center of the region sufficiently far from the boundaries. Towards the center of this region the infinite medium solution is approached, and $\phi \approx \phi^\infty = \frac{q}{\siga}$. Note that this system should not show any significant response to perturbations in the scattering cross section. The following values are chosen for the entire region $x \in [0,10]$: $q=2$, $\siga=1$, and $\sigs=1$. For the response function $q^\dag=1$ for $x\in[4,6]$ and $0$ elsewhere. Therefore the predicted \qoi for this, using the infinite medium approximation is
\begin{equation}
\qoi \approx \bra q^\dag , \phi \ket = \int_4^6 dx \, \frac{q}{\siga} = 4
\end{equation}
The unperturbed scalar fluxes are shown in Figure~\ref{fig:Flux1}. These plots show both the true scalar flux $\phi$ obtained using the SN and VET methods (which should be equal), as well as the ``forward-like'' flux $\varphi$ obtained as the adjoint of the aVET method. Similarly, the scalar adjoint flux $\phi^\dag$ obtained using SN and aVET are shown on the same plot as the ``adjoint-like'' $\varphi^\dag$ from the VET formulation. Looking at the graphs it becomes clear that $\phi^\dag \neq \varphi^\dag$ and $\phi \neq \varphi$, providing a succinct confirmation that the solution of the adjoint of the VET formulation is not the same as the solution to the VET formulation of the transport adjoint. At a glance it may appear that the difference between the scalar flux solutions $\phi$, $\phi^\dag$ and the flux-like $\varphi$, $\varphi^\dag$ is a simple factor of 2 due to source normalization, but that is not the case and plots which take the factor of 2 into account are presented in Appendix~\ref{chap:appx3}
\jcr{what point do you want to make. They only looked different by a factor 2, which one may argue may be changed appropriate by changing the source normalization. So what are you trying to show here?}
\iwh{I tried to clarify this, and added a few additional images to an appendix to prevent overcrowding the main images. I think the point is I just wanted to present a visual confirmation that the adjont of the VET isn't the VET of the adjoint, which is something that wasn't explicitly obvious when I first started working on this. That the solutions we are getting from taking adjoints of VET/aVET are not true "scalar flux" in the transport sense. If this seems extraneous I am willing to drop it and the additional images in the appendix.}

For perturbations, the entire system is perturbed uniformly, resulting in another homogeneous system, so therefore the perturbed \qoi can be easily predicted as $\qoi_p \approx (6-4) \phi^\infty_p = 2 \frac{q_p}{\sigma_{a,p}}$. Given the simple form of the expected flux in the center, the derivatives can be taken to get an expected sensitivity.
\begin{subequations}
\begin{equation}
 \frac{\partial \phi}{\partial q} = \frac{1}{\siga}   
\end{equation}
\begin{equation}
 \frac{\partial \phi}{\partial \siga} = - \frac{q}{\siga^2}   
\end{equation}
\begin{equation}
 \frac{\partial \phi}{\partial \sigs} = 0  
\end{equation}
\end{subequations}

The seven methods of $\qoi$ perturbation calculation were applied to this system for perturbations in $q$, $\sigs$, and $\sigt$. Perturbation in the system parameters were taken in the range of $\pm 10 \%$. The \% response results over this range are plotted in Figure~\ref{fig:Trial1}. $\delta \qoi$ values for selected perturbations are given in Table~\ref{TableT1}.

\begin{figure}[H]
\centering
\begin{subfigure}{.5\textwidth}
  \centering
  \includegraphics[width=.98\linewidth]{figures2/22phi.png}
\end{subfigure}%
\begin{subfigure}{.5\textwidth}
  \centering
  \includegraphics[width=.98\linewidth]{figures2/22phia.png}
\end{subfigure}
\caption{Plots of unperturbed scalar fluxes for the homogeneous system. This include ``forward'' fluxes $\phi$ and $\varphi$ show on the left, and ``adjoint'' fluxes $\phi^\dag$ and $\varphi^\dag$ on the right.}
\label{fig:Flux1}
\end{figure}

\begin{figure}[H]
\centering
\begin{subfigure}{.5\textwidth}
  \centering
  \includegraphics[width=.98\linewidth]{figures2/22qSens.png}
\end{subfigure}%
\begin{subfigure}{.5\textwidth}
  \centering
  \includegraphics[width=.98\linewidth]{figures2/22sigaSens.png}
\end{subfigure}
%
\begin{subfigure}{.5\textwidth}
  \centering
  \includegraphics[width=.98\linewidth]{figures2/22sigsSens.png}
\end{subfigure}%
\begin{subfigure}{.5\textwidth}
  \centering
  \includegraphics[width=.98\linewidth]{figures2/22qsigaSens.png}
\end{subfigure}
\caption{\qoi response to various perturbation scenarios for the homogeneous system under various homogeneous perturbations. For the unperturbed system $q=2$, $\siga=1$, and $\sigs=1$.}
\label{fig:Trial1}
\end{figure}

The unperturbed \qoi value was the same for all methods, agreeing with the prediction. In both the unperturbed and perturbed state, the Eddington is essentially at the infinite medium limit, so the unperturbed Eddington approximation should be a safe assumption in this scenario. The results of the $q$ and $\siga$ perturbations seem to support this. For the source perturbations, no first-order approximation is needed, so all methods appear to give the same result, which is exact. The $\siga$ perturbations begin to show the effects of the first-order approximation, where both the transport and VET adjoint methods depart from the exact forward found $\delta \qoi$. As expected, the system does not show strong sensitivity to $\sigs$ perturbations.

\begin{table}[H]
\centering
  \begin{tabular}{| l | r || r | r | r | r |}
    \hline
    Method  & $\qoi$ & $+10\% q $  & $-10\% \siga $ & $+10\% \sigs $ & $+10\% q,-10\% \siga$ \\ \hline
     SN Fwd 			&3.99976	&0.39998 &0.44419 &5.7577e-05 & 0.88858\\ \hline
     VET Fwd			&3.99976	&0.39998 &0.44428 &2.7131e-05 &0.88868\\ \hline
     SN Adj			    &3.99976	&0.39998 &0.39983 &6.6307e-05 &0.79980\\ \hline
     VET Adj 			&3.99976	&0.39998 &0.39988 &2.8534e-05 &0.79986\\ \hline
     Blended 			&-			&0.39998 &0.39988 &2.8534e-05 &0.79986\\ \hline
     VET $\delta \Edd$ 	&-			&0.39998 &0.39983 &6.124e-05 &0.79980\\ \hline
     aVET				&3.99976 	&0.39998 &0.39980 &4.986e-05	 &0.79978\\ \hline
    \end{tabular}
  \caption{Table of selected $\delta \qoi$ values for the homogeneous system under homogeneous perturbations. The unperturbed $\qoi$ for various methods is given in the first column.}
  \label{TableT1}
\end{table}


%%%%%%%----------------------------------------------------------------------------------------------
\subsection{Homogeneous System, Inhomogeneous Perturbation}

For the next test case, the same initial homogeneous system is used for the unperturbed state. However, we attempt to introduce Eddington perturbations by perturbing the system to an inhomogeneous state, there by introducing a boundary layer into the system. This is done by only perturbing the $q$, $\siga$, $\sigs$ terms on the region $x \in [0,6]$, so that the region of interest is within the perturbed region, but on one of its boundaries. In this perturbed state, the infinite medium limit can no longer be applied. By introducing a $\delta \Edd$, we hope to begin differentiating the transport and VET adjoint methods, as the $\delta \Edd=0$ was the major assumption that had to be made in VET. Response plots are shown in Figure~\ref{fig:Trial2} and selected $\delta \qoi$ values in Table~\ref{TableT2}. Refer to Figure~\ref{fig:Flux1} for the unperturbed fluxes.

\begin{figure}[H]
\centering
\begin{subfigure}{.5\textwidth}
  \centering
  \includegraphics[width=.98\linewidth]{figures2/23qSens.png}
\end{subfigure}%
\begin{subfigure}{.5\textwidth}
  \centering
  \includegraphics[width=.98\linewidth]{figures2/23sigaSens.png}
\end{subfigure}
%
\begin{subfigure}{.5\textwidth}
  \centering
  \includegraphics[width=.98\linewidth]{figures2/23sigsSens.png}
\end{subfigure}%
\begin{subfigure}{.5\textwidth}
  \centering
  \includegraphics[width=.98\linewidth]{figures2/23qsigaSens.png}
\end{subfigure}
\caption{\qoi response to various perturbation scenarios for the homogeneous system under various inhomogeneous perturbations. For the unperturbed system $q=2$, $\siga=1$, and $\sigs=1$.}
\label{fig:Trial2}
\end{figure}

As expected, the introduction of the boundary layer in the perturbed state begins to show differentiation in the selected methods. For source perturbations the transport and VET adjoint methods match their respective forward found values. The blended method matches the SN values for source perturbations and the VET values for cross-section perturbations as designed. The behavior of the blended method with both source and cross-section perturbations shows that it does provide an improvement over the VET method, however the found value still lies closer to the VET value than the transport adjoint value. The $\delta \Edd$ method shows more promise in this trial, as the addition of the $\delta \Edd$ terms begins to reconcile the VET adjoint method with the more exact transport adjoint. The aVET method shows its exact nature for the source perturbation, as well as providing a $\delta \qoi$ value similar to the $\delta \Edd$ approximation method for $\siga$ perturbations, despite the latter leveraging an additional transport solve. 

\begin{table}[H]
\centering
  \begin{tabular}{| l | r || r | r | r | r |}
    \hline
    Method  & $\qoi$ & $+10\% q $  & $-10\% \siga $ & $+10\% \sigs $ & $+10\% q,-10\% \siga$ \\ \hline
     SN Fwd 			&3.99976	&0.36309 &0.39952 &2.9680e-05 & 0.79915\\ \hline
     VET Fwd			&3.99976	&0.35947 &0.39517 &1.5072e-05 &0.79040\\ \hline
     SN Adj  			&3.99976	&0.36309 &0.36301 &3.4051e-05 &0.72610\\ \hline
     VET Adj 			&3.99976	&0.35947 &0.35941 &1.5733e-05 &0.71888\\ \hline
     Blended 			&-			&0.36309 &0.35941 &1.5733e-05 &0.72250\\ \hline
     VET $\delta \Edd$ 	&-			&0.36295 &0.36298 &3.1479e-05 &0.72603\\ \hline
     aVET				&3.99976	&0.36309 &0.36299 &2.6234e-05 &0.72609\\ \hline
    \end{tabular}
  \caption{Table of selected $\delta \qoi$ values for the homogeneous system under inhomogeneous perturbations. The unperturbed $\qoi$ for various methods is given in the first column.}
  \label{TableT2}
\end{table}

%%%%%%%----------------------------------------------------------------------------------------------
\subsection{Shielded Incident Flux}

Next is a simple shielding case to test how the methods deals with a surface source as opposed to the volumetric source of the previous test cases. An isotropic flux is incident on the left boundary $x=0$ of the system with no incident flux on the right boundary, no volumetric source is present. The incident flux passes though a shield from $x=[1,2]$ with $\siga=0.5$ and $\sigs=0.5$. The response is taken on the right side of the shield using a response $q^\dag=1$ for $x \in [3,4]$ and $0$ else. Response plots shown in Figure~\ref{fig:Trial3} and $\delta \qoi$ values in Table~\ref{TableT3}. Also, this test case introduces streaming regions into the problem. For the VET formulations, there is a factor of $\frac{1}{\sigt}$ that is undefined for streaming. To avoid this, a value of $\sigt=10^{-8}$ was used to signify a streaming region. Cross-section perturbations occur in the shielding material, while incident flux perturbations occur only for the flux incident at $x=0$.

\begin{figure}[H]
\centering
\begin{subfigure}{.5\textwidth}
  \centering
  \includegraphics[width=.98\linewidth]{figures2/24phi.png}
\end{subfigure}%
\begin{subfigure}{.5\textwidth}
  \centering
  \includegraphics[width=.98\linewidth]{figures2/24phia.png}
\end{subfigure}
\caption{Plots of unperturbed scalar fluxes for the incident shielding system. Forward fluxes $\phi$ and $\varphi$ show on the left, and adjoint fluxes $\phi^\dag$ and $\varphi^\dag$ on the right.}
\label{Flux3}
\end{figure}

\begin{figure}[H]
\centering
\begin{subfigure}{.5\textwidth}
  \centering
  \includegraphics[width=.98\linewidth]{figures2/24incSens.png}
\end{subfigure}%
\begin{subfigure}{.5\textwidth}
  \centering
  \includegraphics[width=.98\linewidth]{figures2/24sigaSens.png}
\end{subfigure}
%
\begin{subfigure}{.5\textwidth}
  \centering
  \includegraphics[width=.98\linewidth]{figures2/24sigsSens.png}
\end{subfigure}%
\begin{subfigure}{.5\textwidth}
  \centering
  \includegraphics[width=.98\linewidth]{figures2/24incsigaSens.png}
\end{subfigure}
\caption{\qoi response to various perturbation scenarios for the.}
\label{fig:Trial3}
\end{figure}

Response to perturbation in the incident flux behaves as expected, with transport adjoint, blended, and aVET all retrieving the exact $\delta \qoi$. Cross section perturbations begin show dome behavior that differs from that seen in the previous cases. Most notable is the aVET method, which shows to be the worst approximation method for both $\siga$ and $\sigs$ perturbations, which $\delta \qoi$ values that are approaching twice the value found in other methods, this effect is also seen when both the incident flux and the absorption are perturbed. Response to $\sigs$ perturbations are a bit more pronounced here, but still fairly weak. In a higher dimensional system, scattering could be particularly important for shielding problems, as neutrons could scatter around the shield in some way. However, in the tested geometry this is not possible.

\begin{table}[H]
\centering
  \begin{tabular}{| l | r || r | r | r | r |}
    \hline
    Method  & $\qoi$ & $+10\% \psi^- $  & $-10\% \siga $ & $+10\% \sigs $ & $+10\% \psi^-,-10\% \siga$ \\ \hline
     SN Fwd 			&0.234008 	&0.023401 &0.021079 &-0.0067476 & 0.046588\\ \hline
     VET Fwd			&0.234008 	&0.023181 &0.019670 &-0.0066481 &0.044818\\ \hline
     SN Adj  			&0.231931 	&0.023401 &0.019975 &-0.0068956 &0.043376\\ \hline
     VET Adj 			&0.231698 	&0.023181 &0.018981 &-0.0067751 &0.042162\\ \hline
     Blended 			&-			&0.023401 &0.018981 &-0.0067751 &0.042381\\ \hline
     VET $\delta \Edd$ 	&-			&0.023170 &0.020079 &-0.0065100 &0.043249\\ \hline
     aVET				&0.234008 	&0.023401 &0.035869 &-0.012563	&0.059270\\ \hline
    \end{tabular}
  \caption{Table of selected $\delta \qoi$ values for the shielding system under perturbations. The unperturbed $\qoi$ for various methods is given in the first column.}
  \label{TableT3}
\end{table}

%%%%%%%----------------------------------------------------------------------------------------------
\subsection{Reed Problem}
As a final test, a more varied and complex Reed system\cite{buchan} is tested. The system is split into 5 regions of unequal length with properties given below. As for perturbations, $\siga$ experiences perturbations in regions 1 and 4, $\sigs$ is perturbed in regions 4 and 5, and $q$ is perturbed in regions 1 and 4. The system has no incident flux. Response is taken in the right most region of the system $x \in [6,8]$.
\begin{equation*}
\begin{split}
&\text{Region 1: } x \in [0,2), \quad \siga=50, \, 			\sigs=0, \, q=50, \, q^\dag=0 \\
&\text{Region 2: } x \in [2,3), \quad \siga=5, \, 			\sigs=0, \, q=0, \, q^\dag=0 \\
&\text{Region 3: } x \in [3,5), \quad \siga \approx 0, \,	\sigs=0, \, q=0, \, q^\dag=0 \\
&\text{Region 4: } x \in [5,6), \quad \siga=0.1, \, 		\sigs=0.9, \, q=1, \, q^\dag=0 \\
&\text{Region 5: } x \in [6,8], \quad \siga=0.1, \, 		\sigs=0.9, \, q=0, \, q^\dag=1 \\
\end{split}
\end{equation*}

\begin{figure}[H]
\centering
\begin{subfigure}{.5\textwidth}
  \centering
  \includegraphics[width=.98\linewidth]{figures2/7phi.png}
\end{subfigure}%
\begin{subfigure}{.5\textwidth}
  \centering
  \includegraphics[width=.98\linewidth]{figures2/7phia.png}
\end{subfigure}
\caption{Plots of unperturbed scalar fluxes for the Reed problem. Forward fluxes $\phi$ and $\varphi$ show on the left, and adjoint fluxes $\phi^\dag$ and $\varphi^\dag$ on the right.}
\label{Flux4}
\end{figure}

\begin{figure}[H]
\label{Trial4}
\centering
\begin{subfigure}{.5\textwidth}
  \centering
  \includegraphics[width=.98\linewidth]{figures2/7qSens.png}
\end{subfigure}%
\begin{subfigure}{.5\textwidth}
  \centering
  \includegraphics[width=.98\linewidth]{figures2/7sigaSens.png}
\end{subfigure}
%
\begin{subfigure}{.5\textwidth}
  \centering
  \includegraphics[width=.98\linewidth]{figures2/7sigsSens.png}
\end{subfigure}%
\begin{subfigure}{.5\textwidth}
  \centering
  \includegraphics[width=.98\linewidth]{figures2/7qsigaSens.png}
\end{subfigure}
\caption{\qoi response to various perturbation scenarios for the Reed Problem.}
\end{figure}

The effect of scattering perturbations is more pronounced in the Reed problem. The response these scattering perturbations is also where the greatest differentiation between the transport and VET appears. The $\delta \Edd$ approximation method proves quite valuable for resolving difference the transport and VET methods due to scattering perturbations. For perturbations in the other system parameters, the VET method is within $10\%$ of those found by transport adjoint, and even the standard VET adjoint method is close for source perturbations.

\begin{table}[H]
\centering
  \begin{tabular}{| l | r || r | r | r | r |}
    \hline
    Method  & $\qoi$ & $+10\% q $  & $-10\% \siga $ & $+10\% \sigs $ & $+10\% q,-10\% \siga$ \\ \hline
     SN Fwd 			&1.75802 	&0.17580 &0.030945 &0.030235 & 0.20984\\ \hline
     VET Fwd 			&1.75802 	&0.17561 &0.029105 &0.047568 &0.20763\\ \hline
     SN Adj 			&1.75603  	&0.17580 &0.030401 &0.030450 &0.20620\\ \hline
     VET Adj 			&1.75602  	&0.17561 &0.028580 &0.044778 &0.20419\\ \hline
     Blended 			&-	 		&0.17580 &0.028580 &0.044778 &0.20438\\ \hline
     VET $\delta \Edd$ 	&-		 	&0.17573 &0.030348 &0.027743 &0.20608\\ \hline
     aVET		 		&1.75601 	&0.17580 &0.028632 &0.044497 &0.20443\\ \hline
    \end{tabular}
  \caption{Table of selected $\delta \qoi$ values for the Reed system under perturbations. The unperturbed $\qoi$ for various methods is given in the first column.}
  \label{TableT4}
\end{table}

%%%%%%%%%%%%%%%%%%%%%%%%%%%%%%%%%%%%%%%%%%%%%%%%%%%%%%%%%%%%%%%%%%%%%%%%%%%%%%%%%%%%%%%%%%%%%%%%%%%%
%%%%%%%%%%%%%%%%%%%%%%%%%%%%%%%%%%%%%%%%%%%%%%%%%%%%%%%%%%%%%%%%%%%%%%%%%%%%%%%%%%%%%%%%%%%%%%%%%%%%
\chapter{\uppercase {Conclusions and Outlook}} \label{chap:conclusion}
%%%%%%%%%%%%%%%%%%%%%%%%%%%%%%%%%%%%%%%%%%%%%%%%%%%%%%%%%%%%%%%%%%%%%%%%%%%%%%%%%%%%%%%%%%%%%%%%%%%%
%%%%%%%%%%%%%%%%%%%%%%%%%%%%%%%%%%%%%%%%%%%%%%%%%%%%%%%%%%%%%%%%%%%%%%%%%%%%%%%%%%%%%%%%%%%%%%%%%%%%

For the systems tested, the VET method shows promise for use in sensitivity calculations. For the majority of the trials the error between the VET and SN adjoint sensitivities was $<10\%$ of the SN sensitivity, and $<1\%$ of the unperturbed QoI. The blended method demonstrated an approach to increase the accuracy, particularly for source perturbations, at the cost of an additional SN solve to obtain $\phi$. The $\delta \Edd$ approximation approach was even more accurate in most cases, but requires at least 1 extra SN solve for each perturbed system property, making it viable in scenarios where many perturbation scenarios must be tested. 

Scattering perturbations showed possibly the most interesting behavior. The deviation of the VET and SN methods is stronger (relative to the SN sensitivity) in most of the test case when $\sigs$ is perturbed. Additionally, the shielding system presented a scenario where the $\delta \Edd$ approach appeared to introduce more error. Unfortunately, the testing of heavy scatting systems in one spacial dimension can be a bit limited. In a two or three dimensions, the ability for particles to scatter around objects exists in general, while this is not present in one dimension. Expanding the discussed concepts to higher spatial dimensions would be a worthwhile next step, particularly in observing the effects of scattering on the VET method.

For the test cases with only a volumetric source, the aVET method utilizing $\varphi$ showed obvious advantages by way of its exact nature for source perturbations and while having similar accuracy to VET for cross section perturbations. This method however appeared to have difficulties with the incident flux scenario, returning cross section response values significantly higher than any other method. It was able to compute $\delta \qoi$ exact for the incident flux perturbation in the shielding scenario however.


%%%%%%%%%%%%%%%%%%%%%%%%%%%%%%%%%%%%%%%%%%%%%%%%%%%%%%%%%%%%%%%%%%%%%%%%%%%%%%%%%%%%%%%%%%%%%%%%%%%%
%%%%%%%%%%%%%%%%%%%%%%%%%%%%%%%%%%%%%%%%%%%%%%%%%%%%%%%%%%%%%%%%%%%%%%%%%%%%%%%%%%%%%%%%%%%%%%%%%%%%
%%%%%%%%%%%%%%%%%%%%%%%%%%%%%%%%%%%%%%%%%%%%%%%%%%%%%%%%%%%%%%%%%%%%%%%%%%%%%%%%%%%%%%%%%%%%%%%%%%%%
%%%%%%%%%%%%%%%%%%%%%%%%%%%%%%%%%%%%%%%%%%%%%%%%%%%%%%%%%%%%%%%%%%%%%%%%%%%%%%%%%%%%%%%%%%%%%%%%%%%%

%\bibliography{IanProp} 
%\bibliographystyle{ieeetr}

%fix spacing in bibliography, if any...
%%%%%%%%%%%%%%%%%%%%%%%%%%%%%%%%%%%%%%%%%%%%%%%%%%%%%%%%%%%%%
\let\oldbibitem\bibitem
\renewcommand{\bibitem}{\setlength{\itemsep}{0pt}\oldbibitem}
%%%%%%%%%%%%%%%%%%%%%%%%%%%%%%%%%%%%%%%%%%%%%%%%%%%%%%%%%%%%%%%
%The bibliography style declared is the IEEE format. If
%you require a different style, see the document
%bibstyles.pdf included in this package. This file,
%hosted by the University of Vienna, shows several
%bibliography styles and examples of in-text citation
%and a references page.
\bibliographystyle{ieeetr}

\phantomsection
\addcontentsline{toc}{chapter}{REFERENCES}

\renewcommand{\bibname}{{\normalsize\rm REFERENCES}}

%This file is a .bib database that contains the sources.
%This removes the dependency on the previous file
%bibliography.tex.
\bibliography{QoI_MS}




%This next line includes appendices. The file
%appendix.tex contains commands pointing to
%the appendix files; be sure to change these
%pointers if you end up changing the filenames.
%Leave this commented if you will not need
%appendix material.
%\include{data/appendices}

\newpage
\begin{appendices}
\titleformat{\chapter}{\centering\normalsize}{APPENDIX \thechapter}{0em}{\vskip .5\baselineskip\centering}
\renewcommand{\appendixname}{APPENDIX}

%%%%%%%%%%%%%%%%%%%%%%%%%%%%%%%%%%%%%%%%%%%%%%%%%%%%%%%%%%%%%%%%%%%%%%%%%%%%%%%%%%%%%%%%%%%%%%%%%%%%
\chapter{ \uppercase{APPENDIX: Time Dependent VET} } \label{chap:appx1}
%%%%%%%%%%%%%%%%%%%%%%%%%%%%%%%%%%%%%%%%%%%%%%%%%%%%%%%%%%%%%%%%%%%%%%%%%%%%%%%%%%%%%%%%%%%%%%%%%%%%
While out of scope for this writing, a major goal of this method is the application towards time-dependent systems. As such, it is worth taking a brief look at how the VET formulation would look in a time dependent system. As with the steady state, the starting point is the one-group transport equation, now with the time derivative factor.
\begin{equation}
\label{Trans1GTE}
\frac{1}{v} \frac{\partial}{\partial t} \psi(\vr,\vO,t)+ \vO \cdot \grad \psi(\vr,\vO,t) + \sigt(\vr) \psi(\vr,\vO,t) = \frac{1}{4 \pi} \sigs(\vr) \phi(\vr) + q(\vr,\vO,t), \quad \forall \vr \in V,t \geq 0
\end{equation}
\begin{equation}
\label{Trans1GTE_bc}
\psi(\vr,\vO,t) = \psi^{\text{inc}}(\vr,\vO,t) \quad \vr \in \partial V^{-} = \{ \vr \in \partial V, \text{ s.t. }, \vO \cdot \vec{n}(\vr) < 0\}
\end{equation}
\begin{equation}
\label{Trans1GTE_t0}
\psi(\vr,\vO,0) = \psi_0(\vr,\vO)
\end{equation}
The zero-th and first angular moments are taken of the time dependent system. The Eddington Tensor is used in the 1st order equation.
\begin{subequations}
%
\begin{equation}
\label{0amTrans}
\frac{1}{v} \frac{\partial}{\partial t}\phi + \div \vec{J} + (\siga) \phi = \scalSource \,,
\end{equation}
%
\begin{equation}
\label{1amTrans}
\frac{1}{v} \frac{\partial}{\partial t}\vec{J}  + \div \left( \Edd \phi \right) + \sigt \vec{J} = 0 \,.
\end{equation}
%
\end{subequations}
The moments are then combined in the same fashion used in the steady-state case
\begin{equation}
\label{VETTrans}
\frac{1}{v} \frac{\partial}{\partial t}\phi - \div \left( \frac{1}{v \sigt} \frac{\partial}{\partial t}\vec{J} \right)   - \div \left( \frac{1}{\sigt} \div \left( \Edd \phi \right) \right)  + (\siga) \phi = \scalSource.
\end{equation}
Note that the same double divergence term shows up the the time dependent equation as it does in the steady state formulation. So it is conceivable that insight gained from the steady state treatment will carry over to the time-dependent problem. 
\iwh{Not sure on the exact next step to take. The second term is the ugly one. I can define the residual 
\begin{equation}
f=- \div \left( \frac{1}{\sigt} \div \left( \Edd \phi \right) \right)  + (\siga) \phi - \scalSource
\end{equation}
which is the steady state equation, and muse about how that term must be dealt with accurately to produce time evolution. 
Doing sensitivity we want an operator $A \phi = b$ which the above isn't exactly due to the $\vec{J}$. In addition to $\Edd$ is is reasonable to store some vector value like $\vec{g}=\frac{\vec{J}}{\phi}$? That way there is a well defined $A$ operator to perform the adjoint process on.}

%%%%%%%%%%%%%%%%%%%%%%%%%%%%%%%%%%%%%%%%%%%%%%%%%%%%%%%%%%%%%%%%%%%%%%%%%%%%%%%%%%%%%%%%%%%%%%%%%%%%
\chapter{\uppercase{APPENDIX: Detailed VET Boundary Term Derivation}} \label{chap:appx2}
%%%%%%%%%%%%%%%%%%%%%%%%%%%%%%%%%%%%%%%%%%%%%%%%%%%%%%%%%%%%%%%%%%%%%%%%%%%%%%%%%%%%%%%%%%%%%%%%%%%%
Represent the value $\isigt \div \Edd \phi$ by some vector $\vec{v}$. 
\begin{equation}
\begin{split}
- \bra \div \left( \isigt \div \Edd \phi \right), \vefadj \ket
&= - \int_V dV \, \vefadj  \left( \div \left( \isigt \div \Edd \phi \right) \right) \\
&= - \int_V dV \, \vefadj  \left( \div \vec{v}\right)
\end{split}
\end{equation}
Then use a product rule $\div (\vec{v} \varphi^\dag) = \varphi^\dag (\div \vec{v}) + \vec{v} \cdot (\grad \varphi^\dag)  $. Use the divergence theorem on the $\div (\vec{v} \varphi^\dag)$ term to convert to a surface integral
\begin{equation}
\begin{split}
- \int_V dV \, \vefadj  \left( \div \vec{v}\right) &= - \int_V dV \, \div (\vec{v} \varphi^\dag) + \int_V dV \, \vec{v} \cdot (\grad \varphi^\dag) \\
&= - \oint_{\bound} dS \, (\vec{v} \varphi^\dag) \cdot \vn + \int_V dV \, \vec{v} \cdot (\grad \varphi^\dag) \\
&= - \oint_{\bound} dS \, \left( \left( \isigt \div \Edd \phi \right) \varphi^\dag \right) \cdot \vn + \int_V dV \, \left( \isigt \div \Edd \phi \right) \cdot (\grad \varphi^\dag) \\
&= - \sbra \left(  \isigt \div \Edd \phi \right)  \cdot \vn , \varphi^\dag\sket + \int_V dV \, \left( \isigt \div \Edd \phi \right) \cdot (\grad \varphi^\dag) \\
\end{split}
\end{equation}
Turn focus to the remaining volume integral. Define the vector $\vec{u} = \isigt \grad \varphi^\dag$. Breakout $\Edd$ into component vectors $\vec{E}$ 
\begin{equation}
\begin{split}
\int_V dV \, \left( \isigt \div \Edd \phi \right) \cdot (\grad \varphi^\dag) 
& = \int_V dV \, \left( \div \Edd \phi \right) \cdot \vec{u} \\
& = \int_V dV \, \sum_{n=x,y,z} \left( \div \vec{E}_n \phi \right) u_{n} \\
& = \sum_{n=x,y,z} \int_V dV \, \left( \div \vec{E}_n \phi \right) u_{n} \\
& = \sum_{n=x,y,z} \left[ \int_V dV \,  \div \left( \vec{E}_n \phi u_n \right) - \int_V dV (\vec{E}_n \phi) \cdot \grad u_n  \right] \\
& = \sum_{n=x,y,z} \left[ \oint_{\bound} dS \,  \vec{E}_n \phi u_n \cdot \vn - \int_V dV (\vec{E}_n \phi) \cdot \grad u_n  \right] \\
& = \oint_{\bound} dS \,  \phi \left( \vec{E} \cdot \vec{u} \right) \cdot \vn - \int_V dV \phi \left( \Edd : \grad \vec{u} \right) \\
& = \oint_{\bound} dS \,  \phi \left( \vec{E} \cdot \isigt \grad \varphi^\dag \right) \cdot \vn - \int_V dV \phi \left( \Edd : \grad \isigt \grad \varphi^\dag \right) \\
& = \sbra \phi , \vec{E} \cdot \isigt \grad \varphi^\dag \cdot \vn \sket - \bra \phi , \Edd : \grad \isigt \grad \varphi^\dag \ket \\
\end{split}
\end{equation}
Putting everything together
\begin{equation}
\begin{split}
- \bra \div \left( \isigt \div \Edd \phi \right), \vefadj \ket
=& - \bra \phi , \Edd : \grad \isigt \grad \varphi^\dag \ket  
+ \sbra \phi , \vec{E} \cdot \isigt \grad \varphi^\dag \cdot \vn \sket  \\
&- \sbra \left(  \isigt \div \Edd \phi \right)  \cdot \vn , \varphi^\dag \sket\\
\end{split}
\end{equation}

%%%%%%%%%%%%%%%%%%%%%%%%%%%%%%%%%%%%%%%%%%%%%%%%%%%%%%%%%%%%%%%%%%%%%%%%%%%%%%%%%%%%%%%%%%%%%%%%%%%%
\chapter{ \uppercase{ APPENDIX: Additional Flux Graphs showing 2$\varphi$ and 2$\varphi^\dag$}} \label{chap:appx3}
%%%%%%%%%%%%%%%%%%%%%%%%%%%%%%%%%%%%%%%%%%%%%%%%%%%%%%%%%%%%%%%%%%%%%%%%%%%%%%%%%%%%%%%%%%%%%%%%%%%%
Additional graphs showing $\varphi$ and $\varphi^\dag$ scaled by a factor of 2, showing the difference between $\phi$ and $\varphi$ values is more than just the apparent factor of two
\begin{figure}[H]
\centering
\begin{subfigure}{.5\textwidth}
  \centering
  \includegraphics[width=.98\linewidth]{figures2/22phi2.png}
\end{subfigure}%
\begin{subfigure}{.5\textwidth}
  \centering
  \includegraphics[width=.98\linewidth]{figures2/22phia2.png}
\end{subfigure}
\caption{Plots of unperturbed scalar fluxes for the homogeneous system, including $2\varphi$ and $2\varphi^\dag$ }
\label{fig:2Flux1}
\end{figure}

\begin{figure}[H]
\centering
\begin{subfigure}{.5\textwidth}
  \centering
  \includegraphics[width=.98\linewidth]{figures2/24phi2.png}
\end{subfigure}%
\begin{subfigure}{.5\textwidth}
  \centering
  \includegraphics[width=.98\linewidth]{figures2/24phia2.png}
\end{subfigure}
\caption{Plots of unperturbed scalar fluxes for the shielding system, including $2\varphi$ and $2\varphi^\dag$ }
\label{fig:2Flux2}
\end{figure}

\begin{figure}[H]
\centering
\begin{subfigure}{.5\textwidth}
  \centering
  \includegraphics[width=.98\linewidth]{figures2/7phi2.png}
\end{subfigure}%
\begin{subfigure}{.5\textwidth}
  \centering
  \includegraphics[width=.98\linewidth]{figures2/7phia2.png}
\end{subfigure}
\caption{Plots of unperturbed scalar fluxes for the Reed system, including $2\varphi$ and $2\varphi^\dag$ }
\label{fig:2Flux3}
\end{figure}



\end{appendices}

%The next line is the format for inserting new sections.
%Replace the name "newsection"  with the name of your
%new section file.
%\include{data/newsection}


\end{document}
