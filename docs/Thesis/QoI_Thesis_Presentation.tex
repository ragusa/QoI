\documentclass[xcolor={usenames,dvipsnames,svgnames,table}]{beamer}

\mode<presentation>
\usetheme{Madrid}

\usecolortheme[RGB={80,0,0}]{structure}
\useoutertheme[subsection=false]{miniframes}
\useinnertheme{default}

% hide navigation controlls
\setbeamertemplate{navigation symbols}{}

\setbeamercolor{normal text}{fg=black}
\setbeamercovered{dynamic}
\beamertemplatetransparentcovereddynamicmedium
%\usepackage{chronology}
\setbeamertemplate{caption}[numbered]

\definecolor{Maroon}{RGB}{80,0,0}
\definecolor{BurntOrange}{RGB}{204,85,0}


% load macros and prevent authblk from loading
\input{../Presentations/common/macros.tex}
\dontusepackage{authblk}

% load packages, settings and definitions
\input{../Presentations/common/packages.tex}
\input{../Presentations/common/settings.tex}
%\input{../common/definitions.tex}

\newcommand{\vr}{\vec{r}}
\newcommand{\vp}{\vec{p}}
\newcommand{\vOmega}{\vec{\Omega}}
\newcommand{\vJ}{\vec{J}}
\newcommand{\vO}{\vec{\Omega}}
\newcommand{\bra}{\left\langle}
\newcommand{\ket}{\right\rangle}
\newcommand{\sbra}{\left[}
\newcommand{\sket}{\right]}
\newcommand{\braSN}{\left\langle \! \left\langle}
\newcommand{\ketSN}{\right\rangle \! \right\rangle}
\newcommand{\sbraSN}{\left[ \! \left[}
\newcommand{\sketSN}{\right] \! \right]}
\renewcommand{\div}{\vec{\nabla} \cdot}
\newcommand{\grad}{\vec{\nabla}}
\newcommand{\vbeta}{\vec{\beta} }
\newcommand{\pdx}{\frac{\partial}{\partial x}}
\newcommand{\pdy}{\frac{\partial}{\partial y}}
\newcommand{\pdz}{\frac{\partial}{\partial z}}
\newcommand{\intrrr}{\int d^3 r \,}
\newcommand{\intrr}{\int d^2 r \,}
\newcommand{\dEdphi}{\partial_\phi E }
\newcommand{\dEdp}{\partial_p E }
\newcommand{\dBdphi}{\partial_\phi B }
\newcommand{\dBdp}{B }
\newcommand{\adj}{\phi^\dag}
\newcommand{\vefadj}{\varphi^\dag}
\newcommand{\surf}{\int_{\partial V}}
\newcommand{\domain}{V}
\newcommand{\bound}{\partial V}
\newcommand{\vn}{\vec{n}}
\newcommand{\Edd}{\mathbb{E}}
\newcommand{\BEdd}{B}
\newcommand{\sigt}{\sigma_t}
\newcommand{\sigs}{\sigma_s}
\newcommand{\siga}{\sigma_a}
%\newcommand{\isigt}{\sigma_t^{-1}}
%\newcommand{\isigtp}{\sigma_{t,p}^{-1}}
\newcommand{\isigt}{\ell_t}
\newcommand{\isigtp}{\ell_{t,p}}
\newcommand{\angSource}{\frac{q}{4 \pi}}
\newcommand{\angSourcep}{\frac{q_p}{4 \pi}}
\newcommand{\angSourcepd}{\frac{q+\delta q}{4 \pi}}
\newcommand{\angSourced}{\frac{\delta q}{4 \pi}}
\newcommand{\scalSource}{q}
\newcommand{\angResp}{q^\dag}
\newcommand{\scalResp}{q^\dag}
\newcommand{\qoi}{{\it QoI}\xspace}
\newcommand{\tcr}[1]{\textcolor{red}{#1}}
\newcommand{\tcb}[1]{\textcolor{blue}{#1}}
\newcommand{\tcm}[1]{\textcolor{magenta}{#1}}
\newcommand{\tcg}[1]{\textcolor{BlueGreen}{#1}}


% nicer item settings
\setlist[1]{nolistsep,label=\(\textcolor{Maroon}{\blacksquare}\)}
\setlist[2]{nolistsep,label=\(\textcolor{Maroon}{\bullet}\)}

\setenumerate[1]{
	label=\protect\usebeamerfont{enumerate item}%
	\protect\usebeamercolor[fg]{enumerate item}%
	\insertenumlabel.
}

%%%%%%%%%%%%%%%%%%%%%%%%%%%%%%%%%%%%%%%%%%%%%%%
%%% edit to fit your document

% set up pdf support and indexing
\hypersetup{
    pdftitle={Adjoint-Based Sensitivity for Radiation Transport Using an Eddington Tensor Formulation},
    pdfauthor={Ian Halvic},
}

\title[]{Adjoint-Based Sensitivity for Radiation Transport Using an Eddington Tensor Formulation}
\author[ ]{Ian Halvic}
\institute[Texas A\&M]{Department of Nuclear Engineering \\ Texas A\&M University}
\date[3/6/17]

\begin{document}

% title page, do not edit
{
\setbeamertemplate{headline}[default] 
\begin{frame}
\vspace{-1.1cm}
	\begin{figure}[t]
		\centering
			\includegraphics[width=.25\textwidth]{figures2/seal.png}
	\end{figure}
\vspace{-0.75cm}
\titlepage
\end{frame}
}

%%%%%%%%%%%%%%%%%%%%%%%%%%%%%%%%%%%%%%%%%%%%%%%%%%%%%%%%%%%%%%%%%%%%%%%%%%%%%%%%%%%%%%%%%%%%%%%%%%%%%%%%%%%
\section{Introduction}	% define sections here, it is possible to get section slides automatically, but this is not enabled
\subsection{}	% we have to keep these to get the navigation
%%%%%%%%%%%%%%%%%%%%%%%%%%%%%%%%%%%%%%%%%%%%%%%%%%%%%%%%%%%%%%%%%%%%%%%%%%%%%%%%%%%%%%%%%%%%%%%%%%%%%%%%%%%

\begin{frame}[t]\frametitle{Motivation}
\begin{flushleft}
Adjoint methods provide a mechanism for propagating uncertainty in the system variables to the uncertainty in the desired \tcr{quantity of interest} (\qoi). 
\vspace{2mm}
In general adjoint methods 
have the advantages
\begin{itemize}
\item No additional solves required to find the change in \tcr{\qoi, $\delta$\qoi}
\item Change in \qoi is found using a simple inner product
\end{itemize}
\vspace{3mm}
While adjoint methods can greatly reduce the number of system solves required for sensitivity calculations, \tcr{for transport} memory usage may become an issue. To compute $\delta \qoi$, adjoint transport requires storing both the forward angular $\psi$ and adjoint angular $\psi^\dag$ fluxes. This is prohibitive for \tcr{time-dependent transport} where angular fluxes must be stored \tcr{for each timestep}.
\vspace{2mm}
\begin{itemize}
\item consider a \tcr{quasi-diffusion approach to reduce the dimensionality}, so only scalar fluxes are stored
\end{itemize}
\end{flushleft}
\end{frame}
 
 %%%%%%%%%%%%%%%%%%%%%%%%%%%%%%%%%%%%%%%%%%%%%%%%%%%%%%%%%%%%%%%%%%%%%%%%%%%%%%%%%%%%%%%%%%%%%%%%%%%%%%%%%%%

\begin{frame}\frametitle{Approach}
\begin{flushleft}
We will consider a simple transport system
\begin{itemize}
\item steady state, 1 group
\item isotropic sources and response functions
\item isotropic scattering
\end{itemize}
\end{flushleft}
\end{frame}

%%%%%%%%%%%%%%%%%%%%%%%%%%%%%%%%%%%%%%%%%%%%%%%%%%%%%%%%%%%%%%%%%%%%%%%%%%%%%%%%%%%%%%%%%%%%%%%%%%%%%%%%%%%

\begin{frame}\frametitle{Methods Considered}
\begin{flushleft}
\begin{itemize}
\item Present \tcr{transport} equation along with the \tcr{transport adjoint}
\item Formulate a \tcb{Variable Eddington Tensor (VET)} system from the transport equation and \tcb{corresponding adjoint}
\item Refine the VET method in two ways
\begin{itemize} 
\item one leveraging the transport adjoint
\item another estimating Eddington perturbation
\end{itemize}
\item Formulate a Adjoint Variable Eddington Tensor (aVET) system from the \tcr{adjoint transport} equation and corresponding adjoint
\end{itemize}
\end{flushleft}
\end{frame}

%%%%%%%%%%%%%%%%%%%%%%%%%%%%%%%%%%%%%%%%%%%%%%%%%%%%%%%%%%%%%%%%%%%%%%%%%%%%%%%%%%%%%%%%%%%%%%%%%%%%%%%%%%%

%%%%%%%%%%%%%%%%%%%%%%%%%%%%%%%%%%%%%%%%%%%%%%%%%%%%%%%%%%%%%%%%%%%%%%%%%%%%%%%%%%%%%%%%%%%%%%%%%%%%%%%%%%%

\begin{frame}\frametitle{Methods Considered}
\begin{flushleft}
\begin{figure}[H]
\label{Trial1}
\centering
  \centering
  \includegraphics[scale=0.4]{figures2/BigRoadMap.png}
\end{figure}
\end{flushleft}
\end{frame}

%%%%%%%%%%%%%%%%%%%%%%%%%%%%%%%%%%%%%%%%%%%%%%%%%%%%%%%%%%%%%%%%%%%%%%%%%%%%%%%%%%%%%%%%%%%%%%%%%%%%%%%%%%%
\section{Formulations}	% define sections here, it is possible to get section slides automatically, but this is not enabled
\subsection{}	% we have to keep these to get the navigation
%%%%%%%%%%%%%%%%%%%%%%%%%%%%%%%%%%%%%%%%%%%%%%%%%%%%%%%%%%%%%%%%%%%%%%%%%%%%%
%%%%%%%%%%%%%%%%%%%%%%%%%%%%%%

\begin{frame}\frametitle{Transport}
\begin{flushleft}
\begin{figure}[H]
\label{Trial1}
\centering
  \centering
  \includegraphics[scale=0.4]{figures2/BigRoadMap1.png}
\end{figure}
\end{flushleft}
\end{frame}

%%%%%%%%%%%%%%%%%%%%%%%%%%%%%%%%%%%%%%%%%%%%%%%%%%%%%%%%%%%%%%%%%%%%%%%
%%%%%%%%%%%%%%%%%%%%%%%%%%%%%%

\begin{frame}\frametitle{\qoi Computed Using Transport}
 \begin{flushleft}
To begin, consider the one-group steady state transport system with isotropic source $q$ and scattering $\sigs$
\begin{subequations}
\begin{equation}
\label{SNfwd}
 \vO \cdot \grad \psi + \sigt \psi = \frac{\sigs}{4 \pi} \phi + \angSource
\end{equation}
\begin{equation}
\psi(\vr) = \psi^{ \text{inc}}(\vr) \quad \vr \in \partial V^{-} = \{  \vr \in \bound , \quad \vO \cdot \vec{n} < 0 \}
\end{equation}
\end{subequations}
We are concerned with a volumetric and isotropic \qoi given by $\scalResp$
\begin{equation}
\begin{split}
\qoi 	
&:= \braSN \psi , \tcr{\scalResp} \ketSN \equiv \int_V \int_\Omega \psi \scalResp \, d\Omega dV \\
&= \bra \phi , \scalResp \ket \equiv \int_V \phi \scalResp \, dV \\
\end{split}
\end{equation} 
\end{flushleft}
\end{frame}

 %%%%%%%%%%%%%%%%%%%%%%%%%%%%%%%%%%%%%%%%%%%%%%%%%%%%%%%%%%%%%%%%%%%%%%%%%%%%%%%%%%%%%%%%%%%%%%%%%%%%%%%%%%%
 %%%%%%%%%%%%%%%%%%%%%%%%%%%%%%%%%%%%%%%%%%%%%%%%%%%%%%%%%%%%%%%%%%%%%%%%%%%%%%%%%%%%%%%%%%%%%%%%%%%%%%%%%%%

\begin{frame}\frametitle{Sensitivity Using Forward Transport}
 \begin{flushleft}
This presents the most straightforward (and most expensive) way to determine the change in the \qoi due to perturbations in the system parameters
\begin{subequations}
\begin{equation}
\label{SNfwdp}
\vO \cdot \grad \psi_{\tcr{p}} + ( \sigt + \tcb{\delta \sigt}) \psi_{\tcr{p}} = \frac{\sigs+ \tcb{\delta \sigs}}{4 \pi} \phi_{\tcr{p}} + \frac{q + \tcb{\delta q}}{4 \pi}
\end{equation}
\begin{equation}
\psi(\vr) = \psi_{\tcr{p}}^{ \text{inc}}(\vr) \quad \vr \in \partial V^{-} = \{  \vr \in \bound , \quad \vO \cdot \vec{n} < 0 \}
\end{equation}
\end{subequations}
Leading to the simple result
\begin{equation}
\delta \qoi = \bra \phi_{\tcr{p}} , \scalResp \ket - \bra \phi , \scalResp \ket = \bra \delta \phi , \scalResp \ket 
\end{equation} 
\end{flushleft}
\end{frame}

 %%%%%%%%%%%%%%%%%%%%%%%%%%%%%%%%%%%%%%%%%%%%%%%%%%%%%%%%%%%%%%%%%%%%%%%%%%%%%%%%%%%%%%%%%%%%%%%%%%%%%%%%%%%
 
 
 %%%%%%%%%%%%%%%%%%%%%%%%%%%%%%%%%%%%%%%%%%%%%%%%%%%%%%%%%%%%%%%%%%%%%%%%%%%%%%%%%%%%%%%%%%%%%%%%%%%%%%%%%%%

\begin{frame}\frametitle{Transport Adjoint and \qoi}
 \begin{flushleft}
To avoid a forward solve for every perturbation case, an adjoint method may be employed.\cite{Marchuk} The adjoint system is defined 
\begin{subequations}
\begin{equation}
- \vO \cdot \grad \psi^\dag + \sigt \psi^\dag = \frac{\sigs}{4 \pi} \phi^\dag + \scalResp
\end{equation}
\begin{equation}
\psi^\dag(\vr) = \psi^{\dag \text{out}}(\vr)=0 \quad \vr \in \partial V^{+} = \{  \vr \in \bound , \quad \vO \cdot \vec{n} > 0 \}
\end{equation}
\end{subequations}
Through the standard adjoint method a surface term is added in the \qoi
\begin{equation}
\begin{split}
\qoi 
&= \bra \phi^\dag , \angSource \ket - \oint_{\partial V} \int_\Omega (\vO \cdot \vn) \psi^\dag  \psi \, d\Omega dS \\
&= \bra \phi^\dag , \angSource \ket - \sbraSN \psi^\dag , \psi \sketSN \\
\end{split}
\end{equation} 
\end{flushleft}
\end{frame}

 %%%%%%%%%%%%%%%%%%%%%%%%%%%%%%%%%%%%%%%%%%%%%%%%%%%%%%%%%%%%%%%%%%%%%%%%%%%%%%%%%%%%%%%%%%%%%%%%%%%%%%%%%%%\\
  %%%%%%%%%%%%%%%%%%%%%%%%%%%%%%%%%%%%%%%%%%%%%%%%%%%%%%%%%%%%%%%%%%%%%%%%%%%%%%%%%%%%%%%%%%%%%%%%%%%%%%%%%%%

\begin{frame}\frametitle{Sensitivity Using Adjoint Transport}
 \begin{flushleft}
 Multiply by the perturbed $\psi_p$ by $\scalResp$ and integrate. First-order approximation used on last step.
\begin{equation}
\label{snSensPart}
\begin{split}
\qoi_p 
&:=\braSN \psi_p , \tcg{\angResp} \ketSN \\
&=\braSN \psi_p , \tcg{- \vO \cdot \grad \psi^\dag + \sigt \psi^\dag - \frac{\sigs}{4 \pi} \phi^\dag  }\ketSN \\
&=\braSN  \vO \cdot \grad \psi_p + \sigt \psi_p - \frac{\sigs}{4 \pi} \phi_p , \psi^\dag  \ketSN - \sbraSN \psi_p, \psi^\dag \sketSN \\
&= \braSN \tcm{ \vO \cdot \grad \psi_p + \sigma_{t,p}\psi_p} - \delta\sigt \tcr{\psi_p} \tcm{- \frac{\sigma_{s,p}}{4 \pi} \phi_p}
+\frac{\delta \sigs}{4 \pi} \tcr{\phi_p}
 , \psi^\dag  \ketSN - \sbraSN \psi_p, \psi^\dag \sketSN \\
&\tcr{\approx} \braSN \tcm{ \angSourcepd } - \delta\sigt \tcb{\psi} + \frac{\delta \sigs}{4 \pi} \tcb{\phi}
 , \psi^\dag  \ketSN - \sbraSN \psi_p, \psi^\dag \sketSN
\end{split}
\end{equation}

\end{flushleft}
\end{frame}

 %%%%%%%%%%%%%%%%%%%%%%%%%%%%%%%%%%%%%%%%%%%%%%%%%%%%%%%%%%%%%%%%%%%%%%%%%%%%%%%%%%%%%%%%%%%%%%%%%%%%%%%%%%%
 \begin{frame}\frametitle{Transport Adjoint}
 \begin{flushleft}
Perform $\qoi_p - \qoi$ to obtain the change in $\qoi$
\begin{equation}
\label{snSens}
\begin{split}
\delta \qoi 
&\approx \braSN  \angSourced - \delta\sigt\psi + \frac{\delta \sigs}{4 \pi} \phi
 , \psi^\dag  \ketSN - \sbraSN \delta \psi, \psi^\dag \sketSN \\
 &= \bra \angSourced, \phi^\dag \ket - \tcr{\braSN \delta\sigt\psi , \psi^\dag \ketSN} + \bra \frac{\delta \sigs}{4 \pi} \phi
 , \phi^\dag \ket - \sbraSN \delta \psi^{\text{inc}}, \psi^\dag \sketSN \\
\end{split}
\end{equation}
Pros
\begin{itemize}
\item Do not need SN solve for each perturbation scenario, just 2 SN solves to get unperturbed fluxes $\psi$ and $\psi^\dag$
\item Still exact for source (volumetric and boundary) perturbations, $\delta q$ and $\delta \psi^{\text{inc}}$
\end{itemize}
Cons
\begin{itemize}
\item Only first-order perturbation accurate ($\delta \sigs \delta \phi=0$, $\delta \sigt \delta \phi=0$)
\item Need to store angular dependent $\psi$ and $\psi^\dag$, which is prohibitive for time-dependent problems
\end{itemize}
\end{flushleft}
\end{frame}

 %%%%%%%%%%%%%%%%%%%%%%%%%%%%%%%%%%%%%%%%%%%%%%%%%%%%%%%%%%%%%%%%%%%%%%%%%%%%%%%%%%%%%%%%%%%%%%%%%%%%%%%%%%%
 %%%%%%%%%%%%%%%%%%%%%%%%%%%%%%

\begin{frame}\frametitle{VET Formulation}
\begin{flushleft}
\begin{figure}[H]
\label{Trial1}
\centering
  \centering
  \includegraphics[scale=0.4]{figures2/BigRoadMap2.png}
\end{figure}
\end{flushleft}
\end{frame}

%%%%%%%%%%%%%%%%%%%%%%%%%%%%%%%%%%%%%%%%%%%%%%%%%%%%%%%%%%%%%%%%%%%%%%%
 %%%%%%%%%%%%%%%%%%%%%%%%%%%%%%%%%%%%%%%%%%%%%%%%%%%%%%%%%%%%%%%%%%%%%%%%%%%%%%%%%%%%%%%%%%%%%%%%%%%%%%%%%%%

\begin{frame}\frametitle{VET Formulation}
 \begin{flushleft}
Formulate the P1 equations and define an Eddington Tensor $\Edd$ and a Boundary Eddington Factor $\BEdd$ 
\begin{equation}
\div \vec{J} + (\sigt-\sigs) \phi = \scalSource
, \quad \quad 
\div \left(  \int d\Omega \vO \vO \psi \right) + \sigt \vec{J} = 0 
\end{equation}
\begin{equation}
\Edd(\vr)=\frac{\int d\Omega \vO \vO \psi(\vr,\vO)}{\phi(\vr)}
, \, \vr \in V \quad \quad 
\BEdd(\vr) = \frac{\int_{4 \pi} d\Omega \, | \vO \cdot \vn | \psi}{\phi(\vr)} , \, \vr \in \bound
\end{equation}
Combine into the VET formulation \cite{Miften}
\begin{subequations}
\begin{equation}
\label{VEFForm}
- \div \left( \frac{1}{\sigt}\div \Edd \phi \right) + \siga \phi = \scalSource \, \quad \vr \in V 
\end{equation}
\begin{equation}
2 J^{\text{inc}} = \BEdd \phi + \vn \cdot \frac{1}{\sigt} \div \Edd \phi \,\quad  \vr \in \bound
\end{equation}
\end{subequations}
If exact $\Edd$ and $\BEdd$ are know, then $\phi^{\text{VET}} = \phi^{\text{SN}}$
\end{flushleft}
\end{frame}

%%%%%%%%%%%%%%%%%%%%%%%%%%%%%%%%%%%%%%%%%%%%%%%%%%%%%%%%%%%%%%%%%%%%%%%%%%%%%%%%%%%%%%%%%%%%%%%%%%%%%%%%%%%
%%%%%%%%%%%%%%%%%%%%%%%%%%%%%%%%%%%%%%%%%%%%%%%%%%%%%%%%%%%%%%%%%%%%%%%%%%%%%%%%%%%%%%%%%%%%%%%%%%%%%%%%%%%

\begin{frame}\frametitle{VET Adjoint Formulation}
 \begin{flushleft}
To derive an expression for the VET adjoint equation $\vefadj$, start with the forward VET balance equation, Eq.~\eqref{VEFForm}
\begin{equation}
\label{VEFadjFormDeriv}
\begin{split}
\bra \scalSource , \vefadj \ket &= - \bra \div \left( \frac{1}{\sigt}\div \Edd \phi \right), \vefadj \ket +  \bra \siga \phi, \vefadj \ket   \\
&= \bra \frac{1}{\sigt}\div \Edd \phi, \grad \vefadj \ket  +  \bra  \phi, \siga \vefadj \ket - \sbra \vn \cdot \frac{1}{\sigt}\div \Edd \phi, \vefadj \sket   \\
 &=  \bra \phi, \tcb{\Edd : \left( - \grad \left( \frac{1}{\sigt}\grad \vefadj \right) \right) + \siga \vefadj } \ket \\
 & - \sbra \vn \cdot  \frac{1}{\sigt}\div \Edd \phi, \vefadj \sket + \sbra \phi, \vn \cdot  \Edd \cdot \frac{1}{\sigt} \grad \vefadj \sket \\
\end{split}
\end{equation}
Where the inner product $\sbra f ,g \sket = \oint_{\partial V} fg \, dS$ has been defined
\end{flushleft}
\end{frame}

 %%%%%%%%%%%%%%%%%%%%%%%%%%%%%%%%%%%%%%%%%%%%%%%%%%%%%%%%%%%%%%%%%%%%%%%%%%%%%%%%%%%%%%%%%%%%%%%%%%%%%%%%%%%
 %%%%%%%%%%%%%%%%%%%%%%%%%%%%%%%%%%%%%%%%%%%%%%%%%%%%%%%%%%%%%%%%%%%%%%%%%%%%%%%%%%%%%%%%%%%%%%%%%%%%%%%%%%%

\begin{frame}\frametitle{VET Adjoint Formulation}
 \begin{flushleft}
From the previous balance equation, an obvious form of the VET adjoint emerges
\begin{subequations}
\begin{equation}
\label{adjForm}
- \Edd : \left( \grad \left( \frac{1}{\sigt}\grad \vefadj \right) \right) + \siga \vefadj = \scalResp,    \quad \vr \in V
\end{equation}
\begin{equation}
\label{adjVETBC}
2J^{\dag,\text{out}} = B \vefadj+ \vn \cdot
\Edd \cdot \frac{1}{\sigma_{t} } \vec{\nabla} \vefadj,    \quad \vr \in \bound
\end{equation}
\end{subequations}
Choosing $J^{\dag,\text{out}} =0$ results in the relatively simple \qoi expression using this new formulation
\begin{equation}
\label{adjVETqoi}
\qoi=\bra \scalSource , \vefadj \ket  + \sbra \vefadj, 2J^{\text{inc}} \sket
\end{equation}
Of particular note is that $ \phi^\dag \neq \vefadj$
\end{flushleft}
\end{frame}

 %%%%%%%%%%%%%%%%%%%%%%%%%%%%%%%%%%%%%%%%%%%%%%%%%%%%%%%%%%%%%%%%%%%%%%%%%%%%%%%%%%%%%%%%%%%%%%%%%%%%%%%%%%%

 %%%%%%%%%%%%%%%%%%%%%%%%%%%%%%%%%%%%%%%%%%%%%%%%%%%%%%%%%%%%%%%%%%%%%%%%%%%%%%%%%%%%%%%%%%%%%%%%%%%%%%%%%%%

\begin{frame}\frametitle{Sensitivity Using VET Adjoint}
 \begin{flushleft}
Now an assumption is made that the \tcr{$\Edd$ and $\BEdd$ remain unperturbed under system perturbation}. Note $\ell_t = \frac{1}{\sigt}$
\begin{equation}
\label{VETSensDeriv}
\begin{split}
\qoi_p = &\bra \tcb{\phi_p} , \scalResp \ket \\
       = &\bra \tcb{\phi_p} , - \Edd : \left( \grad \isigt \grad \varphi^\dag \right) + \siga \vefadj \ket \\
\tcr{\approx} & \bra \scalSource + \delta \scalSource + \div \delta \isigt \div \left( \Edd \tcr{\phi} \right) - \delta \siga \tcr{\phi}, \vefadj \ket - \sbra \tcb{\phi_p}, \Edd \cdot \isigt \grad \vefadj \sket \\
&+ \sbra \vefadj, \isigt \div \Edd \tcb{\phi_p} \sket \\
=& \bra q, \vefadj \ket  + \bra \delta \scalSource + \div \delta \isigt \div \left( \Edd \tcr{\phi} \right)  - \delta \siga \tcr{\phi}, \vefadj \ket \\
& - \sbra \tcb{\phi_p}, \Edd \cdot \isigt \grad \vefadj \sket + \sbra \vefadj, \isigt \div \Edd \tcb{\phi_p} \sket 
\end{split}
\end{equation}

\end{flushleft}
\end{frame}

 %%%%%%%%%%%%%%%%%%%%%%%%%%%%%%%%%%%%%%%%%%%%%%%%%%%%%%%%%%%%%%%%%%%%%%%%%%%%%%%%%%%%%%%%%%%%%%%%%%%%%%%%%%%

  %%%%%%%%%%%%%%%%%%%%%%%%%%%%%%%%%%%%%%%%%%%%%%%%%%%%%%%%%%%%%%%%%%%%%%%%%%%%%%%%%%%%%%%%%%%%%%%%%%%%%%%%%%%

\begin{frame}\frametitle{Sensitivity Using VET Adjoint}
 \begin{flushleft}
 Subtracting off the unperturbed \qoi in Eq.~\eqref{adjVETqoi} and cleaning up some boundary terms results in
 \begin{equation}
\label{VETsens}
\begin{split}
\delta \qoi =&  \bra \delta \scalSource - \delta \siga \phi, \vefadj \ket  - \bra \delta \isigt \div \left( \Edd \phi \right) , \grad \vefadj \ket + \sbra \vefadj, 2 \delta J^{\text{inc}} \sket \\
\end{split}
\end{equation}
Pros
\begin{itemize}
\item Requires only 1 SN solve to get $\Edd$ and $\phi$, and 1 scalar VET solve for $\varphi^\dag$
\item No angular fluxes stored, only $\Edd$
\end{itemize}
Cons
\begin{itemize}
\item Only first order perturbation accurate (as was SN adjoint)
\item Major assumption that $\delta \Edd=0$, which is not always true. This means that the method in general is not exact for source perturbations $\delta q$ and $\delta \psi^{\text{inc}}$, which was the case for transport
\end{itemize}
\end{flushleft}
\end{frame}
 %%%%%%%%%%%%%%%%%%%%%%%%%%%%%%%%%%%%%%%%%%%%%%%%%%%%%%%%%%%%%%%%%%%%%%%%%%%%%%%%%%%%%%%%%%%%%%%%%%%%%%%%%%%

\begin{frame}\frametitle{Refinement Option 1: ``Blended'' Approach}
 \begin{flushleft}
Form a $\delta \qoi$ inner-product by picking the exact source terms from \tcr{SN adjoint Eq.~\eqref{snSens}} and cross-section terms from \tcb{VET formulation adjoint Eq.~\eqref{VETsens}}
\begin{equation}
\label{Blendsens}
\begin{split}
\delta \qoi =&  \tcr{\bra \angSourced , \phi^\dag \ket} \tcb{- \bra \delta \siga \phi, \vefadj \ket - \bra \delta \isigt \div \left( \Edd \phi \right) , \grad \vefadj \ket} \tcr{
- \sbraSN \delta \psi^\text{inc}, \psi^\dag \sketSN }\\
\end{split}
\end{equation}
Pros
\begin{itemize}
\item Requires 2 transport solves to get $\Edd$, $\phi$ and $\phi^\dag$ then one VET solve for $\varphi^\dag$
\item No angular fluxes stored
\item Still exact for source perturbations (equivalent to transport adjoint)
\end{itemize}
Cons
\begin{itemize}
%\item Lacks rigorous derivation
\item Could experience issues when both source and cross-section perturbation are present, compared to pure VET adjoint
\end{itemize}
\end{flushleft}
\end{frame}

%%%%%%%%%%%%%%%%%%%%%%%%%%%%%%%%%%%%%%%%%%%%%%%%%%%%%%%%%%%%%%%%%%%%%%%%%%%%%%%%%%%%%%%%%%%%%%%%%%%%%%%%%%%
 
%%%%%%%%%%%%%%%%%%%%%%%%%%%%%%%%%%%%%%%%%%%%%%%%%%%%%%%%%%%%%%%%%%%%%%%%%%%%%%%%%%%%%%%%%%%%%%%%%%%%%%%%%%%

\begin{frame}\frametitle{Refinement Option 2: Estimate $\delta \Edd$}
 \begin{flushleft}
Attempt to approximate by taking the slope with respect to perturbation parameters $\vp$
\begin{equation}
\delta \Edd \approx \frac{\partial \Edd}{\partial \vp} \cdot \delta \vp \approx \left( \frac{\Edd(\vp_1) - \Edd(\vp_0)}{\vp_1 - \vp_0} \right) \cdot \delta \vp
\end{equation}
 \begin{equation}
\label{VETsens}
\begin{split}
\delta \qoi =&  \bra \delta \scalSource - \delta \siga \phi, \vefadj \ket  - \bra \delta \isigt \div \left( \Edd \phi \right) , \grad \vefadj \ket + \sbra \vefadj, 2 \delta J^{\text{inc}} \sket \\
& - \bra  \isigt \div \left( \tcr{\delta \Edd} \phi \right), \grad \vefadj \ket
- \sbra \vefadj, \phi \tcr{\delta \BEdd} \sket
\end{split}
\end{equation}
Pros
\begin{itemize}
\item Attempts to resolve errors from the $\delta \Edd = 0$ assumption in VET adjoint
\end{itemize}
Cons
\begin{itemize}
\item Requires an additional transport solve for $\Edd(\vp_1)$. Pointless for only testing one perturbation scenario, potentially powerful for testing many perturbation scenarios
\end{itemize}
\end{flushleft}
\end{frame}

 %%%%%%%%%%%%%%%%%%%%%%%%%%%%%%%%%%%%%%%%%%%%%%%%%%%%%%%%%%%%%%%%%%%%%%%%%%%%%%%%%%%%%%%%%%%%%%%%%%%%%%%%%%%
 %%%%%%%%%%%%%%%%%%%%%%%%%%%%%%%
 %%%%%%%%%%%%%%%%%%%%%%%%%%%%%%

\begin{frame}\frametitle{Adjoint-VET (Alternate VET)}
\begin{flushleft}
\begin{figure}[H]
\label{Trial1}
\centering
  \centering
  \includegraphics[scale=0.4]{figures2/BigRoadMap3.png}
\end{figure}
\end{flushleft}
\end{frame}

%%%%%%%%%%%%%%%%%%%%%%%%%%%%%%%%%%%%%%%%%%%%%%%%%%%%%%%%%%%%%%%%%%%%%%%
 %%%%%%%%%%%%%%%%%%%%%%%%%%%%%%%%%%%%%%%%%%%%%%%%%%%%%%%%%%%%%%%%%%%%%%%%%%%%%%%%%%%%%%%%%%%%%%%%%%%%%%%%%%%

\begin{frame}\frametitle{Adjoint-VET (aVET)}
\begin{flushleft}
Reverse the order of the VET derivation, start with \tcr{adjoint} P1 and work towards an alternate forward $\varphi$
\begin{equation}
\label{0amAlt}
\tcr{-}\div \vec{J}^\dag + (\sigt-\sigs) \phi^\dag  = \tcr{4\pi} \scalResp 
, \quad \quad 
\tcr{-}\div \left(  \int d\Omega \vO \vO \psi^\dag  \right) + \sigt \vec{J}^\dag  = 0
\end{equation}
\begin{equation}
\Edd^\dag(\vr)=\frac{\int d\Omega \vO \vO \psi^\dag(\vr,\vO)}{\phi^\dag(\vr)}
, \quad \quad 
\BEdd^\dag(\vr) = \frac{\int_{4 \pi} d\Omega \, | \vO \cdot \vn | \psi^\dag}{\phi^\dag(\vr)} \, \vr \in \bound
\end{equation}
Gives the familiar form
\begin{subequations}
\begin{equation}
\label{TranAdjVEFForm}
- \div \left( \frac{1}{\sigt}\div \Edd^\dag \phi^\dag  \right) + \siga \phi^\dag  = 4\pi \scalResp  \,.
\end{equation}
\begin{equation}
2 J^{^\dag,\text{out}}(\vr) = \BEdd^\dag(\vr) \phi^\dag(\vr)  + \vn \cdot \frac{1}{\sigt} \div \Edd^\dag  \phi^\dag  \,.
\end{equation}
\end{subequations}
\end{flushleft}
\end{frame}

 %%%%%%%%%%%%%%%%%%%%%%%%%%%%%%%%%%%%%%%%%%%%%%%%%%%%%%%%%%%%%%%%%%%%%%%%%%%%%%%%%%%%%%%%%%%%%%%%%%%%%%%%%%%
 
 %%%%%%%%%%%%%%%%%%%%%%%%%%%%%%%%%%%%%%%%%%%%%%%%%%%%%%%%%%%%%%%%%%%%%%%%%%%%%%%%%%%%%%%%%%%%%%%%%%%%%%%%%%%

\begin{frame}\frametitle{Adjoint-VET (aVET)}
 \begin{flushleft}
Adjoint of this Alternate VET is proposed
\begin{subequations}
\begin{equation}
\label{ForwardVEFAlt}
- \Edd^\dag : \grad \left( \frac{1}{\sigt}\grad \varphi \right) + \siga \varphi  = \tcb{\angSource}, \quad \vr \in V  \,.
\end{equation}
\begin{equation}
2 J^{\text{inc}}(\vr) = \BEdd^\dag(\vr) \varphi(\vr) + \Edd^\dag \cdot \frac{1}{\sigt} \grad \varphi, \quad \vr \in \bound \,
\end{equation}
\end{subequations}
\qoi derived starting with basic transport definition
\begin{equation}
\label{AdjQoIAltExpand}
\begin{split}
QoI = \bra \phi , \scalResp \ket &= \bra \phi^\dag , \tcb{\angSource} \ket - \sbraSN \psi^\dag,  \psi \sketSN \\
&= \bra \phi^\dag , - \Edd^\dag : \grad \left( \frac{1}{\sigt}\grad \varphi \right) + \siga \varphi \ket - \sbraSN \psi^\dag,  \psi \sketSN \\
&= \bra 4\pi \scalResp  ,\varphi \ket - \sbraSN \psi^\dag,  \psi \sketSN  
- \sbra \Edd^\dag \cdot \frac{1}{\sigt}\grad \varphi,  \phi^\dag \sket 
+ \sbra \frac{1}{\sigt} \div \Edd^\dag \phi^\dag,  \varphi \sket \\
&=  \bra 4\pi \scalResp  ,\varphi \ket - \sbraSN \psi^\dag,  \psi \sketSN - \sbra \phi^\dag, 2J^{\text{inc}} \sket + \sbra \varphi , 2 J^{\dag,\text{out}} \sket
\end{split}
\end{equation}
\end{flushleft}
\end{frame}

 %%%%%%%%%%%%%%%%%%%%%%%%%%%%%%%%%%%%%%%%%%%%%%%%%%%%%%%%%%%%%%%%%%%%%%%%%%%%%%%%%%%%%%%%%%%%%%%%%%%%%%%%%%%
 
%%%%%%%%%%%%%%%%%%%%%%%%%%%%%%%%%%%%%%%%%%%%%%%%%%%%%%%%%%%%%%%%%%%%%%%%%%%%%%%%%%%%%%%%%%%%%%%%%%%%%%%%%%%

\begin{frame}\frametitle{Sensitivity Using aVET}
\begin{flushleft}
Perturbations are introduced in the adjoint transport system, which perturbs the aVET system, but the adjoint aVET remains unperturbed. Assumes unperturbed $\Edd^\dag$ and $\BEdd^\dag$.
\begin{equation}
\begin{split}
QoI_p 
&= \bra \phi^\dag_p , \angSourcepd \ket  - \sbraSN \psi^\dag_p,  \psi \sketSN  \\
&= \bra \phi^\dag_p , - \Edd^\dag : \grad \left( \frac{1}{\sigt}\grad \varphi \right) + \siga \varphi \ket + \bra \phi^\dag_p , \angSourced \ket - \sbraSN \psi^\dag_p,  \psi \sketSN  \\
&\tcr{\approx} \bra 4 \pi \scalResp + 4 \pi \tcb{\delta \scalResp} , \varphi \ket + \bra\div \left( \delta \isigt \div \Edd^\dag \tcr{\phi^\dag}  \right), \varphi \ket - \bra \delta \siga \tcr{\phi^\dag} , \varphi \ket  \\
& \quad + \bra \tcr{\phi^\dag} , \angSourced \ket - \sbra \phi_p^\dag, 2J^{\text{inc}} \sket + \sbra \varphi , 2 J^{\dag,\text{out}} - \delta \isigt \div \Edd \phi^\dag \sket  -\sbraSN \psi^\dag,  \psi_p \sketSN   \\
\end{split}
\end{equation} 
\end{flushleft}
\end{frame}

 %%%%%%%%%%%%%%%%%%%%%%%%%%%%%%%%%%%%%%%%%%%%%%%%%%%%%%%%%%%%%%%%%%%%%%%%%%%%%%%%%%%%%%%%%%%%%%%%%%%%%%%%%%%
 
  %%%%%%%%%%%%%%%%%%%%%%%%%%%%%%%%%%%%%%%%%%%%%%%%%%%%%%%%%%%%%%%%%%%%%%%%%%%%%%%%%%%%%%%%%%%%%%%%%%%%%%%%%%%

\begin{frame}\frametitle{Adjoint-VET (Alternate VET)}
\begin{flushleft}
Some cleanup gives 
\begin{equation}
\begin{split}
\delta \qoi &= \bra \tcb{\delta q^\dag}, \varphi \ket - \bra\left( \delta \isigt \div \Edd^\dag \phi^\dag  \right), \grad \varphi \ket \
- \bra \delta \siga \phi^\dag , \varphi \ket + \bra \delta q , \phi \ket \\
&\quad - \sbraSN \delta \psi^{\text{inc}}, \tcr{\psi^{\dag \text{inc}}_p}\sketSN
\end{split}
\end{equation}
Pros
\begin{itemize}
\item Requires 1 transport solve to get $\Edd^\dag$ and $\phi^\dag$ then one VET solve for $\varphi$
\item No angular fluxes stored
\item Still exact for volumetric source perturbations $\delta q$
\end{itemize}
Cons
\begin{itemize}
\item Due to boundary term, cannot be used with incident flux systems. 
\item Assumes an unperturbed $\Edd^\dag$ and $\BEdd^\dag$
\item Not exact for perturbed response $\delta q^\dag$, which was trivial for forward transport derived methods. $\bra \phi, \delta q^\dag \ket$
\end{itemize}
\end{flushleft}
\end{frame}

%%%%%%%%%%%%%%%%%%%%%%%%%%%%%%%%%%%%%%%%%%%%%%%%%%%%%%%%%%%%%%%%%%%%%%%%%%%%%%%%%%%%%%%%%%%%%%%%%%%%%%%%%%%
\begin{frame}\frametitle{Summary of methods}
\begin{flushleft}
\begin{table}[H]
\centering
  \begin{tabular}{| l | l |}
    \hline
    Method  &  $\delta \qoi$ Inner Product \\ \hline
     Transport 			&$\bra \tcr{\phi_{p}} , \scalResp \ket - \bra \phi , \scalResp \ket = \bra \delta \phi , \scalResp \ket $ \\ \hline
     Trans Adj  			&$\bra \angSourced, \phi^\dag \ket - \tcr{\braSN \delta\sigt\psi , \psi^\dag \ketSN} + \bra \frac{\delta \sigs}{4 \pi} \phi
 , \phi^\dag \ket - \sbraSN \delta \psi^{\text{inc}}, \psi^\dag \sketSN$\\ \hline
     VET Adj			&$ \bra \delta \scalSource, \vefadj  \ket - \bra \delta \siga \phi, \vefadj \ket  - \bra \delta \isigt \div \left( \Edd \phi \right) , \grad \vefadj \ket + \sbra \vefadj, 2 \delta J^{\text{inc}} \sket$\\ \hline
     VET Blend			&$\bra \angSourced , \phi^\dag \ket - \bra \delta \siga \phi, \vefadj \ket - \bra \delta \isigt \div \left( \Edd \phi \right) , \grad \vefadj \ket 
- \sbraSN \delta \psi^\text{inc}, \psi^\dag \sketSN $\\ \hline
     VET $\delta \Edd$ 	&VET Adj $- \bra  \isigt \div \left( \tcr{\delta \Edd} \phi \right), \grad \vefadj \ket
- \sbra \vefadj, \phi \tcr{\delta \BEdd} \sket$ \\ \hline
     aVET			&$
     \begin{array}{lcl}
     &\bra \tcb{\delta q^\dag}, \varphi \ket - \bra\left( \delta \isigt \div \Edd^\dag \phi^\dag  \right), \grad \varphi \ket \
- \bra \delta \siga \phi^\dag , \varphi \ket \\ &+ \bra \delta q , \phi \ket - \sbraSN \delta \psi^{\text{inc}}, \tcr{\psi^{\dag \text{inc}}_p}\sketSN 
\end{array}$ \\ \hline
    \end{tabular}
  \caption{Summary of Methods. For response perturbation $\delta q^\dag$, the straightforward $\bra \phi , \delta q^\dag \ket $ had been omitted from all methods except aVET.}
\end{table}
\end{flushleft}
\end{frame}
%%%%%%%%%%%%%%%%%%%%%%%%%%%%%%%%%%%%%%%%%%%%%%%%%%%%%%%%%%%%%%%%%%%%%%%%%%%%%%%%%%%%%%%%%%%%%%%%%%%%%%%%%%%
\section{Test Cases}	% define sections here, it is possible to get section slides automatically, but this is not enabled
\subsection{}	% we have to keep these to get the navigation
%%%%%%%%%%%%%%%%%%%%%%%%%%%%%%%%%%%%%%%%%%%%%%%%%%%%%%%%%%%%%%%%%%%%%%%%%%%%%%%%%%%%%%%%%%%%%%%%%%%%%%%%%%%
 %%%%%%%%%%%%%%%%%%%%%%%%%%%%%%%%%%%%%%%%%%%%%%%%%%%%%%%%%%%%%%%%%%%%%%%%%%%%%%%%%%%%%%%%%%%%%%%%%%%%%%%%%%%

\begin{frame}\frametitle{Solvers}
\begin{flushleft}
Transport solver ($\psi$, $\psi^\dag$, $\phi$, $\phi^\dag$)
\begin{itemize}
\item 1D slab geometry solver
\item Iterative SN method; N=8
\item Uniform grid, 2000 elements
\end{itemize}
VET solver ($\phi$, $\phi^\dag$, $\varphi$, $\varphi^\dag$)
\begin{itemize}
\item 1D slab geometry solver
\item Discontinuous Galerkin FEM
\item Uniform grid, 2000 elements
\end{itemize}
Note that $\delta \siga$ and $\delta \sigs$ are specified for each problem, and $\delta \sigt=\delta \siga+\delta \sigs$ is computed.
\end{flushleft}
\end{frame}

 %%%%%%%%%%%%%%%%%%%%%%%%%%%%%%%%%%%%%%%%%%%%%%%%%%%%%%%%%%%%%%%%%%%%%%%%%%%%%%%%%%%%%%%%%%%%%%%%%%%%%%%%%%%

\begin{frame}\frametitle{Homogeneous System, Homogeneous Perturbation}
        \begin{flushleft}
             Homogeneous System with Homogeneous Perturbation
          	 \begin{itemize}
			      \item $\siga=1$, $\sigs=1$
			      \item $q=2$, $\psi^{\text{inc}}=0$
			      \item $q^\dag=1$ for $4 \leq x \leq 6$, so $\qoi = \int_4^6 \phi \, dx$
			      \item Perturbed throughout
  			 \end{itemize} 
            \end{flushleft}
\begin{figure}[H]
\label{FLux1}
\centering
\begin{subfigure}{0.45\textwidth}
  \centering
  \includegraphics[width=.98\linewidth]{figures2/22phi.png}
  \label{T1:sfig1}
\end{subfigure}
%
\begin{subfigure}{0.45\textwidth}
  \centering
  \includegraphics[width=.98\linewidth]{figures2/22phia.png}
  \label{T1:sfig2}
\end{subfigure}
\end{figure}
\end{frame}

 %%%%%%%%%%%%%%%%%%%%%%%%%%%%%%%%%%%%%%%%%%%%%%%%%%%%%%%%%%%%%%%%%%%%%%%%%%%%%%%%%%%%%%%%%%%%%%%%%%%%%%%%%%%

\begin{frame}\frametitle{Homogeneous System, Homogeneous Perturbation}
 \begin{flushleft}

\begin{figure}[H]
\label{Trial1}
\centering
\begin{subfigure}{.5\textheight}
  \centering
  \includegraphics[width=.98\linewidth]{figures2/22qSens.png}
  \label{T1:sfig1}
\end{subfigure}%
\begin{subfigure}{.5\textheight}
  \centering
  \includegraphics[width=.98\linewidth]{figures2/22sigaSens.png}
  \label{T1:sfig2}
\end{subfigure}
%
\begin{subfigure}{.5\textheight}
  \centering
  \includegraphics[width=.98\linewidth]{figures2/22sigsSens.png}
  \label{T1:sfig3}
\end{subfigure}%
\begin{subfigure}{.5\textheight}
  \centering
  \includegraphics[width=.98\linewidth]{figures2/22qsigaSens.png}
  \label{T1:sfig4}
\end{subfigure}
\end{figure}
\end{flushleft}
\end{frame}

 %%%%%%%%%%%%%%%%%%%%%%%%%%%%%%%%%%%%%%%%%%%%%%%%%%%%%%%%%%%%%%%%%%%%%%%%%%%%%%%%%%%%%%%%%%%%%%%%%%%%%%%%%%%

\begin{frame}\frametitle{Homogeneous System, Homogeneous Perturbation}
\begin{flushleft}
\begin{itemize}
\item Infinite medium approximation in both unperturbed and perturbed. $\phi_\infty=\frac{q}{\siga}$
\item All methods exact for $q$ perturbations ($\delta \Edd \approx 0$ valid here).
\item All adjoint methods show first-order approximation error for $\siga$ perturbations.
\item Low sensitivity to $\sigs$ perturbations.
\end{itemize}
\end{flushleft}
\end{frame}

 %%%%%%%%%%%%%%%%%%%%%%%%%%%%%%%%%%%%%%%%%%%%%%%%%%%%%%%%%%%%%%%%%%%%%%%%%%%%%%%%%%%%%%%%%%%%%%%%%%%%%%%%%%%

\begin{frame}\frametitle{Homogeneous System, Non-Homogeneous Perturbation}
        \begin{flushleft}
             Homogeneous System with Non-Homogeneous Perturbation
          	 \begin{itemize}
			      \item $\siga=1$, $\sigs=1$
			      \item $q=2$, $\psi^{\text{inc}}=0$
			      \item $\qoi = \int_4^6 \phi \, dx$. Means same adjoint flux as previous case.
			      \item \textcolor{red}{Perturbed for $0 \leq x \leq 6$}
  			 \end{itemize} 
            \end{flushleft}
\begin{figure}[H]
\label{Flux2}
\centering
\begin{subfigure}{0.45\textwidth}
  \centering
  \includegraphics[width=.98\linewidth]{figures2/23phip.png}
  \label{T1:sfig1}
\end{subfigure}
%
\begin{subfigure}{0.45\textwidth}
  \centering
  \includegraphics[width=.98\linewidth]{figures2/23deltaE.png}
  \label{T1:sfig2}
\end{subfigure}
\end{figure}
\end{frame}

 %%%%%%%%%%%%%%%%%%%%%%%%%%%%%%%%%%%%%%%%%%%%%%%%%%%%%%%%%%%%%%%%%%%%%%%%%%%%%%%%%%%%%%%%%%%%%%%%%%%%%%%%%%%

\begin{frame}\frametitle{Homogeneous System, Non-Homogeneous Perturbation}
 \begin{flushleft}

\begin{figure}[H]
\label{Trial2}
\begin{minipage}{.35\textwidth}
\centering
\begin{subfigure}{.45\textheight}
  \centering
  \includegraphics[width=.98\linewidth]{figures2/23qSens.png}
  \label{T1:sfig1}
\end{subfigure}%

\begin{subfigure}{.45\textheight}
  \centering
  \includegraphics[width=.98\linewidth]{figures2/23sigsSens.png}
\end{subfigure}
\end{minipage}
\begin{minipage}{.60\textwidth}
\begin{subfigure}{1\textwidth}
  \centering
  \includegraphics[width=.98\linewidth]{figures2/23sigaSens.png}
\end{subfigure}%
%\begin{subfigure}{.5\textheight}
%  \centering
%  \includegraphics[width=.98\linewidth]{figures2/23qsigaSens.png}
%  \label{T1:sfig4}
%\end{subfigure}
\end{minipage}
\end{figure}
\end{flushleft}
\end{frame}

 %%%%%%%%%%%%%%%%%%%%%%%%%%%%%%%%%%%%%%%%%%%%%%%%%%%%%%%%%%%%%%%%%%%%%%%%%%%%%%%%%%%%%%%%%%%%%%%%%%%%%%%%%%%

\begin{frame}\frametitle{Homogeneous System, Non-Homogeneous Perturbation}
 \begin{flushleft}
\begin{itemize}
\item Infinite medium approximation in only unperturbed.
\item For source perturbations: Transport Adjdoint, blended, and aVET agree with transport forward.
\item For $\sigma$ perturbations, adjoint methods begin to show first order approximation error.
\item For $\sigma$ perturbations VET/blended adjoint are within $1 \%$ of transport adjoint. Eddington approximation and aVET nearly equal to transport adjoint $< 0.1 \%$ 
\item Still low sensitivity to $\sigs$ perturbations.
\end{itemize}
\end{flushleft}
\end{frame}

 %%%%%%%%%%%%%%%%%%%%%%%%%%%%%%%%%%%%%%%%%%%%%%%%%%%%%%%%%%%%%%%%%%%%%%%%%%%%%%%%%%%%%%%%%%%%%%%%%%%%%%%%%%%
 %%%%%%%%%%%%%%%%%%%%%%%%%%%%%%%%%%%%%%%%%%%%%%%%%%%%%%%%%%%%%%%%%%%%%%%%%%%%%%%%%%%%%%%%%%%%%%%%%%%%%%%%%%%

\begin{frame}\frametitle{Shielded Incident Isotropic Flux}
        \begin{flushleft}
             Shielded system with isotropic incident flux
          	 \begin{itemize}
			      \item Shield $\siga=0.5$, $\sigs=0.5$ for $1 \leq x  \leq 2$ (Vacuum else, $\sigt=10^{-8}$)
			      \item $q=0$, $\psi^{\text{inc}}=1$ for $\mu > 0$; $0$ else
			      \item $\qoi = \int_3^4 \phi \, dx$
			      \item $\sigs$, $\siga$ perturbed in shield
  			 \end{itemize} 
            \end{flushleft}
\begin{figure}[H]
\label{Flux3}
\centering
\begin{subfigure}{0.45\textwidth}
  \centering
  \includegraphics[width=.98\linewidth]{figures2/24phi.png}
  \label{T1:sfig1}
\end{subfigure}
%
\begin{subfigure}{0.45\textwidth}
  \centering
  \includegraphics[width=.98\linewidth]{figures2/24phia.png}
  \label{T1:sfig2}
\end{subfigure}
\end{figure}
\end{frame}

 %%%%%%%%%%%%%%%%%%%%%%%%%%%%%%%%%%%%%%%%%%%%%%%%%%%%%%%%%%%%%%%%%%%%%%%%%%%%%%%%%%%%%%%%%%%%%%%%%%%%%%%%%%%
\begin{frame}\frametitle{Shielded Incident Isotropic Flux}
 \begin{flushleft}

\begin{figure}[H]
\label{Trial3}
\begin{minipage}{.35\textwidth}
\centering
\begin{subfigure}{.45\textheight}
  \centering
  \includegraphics[width=.98\linewidth]{figures2/24incSens.png}
  \label{T1:sfig1}
\end{subfigure}%

\begin{subfigure}{.45\textheight}
  \centering
  \includegraphics[width=.98\linewidth]{figures2/24sigsSensNoavet.png}
\end{subfigure}
\end{minipage}
\begin{minipage}{.60\textwidth}
\begin{subfigure}{1\textwidth}
  \centering
  \includegraphics[width=.98\linewidth]{figures2/24sigaSensNoavet.png}
\end{subfigure}%
%\begin{subfigure}{.5\textheight}
%  \centering
%  \includegraphics[width=.98\linewidth]{figures2/24incsigaSens.png}
%  \label{T3:sfig4}
%\end{subfigure}
\end{minipage}
\end{figure}
\end{flushleft}
\end{frame}

 %%%%%%%%%%%%%%%%%%%%%%%%%%%%%%%%%%%%%%%%%%%%%%%%%%%%%%%%%%%%%%%%%%%%%%%%%%%%%%%%%%%%%%%%%%%%%%%%%%%%%%%%%%%

\begin{frame}\frametitle{Shielded Incident Isotropic Flux}
 \begin{flushleft}
\begin{itemize}
\item For incident flux perturbations: Transport Adjdoint, blended, and aVET agree with transport forward. Slight deviation in others.
\item For source perturbations  VET adjoint shows minor deviations $\approx 1 \%$.
\item About $5 \%$ error between VET adjoint and transport adjoint for $\delta \siga$. $\delta E$ approximation drops this to $0.5\%$.
\item Somewhat stronger sensitivity to $\sigs$ perturbations, but still relatively weak.
\item aVET appears to have strong deviations in this test for cross-section perturbations.
\end{itemize}
\end{flushleft}
\end{frame}

 %%%%%%%%%%%%%%%%%%%%%%%%%%%%%%%%%%%%%%%%%%%%%%%%%%%%%%%%%%%%%%%%%%%%%%%%%%%%%%%%%%%%%%%%%%%%%%%%%%%%%%%%%%%

%%%%%%%%%%%%%%%%%%%%%%%%%%%%%%%%%%%%%%%%%%%%%%%%%%%%%%%%%%%%%%%%%%%%%%%%%%%%%%%%%%%%%%%%%%%%%%%%%%%%%%%%%%%
%%%%%%%%%%%%%%%%%%%%%%%%%%%%%%%%%%%%%%%%%%%%%%%%%%%%%%%%%%%%%%%%%%%%%%%%%%%%%%%%%%%%%%%%%%%%%%%%%%%%%%%%%%%

\begin{frame}\frametitle{Shielded Incident Beam}
        \begin{flushleft}
             Shielded system with grazing incident beam $\mu=0.1834$
          	 \begin{itemize}
			      \item Shield $\siga=0.5$, $\sigs=0.5$ for $1 \leq x  \leq 2$ (Vacuum else, $\sigt=10^{-8}$)
			      \item $q=0$, $\psi^{\text{inc}}=2.7571$ for $\mu=0.1834$ ($N=5$); $0$ else
			      \item $\qoi = \int_3^4 \phi \, dx$
			      \item $\sigs$, $\siga$ perturbed in shield
  			 \end{itemize} 
            \end{flushleft}
\begin{figure}[H]
\label{Flux4}
\centering
\begin{subfigure}{0.45\textwidth}
  \centering
  \includegraphics[width=.98\linewidth]{figures2/25phi.png}
  \label{T1:sfig1}
\end{subfigure}
%
\begin{subfigure}{0.45\textwidth}
  \centering
  \includegraphics[width=.98\linewidth]{figures2/25phia.png}
  \label{T1:sfig2}
\end{subfigure}
\end{figure}
\end{frame}

 %%%%%%%%%%%%%%%%%%%%%%%%%%%%%%%%%%%%%%%%%%%%%%%%%%%%%%%%%%%%%%%%%%%%%%%%%%%%%%%%%%%%%%%%%%%%%%%%%%%%%%%%%%%
\begin{frame}\frametitle{Shielded Incident Beam}
 \begin{flushleft}

\begin{figure}[H]
\label{Trial4}
\begin{minipage}{.35\textwidth}
\centering
\begin{subfigure}{.45\textheight}
  \centering
  \includegraphics[width=.98\linewidth]{figures2/25incSens.png}
  \label{T1:sfig1}
\end{subfigure}%

\begin{subfigure}{.45\textheight}
  \centering
  \includegraphics[width=.98\linewidth]{figures2/25sigsSensNoavet.png}
\end{subfigure}
\end{minipage}
\begin{minipage}{.60\textwidth}
\begin{subfigure}{1\textwidth}
  \centering
  \includegraphics[width=.98\linewidth]{figures2/25sigaSensNoavet.png}
\end{subfigure}%
%\begin{subfigure}{.5\textheight}
%  \centering
%  \includegraphics[width=.98\linewidth]{figures2/25incsigaSens.png}
%\end{subfigure}
\end{minipage}
\end{figure}
\end{flushleft}
\end{frame}

%%%%%%%%%%%%%%%%%%%%%%%%%%%%%%%%%%%%%%%%%%%%%%%%%%%%%%%%%%%%%%%%%%%%%%%%%%%%%%%%%%%%%%%%%%%%%%%%%%%%%%%%%%%

\begin{frame}\frametitle{Shielded Incident Beam}
 \begin{flushleft}
\begin{itemize}
\item VET/blended method was worse for cross section perturbations, particularly scattering, when compared to the isotropic incident.
\item $\delta \Edd$ approximation fared well and nearly reconciled transport adjoint and VET adjoint.
\end{itemize}
\end{flushleft}
\end{frame}

%%%%%%%%%%%%%%%%%%%%%%%%%%%%%%%%%%%%%%%%%%%%%%%%%%%%%%%%%%%%%%%%%%%%%%%%%%%%%%%%%%%%%%%%%%%%%%%%%%%%%%%%%%%%
%
%\begin{frame}\frametitle{Reed Problem}
%        \begin{flushleft}
%             More complex Reed system
%          	 \begin{itemize}
%				\item $x \in [0,2), \quad \siga=50, \, 			\sigs=0, \, q=50, \, q^\dag=0 $\\
%				\item $x \in [2,3), \quad \siga=5, \, 			\sigs=0, \, q=0, \, q^\dag=0$ \\
%				\item $x \in [3,5), \quad \siga \approx 0, \,	\sigs=0, \, q=0, \, q^\dag=0$ \\
%				\item $x \in [5,6), \quad \siga=0.1, \, 		\sigs=0.9, \, q=1, \, q^\dag=0$ \\
%				\item $x \in [6,8], \quad \siga=0.1, \, 		\sigs=0.9, \, q=0, \, \tcr{q^\dag=1} $\\
%  			 \end{itemize} 
%            \end{flushleft}
%\begin{figure}[H]
%\label{Trial5}
%\centering
%\begin{subfigure}{0.45\textwidth}
%  \centering
%  \includegraphics[width=.98\linewidth]{figures2/7phi.png}
%  \label{T1:sfig1}
%\end{subfigure}
%%
%\begin{subfigure}{0.45\textwidth}
%  \centering
%  \includegraphics[width=.98\linewidth]{figures2/7phia.png}
%  \label{T1:sfig2}
%\end{subfigure}
%\end{figure}
%\end{frame}
%%%%%%%%%%%%%%%%%%%%%%%%%%%%%%%%%%%%%%%%%%%%%%%%%%%%%%%%%%%%%%%%%%%%%%%%%%%%%%%%%%%%%%%%%%%%%%%%%%%%%%%%%%%%
%\begin{frame}\frametitle{Reed Problem}
% \begin{flushleft}
%
%\begin{figure}[H]
%\label{Trial5}
%\centering
%\begin{subfigure}{.5\textheight}
%  \centering
%  \includegraphics[width=.98\linewidth]{figures2/7qSens.png}
%  \label{T3:sfig1}
%\end{subfigure}%
%\begin{subfigure}{.5\textheight}
%  \centering
%  \includegraphics[width=.98\linewidth]{figures2/7sigaSens.png}
%  \label{T3:sfig2}
%\end{subfigure}
%%
%\begin{subfigure}{.5\textheight}
%  \centering
%  \includegraphics[width=.98\linewidth]{figures2/7sigsSens.png}
%  \label{T3:sfig3}
%\end{subfigure}%
%\begin{subfigure}{.5\textheight}
%  \centering
%  \includegraphics[width=.98\linewidth]{figures2/7qsigaSens.png}
%  \label{T3:sfig4}
%\end{subfigure}
%\end{figure}
%\end{flushleft}
%\end{frame}
%
%%%%%%%%%%%%%%%%%%%%%%%%%%%%%%%%%%%%%%%%%%%%%%%%%%%%%%%%%%%%%%%%%%%%%%%%%%%%%%%%%%%%%%%%%%%%%%%%%%%%%%%%%%%%
%
%\begin{frame}\frametitle{Reed Problem}
% \begin{flushleft}
%\begin{itemize}
%\item Source perturbations still behave as expected.
%\item Effect of $\delta \sigs$ more pronounced. VET forward, VET adjoint, blended, and aVET show deviations from transport methods. Eddington approximations reduces this. $\delta \Edd =0$ and $\delta \Edd^\dag=0$ approximation seemingly not valid here.
%\end{itemize}
%\end{flushleft}
%\end{frame}
%
%%%%%%%%%%%%%%%%%%%%%%%%%%%%%%%%%%%%%%%%%%%%%%%%%%%%%%%%%%%%%%%%%%%%%%%%%%%%%%%%%%%%%%%%%%%%%%%%%%%%%%%%%%%%
%%%%%%%%%%%%%%%%%%%%%%%%%%%%%%%%%%%%%%%%%%%%%%%%%%%%%%%%%%%%%%%%%%%%%%%%%%%%%%%%%%%%%%%%%%%%%%%%%%%%%%%%%%%%
%
%%%%%%%%%%%%%%%%%%%%%%%%%%%%%%%%%%%%%%%%%%%%%%%%%%%%%%%%%%%%%%%%%%%%%%%%%%%%%%%%%%%%%%%%%%%%%%%%%%%%%%%%%%%

\begin{frame}\frametitle{Reed Problem: QoI comparison}
        \begin{flushleft}
             More complex Reed system\cite{buchan}. For this consider two separate QoI scenarios corresponding to \tcb{$q^\dag_1$} and \tcr{$q^\dag_2$}. We focus scattering perturbations.
          	 \begin{itemize}
				\item $x \in [0,2), \quad \siga=50, \, 			\sigs=0, \, q=50, \, q^\dag=0 $\\
				\item $x \in [2,3), \quad \siga=5, \, 			\sigs=0, \, q=0, \, q^\dag=0$ \\
				\item $x \in [3,5), \quad \siga \approx 0, \,	\sigs=0, \, q=0, \, \tcb{q^\dag_1=0}, \tcr{q^\dag_2=1}$, \\
				\item $x \in [5,6), \quad \siga=0.1, \, 		\sigs=0.9, \, q=1, \, q^\dag=0$ \\
				\item $x \in [6,8], \quad \siga=0.1, \, 		\sigs=0.9, \, q=0, \, \tcb{q^\dag_1=1}, \tcr{q^\dag_2=0} $
  			 \end{itemize} 
            \end{flushleft}
\begin{figure}[H]
\label{Trial5}
\centering
\begin{subfigure}{0.35\textwidth}
  \centering
  \includegraphics[width=.98\linewidth]{figures2/7phiwp.png}
  \label{T1:sfig1}
\end{subfigure}
%
\begin{subfigure}{0.35\textwidth}
  \centering
  \includegraphics[width=.98\linewidth]{figures2/7Ewp.png}
  \label{T1:sfig2}
\end{subfigure}
\end{figure}
\end{frame}
%%%%%%%%%%%%%%%%%%%%%%%%%%%%%%%%%%%%%%%%%%%%%%%%%%%%%%%%%%%%%%%%%%%%%%%%%%%%%%%%%%%%%%%%%%%%%%%%%%%%%%%%%%
\begin{frame}\frametitle{Reed Problem: QoI comparison}
 \begin{flushleft}

\begin{figure}[H]
\label{Trial5}
\centering
\begin{subfigure}{.5\textheight}
  \centering
  \includegraphics[width=.98\linewidth]{figures2/7phia.png}
  \label{T3:sfig1}
\end{subfigure}%
\begin{subfigure}{.5\textheight}
  \centering
  \includegraphics[width=.98\linewidth]{figures2/777phia.png}
  \label{T3:sfig2}
\end{subfigure}
%
\begin{subfigure}{.5\textheight}
  \centering
  \includegraphics[width=.98\linewidth]{figures2/7sigsSens.png}
  \label{T3:sfig3}
\end{subfigure}%
\begin{subfigure}{.5\textheight}
  \centering
  \includegraphics[width=.98\linewidth]{figures2/777sigsSens.png}
  \label{T3:sfig4}
\end{subfigure}
\end{figure}
\end{flushleft}
\end{frame}

%%%%%%%%%%%%%%%%%%%%%%%%%%%%%%%%%%%%%%%%%%%%%%%%%%%%%%%%%%%%%%%%%%%%%%%%%%%%%%%%%%%%%%%%%%%%%%%%%%%%%%%%%%%

%%%%%%%%%%%%%%%%%%%%%%%%%%%%%%%%%%%%%%%%%%%%%%%%%%%%%%%%%%%%%%%%%%%%%%%%%%%%%%%%%%%%%%%%%%%%%%%%%%%%%%%%%%%

\begin{frame}\frametitle{Reed Problem: QoI comparison}
 \begin{flushleft}
\begin{itemize}
\item \qoi in scattering section fared fairly well for Eddington formulations, void \qoi did not
\item In void \qoi, the $\delta \Edd$ approximation method reconciled quite a bit of the error between VET and transport. Indicating there is a large contribution from $ - \bra  \isigt \div \left( \delta \Edd \phi \right),\tcr{ \grad \vefadj_2 }\ket$. Note that in the void $\isigt$ is large 
\item LHS of above product is the same for both scenarios, difference comes from behavior of $\tcr{ \grad \vefadj_2 }$
\end{itemize}
\end{flushleft}
\end{frame}

%%%%%%%%%%%%%%%%%%%%%%%%%%%%%%%%%%%%%%%%%%%%%%%%%%%%%%%%%%%%%%%%%%%%%%%%%%%%%%%%%%%%%%%%%%%%%%%%%%%%%%%%%%%

%%%%%%%%%%%%%%%%%%%%%%%%%%%%%%%%%%%%%%%%%%%%%%%%%%%%%%%%%%%%%%%%%%%%%%%%%%%%%%%%%%%%%%%%%%%%%%%%%%%%%%%%%%%
\begin{frame}\frametitle{Reed Problem: QoI comparison term by term}
\begin{flushleft}
\begin{table}[H]
\centering
  \begin{tabular}{| l | r | r |}
    \hline
    Term  &  Right $q^\dag$ & Void  $q^\dag$\\ \hline
    Unperturbed \qoi																	& 1.53202  & 2.21022 \\ \hline 
     $+ \bra \delta \scalSource, \vefadj  \ket$											& 0 & 0\\ \hline
     $- \bra \delta \siga \phi, \vefadj \ket$ 											& 0 & 0 \\ \hline
     $- \bra \delta \isigt \div \left( \Edd \phi \right) , \grad \vefadj \ket$			& 0.0136565 & 0.00270713 \\ \hline
     $+ \sbra \vefadj, 2 \delta J^{\text{inc}} \sket$										& 0 & 0\\ \hline
     $- \bra  \isigt \div \left( \tcr{\delta \Edd} \phi \right), \grad \vefadj \ket$	& 0.00157771 & -0.0217348\\ \hline
     $- \sbra \vefadj, \phi \tcr{\delta \BEdd} \sket$									& 0.000122458 & 2.3291e-05 \\ \hline
    \end{tabular}
  \caption{Term by Term comparison of VET method and $\delta \Edd$ approximation method for both the Reed QoI's under a $+10\%$ scattering perturbation.}
\end{table}
\end{flushleft}
\end{frame}
%%%%%%%%%%%%%%%%%%%%%%%%%%%%%%%%%%%%%%%%%%%%%%%%%%%%%%%%%%%%%%%%%%%%%%%%%%%%%%%%%%%%%%%%%%%%%%%%%%%%%%%%%%%

%%%%%%%%%%%%%%%%%%%%%%%%%%%%%%%%%%%%%%%%%%%%%%%%%%%%%%%%%%%%%%%%%%%%%%%%%%%%%%%%%%%%%%%%%%%%%%%%%%%%%%%%%%%
\section{Wrap Up}	% define sections here, it is possible to get section slides automatically, but this is not enabled
\subsection{}	% we have to keep these to get the navigation
%%%%%%%%%%%%%%%%%%%%%%%%%%%%%%%%%%%%%%%%%%%%%%%%%%%%%%%%%%%%%%%%%%%%%%%%%%%%%%%%%%%%%%%%%%%%%%%%%%%%%%%%%%%
 %%%%%%%%%%%%%%%%%%%%%%%%%%%%%%%%%%%%%%%%%%%%%%%%%%%%%%%%%%%%%%%%%%%%%%%%%%%%%%%%%%%%%%%%%%%%%%%%%%%%%%%%%%%

\begin{frame}\frametitle{Takeaways}
\begin{flushleft}
\begin{itemize}
\item All 4 VET formulations show ability to approximate $\delta \qoi$ in many of the test cases, response function in void shows issues for cross-section perturbations.

\item Blended method may offer a quick refinement over the straightforward VET, at the expense of an additional SN solve to get transport adjoint.

\item $\delta \Edd$ method appears to work quite well, but requires additional SN solves to parameterize the perturbation space. 

\item The Alternate VET method using $\varphi$ can deal with source perturbations exactly, but cannot be used when there is an incident flux specified.

\end{itemize}
\end{flushleft}
\end{frame}

\begin{frame}\frametitle{Looking Forward}
\begin{flushleft}
\begin{itemize}
\item Extend tests to 2D/3D. Strong sensitivity to $\sigs$ is difficult to test in the 1D system. Inability to scatter ``around'' obstacles.
\item Potentially extend $\delta \Edd$ method to the aVET formulation.
\item Focus on streaming/void problems.
\end{itemize}
\end{flushleft}
\end{frame}

\begin{frame}\frametitle{References}
\begin{flushleft}
\bibliography{QoI_MS} 
\bibliographystyle{ieeetr}
\end{flushleft}
\end{frame}

\begin{frame}\frametitle{Acknowledgments}
\begin{flushleft}
\begin{itemize}
\item Would like to thank my advisor Professor Jean C. Ragusa for invaluable help and heading my defense committee
\item Also extend thanks to my defense committee members Professor Marvin L. Adams and Bojan Popov  
\item Graduate study was supported by a  graduate assistantship from Texas A\&M University and DOE Grant DE-NA0002376 
\end{itemize}
\end{flushleft}
\end{frame}

%%%%%%%%%%%%%%%%%%%%%%%%%%%%%%%%%%%%%%%%%%%%%%%%%%%%%%%%%%%%%%%%%%%%%%%%%%%%%%%%%%%%%%%%%%%%%%%%%%%%%%%%%%%
\begin{frame}\frametitle{Appendix: VET Boundary Condition Derivation}
 \begin{flushleft}
\begin{equation}
\begin{split}
2 J^{\text{inc}}(\vr) 
&\equiv  2 \int_{\vO \cdot \vn <0 }  d \Omega\, | \vO \cdot \vec{n} | \psi^{\text{inc}}(\vr,\vO) \\
&= 2\int_{\vO \cdot \vn <0 } d \Omega\,  | \vO \cdot \vn |  \psi(\vr,\vO)\\
&= \int_{4\pi} d \Omega\,  \left( | \vO \cdot \vn |- \vO\cdot\vn\right)  \psi(\vr,\vO) \\
&= \BEdd(\vr) \phi(\vr) - \vn \cdot \textcolor{red}{\vJ} \\
&= \BEdd \phi + \vn \cdot \frac{1}{\sigt} \div \Edd \phi 
\end{split}
\end{equation}
\end{flushleft}
\end{frame}
%%%%%%%%%%%%%%%%%%%%%%%%%%%%%%%%%%%%%%%%%%%%%%%%%%%%%%%%%%%%%%%%%%%%%%%%%%%%%%%%%%%%%%%%%%%%%%%%%%%%%%%%%%%
 %%%%%%%%%%%%%%%%%%%%%%%%%%%%%%%%%%%%%%%%%%%%%%%%%%%%%%%%%%%%%%%%%%%%%%%%%%%%%%%%%%%%%%%%%%%%%%%%%%%%%%%%%%%

\begin{frame}\frametitle{Homogeneous System, Homogeneous Perturbation}
 \begin{flushleft}
\begin{table}[H]
\label{TableT1}
\centering
  \begin{tabular}{| l | r | r | r | r |}
    \hline
    Method  &  $+10\% q $  & $-10\% \siga $ & $+10\% \sigs $ & $+10\% q,-10\% \siga$ \\ \hline
     SN Fwd 			&0.39998 &0.44419 &5.7577e-05 & 0.88858\\ \hline
     VET Fwd			&0.39998 &0.44428 &2.7131e-05 &0.88868\\ \hline
     SN Adj			&0.39998 &0.39983 &6.6307e-05 &0.79980\\ \hline
     VET Adj 			&0.39998 &0.39988 &2.8534e-05 &0.79986\\ \hline
     Blended 			&0.39998 &0.39988 &2.8534e-05 &0.79986\\ \hline
     VET $\delta \Edd$ 	&0.39998 &0.39983 &5.9537e-05 &0.79981\\ \hline
     VET Alt			&0.39998 &0.39980  &4.986e-05	 &0.79978\\ \hline
    \end{tabular}
  \caption{Table of selected results for the homogeneous system under homogeneous perturbations. Values given are absolute $\delta \Edd$. }
\end{table}
\end{flushleft}
\end{frame}

\begin{frame}\frametitle{Homogeneous System, Non-Homogeneous Perturbation}
 \begin{flushleft}
\begin{table}[H]
\centering
  \begin{tabular}{| l | r | r | r | r |}
    \hline
    Method  &  $+10\% q $  & $-10\% \siga $ & $+10\% \sigs $ & $+10\% q,-10\% \siga$ \\ \hline
     SN Fwd 			&0.36309 &0.39952 &2.9680e-05 & 0.79915\\ \hline
     VET Fwd			&0.35947 &0.39517 &1.5072e-05 &0.79040\\ \hline
     SN Adj  			&0.36309 &0.36301 &3.4051e-05 &0.72610\\ \hline
     VET Adj 			&0.35947 &0.35941 &1.5733e-05 &0.71888\\ \hline
     Blended 			&0.36309 &0.35941 &1.5733e-05 &0.72250\\ \hline
     VET $\delta \Edd$ 	&0.36287 &0.36290 &3.0640e-05 &0.72586\\ \hline
     VET Alt			&0.36309 &0.36299 &2.6234e-05 &0.72609\\ \hline
    \end{tabular}
  \caption{Table of selected results for the homogeneous system under inhomogeneous perturbations. Values given are absolute $\delta \Edd$. }
\end{table}
\end{flushleft}
\end{frame}

\begin{frame}\frametitle{Shielded Isotropic Flux}
 \begin{flushleft}
\begin{table}[H]
\centering
  \begin{tabular}{| l | r | r | r | r |}
    \hline
    Method  &  $+10\% \psi^- $  & $-10\% \siga $ & $+10\% \sigs $ & $+10\% \psi^-,-10\% \siga$ \\ \hline
     SN Fwd 			&0.023401 &0.021079 &-0.0067476 & 0.046588\\ \hline
     VET Fwd			&0.023181 &0.019670 &-0.0066481 &0.044818\\ \hline
     SN Adj  			&0.023401 &0.019975 &-0.0068956 &0.043376\\ \hline
     VET Adj 			&0.023181 &0.018981 &-0.0067751 &0.042162\\ \hline
     Blended 			&0.023401 &0.018981 &-0.0067751 &0.042381\\ \hline
     VET $\delta \Edd$ 	&0.023181 &0.020070 &-0.0065156 &0.043251\\ \hline
     VET Alt			&0.023401 &0.035869 &-0.012563	&0.059270\\ \hline
    \end{tabular}
  \caption{Table of selected results for the shielding system under perturbations. Values given are absolute $\delta \Edd$. }
\end{table}
\end{flushleft}
\end{frame}

%%%%%%%%%%%%%%%%%%%%%%%%%%%%%%%%%%%%%%%%%%%%%%%%%%%%%%%%%%%%%%%%%%%%%%%%%%%%%%%%%%%%%%%%%%%%%%%%%%%%%%%%%%%%%%%%%%%%%%%%%%%%%%%%%%%%%%%%


\end{document}