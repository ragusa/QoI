\documentclass[xcolor={usenames,dvipsnames,svgnames,table}]{beamer}

%Load the myriad packages
\usepackage[orientation=landscape,size=a0,scale=1.4]{beamerposter}

% load macros and prevent authblk from loading
\input{../Presentations/common/macros.tex}
\dontusepackage{authblk}

% load packages, settings and definitions
\input{../Presentations/common/packages.tex}
\usepackage{exscale} 		% needs to stay at the end of package loading to fix integral signs
\input{../Presentations/common/settings.tex}
\input{../Presentations/common/definitions.tex}

\newcommand{\vr}{\vec{r}}
\newcommand{\vOmega}{\vec{\Omega}}
\newcommand{\vJ}{\vec{J}}
\newcommand{\vO}{\vec{\Omega}}
\newcommand{\bra}{\left\langle}
\newcommand{\ket}{\right\rangle}
\newcommand{\sbra}{\left[}
\newcommand{\sket}{\right]}
\newcommand{\vbeta}{\vec{\beta} }
\newcommand{\pdx}{\frac{\partial}{\partial x}}
\newcommand{\pdy}{\frac{\partial}{\partial y}}
\newcommand{\pdz}{\frac{\partial}{\partial z}}
\newcommand{\intrrr}{\int d^3 r \,}
\newcommand{\intrr}{\int d^2 r \,}
\newcommand{\dEdphi}{\partial_\phi E }
\newcommand{\dEdp}{\partial_p E }
\newcommand{\dBdphi}{\partial_\phi B }
\newcommand{\dBdp}{B }
\newcommand{\surf}{\int_{\partial V}}
\newcommand{\bound}{\partial V}
\newcommand{\vn}{\vec{n}}
\newcommand{\Edd}{\mathbb{E}}
\newcommand{\BEdd}{B}
\newcommand{\isigt}{c}
% why \newcommand{\angSource}{q_\Omega}
\newcommand{\angSource}{q}
\newcommand{\scalSource}{q}
\newcommand{\angResp}{q^\dag}
\newcommand{\scalResp}{q^\dag}
\newcommand{\qoi}{{\it QoI}\xspace}

\newcommand{\tcr}[1]{\textcolor{red}{#1}}

%%%%%%%%%%%%%%%%%%%%%%%%%%%%%%%%%%%%%%%%%%%%%%%%%%%%%%%%%%%%%%%

\mode<presentation>{\usetheme{TAMU}}

\title[<short>]{Adjoint-based Sensitivity for Radiation Transport Using an Eddington Tensor Formulation}
\author[<short>]{Ian Halvic and Jean Ragusa}
\institute{Department of Nuclear Engineering, Texas A\&M University, College Station, TX, USA 77843}

\newlength{\columnheight}
\setlength{\columnheight}{70cm}

%%%%%%%%%%%%%%%%%%%%%%%%%%%%%%%%%%%%%%%%%%%%%%%%%%%%%%%%%%%%%%%%%%%%%%%%%%%%%%%%%%%%%%%%%%%%%
\begin{document}
\begin{frame}
	\begin{columns}

	%%%%%%%%%%%%%%%%%%%%%%%%%%%%%%%%%%%%%%%%%%%%%%%%%    %%%%%%%%%%%%%%%%%%%%%%%%%%%%%%%%%%%%%%%%%%%%%%%%%    %%%%%%%%%%%%%%%%%%%%%%%%%%%%%%%%%%%%%%%%%%%%%%%%%
	\begin{column}{.33\textwidth}
		\begin{beamercolorbox}[center,wd=\textwidth]{postercolumn}
			\begin{minipage}[T]{0.95\textwidth} % tweaks the width, makes a new \textwidth
			\parbox[t][\columnheight]{\textwidth}{ % must be some better way to set the the height, width and textwidth simultaneously
			    % Since all columns are the same length, it is all nice and tidy.  You have to get the height empirically

			    %%%%%%%%%%%%%%%%%%%%%%%%%%%%%%%%%%%%%%%%%%%%%%%%%
			    \begin{block}{Motivation}
					Predictive scientific computing requires Quantities of Interest (QoI) and sensitivities to uncertain parameters ($\delta$QoI) 
					\begin{itemize}
					\item \tcr{QoI}$=\langle r, \psi \rangle$ (inner product over whole phase-space);\\ $r=$response function; $\psi$ is the forward angular flux, obtained from solving the neutron transport equation $L \psi = H\psi +q$.
					\item \tcr{$\delta$QoI}$=\langle r, \psi^{\text{pert}}-\psi \rangle$, with $\psi^{\text{pert}}$ the {\bf perturbed} angular flux due to a perturbation in $L$ (e.g., $\sigma_t$), in $H$ (e.g., $\sigma_s$), $q$, or boundary conditions.
					\item \tcr{Issue}: solving the transport equation is cost-prohibitive.
					\end{itemize}
					\vspace{1cm}
{\bf Proposed approach:}
\begin{itemize}
  \item Use \tcr{first-order-accurate adjoint-based formulation} (Sn adjoint transport).\\
	Only requires \tcr{2} full transport solve (one forward, one adjoint) to obtain QoI and $\delta$QoI.
	\item This is still cost-prohibitive for time-dependent transport problems: forward and adjoint solutions in space, energy, and angle must be stored at multiple time moments for subsequent retrieval.
	\item \tcr{This study}: to circumvent the storage issues by employing a Variable Eddington Tensor (\tcr{VET}) approach, which only requires storing scalar fluxes.
	\item \tcr{Analysis carried out}: Whether VET is a reasonable substitute for Sn adjoint transport.
\end{itemize}

			    \end{block}
			    \vfill


			    %%%%%%%%%%%%%%%%%%%%%%%%%%%%%%%%%%%%%%%%%%%%%%%%%
			    \begin{block}{First-order Sensitivity using Sn Transport}
Sn adjoint solve for {\bf nominal} (i.e., unperturbed) parameter values 			    
\begin{equation}
\label{snAdj}
%\begin{split}
%&- \vO_d \cdot \grad \psi^\dag_d + \sigt \psi^\dag_d = \frac{\sigs}{4 \pi} \phi^\dag + \angResp_d \\
%&\psi^\dag(\vr) = \psi^{\dag \text{out}}(\vr)=0 \quad \vr \in \partial V^{+} = \{  \vr \in \bound , \quad \vO \cdot \vec{n} > 0 \}
%\end{split}
L^\dag \psi^\dag = H^\dag \psi^\dag + \tcr{r}
\end{equation}
First-order perturbation sensitivity to be calculated as follows:
\begin{equation}
\label{snSens}
\delta QoI = \bra \delta \scalSource - \delta \sigt \psi + \frac{\delta\sigs}{4 \pi} \phi , \psi^\dag  \ket - \sbra \delta \psi^{\text{inc}}, \psi^\dag \sket_- \,
\end{equation}
\tcr{Only} requires unperturbed forward and adjoint angular fluxes.
			    \end{block}
			    \vfill
			    %%%%%%%%%%%%%%%%%%%%%%%%%%%%%%%%%%%%%%%%%%%%%%%%%




			    %%%%%%%%%%%%%%%%%%%%%%%%%%%%%%%%%%%%%%%%%%%%%%%%%
\begin{block}{Eddington Formulation}
In an effort to remove the angular dependence of the system completely, an Eddington Tensor $\Edd$ is introduced:
%, along with a Boundary Eddington Factor $\BEdd$. These two values are the primary tools of the VET formulation.
\begin{equation}
\label{EddDef}
\Edd(\vr)=\frac{\int d\Omega \vO \vO \psi(\vr,\vO)}{\phi(\vr)}
\end{equation}
%\begin{equation}
%\BEdd(\vr) = \frac{\int_{4 \pi} d\Omega \, | \vO \cdot \vn | \psi}{\int_{4\pi} d\Omega \, \psi} \quad , \vr \in \bound 
%\end{equation}
With the \tcr{nominal} (i.e., unperturbed) Eddington values in hand, the Sn transport equation can be converted to the form shown below. 
\begin{equation}
\label{VEFForm}
- \div \left( \frac{1}{\sigt}\div \Edd \phi \right) + \siga \phi = \scalSource \quad + \text{B.C.}
\end{equation}

\end{block}
\vfill

			    %%%%%%%%%%%%%%%%%%%%%%%%%%%%%%%%%%%%%%%%%%%%%%%%%
			}
			\end{minipage}
		\end{beamercolorbox}
	\end{column}

	%%%%%%%%%%%%%%%%%%%%%%%%%%%%%%%%%%%%%%%%%%%%%%%%%    %%%%%%%%%%%%%%%%%%%%%%%%%%%%%%%%%%%%%%%%%%%%%%%%%
	%%%%%%%%%%%%%%%%%%%%%%%%%%%%%%%%%%%%%%%%%%%%%%%%%
	\begin{column}{.33\textwidth}
		\begin{beamercolorbox}[center,wd=\textwidth]{postercolumn}
			\begin{minipage}[T]{0.95\textwidth} % tweaks the width, makes a new \textwidth
			\parbox[t][\columnheight]{\textwidth}{ % must be some better way to set the the height, width and textwidth simultaneously
			         % Since all columns are the same length, it is all nice and tidy.  You have to get the height empirically


			    %%%%%%%%%%%%%%%%%%%%%%%%%%%%%%%%%%%%%%%%%%%%%%%%%
			    \begin{block}{VET Sensitivity}
Since VET formulation generates a new forward equation to describe the system, a new adjoint corresponding to Eq.~\eqref{VEFForm} can be obtained. 
\begin{equation}
\label{adjForm}
%\begin{split}
%&
- \Edd : \left( \grad \left( \frac{1}{\sigt}\grad \varphi^\dag \right) \right) + \siga \varphi^\dag = \scalResp %\\
%&2J^{\dag,\text{out}} = B \phi^\dag + 
%\Edd \cdot \frac{1}{\sigma_{t} } \vec{\nabla} \phi^\dag   \quad \vr \in \bound
%\end{split}
\end{equation}
The adjoint flux in this formulation is denoted by $\varphi^\dag$.\\
%{\bf Note}:  This adjoint scalar flux is not the adjoint one would obtain from the Sn adjoint equation (the latter is denoted by $\phi^\dag$).

\vspace{1cm}
If the assumption is made that the Eddington factor remains unperturbed, a sensitivity inner product can be derived.


\begin{equation}
\label{SensIP1}
\boxed{
\begin{split}
\delta \qoi =&  \bra \delta q , \varphi^\dag \ket - \bra \delta \isigt \div (\Edd \phi), \grad \varphi^\dag \ket + \bra \delta \sigma_a \phi, \varphi^\dag \ket + \sbra \varphi^\dag, 2 \delta J^{\text{inc}} \sket \\
\end{split}
}
\end{equation}

The Sn-adjoint method is exact for perturbations in $q$ and $\psi^\text{inc}$ and does not require storing $\psi$ through the volume. This leads to a ``blended'' sensitivity expression which also uses Sn-adjoint solution $\phi^\dag$.

\begin{equation}
\label{SensIP2}
\boxed{
\delta \qoi =  \bra \delta q , \phi^\dag \ket - \bra \delta \isigt \div (\Edd \phi), \grad \varphi^\dag \ket + \bra \delta \sigma_a \phi, \varphi^\dag \ket - \sbra \delta \psi^{\text{inc}}, \psi^\dag \sket_- 
}
\end{equation}


			    \end{block}
			    \vfill
			    %%%%%%%%%%%%%%%%%%%%%%%%%%%%%%%%%%%%%%%%%%%%%%%%%
			    \begin{block}{Eddington estimation}
			    When many perturbation scenarios must be computed, can attempt to predict the Eddington perturbation $\delta \Edd$, at the cost of additional Sn solves. 
\begin{equation}
	\delta \Edd \approx \frac{\partial \Edd}{\partial p} \delta p \approx \frac{\Edd(p_1) - \Edd(p_0)}{p_1 - p_0} \delta p	     
\end{equation}

The method requires one additional forward Sn solve per perturbed variable $p$ to obtain $\Edd(p_1)$.	
The approximation was applied to a 5 region test system (Reed problem).

\begin{figure}[H]
\label{Case113syst}
\centering
\begin{subfigure}{.33\textwidth}
  \centering
  \includegraphics[width=.98\linewidth]{posterfigures/7phi.png}
  \caption{$\phi$ and $\phi_p$ for $+10\%$ $\sigs$}
  \label{fig:sfig1}
\end{subfigure}%
\begin{subfigure}{.33\textwidth}
  \centering
  \includegraphics[width=.98\linewidth]{posterfigures/7E.png}
    \caption{$\Edd$ and perturbed $\Edd$}
  \label{fig:sfig2}
\end{subfigure}%
\begin{subfigure}{.33\textwidth}
  \centering
  \includegraphics[width=.98\linewidth]{posterfigures/7deltaE.png}
    \caption{Estimation of $\delta \Edd$}
  \label{fig:sfig5}
\end{subfigure}%
\label{fig:fig}
\end{figure}		    

The full sensitivity expression using the estimated $\delta \Edd$ is then
\begin{equation}
\label{SensIP3}
\boxed{
\begin{split}
\delta \qoi =&  \bra \delta q , \varphi^\dag \ket - \bra \delta \isigt \div (\Edd \phi), \grad \varphi^\dag \ket + \bra \delta \sigma_a \phi, \varphi^\dag \ket + \sbra \varphi^\dag, 2 \delta J^{\text{inc}} \sket \\
& \tcr{ - \bra  \isigt \div \left( \delta \Edd \phi \right), \grad \varphi^\dag \ket
- \sbra \varphi^\dag, \phi \delta \BEdd \sket}
\end{split}
}
\end{equation}	 
		    	\vfill
			    \end{block}
			    \vfill
			    %%%%%%%%%%%%%%%%%%%%%%%%%%%%%%%%%%%%%%%%%%%%%%%%%
			}
			\end{minipage}
		\end{beamercolorbox}
	\end{column}

	%%%%%%%%%%%%%%%%%%%%%%%%%%%%%%%%%%%%%%%%%%%%%%%%%    %%%%%%%%%%%%%%%%%%%%%%%%%%%%%%%%%%%%%%%%%%%%%%%%%
	%%%%%%%%%%%%%%%%%%%%%%%%%%%%%%%%%%%%%%%%%%%%%%%%%
	\begin{column}{.33\textwidth}
		\begin{beamercolorbox}[center,wd=\textwidth]{postercolumn}
			\begin{minipage}[T]{0.95\textwidth} % tweaks the width, makes a new \textwidth
			\parbox[t][\columnheight]{\textwidth}{ % must be some better way to set the the height, width and textwidth simultaneously
			         % Since all columns are the same length, it is all nice and tidy.  You have to get the height empirically

			    %%%%%%%%%%%%%%%%%%%%%%%%%%%%%%%%%%%%%%%%%%%%%%%%%
			    \begin{block}{Eddington estimation continued}		    
			    
Sensitivity plots for the Reed Problem due to perturbations in the source and $\siga$, then for $\sigs$. Two \qoi 's tested: $\qoi_1$ in scatter and $\qoi_2$ in void.
% The Sensitivity inner products laid out in Eqs.~\eqref{SensIP1}, ~\eqref{SensIP2}, ~\eqref{SensIP3} are tested against two individual forward solves.
 
\begin{figure}[H]
\label{Case113Sens}
\centering
\begin{subfigure}{.4\textwidth}
  \centering
  \includegraphics[width=.98\linewidth]{posterfigures/774phia.png}  
  \caption{$\phi^\dag$ for scattering $\qoi_1$}
  \label{fig:sfig1}
\end{subfigure}%
\begin{subfigure}{.4\textwidth}
  \centering
  \includegraphics[width=.98\linewidth]{posterfigures/772phia.png}
  \caption{$\phi^\dag$ for void $\qoi_2$}
  \label{fig:sfig2}
\end{subfigure}

\begin{subfigure}{.4\textwidth}
  \centering
  \includegraphics[width=.98\linewidth]{posterfigures/774qsigaSens.png} 
  \caption{source/$\siga$ sensitivity for $\qoi_1$}
  \label{fig:sfig5}
\end{subfigure}%
\begin{subfigure}{.4\textwidth}
  \centering
  \includegraphics[width=.98\linewidth]{posterfigures/772qsigaSens.png}
  \caption{source/$\siga$ sensitivity for $\qoi_2$}
  \label{fig:sfig4}
\end{subfigure}

\begin{subfigure}{.4\textwidth}
  \centering
  \includegraphics[width=.98\linewidth]{posterfigures/774sigsSens.png} 
  \caption{$\sigs$ sensitivity for $\qoi_1$}
  \label{fig:sfig5}
\end{subfigure}%
\begin{subfigure}{.4\textwidth}
  \centering
  \includegraphics[width=.98\linewidth]{posterfigures/772sigsSens.png}
  \caption{$\sigs$ sensitivity for $\qoi_2$}
  \label{fig:sfig4}
\end{subfigure}
\label{fig:fig}
\end{figure}			    
			    
			    
			    \end{block}
			    \vfill
			    %%%%%%%%%%%%%%%%%%%%%%%%%%%%%%%%%%%%%%%%%%%%%%%%%
			    \begin{block}{Conclusion}			    

$\bullet$ This VET adjoint can be used to compute response sensitivity without storing angular flux if assumptions are made about perturbation of the Eddington, though scattering and void regions can cause large error between transport and VET adjoint


$\bullet$ At the additional cost of solving and storing the transport adjoint scalar flux, exact sensitivity of source perturbations can be obtained, while using VET adjoint for cross-section perturbations

$\bullet$ In situations where numerous perturbation cases must be considered, sampling Eddington values using additional forward solves can increase the VET adjoint and possibly deal with void regions

$\bullet$ Tests should be extended to higher dimensions to test more complex scatting scenarios 
			    \end{block}
			    \vfill
			}
			\end{minipage}
		\end{beamercolorbox}
	\end{column}

	\end{columns}
\end{frame}
\end{document}

