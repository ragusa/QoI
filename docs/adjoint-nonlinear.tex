\documentclass[11pt]{article}
% PACKAGES
\usepackage{graphicx}
\usepackage{tabls}
\usepackage{afterpage}
\usepackage{amsmath}
\usepackage{amsfonts}
\usepackage{amssymb}
\usepackage{amstext}
\usepackage{amsbsy}
\usepackage{epsfig}
%\usepackage{cites}
\usepackage{epsf}
\usepackage{float} 
\usepackage{color} 

\usepackage{array}
\usepackage[section]{placeins} % force � mettre l'image o� on veut
\usepackage{lscape} %utilisation du mode paysage
\usepackage{xspace}

%\usepackage[pdftex,bookmarks=true]{hyperref}
%\usepackage{hyperref}
\usepackage{url}
\usepackage{verbatim}
%\usepackage[all]{hypcap}

\usepackage[labelsep=quad]{caption} % needed by the breakalgo environment

\usepackage{ifthen}
\usepackage{subfig}

\usepackage{algorithmic}
\usepackage{algorithm}
\usepackage{listings}
%\usepackage[noprefix]{nomencl}  % for nomenclature

%-----------------------------------------------------------
% NEW  DEFINITIONS
% margin par
\newcommand{\mt}[1]{\marginpar{ {\footnotesize #1} }}
% vector shortcuts
\newcommand{\vo}{\vec{\Omega}}
\newcommand{\vr}{\vec{r}}
\newcommand{\vn}{\vec{n}}
\newcommand{\vnk}{\vec{\mathbf{n}}}

% More Quick Commands
% 
\newcommand{\bi}{\begin{itemize}}
\newcommand{\ei}{\end{itemize}}
\newcommand{\ben}{\begin{enumerate}}
\newcommand{\een}{\end{enumerate}}

\renewcommand{\div}{\vec{\nabla}\! \cdot \!}
\newcommand{\grad}{\vec{\nabla}}
\newcommand{\oa}[1]{\fn{\frac{1}{3}\hat{\Omega}\!\cdot\!\overrightarrow{A_{#1}}}}

% common reference commands
\newcommand{\eqt}[1]{Eq.~(\ref{#1})}                     % equation
\newcommand{\fig}[1]{Fig.~\ref{#1}}                      % figure
\newcommand{\tbl}[1]{Table~\ref{#1}}                     % table

%
% Equation beginnings and endings
%
\newcommand{\bea}{\begin{eqnarray}}
\newcommand{\eea}{\end{eqnarray}}

\newcommand{\be}{\begin{equation}}
\newcommand{\ee}{\end{equation}}
\newcommand{\beas}{\begin{eqnarray*}}
\newcommand{\eeas}{\end{eqnarray*}}
\newcommand{\bdm}{\begin{displaymath}}
\newcommand{\edm}{\end{displaymath}}
%
\newcommand{\vj}{\vec{J}}
\newcommand{\sa}[1]{\sigma_{a #1}}
\newcommand{\vl}{\vec{\lambda}}
\newcommand{\vdj}{\delta \vec{J}}
\newcommand{\dphi}{\delta \Phi}
\newcommand{\lmax}{\ensuremath{L_{\textit{max}}}\xspace}
\newcommand{\pmax}{\ensuremath{p_{\textit{max}}}\xspace}
\newcommand{\sddx}{\frac{d}{dx}}
\newcommand{\sddt}{\frac{d}{dt}}
\newcommand{\der}[2]{\frac{\partial #1}{\partial #2}}
\newcommand{\vF}{\vec{F}}

\renewcommand{\O}{\mathcal{O}}
\newcommand{\mc}[1]{\mathcal{#1}}
\newcommand{\us}{{u^\ast}}

\newcommand{\ti}{ {t^{\text{init}}} }
\newcommand{\te}{ {t^{\text{end }}} }

%-----------------------------------------------------------
\addtolength{\hoffset}{-2.0cm}
\addtolength{\textwidth}{4cm}
\addtolength{\textheight}{4.0cm}
\addtolength{\voffset}{-1.8cm}
\addtolength{\headsep}{-0.3cm}

\setlength{\parindent}{0pt}
\setlength{\parskip}{1.8ex plus 0.5ex minus 0.2ex}

\linespread{1.1}
%-----------------------------------------------------------
\begin{document}
%-----------------------------------------------------------
%%%%%%%%%%%%%%%%%%%%%%%%%%%%%%%%%%%%%%%%%%%%%%%%%%%%%%%%%%%%%%%%%%%%%%%%%%%%%%%%%%%%%%%%
%%%%%%%%%%%%%%%%%%%%%%%%%%%%%%%%%%%%%%%%%%%%%%%%%%%%%%%%%%%%%%%%%%%%%%%%%%%%%%%%%%%%%%%%

%%%%%%%%%%%%%%%%%%%%%%%%%%%%%%%%%%%%%%%%%%%%%%%%%%%%%%%%%%%%%%%%%%%%%%%%%%%%%%%%%%%%%%%%%%%%%%%%%%%%
%%%%%%%%%%%%%%%%%%%%%%%%%%%%%%%%%%%%%%%%%%%%%%%%%%%%%%%%%%%%%%%%%%%%%%%%%%%%%%%%%%%%%%%%%%%%%%%%%%%%
\begin{center}
{\huge Adjoint-based sensitivity coefficients for nonlinear problems}\\
{\Large Jean C. Ragusa, TAMU}\\
{\Large \textcolor{red}{No distribution please}}\\
\end{center}
%%%%%%%%%%%%%%%%%%%%%%%%%%%%%%%%%%%%%%%%%%%%%%%%%%%%%%%%%%%%%%%%%%%%%%%%%%%%%%%%%%%%%%%%%%%%%%%%%%%%

\tableofcontents
\pagebreak

%%%%%%%%%%%%%%%%%%%%%%%%%%%%%%%%%%%%%%%%%%%%%%%%%%%%%%%%%%%%%%%%%%%%%%%%%%%%%%%%%%%%%%%%
%%%%%%%%%%%%%%%%%%%%%%%%%%%%%%%%%%%%%%%%%%%%%%%%%%%%%%%%%%%%%%%%%%%%%%%%%%%%%%%%%%%%%%%%
\section{Introduction}
%%%%%%%%%%%%%%%%%%%%%%%%%%%%%%%%%%%%%%%%%%%%%%%%%%%%%%%%%%%%%%%%%%%%%%%%%%%%%%%%%%%%%%%%
%%%%%%%%%%%%%%%%%%%%%%%%%%%%%%%%%%%%%%%%%%%%%%%%%%%%%%%%%%%%%%%%%%%%%%%%%%%%%%%%%%%%%%%%

%%%%%%%%%%%%%%%%%%%%%%%%%%%%%%%%%%%%%%%%%%%%%%%%%%%%%%%%%%%%%%%%%%%%%%%%%%%%%%%%%%%%%%%%
\subsection{Governing PDE}
%%%%%%%%%%%%%%%%%%%%%%%%%%%%%%%%%%%%%%%%%%%%%%%%%%%%%%%%%%%%%%%%%%%%%%%%%%%%%%%%%%%%%%%%

%----------------------
\subsubsection{General Case}
%----------------------

Consider an operator $\O$ dependent on a set of parameters $p$. The operator is assumed
to be nonlinear, so it also depend on the solution $u$. Let $X$ be the domain and $\partial X=\Gamma$
its boundary. We denote by the subscript $0$ the unperturbed state. $u_0$ is solution of the following
sets of equations:
\begin{align}
\O(u_0,p_0) u_0 &= Q(p_0), \qquad x\in X \label{eq:gov} \\
\mc{C}(u_0,p_0) u_0 &=u_\Gamma(p_0), \qquad x \in \partial X=\Gamma \label{eq:BCgov}
\end{align}

When dealing with transient problems, I still single out the time derivative as follows:
\be
\O(u,p) = \mc{A}(u,p)\partial_t + \mc{B}(u,p)
\ee

Later, we will consider a perturbation of the form
\be
p = p_0 + \epsilon p_1
\ee
that leads to a perturbation in $u$
\be
u=u_0 + \epsilon u_1 + \ldots
\ee

%----------------------
\subsubsection{Linear diffusion}
%----------------------

For simplicity, consider the linear 1-g neutron diffusion (in a 1-D slab for the numerical results)

\be
-\div D(p) \grad u  + \Sigma_a(p) u = Q(p) \qquad \text{for } a<x<b
\ee
with vacuum BC
\be
u(a)=u(b)=0
\ee

A transient version of this problem is
\be
\frac{1}{v(p)} \partial_t u -\div D(p) \grad u  + \Sigma_a(p) u = Q(p) \qquad \text{for } a \le x \le b
\ee
with given initial conditions $u^{\text{init}}(x,0)$.

%----------------------
\subsubsection{Nonlinear heat conduction}
%----------------------

For simplicity, consider the nonlinear heat conduction (in a 1-D slab for the numerical results)
\be
-\div k(u,p)  \grad u = q \qquad \text{for } a<x<b
\label{eq:heat_conduction}
\ee
with boundary conditions
\begin{align}
u(a) &= u^D \\
\varphi_u(b) &=  -k \grad u \cdot \vec{n} |_{x=b}= -k \partial_n u |_{x=b} = \phi^N
\end{align}
(a Dirichlet BC at $a$ and a Neumann BC at $b$).

Let the operator $\O$ be such that
\be
\O(u,p) u = -\div k(u)  \grad u
\ee
that is, $\O=-\div k(u)  \grad \cdot$, and $p$ be an uncertain parameter (here, the conductivity 
$\longrightarrow k(u,p)$). To give an idea, we can take as an example 
\be
k(u,\alpha,\,\beta,\gamma) = \frac{\alpha}{\beta + \gamma u}
\ee
where the coefficients $\alpha,\,\beta,$ and $\gamma$ are the ``uncertain'' parameters.


A transient version of this problem is
\be
\rho C_p(p) \partial_t u -\div k(u,p)  \grad u = q \qquad \text{for } a \le x \le b
\label{eq:heat_conduction_transient}
\ee
with boundary conditions
\begin{align}
u(a,t) &= u^D(t) \\
\varphi_u(b,t) &=  -k \grad u \cdot \vec{n} |_{x=b}= -k \partial_n u |_{x=b} = \phi^N(t)
\end{align}


and a given initial condition $u^{\text{init}}(x,0)$.

%----------------------
\subsubsection{Coupled nonlinear problem: radiation diffusion and material temperature}
%----------------------

\begin{align}
\frac{1}{c} \partial_t E - \div D_E(T,E) \grad E + \sigma_a(T)(E-acT^4) &= 0 \\
C_p \partial_t T -\div D_T(T) \grad T +\sigma_a(T)(acT^4-E)            &= 0
\end{align}

supplemented with appropriate boundary and initial conditions.

I have omitted to explicitly mention at possible dependence on a generic parameter $p$.

%%%%%%%%%%%%%%%%%%%%%%%%%%%%%%%%%%%%%%%%%%%%%%%%%%%%%%%%%%%%%%%%%%%%%%%%%%%%%%%%%%%%%%%%
\subsection{Quantity of interest: QoI}
%%%%%%%%%%%%%%%%%%%%%%%%%%%%%%%%%%%%%%%%%%%%%%%%%%%%%%%%%%%%%%%%%%%%%%%%%%%%%%%%%%%%%%%%

%----------------------
\subsubsection{Steady-state QoI}
%----------------------

We are interested in the following quantity of interest
\be
QoI[u,p] = (\sigma, u) = \int_X  dx \, \sigma(x)u(x)
\ee

Examples of $\sigma(x)$:
\begin{table*}[!h]
	
	\begin{center}
		\begin{tabular}{|l||c|}
		\hline
	local value at $x_0$ & $\delta(x-x_0)$ \\ \hline
	derivative value at $x_0$ & $\delta(x-x_0) \frac{\partial}{\partial x}$ \\ \hline
	average over region of interest $\mathcal{R}$ & $\frac{1}{\tt{meas}(\mathcal{R})} \chi(x)$ with $\chi(x)=1$ if $x\in\mathcal{R}$ and 0 otherwise \\ \hline
		\end{tabular}
	\end{center}
	\caption{Examples of response functions}
	\label{tab:ExamplesOfResponseFunctions}
\end{table*}

%----------------------
\subsubsection{Transient QoI}
%----------------------

For transient problems, this becomes
\be
QoI[u,p] = (\sigma, u) = \int_\ti ^\te dt \int_X  dx \, \sigma(x,t)u(x,t)
\ee

The time dependence of $\sigma(x,t)$ can be 
\begin{itemize}
\item $\frac{1}{t^{\text{end}}-t^{\text{init}}}$: i.e., time average over the entire simulation time 
\item $\frac{1}{t_2-t_1}$: i.e., time average over a subset over the simulation period
\item $\delta(t-t_0)$: i.e., an instantaneous quantity at time $t_0$
\item \ldots
\end{itemize}

%%%%%%%%%%%%%%%%%%%%%%%%%%%%%%%%%%%%%%%%%%%%%%%%%%%%%%%%%%%%%%%%%%%%%%%%%%%%%%%%%%%%%%%%
\subsection{Adjoint operator}
%%%%%%%%%%%%%%%%%%%%%%%%%%%%%%%%%%%%%%%%%%%%%%%%%%%%%%%%%%%%%%%%%%%%%%%%%%%%%%%%%%%%%%%%

%----------------------
\subsubsection{General Case}
%----------------------

Multiply \eqt{eq:gov} by $\us(x,t)$ and integrate over space (and time if transient)
\be
(\us,\O(u,p)u) = (\O^\ast(u,p)\us,u) + W(\us,u)
\ee
where the adjoint (linear) problem is {\bf defined} as
\be
\O^\ast(u,p)\us = \sigma
\ee
%
$W(\us,u)$ is the result of integration by parts in space (and time, if transient). 

%----------------------
\paragraph{Temporal portion of the operator:}
%----------------------

\be
(\us,\mc{A} \partial_t u ) 
=  \int_\ti^\te dt \int_X dx \, \us \mc{A} \partial_t u
= -\int_\ti^\te dt \int_X dx \, u  \partial_t (\mc{A}\us) + 
\left. \int_X dx \, u \mc{A} \us \right|_\ti^\te 
\ee

Note the time reversal.
If $\mc{A}$ is independent of time, then we also have 
\be
(\us,\mc{A} \partial_t u )  
= -(u , \mc{A} \partial_t \us ) + 
\left. \int_X dx \, u \mc{A} \us \right|_\ti^\te 
\ee

%----------------------
\subsubsection{Linear diffusion}
%----------------------
Do it (see below for a worked out example)

%----------------------
\subsubsection{Nonlinear heat conduction}
%----------------------


Multiply \eqt{eq:gov} by $\us(x)$ and integrate by parts (twice)
\be
(\us,\O(u,p)u) = (\O^\ast(u,p)\us,u) + W(\us,u)
\ee
where the adjoint (linear) problem is
\be
\O^\ast(u,p)\us = -\div k(u)  \grad \us = \sigma
\ee
%
$W(\us,u)$ is
\be
W(\us,u)=u(b)\varphi_\us(b)-u(a)\varphi_\us(a) -\left( \us(b)\varphi_u(b) -\us(a)\varphi_u(a)\right)
\ee
%where we recall that
%\begin{align}
%\varphi_u   &= - k \grad u   \cdot \vec{n} \\
%\varphi_\us &= + k \grad \us \cdot \vec{n} 
%\end{align}
%[{\bf caveat:} change of sign]. 
I prefer a more concise notation. Introduction
\be
\mc{L} = n \cdot D\grad
\ee where $n$ is the {\bf outward} unit normal vector on the boundary, 
we have
\be
(\us, \O(u,p)u) + \langle \us, \mc{L} u \rangle = (u, \O^\ast(u,p)\us) + \langle u, \mc{L} \us \rangle 
\ee
with $\langle \  , \, \rangle = \int_{\Gamma}\ $.

Thus,
\be
QoI[u,p] = (\sigma, u) = (\O^\ast(u,p)\us, u) = (\us,\O(u,p)u) - W(\us,u) = (\us,Q(p)) - W(\us,u)
\ee
{\bf Note:} the QoI can be evaluated using $\us$. The fact that $u$ and $\us$ appear in $W$ is ok 
because $W$ only needs boundary values, which should be known.

%----------------------
\subsubsection{Coupled nonlinear problem: radiation diffusion and material temperature}
%----------------------

% A ``tad'' more complex. Take a stab at it.

The operators of the forward equation are
\be
\mc{A} = 
\begin{pmatrix}
\frac{1}{c} &  0 \\
0           &   C_p
\end{pmatrix}
\qquad
\mc{B} = 
\begin{pmatrix}
\mc{B}_E      &  \mc{B}_{ET} \\
\mc{B}_{TE}   &  \mc{B}_{T} 
\end{pmatrix}
\ee
with 
\begin{align}
\mc{B}_{E}  &= -\div D_E \grad + \sigma_a \\
\mc{B}_{ET} &= -\sigma_a acT^3 \\
\mc{B}_{T}  &= -\div D_T \grad + \sigma_a acT^3 \\
\mc{B}_{TE} &= \sigma_a  
\end{align}

The duality expressions are:
\begin{itemize}

\item
\be
(E^\ast, \mc{B}_{E}E) + \langle E^\ast, \mc{L}_E E \rangle = (E, \mc{B}_{E}^\ast E^\ast) + \langle E, \mc{L}_E E^\ast \rangle 
\ee
with $\mc{B}_{E}^\ast = \mc{B}_{E}$ and $\mc{L}_E = n \cdot D_E \grad$.

\item
\be
(T^\ast, \mc{B}_{T}T) + \langle T^\ast, \mc{L}_T T \rangle = (T, \mc{B}_{T}^\ast T^\ast) + \langle T, \mc{L}_T T^\ast \rangle 
\ee
with $\mc{B}_{T}^\ast = \mc{B}_{T}$ and $\mc{L}_T = n \cdot D_T \grad$.

\item
\be
(E^\ast, \mc{B}_{ET}T)  = (T, \mc{B}_{TE}^\ast E^\ast) 
\ee
with $\mc{B}_{TE}^\ast = \mc{B}_{ET}$

\item
\be
(T^\ast, \mc{B}_{TE}E)  = (E, \mc{B}_{ET}^\ast T^\ast) 
\ee
with $\mc{B}_{ET}^\ast = \mc{B}_{TE}$

\end{itemize}

%%%%%%%%%%%%%%%%%%%%%%%%%%%%%%%%%%%%%%%%%%%%%%%%%%%%%%%%%%%%%%%%%%%%%%%%%%%%%%%%%%%%%%%%
%%%%%%%%%%%%%%%%%%%%%%%%%%%%%%%%%%%%%%%%%%%%%%%%%%%%%%%%%%%%%%%%%%%%%%%%%%%%%%%%%%%%%%%%
\section{Sensitivity coefficients}
%%%%%%%%%%%%%%%%%%%%%%%%%%%%%%%%%%%%%%%%%%%%%%%%%%%%%%%%%%%%%%%%%%%%%%%%%%%%%%%%%%%%%%%%
%%%%%%%%%%%%%%%%%%%%%%%%%%%%%%%%%%%%%%%%%%%%%%%%%%%%%%%%%%%%%%%%%%%%%%%%%%%%%%%%%%%%%%%%
Let the ``unperturbed'' state be
\begin{align}
\O(u_0,p_0) u_0 &= Q(p_0), \qquad x\in X \\
\mc{C}(u_0,p_0) u_0 &=u_\Gamma(p_0), \qquad x \in \partial X=\Gamma
\end{align}

We consider a perturbation
\be
p = p_0 + \epsilon p_1
\ee
that leads to a perturbation in $u$
\be
u=u_0 + \epsilon u_1 + O(\epsilon ^2)
\ee

%%%%%%%%%%%%%%%%%%%%%%%%%%%%%%%%%%%%%%%%%%%%%%%%%%%%%%%%%%%%%%%%%%%%%%%%%%%%%%%%%%%%%%%%
\subsection{Direct approach}
%%%%%%%%%%%%%%%%%%%%%%%%%%%%%%%%%%%%%%%%%%%%%%%%%%%%%%%%%%%%%%%%%%%%%%%%%%%%%%%%%%%%%%%%

A solution consists in directly solving
the perturbed system
\begin{align}
\O(u,p) u &= Q(p), \qquad x\in X \\
\mc{C}(u,p) u &=u_\Gamma(p), \qquad x \in \partial X=\Gamma
\end{align}

and then an expression for the sensitivity coefficient is:
\be
S_p  = \lim_{\epsilon \rightarrow 0^+} \frac{ (\sigma, u_0+\epsilon u_1) - (\sigma, u_0)}{\epsilon} = (\sigma,u_1)
\ee
(assuming $\sigma$ independent of $p$). 

If $\sigma$ is {\bf not} independent of $p$, we have
\be
S_p  = \lim_{\epsilon \rightarrow 0^+} \frac{ (\sigma(p_0) + \epsilon \partial _p \sigma |_{p_0} p_1, u_0+\epsilon u_1) - (\sigma(p_0), u_0)}{\epsilon} = (\sigma,u_1) +  (\partial _p \sigma p_1, u_0)
\ee

%%%%%%%%%%%%%%%%%%%%%%%%%%%%%%%%%%%%%%%%%%%%%%%%%%%%%%%%%%%%%%%%%%%%%%%%%%%%%%%%%%%%%%%%
\subsection{Adjoint magic}
%%%%%%%%%%%%%%%%%%%%%%%%%%%%%%%%%%%%%%%%%%%%%%%%%%%%%%%%%%%%%%%%%%%%%%%%%%%%%%%%%%%%%%%%


From the sensitivity coefficient expression, we would like to have an adjoint operator such that
\be
\O^\ast \us = \sigma
\ee
so that
\be
(\sigma,u_1) = (\O_1^\ast \us , u_1 ) =  ( \us , \O_1 u_1 ) - W_1(\us,u_1)   =  ( \us ,rhs_1 ) - W_1(\us,u_1)  
\ee
where $W_1$ will denote the concomitant associated with $\O_1$.\\

What is that special equation for the forward perturbation $u_1$ whose adjoint operator we need?
Note $u_1$ satisfies $\O_1 u_1=rhs_1$.

%%%%%%%%%%%%%%%%%%%%%%%%%%%%%%%%%%%%%%%%%%%%%%%%%%%%%%%%%%%%%%%%%%%%%%%%%%%%%%%%%%%%%%%%
\subsection{Equation for the perturbation $u_1=D_pu$}
%%%%%%%%%%%%%%%%%%%%%%%%%%%%%%%%%%%%%%%%%%%%%%%%%%%%%%%%%%%%%%%%%%%%%%%%%%%%%%%%%%%%%%%%


The equation for the perturbation in $u$, i.e., the $u_1$ unknown, is simply obtained by
taking the difference of the perturbed and unperturbed systems, dividing it by $\epsilon$, and 
letting $\epsilon \rightarrow 0$. This yields:
\begin{align}
\O(u_0,p_0) u_1 + \partial_u\O(u_0,p_0;u_1)u_0 + \partial_p \O(u_0,p_0;p_1) u_0 &= \partial_p Q p_1  \\
\mc{C}(u_0,p_0) u_1 +  \partial_u \mc{C}(u_0,p_0;u_1)u_0 + \partial_p \mc{C}(u_0,p_0;p_1)u_0  &=\partial_p u_\Gamma p_1
\end{align}
%
where the notation $\partial_u\O(u_0,p_0;u_1)$ means that the operator $\partial_u\O$ is evaluated at
$(u_0,p_0)$ and acts on $u_1$. 

We have defined:
\be
\partial_u\O(u_0,p_0;u_1) 
\equiv \lim_{\epsilon \rightarrow 0} \frac{\O(u_0+\epsilon u_1,p_0) - \O(u_0,p_0)}{\epsilon} 
\ee
and
\be
\partial_p\O(u_0,p_0;p_1) 
\equiv \lim_{\epsilon \rightarrow 0} \frac{\O(u_0,p_0+\epsilon p_1) - \O(u_0,p_0)}{\epsilon} 
\ee

%----------------------
\subsubsection{Linear case (linear diffusion)}
%----------------------
TO DO. It's simple.

%----------------------
\subsubsection{Nonlinear case (heat conduction)}
%----------------------
Let us go back to our example to try to clarify a bit. What is $\partial_u\O(u_0,p_0;u_1)u_0$?
From the definition, we have
\begin{align}
\partial_u\O(u_0,p_0;u_1) 
&\equiv \lim_{\epsilon \rightarrow 0} \frac{\O(u_0+\epsilon u_1,p_0) - \O(u_0,p_0)}{\epsilon} \notag \\
&= \lim_{\epsilon \rightarrow 0} \frac{-\div k(u_0+\epsilon u_1,p_0)\grad \cdot +\div k(u_0,p_0)\grad \cdot}{\epsilon} \notag \\
&= -\div \partial_u k(u_0,p_0) u_1\grad \cdot 
\end{align}
So,
\be
\partial_u\O(u_0,p_0;u_1)u_0 =  -\div \left( \partial_u k(u_0,p_0) u_1\grad u_0 \right)
\ee

Next term: What is $\partial_p\O(u_0,p_0;u_1)u_0$?
\begin{align}
\partial_p\O(u_0,p_0;p_1) 
&\equiv \lim_{\epsilon \rightarrow 0} \frac{\O(u_0,p_0+\epsilon p_1) - \O(u_0,p_0)}{\epsilon} \notag \\
&= \lim_{\epsilon \rightarrow 0} \frac{-\div k(u_0,p_0+\epsilon p_1)\grad \cdot +\div k(u_0,p_0)\grad \cdot}{\epsilon} \notag \\
&= -\div \partial_p k(u_0,p_0) p_1\grad \cdot 
\end{align}
So,
\be
\partial_p\O(u_0,p_0;u_1)u_0 =  -\div \left( \partial_p k(u_0,p_0) p_1\grad u_0 \right)
\ee


%----------------------
\subsubsection{Closing on the adjoint magic}
%----------------------

Going back to the general problem and using the $\O_1$ notation to simplify the 
expressions, we have the following {\bf linear} system for $u_1$
\begin{align}
\O_1(u_0,p_0) u_1 &= -\partial_p \O(u_0,p_0;p_1) u_0 +\partial_p Q p_1 \label{eq:pertsystV}\\
\mc{C}_1(u_0,p_0) u_1 &= -\partial_p \mc{C}(u_0,p_0;p_1) u_0 +\partial_p u_\Gamma p_1 \label{eq:pertsystBD}
\end{align}
with
\be
\O_1(u_0,p_0) u_1 \equiv \O(u_0,p_0) u_1 + \partial_u \O(u_0,p_0;u_1)  u_0
\ee
\be
\mc{C}_1(u_0,p_0) u_1 \equiv \mc{C}(u_0,p_0) u_1 + \partial_u \mc{C}(u_0,p_0;u_1)  u_0
\ee

%----------------------
\subsection{Back to the adjoint}
%----------------------

The adjoint equations of \eqt{eq:pertsystV} and \eqt{eq:pertsystBD}  are
%
\begin{align}
\O_1^\ast(u_0,p_0) u^\ast     &= \sigma  \label{eq:adjV} \\ % e_u
\mc{C}^\ast_1(u_0,p_0) u^\ast &= u^\ast_\Gamma \label{eq:adjBD} 
\end{align}
%with $u^\ast_\Gamma(p)$.

Since the $\O_1$ and $\mc{C}_1$ operators are linear, their transpose versions are easy to obtain. 
We also have a duality relation for these operators:
\be
(\us,\O_1(u,p)u) = (\O_1^\ast(u,p)\us,u) + W_1(\us,u)
\ee
Note, the concomitant is denoted by $W_1$ and $W_1 \ne W$.

The sensitivity coefficient is 
%
\begin{align}
S_p &= (\sigma, u_1) \notag \\
&= (\O_1^\ast(u_0,p_0) \us, u_1)  \notag \\
&= ( \us, \O_1(u_0,p_0) u_1) -W_1(\us,u_1) \notag \\
&= ( \us, -\partial_p \O(u_0,p_0;p_1) u_0 +\partial_p Q p_1) -W_1(\us,u_1) 
\end{align}

With our heat conduction example, we have
\be
S_p = \left( \us,  -\div \left( \partial_p k(u_0,p_0) p_1\grad u_0 \right) +\partial_p Q p_1 \right) -W_1(\us,u_1) 
\label{eq:dotprod}
\ee

Evaluating $S_p$ requires, therefore,
\ben
\item
computing $\us$ from \eqt{eq:adjV}  and \eqt{eq:adjBD} 
\item
evaluating scalar products in \eqt{eq:dotprod}
\item
what about $u_1$ at the boundary? (it is used in $W_1$ but it is 
also part of the specifications of the problem, so it must be known!)
\item
practically, $p_1$ is user-specified.
\een

%\begin{align}
%S_p = D_p\mc{F}[u_0,p_0;u_1,p_1] &= \lim_{\epsilon \rightarrow 0^+} \frac{\mc{F}[u+\epsilon u_1,p_0+\epsilon p_1] - \mc{F}[u_0,p_0] }{\epsilon} \notag \\
%&= \partial_u \mc{F}[u_0,p_0;u_1] + \partial_p \mc{F}[u_0,p_0;p_1] 
%\end{align}
%$\partial_u \mc{F}[u_0,p_0;u_1]$ acts on $u_1$ and is evaluated at $u_0,p_0$.\\
%$\partial_p \mc{F}[u_0,p_0;p_1]$ acts on $p_1$ and is evaluated at $u_0,p_0$.
%
%By definition of $\mc{F}$, we have
%\begin{align}
%S_p = (\B^\ast(u,p)\us, u) &= \left(\us,D_p(\B(u,p)u)\right) - W(\us,u_1) \notag\\
%&= \left(\us,D_p \B(u,p)u_0\right) + \left(\us,\B(u_0,p_0)u_1\right) - W(\us,u_1) \notag\\
%&= \left(\us,(\partial_u \B(u,p))u_0\right) + \left(\us,(\partial_p \B(u,p))u_0\right) + \left(\us,\B(u_0,p_0)u_1\right) - W(\us,u_1) \notag\\
%\end{align}


%%%%%%%%%%%%%%%%%%%%%%%%%%%%%%%%%%%%%%%%%%%%%%%%%%%%%%%%%%%%%%%%%%%%%%%%%%%%%%%%%%%%%%%%
\subsection{Radiation diffusion with material temperature}
%%%%%%%%%%%%%%%%%%%%%%%%%%%%%%%%%%%%%%%%%%%%%%%%%%%%%%%%%%%%%%%%%%%%%%%%%%%%%%%%%%%%%%%%


\begin{align}
\O_1(u_0,p_0) u_1 & \equiv \O(u_0,p_0) u_1 + \partial_u \O(u_0,p_0;u_1)  u_0 \notag \\
                  & =(\mc{A}\partial_t + \mc{B}) u_1 + \partial_u \mc{A}(u_0,p_0;u_1)  \partial_t u_0 + \partial_u \mc{B}(u_0,p_0;u_1)  u_0
\end{align}


%%%%%%%%%%%%%%%%%%%%%%%%%%%%%%%%%%%%%%%%%%%%%%%%%%%%%%%%%%%%%%%%%%%%%%%%%%%%%%%%%%%%%%%%
%%%%%%%%%%%%%%%%%%%%%%%%%%%%%%%%%%%%%%%%%%%%%%%%%%%%%%%%%%%%%%%%%%%%%%%%%%%%%%%%%%%%%%%%
%------------------------------------------------------
\end{document}
