\documentclass[11pt]{article}
% PACKAGES
\usepackage{graphicx}
\usepackage{tabls}
\usepackage{afterpage}
\usepackage{amsmath}
\usepackage{amsfonts}
\usepackage{amssymb}
\usepackage{dsfont}
\usepackage{amstext}
\usepackage{amsbsy}
\usepackage{epsfig}
%\usepackage{cites}
\usepackage{epsf}
\usepackage{float} 
\usepackage{color} 

\usepackage{array}
\usepackage[section]{placeins} % force � mettre l'image o� on veut
\usepackage{lscape} %utilisation du mode paysage
\usepackage{xspace}

%\usepackage[pdftex,bookmarks=true]{hyperref}
%\usepackage{hyperref}
\usepackage{url}
\usepackage{verbatim}
%\usepackage[all]{hypcap}

\usepackage[labelsep=quad]{caption} % needed by the breakalgo environment

\usepackage{ifthen}
\usepackage{subfig}

\usepackage{algorithmic}
\usepackage{algorithm}
\usepackage{listings}
%\usepackage[noprefix]{nomencl}  % for nomenclature

%-----------------------------------------------------------
% NEW  DEFINITIONS
% margin par
\newcommand{\mt}[1]{\marginpar{ {\footnotesize #1} }}
% vector shortcuts
\newcommand{\vo}{\vec{\Omega}}
\newcommand{\vr}{\vec{r}}
\newcommand{\vn}{\vec{n}}
\newcommand{\vnk}{\vec{\mathbf{n}}}

% More Quick Commands
% 
\newcommand{\bi}{\begin{itemize}}
\newcommand{\ei}{\end{itemize}}
\newcommand{\ben}{\begin{enumerate}}
\newcommand{\een}{\end{enumerate}}

\renewcommand{\div}{\vec{\nabla}\! \cdot \!}
\newcommand{\grad}{\vec{\nabla}}
\newcommand{\oa}[1]{\fn{\frac{1}{3}\hat{\Omega}\!\cdot\!\overrightarrow{A_{#1}}}}

% common reference commands
\newcommand{\eqt}[1]{Eq.~(\ref{#1})}                     % equation
\newcommand{\fig}[1]{Fig.~\ref{#1}}                      % figure
\newcommand{\tbl}[1]{Table~\ref{#1}}                     % table

%
% Equation beginnings and endings
%
\newcommand{\bea}{\begin{eqnarray}}
\newcommand{\eea}{\end{eqnarray}}

\newcommand{\be}{\begin{equation}}
\newcommand{\ee}{\end{equation}}
\newcommand{\beas}{\begin{eqnarray*}}
\newcommand{\eeas}{\end{eqnarray*}}
\newcommand{\bdm}{\begin{displaymath}}
\newcommand{\edm}{\end{displaymath}}
%
\newcommand{\vj}{\vec{J}}
\newcommand{\sa}[1]{\sigma_{a #1}}
\newcommand{\vl}{\vec{\lambda}}
\newcommand{\vdj}{\delta \vec{J}}
\newcommand{\dphi}{\delta \Phi}
\newcommand{\lmax}{\ensuremath{L_{\textit{max}}}\xspace}
\newcommand{\pmax}{\ensuremath{p_{\textit{max}}}\xspace}
\newcommand{\sddx}{\frac{d}{dx}}
\newcommand{\sddt}{\frac{d}{dt}}
\newcommand{\der}[2]{\frac{\partial #1}{\partial #2}}
\newcommand{\vF}{\vec{F}}

\renewcommand{\O}{\mathcal{O}}
\newcommand{\mc}[1]{\mathcal{#1}}
\newcommand{\us}{{u^\ast}}

\newcommand{\ti}{ {t^{\text{init}}} }
\newcommand{\te}{ {t^{\text{end }}} }

%-----------------------------------------------------------
\addtolength{\hoffset}{-2.0cm}
\addtolength{\textwidth}{4cm}
\addtolength{\textheight}{4.0cm}
\addtolength{\voffset}{-1.8cm}
\addtolength{\headsep}{-0.3cm}

\setlength{\parindent}{0pt}
\setlength{\parskip}{1.8ex plus 0.5ex minus 0.2ex}

\linespread{1.1}
%-----------------------------------------------------------
\begin{document}
%-----------------------------------------------------------
%%%%%%%%%%%%%%%%%%%%%%%%%%%%%%%%%%%%%%%%%%%%%%%%%%%%%%%%%%%%%%%%%%%%%%%%%%%%%%%%%%%%%%%%
%%%%%%%%%%%%%%%%%%%%%%%%%%%%%%%%%%%%%%%%%%%%%%%%%%%%%%%%%%%%%%%%%%%%%%%%%%%%%%%%%%%%%%%%

I would like to understand the connection between the continuum and discrete form of adjoint-based quantities
of interest.\\


%%%%%%%%%%%%%%%%%%%%%%%%%%%%%%%%%%%%%%%%%%%%%%%%%%%%%%%%%%%%%%%%%%%%%%%%%%%%%%%%%%%%%%%%
\section{Governing law}
%%%%%%%%%%%%%%%%%%%%%%%%%%%%%%%%%%%%%%%%%%%%%%%%%%%%%%%%%%%%%%%%%%%%%%%%%%%%%%%%%%%%%%%%

Consider the following linear diffusion operator (in a 1-D slab for the numerical results)
\be
\mc{B}(p)u = -\div D(p) \grad u  = q(p) \qquad \text{for } x \in X=[a,b]
\ee
with vacuum BC on $\Gamma=\partial X$
\be
u(a)=u_{left} \quad u(b)=u_{right}
\ee

%%%%%%%%%%%%%%%%%%%%%%%%%%%%%%%%%%%%%%%%%%%%%%%%%%%%%%%%%%%%%%%%%%%%%%%%%%%%%%%%%%%%%%%%
\section{Duality}
%%%%%%%%%%%%%%%%%%%%%%%%%%%%%%%%%%%%%%%%%%%%%%%%%%%%%%%%%%%%%%%%%%%%%%%%%%%%%%%%%%%%%%%%


Duality relationship for our problem:
\be
(\us,\mc{B}u) = (\mc{B}^\ast\us,u) + W(\us,u)
\ee
where
\be
\mc{B}^\ast= \mc{B} =  -\div D \grad  
\ee
\be
W(\us,u) = \langle u, \mc{L} \us \rangle  -  \langle \us, \mc{L} u \rangle
\ee
with $\mc{L} = n \cdot D\grad$ and $\langle \  , \, \rangle = \int_{\Gamma}\ $.


%%%%%%%%%%%%%%%%%%%%%%%%%%%%%%%%%%%%%%%%%%%%%%%%%%%%%%%%%%%%%%%%%%%%%%%%%%%%%%%%%%%%%%%%
\section{Quantity of interest}
%%%%%%%%%%%%%%%%%%%%%%%%%%%%%%%%%%%%%%%%%%%%%%%%%%%%%%%%%%%%%%%%%%%%%%%%%%%%%%%%%%%%%%%%


Say I am interested in the integral of $u$ over the domain, that is, in the following quantity of interest
\be
QoI[u,p] = \left(\mathds{1}, u \right) = \int_X  dx \, u(x) = \int_a^b dx \, u(x)
\ee

%%%%%%%%%%%%%%%%%%%%%%%%%%%%%%%%%%%%%%%%%%%%%%%%%%%%%%%%%%%%%%%%%%%%%%%%%%%%%%%%%%%%%%%%
\section{1D simplistic example}
%%%%%%%%%%%%%%%%%%%%%%%%%%%%%%%%%%%%%%%%%%%%%%%%%%%%%%%%%%%%%%%%%%%%%%%%%%%%%%%%%%%%%%%%

%------------------------------------------------------
\subsection{Analytical answers}
%------------------------------------------------------
We use $a=0$, $b=L$,$D=1$, $q$ uniform in space. Thus the analytical solution is:
\begin{equation}
u(x)= \alpha x^2 + \beta x + \gamma
\end{equation}
with $\alpha = -q/2$, $\gamma=u_{left}$, $\beta=\frac{u_{right}-u_{left}+qL^2/2}{L}$.\\
The analytical QoI is
\begin{equation}
\int_a^b dx \, u(x) =\frac{6(u_{right}+u_{left})L+qL^3}{12}
\end{equation}

%------------------------------------------------------
\subsection{FEM formalism}
%------------------------------------------------------
We discretize the governing law with standard continuous FEM and strongly enforced the Dirichlet boundary conditions. 
The resulting linear system is:
\be
Au=b
\ee
where
\begin{align}
A_{ij} &= \int_X \grad \varphi_i D \grad \varphi_j  - \int_{\partial X} \varphi_i D \grad \varphi_j \cdot \vn \\
b_i    &=  \int_X \varphi_i q 
\end{align}
and
\begin{align}
A_{1j} &= \delta_{1j}  \\
A_{nj} &= \delta_{nj}  \\
b_1    &=  u_{left}    \\
b_n    &=  u_{right}
\end{align}

The adjoint problem is
\be
A^\ast\us=b^\ast
\ee
where
\begin{align}
A^\ast_{ij} &= \int_X \grad \varphi_i D \grad \varphi_j  - \int_{\partial X} \varphi_i D \grad \varphi_j \cdot \vn \\
b^\ast_i    &=  \int_X \varphi_i 1 
\end{align}
and
\begin{align}
A^\ast_{1j} &= \delta_{1j}  \\
A^\ast_{nj} &= \delta_{nj}  \\
b^\ast_1    &=  \us_{left}    \\
b^\ast_n    &=  \us_{right}
\end{align}


\subsubsection{FEM tests 1: zero-Dirichlet BC for $u$ and $\us$}
%------------------------------------------------------
Here, we pick $u_{right}=u_{left}=0$ and $\us_{left}=\us_{right}=0$. Thus $W(\us,u)=0$ and duality gives us exactly
\be
(\us,\mc{B}u) = (\mc{B}^\ast\us,u) 
\ee
Recall that $ \mc{B} =  -\div D \grad  $. From here, we can play the FEM game and integrate by parts the Laplacian and we also have 
\be
(\grad \us, D \grad u ) = (\grad u, D \grad \us ) 
\ee

\begin{table*}[!htpb]
\begin{center}
\begin{tabular}{|l||c|}
\hline
	analytical QoI                                     & 250.0 \\ \hline
	forward numerical $ \left(\mathds{1}, u \right) $  &  249.999990018411 \\ \hline
	adjoint numerical $ \left(\us       , q \right) $  &  249.999990018412 \\ \hline
\end{tabular}
\end{center}
\caption{$q=3$, $L=10$, $u_{right}=u_{left}=0$, $\us_{left}=\us_{right}=0$, linear FEM with 5,000 cells for numerical approximations}
%\label{tab:ExamplesOfResponseFunctions}
\end{table*}


\subsubsection{FEM tests 1: nonzero-Dirichlet BC for $u$ {\bf but} zero-Dirichlet for $\us$}
%------------------------------------------------------
Here, we pick $u_{right}=u_{left}=1$. The QoI of interest being the integral of $u$, its new value should be greater than the previous value by $1 \times L$ (i.e., $250 \longrightarrow 260$) and we have that analytically.  

\begin{table*}[!htpb]
\begin{center}
\begin{tabular}{|l||c|}
\hline
	analytical QoI                                     & 260.0 \\ \hline
	forward numerical $ \left(\mathds{1}, u \right) $  &  259.999990018996 \\ \hline
	adjoint numerical $ \left(\us       , q \right) $  &  249.999990018412 \\ \hline
\end{tabular}
\end{center}
\caption{$q=3$, $L=10$, $u_{right}=u_{left}=1$, $\us_{left}=\us_{right}=0$, linear FEM with 5,000 cells for numerical approximations}
%\label{tab:ExamplesOfResponseFunctions}
\end{table*}


%%%%%%%%%%%%%%%%%%%%%%%%%%%%%%%%%%%%%%%%%%%%%%%%%%%%%%%%%%%%%%%%%%%%%%%%%%%%%%%%%%%%%%%%
%%%%%%%%%%%%%%%%%%%%%%%%%%%%%%%%%%%%%%%%%%%%%%%%%%%%%%%%%%%%%%%%%%%%%%%%%%%%%%%%%%%%%%%%
%------------------------------------------------------
\end{document}
