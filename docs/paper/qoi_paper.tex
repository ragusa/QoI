\documentclass[review]{elsarticle}

\usepackage{lineno,hyperref}
\usepackage{amsmath, amsthm, amssymb, amsfonts, booktabs, hyperref, graphicx, float, esint, xcolor, subcaption, xspace}

\usepackage{tikz}
\usepackage{rotating}
\usetikzlibrary{arrows.meta, patterns, shapes.misc, calc}

\modulolinenumbers[5]

\journal{Journal of \LaTeX\ Templates}

%%%%%%%%%%%%%%%%%%%%%%%
%% Elsevier bibliography styles
%%%%%%%%%%%%%%%%%%%%%%%
%% To change the style, put a % in front of the second line of the current style and
%% remove the % from the second line of the style you would like to use.
%%%%%%%%%%%%%%%%%%%%%%%

%% Numbered
%\bibliographystyle{model1-num-names}

%% Numbered without titles
%\bibliographystyle{model1a-num-names}

%% Harvard
%\bibliographystyle{model2-names.bst}\biboptions{authoryear}

%% Vancouver numbered
%\usepackage{numcompress}\bibliographystyle{model3-num-names}

%% Vancouver name/year
%\usepackage{numcompress}\bibliographystyle{model4-names}\biboptions{authoryear}

%% APA style
%\bibliographystyle{model5-names}\biboptions{authoryear}

%% AMA style
%\usepackage{numcompress}\bibliographystyle{model6-num-names}

%% `Elsevier LaTeX' style
\bibliographystyle{elsarticle-num}
%%%%%%%%%%%%%%%%%%%%%%%

%-----------------------------------------------------------
%-----------------------------------------------------------

\newcommand{\vr}{\vec{r}}
\newcommand{\vp}{\vec{p}}
\newcommand{\vOmega}{\vec{\Omega}}
\newcommand{\vJ}{\vec{J}}
\newcommand{\vO}{\vec{\Omega}}
\newcommand{\bra}{\left\langle}
\newcommand{\ket}{\right\rangle}
\newcommand{\braSN}{\left\langle \! \left\langle}
\newcommand{\ketSN}{\right\rangle \! \right\rangle}
\newcommand{\sbraSN}{\left[ \! \left[}
\newcommand{\sketSN}{\right] \! \right]}
\newcommand{\sbra}{\left[}
\newcommand{\sket}{\right]}
\renewcommand{\div}{\vec{\nabla} \cdot}
\newcommand{\grad}{\vec{\nabla}}
\newcommand{\vbeta}{\vec{\beta} }
\newcommand{\pdx}{\frac{\partial}{\partial x}}
\newcommand{\pdy}{\frac{\partial}{\partial y}}
\newcommand{\pdz}{\frac{\partial}{\partial z}}
\newcommand{\intrrr}{\int d^3 r \,}
\newcommand{\intrr}{\int d^2 r \,}
\newcommand{\dEdphi}{\partial_\phi E }
\newcommand{\dEdp}{\partial_p E }
\newcommand{\dBdphi}{\partial_\phi B }
\newcommand{\dBdp}{B }
\newcommand{\adj}{\phi^\dag}
\newcommand{\vefadj}{\varphi^\dag}
\newcommand{\surf}{\int_{\partial V}}
\newcommand{\domain}{V}
\newcommand{\bound}{\partial V}
\newcommand{\vn}{\vec{n}}
\newcommand{\Edd}{\mathbb{E}}
\newcommand{\BEdd}{B}
\newcommand{\sigt}{\sigma_t}
\newcommand{\sigs}{\sigma_s}
\newcommand{\siga}{\sigma_a}
%\newcommand{\isigt}{\sigma_t^{-1}}
%\newcommand{\isigtp}{\sigma_{t,p}^{-1}}
\newcommand{\isigt}{\ell_t}
\newcommand{\isigtp}{\ell_{t,p}}
\newcommand{\angSource}{\frac{q}{4 \pi}}
\newcommand{\angSourcep}{\frac{q_p}{4 \pi}}
\newcommand{\angSourcepd}{\frac{q_p+\delta q_p}{4 \pi}}
\newcommand{\angSourced}{\frac{\delta q}{4 \pi}}
\newcommand{\scalSource}{q}
\newcommand{\angResp}{q^\dag}
\newcommand{\scalResp}{q^\dag}
\newcommand{\qoi}{{\it QoI}\xspace}


\newcommand{\comment}[2]{\marginpar{\textcolor{#2}{$\star$}}\textcolor{#2}{#1}\newline}

%-----------------------------------------------------------
%-----------------------------------------------------------
\usepackage{ifthen}
\newboolean{draftversion}
\setboolean{draftversion}{true}
%-----------------------------------------------------------
%----------------------------------------------------------

\ifthenelse{\boolean{draftversion}}
{
\newcommand{\iwh}[1]{\comment{#1}{red}}
\newcommand{\jcr}[1]{\comment{#1}{blue}}
\newcommand{\todo}[1]{\comment{#1}{purple}}
}
{
\newcommand{\iwh}[1]{\phantom{a}}
\newcommand{\jcr}[1]{\phantom{a}}
\newcommand{\todo}[1]{\phantom{a}}
}
\newcommand{\tcr}[1]{\textcolor{red}{#1}}


%%%%%%%%%%%%%%%%%%%%%%%%%%%%%%%%%%%%%%%%%%%%%%%%%%%%%%%%%%%%%%%%%%%%%
%%%%%%%%%%%%%%%%%%%%%%%%%%%%%%%%%%%%%%%%%%%%%%%%%%%%%%%%%%%%%%%%%%%%%
\begin{document}
%%%%%%%%%%%%%%%%%%%%%%%%%%%%%%%%%%%%%%%%%%%%%%%%%%%%%%%%%%%%%%%%%%%%%
%%%%%%%%%%%%%%%%%%%%%%%%%%%%%%%%%%%%%%%%%%%%%%%%%%%%%%%%%%%%%%%%%%%%%

\begin{frontmatter}

\title{Adjoint-based Sensitivity for Radiation Transport Using an Eddington Tensor Formulation}

%% Group authors per affiliation:
%\author{Elsevier\fnref{myfootnote}}
%\address{Radarweg 29, Amsterdam}

%% or include affiliations in footnotes:
%\author[mymainaddress,mysecondaryaddress]{Elsevier Inc1}
\author{Ian W. Halvic}
\ead{iwhalvic@tamu.edu}
\author{Jean C. Ragusa}
\ead{jean.ragusa@tamu.edu}

\address{Department of Nuclear Engineering, Texas A\&M University, College Station, TX, USA}
%\footnote{Corresponding author}

\begin{abstract}
The abstract ... 
\end{abstract}

\begin{keyword}
\texttt{elsarticle.cls}\sep \LaTeX\sep Elsevier \sep template
\MSC[2010] 00-01\sep  99-00
\end{keyword}

\end{frontmatter}

% \linenumbers

%%%%%%%%%%%%%%%%%%%%%%%%%%%%%%%%%%%%%%%%%%%%%%%%%%%%%%%%%%%%%%%%%%%%%
%%%%%%%%%%%%%%%%%%%%%%%%%%%%%%%%%%%%%%%%%%%%%%%%%%%%%%%%%%%%%%%%%%%%%

%%%%%%%%%%%%%%%%%%%%%%%%%%%%%%%%%%%%%%%%%%%%%%%%%%%%%%%%%%%%%%%%%%%%%
%%%%%%%%%%%%%%%%%%%%%%%%%%%%%%%%%%%%%%%%%%%%%%%%%%%%%%%%%%%%%%%%%%%%%
\section{Introduction}
%%%%%%%%%%%%%%%%%%%%%%%%%%%%%%%%%%%%%%%%%%%%%%%%%%%%%%%%%%%%%%%%%%%%%
%%%%%%%%%%%%%%%%%%%%%%%%%%%%%%%%%%%%%%%%%%%%%%%%%%%%%%%%%%%%%%%%%%%%%
\iwh{
The main parts that I think I want to touch on in the Introduction/Background are:
\begin{itemize}
\item Lit review focus on the purpose (Including VVUQ needs), strengths, and shortcomings of adjoint methods. I think this was a somewhat weak part in the thesis write up, so probably want to put a bit more into it this time around. In this setting, is it more appropriate to make the general explanation of the adjoint method much more brief? That is to say have a statement of $\delta \qoi \approx \braSN \delta q - \delta \mathbf{A} \psi , \psi^\dag \ketSN + \delta BC$ and a reference as opposed to doing the whole general derivation of a first order perturbation using a general operator $\mathbf{A}$.
\item Presentation of the transport equation and the transport adjoint. Also introduce the $\delta \qoi$ innerproduct for the transport system using the adjoint.
\item The issues using the adjoint in this method for time dependent transport, namely the $\braSN  \delta \sigt \psi , \psi^\dag \ketSN$ term which involves storing the angular flux solutions. 
\item At some point we need to do the transition from time dependent to steady state, and the justification for beginning with the steady state problem in the hope of informing the behavior of the time dependent problem
\end{itemize}
Main question(s) I will have are probably about what can be presented without derivation. I assume I can present the transport equation, but can the adjoint be presented just as simply? If so, can the first-order sensitivity inner-product then be presented without demonstrating the ad joint process (instead just citing the appropriate paper). Key equations to introduce are related to transport adjoint sensitivity:
Forward
\begin{subequations}\label{eqs:TransportSystem}
\begin{equation}
\label{SS1GTE}
\vO \cdot \grad \psi(\vr,\vO) + \sigt(\vr) \psi(\vr,\vO) = \frac{1}{4 \pi} \sigs(\vr) \phi(\vr) + \frac{1}{4 \pi} q(\vr)\, ,
\end{equation}
\begin{equation}
\label{SS1GTE_bc}
\psi(\vr,\vO) = \psi^{\text{inc}}(\vr,\vO) \quad \vr \in \partial V^{-} = \{ \vr \in \partial V,  \vO \cdot \vec{n}(\vr) < 0\}
\end{equation}
\end{subequations}
Adjoint
\begin{subequations}\label{eqs:TransportAdjSystem}
\begin{equation}
\label{SS1GTE}
- \vO \cdot \grad \psi^\dag(\vr,\vO) + \sigt(\vr) \psi^\dag(\vr,\vO) = \frac{1}{4 \pi} \sigs(\vr) \phi^\dag(\vr) + q^\dag(\vr)\, ,
\end{equation}
\begin{equation}
\label{SS1GTE_bc}
\psi^\dag(\vr,\vO) = \psi^{\dag,\text{out}}(\vr,\vO) \quad \vr \in \partial V^{+} = \{ \vr \in \partial V,  \vO \cdot \vec{n}(\vr) > 0\}
\end{equation}
\end{subequations}
Sensitivity
\begin{equation}
\delta \qoi = \bra \angSourced  + \frac{\delta\sigs}{4 \pi} \phi , \phi^\dag  \ket - \braSN  \delta \sigt \psi , \psi^\dag \ketSN - \sbraSN \delta \psi^{\text{inc}}, \psi^\dag \sketSN_- \,.
\end{equation}
}

\subsection{VVUQ and Adjoint Methods}

\subsection{Sensitivity Using Transport Adjoint Formulation}

%%%%%%%%%%%%%%%%%%%%%%%%%%%%%%%%%%%%%%%%%%%%%%%%%%%%%%%%%%%%%%%%%%%%%
%%%%%%%%%%%%%%%%%%%%%%%%%%%%%%%%%%%%%%%%%%%%%%%%%%%%%%%%%%%%%%%%%%%%%
\section{Variable Eddington Tensor Method}
\label{sec:VET}
%%%%%%%%%%%%%%%%%%%%%%%%%%%%%%%%%%%%%%%%%%%%%%%%%%%%%%%%%%%%%%%%%%%%%
%%%%%%%%%%%%%%%%%%%%%%%%%%%%%%%%%%%%%%%%%%%%%%%%%%%%%%%%%%%%%%%%%%%%%
\iwh{
Section comes from the bulk of my thesis, the general flow I expect to be similar to my thesis, but without the weird ``Eddington of the adjoint'' thing. Also may drop the ``blended'' method. In general these are what I wish to touch on
\begin{itemize}
\item Formation of the P1 equations and using them to form the VET formulation of the forward system. Not entirely sure how much detail I should go into the boundary condition derivation. As opposed to the thesis, for consistency I may the substitution of $\isigt = 1/\sigt$ here, rather than waiting to introduce $\delta$ terms.
\item Apply adjoint process to the VET formulation to derive the VET adjoint equation.
\item Derive the perturbation sensitivity inner product for VET adjoint (assuming $\delta \Edd=0$). This is a bit long winded in the thesis, so I plan on trying to pare this back a tad.
\item Present the simple method by which we can attempt to approximate $\delta \Edd$ using additional transport solves. Question: As opposed to the thesis, should we maybe point out the major effect the $\isigt$ term has preemptively, instead of waiting for results? I think in the results the 2D/3D application of this could be a bit more elegant, where we attempt to only approximate $\delta \Edd$ in/near reagions of very high $\isigt$ (voids) to cut down on Eddington storage using this method.
\item Should we include the ``blended'' method?
\end{itemize}
This is basically all legwork to get to the sensitivity inner-product, so I may need some advice on what mathematical steps I can skip over, and what I should write out explicitly. The key equations I want to hit in this formulation are:
P1 equations
\begin{subequations}
%
\begin{equation}
\label{0am}
\div \vec{J} + (\siga) \phi = \scalSource \,,
\end{equation}
\begin{equation}
\label{1am}
\div \left(  \int d\Omega \vO \vO \psi \right) + \sigt \vec{J} = 0 \,.
\end{equation}
%
\end{subequations}
VET system
\begin{subequations} \label{eqs:EddingtonSystem}
\begin{equation} \label{eq:EddingtonVol}
- \div \left( \isigt(\vr)\div \Edd(\vr) \phi(\vr) \right) + \siga(\vr) \phi(\vr) = \scalSource(\vr) \,,
\end{equation}
\begin{equation} \label{eq:EddingtonBC}
2 J^{\text{inc}}(\vr) = \BEdd(\vr) \phi(\vr) + \vn \cdot \isigt(\vr) \div \Edd(\vr) \phi(\vr)  \quad \vr \in \bound \,.
\end{equation}
\end{subequations}
The adjoint of Eq.~\eqref{eqs:EddingtonSystem}
\begin{subequations}\label{eqs:EddingtonAdjSystem}
\begin{equation}\label{eq:EddingtonAdjVol}
- \Edd : \grad \left( \isigt(\vr)\grad \vefadj \right)  + \siga \vefadj = \scalResp
\end{equation}
\begin{equation}\label{eq:EddingtonAdjBC}
0 = B \vefadj+ \vn \cdot
\Edd \cdot \isigt(\vr) \vec{\nabla} \vefadj    \quad \vr \in \bound
\end{equation}
\end{subequations}
Sensitivity Inner Product
\begin{equation}\label{eqs:sensitivity}
\delta \qoi =  \bra \delta \scalSource - \delta \siga \phi, \vefadj \ket  - \bra \delta \isigt \div \left( \Edd \phi \right) , \grad \vefadj \ket
 + \sbra \vefadj, 2 \delta J^{\text{inc}} \sket \,.
\end{equation}
Eddington Approximation
\begin{equation}
\label{Eddapprox}
\frac{\partial \Edd}{\partial \vp} \approx \frac{\Edd(\vp_1) - \Edd(\vp_0)}{\vp_1 - \vp_0}
\end{equation}
Sensitivity with Eddington Approx
\begin{equation}
\label{EddErr}
 Eq.~\eqref{eqs:sensitivity}- \bra  \isigt \div \left( \delta \Edd \phi \right), \grad \vefadj \ket
- \sbra \vefadj, \delta \BEdd \phi \sket \,.
\end{equation} 
.}

\subsection{VET and Adjoint Formulation}

\subsection{Sensitivity Using VET}

\subsection{Refinement by Approximation of $\delta \Edd$}

%%%%%%%%%%%%%%%%%%%%%%%%%%%%%%%%%%%%%%%%%%%%%%%%%%%%%%%%%%%%%%%%%%%%%
%%%%%%%%%%%%%%%%%%%%%%%%%%%%%%%%%%%%%%%%%%%%%%%%%%%%%%%%%%%%%%%%%%%%%
\section{Results and Analysis}
\label{sec:RnA}
%%%%%%%%%%%%%%%%%%%%%%%%%%%%%%%%%%%%%%%%%%%%%%%%%%%%%%%%%%%%%%%%%%%%%
%%%%%%%%%%%%%%%%%%%%%%%%%%%%%%%%%%%%%%%%%%%%%%%%%%%%%%%%%%%%%%%%%%%%%
\iwh{Probably the most TBD section, as the most interesting results will be the 2D ones. Major things to touch on:
\begin{itemize}
\item 1D Reed problem. From the thesis, the Reed problem proved to be the most robust and most interesting. The important effect we see is here is how voids can begin to cause issues with the VET. In the thesis, by the time we are at the Reed problem we have already taken a look at source and absorption perturbation in the more simple cases. We may need to instead take a brief glance at those perturbations in the Reed system (incident flux perturbations may need dealt with). However, assuming we have the 2D results to look forward to, we may not need to spend a whole lot of time on the Reed results. 
\item MOOSE 2D results. To be determined, but probably much more interesting. Start with some non streaming problems and hopefully things looks acceptable. Introduce streaming regions near \qoi regions and note the effect. Attempt to approximate $\delta \Edd$ and see the results. 
\item Note that with the $\delta E$ approximation method, we have changed our original memory intensive transport adjoint where we need to store $\psi,\psi^\dag$, for a somewhat more flexible system where we now need to store $\Edd$ and $\delta \Edd$ values. Identifying regions/perturbations where $\delta \Edd$ has the strongest effect may be key to using memory wisely.
\end{itemize}
.}

\subsection{1D Reed System}

\subsection{2D MOOSE Results}


%%%%%%%%%%%%%%%%%%%%%%%%%%%%%%%%%%%%%%%%%%%%%%%%%%%%%%%%%%%%%%%%%%%%%
%%%%%%%%%%%%%%%%%%%%%%%%%%%%%%%%%%%%%%%%%%%%%%%%%%%%%%%%%%%%%%%%%%%%%
\section*{References}

\bibliography{QoI_bib}

%%%%%%%%%%%%%%%%%%%%%%%%%%%%%%%%%%%%%%%%%%%%%%%%%%%%%%%%%%%%%%%%%%%%%
%%%%%%%%%%%%%%%%%%%%%%%%%%%%%%%%%%%%%%%%%%%%%%%%%%%%%%%%%%%%%%%%%%%%%
\end{document}
%%%%%%%%%%%%%%%%%%%%%%%%%%%%%%%%%%%%%%%%%%%%%%%%%%%%%%%%%%%%%%%%%%%%%
%%%%%%%%%%%%%%%%%%%%%%%%%%%%%%%%%%%%%%%%%%%%%%%%%%%%%%%%%%%%%%%%%%%%%
