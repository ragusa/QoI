\title{Methods and their Adjoints}
\author{Jared Geer}
\date{\today}

\documentclass[12pt]{article}
\usepackage{url}
\usepackage{amsmath}
\usepackage{bm}
\begin{document}
\maketitle
\section{Methods}
\begin{equation}\label{ForwardProblem}
\frac{\partial u}{\partial t} = A u + b
\end{equation}
\begin{equation}\label{AdjointProblem}
-\frac{\partial u^*}{\partial t} = A^* u^* + r
\end{equation}
\subsection{Continuum}
\subsection{Forward Euler's Method}
The forward problem is discretized as
\begin{equation}\label{FEulerDiscrete}
(I-A_{n+1})u_{n+1} = u_n + \tau b_{n+1}
\end{equation}
In matrix form that is
\begin{equation}
\begin{bmatrix}
I\\
-I & I - \tau A_1\\
& \ddots & \ddots\\
& & -I & I - \tau A_{n-1}\\
& & & -I & I - \tau A_{n}\\
\end{bmatrix}
u = 
\begin{bmatrix}
u_0 \\
\tau b_1 \\
\vdots \\
\tau b_{n-1} \\
\tau b_n
\end{bmatrix}
\end{equation}

With malice aforethought we modify the 0th row to conform with the diagonal. I.e.
\begin{equation}
\begin{bmatrix}
I\bm{-\tau A_0}\\
-I & I - \tau A_1\\
& \ddots & \ddots\\
& & -I & I - \tau A_{n-1}\\
& & & -I & I - \tau A_{n}\\
\end{bmatrix}
u = 
\begin{bmatrix}
u_0 \bm{- \tau A_0 u_0}\\
\tau b_1 \\
\vdots \\
\tau b_{n-1} \\
\tau b_n
\end{bmatrix}
\end{equation}
Out requirement is that $ A^* := A^t $ which means the equations for the adjoint are given by
\begin{equation}
\begin{bmatrix}
I-\tau A^*_0 & -I\\
& I - \tau A^*_1  & -I\\
& & \ddots & \ddots\\
& & & I - \tau A^*_{n-1} & -I\\
& & & & I - \tau A^*_{n}\\
\end{bmatrix}
u^* = 
\begin{bmatrix}
\tau r_0\\
\tau r_1 \\
\vdots \\
\tau r_{n-1} \\
u^*_n - \tau A^*_n u^*_n
\end{bmatrix}
\end{equation}
which leads to the equation
\begin{equation}\label{BEulerDiscrete}
(I-\tau A^*_{n-1})u^*_{n-1} - u^*_n = \tau r_n-1
\end{equation}
Which is the Backwards Euler Method.

To calculate the QoI
\begin{eqnarray}
u^* A u & = & u^* b \\
 & = & \sum_{n=1}^N \left(u^*_n \tau b_n\right) \bm{+ u^*_0(I-\tau A_0)u_0} \\
 & = & u r \\
 & = & \sum_{n=0}^{N-1}\left( u_n \tau r_n\right) \bm{+ u_n(I-\tau A_N)u^*_N}
\end{eqnarray}
where the extra bolded terms is due to the modification done to make the A matrix's transpose equivalent to the adjoint's.\\

The QoI is $\sum_{n=0}^{N-1}\left( u_n \tau r_n\right)$ which is thus equivalent to
\begin{equation}\label{FEulerQoIa}
QoI = \sum_{n=0}^{N-1}\left( u_n \tau r_n\right) = \sum_{n=1}^N \left(u^*_n \tau b_n\right) + u^*_0(I-\tau A_0)u_0 - u_n(I-\tau A_N)u^*_N
\end{equation}
As the number of steps approaches infinity (the continuum) the step size, $\tau$, approaches zero, thus
\begin{equation}
QoI_{n\rightarrow\infty} = \sum_{n=1}^N \left(u^*_n \tau b_n\right) + u^*_0 u_0 - u_n u^*_N
\end{equation}
Which is equivalent to the continuum.
\subsection{Backward Euler's Method}
\begin{equation}\label{BEulerDiscrete}
(I-\tau A^*_{n-1})u^*_{n-1} - u^*_n = \tau r_n-1
\end{equation}
\subsection{Crank–Nicolson Method}
The forward problem is discretized as
\begin{equation}\label{CNdiscrete}
\left(\frac{1}{\tau}-\frac{A_{n+1}}{2}\right)u_{n+1} + \left(-\frac{1}{\tau}-\frac{A_n}{2}\right)u_n = \tau \frac{b_{n+1} + b_n}{2}
\end{equation}
\begin{equation}
\begin{bmatrix}
\frac{1}{\tau}-\frac{A_0}{2}\\
-\frac{1}{\tau}-\frac{A_0}{2} & \frac{1}{\tau}-\frac{A_1}{2}\\
& \ddots & \ddots\\
& & -\frac{1}{\tau}-\frac{A_{n-2}}{2} & \frac{1}{\tau}-\frac{A_{n-1}}{2} \\
& & & -\frac{1}{\tau}-\frac{A_{n-1}}{2} & \frac{1}{\tau}-\frac{A_{n}}{2} \\
\end{bmatrix}
u = 
\begin{bmatrix}
\left(\frac{1}{\tau}-\frac{A_0}{2}\right)u_0 \\
\tau \frac{b_1 + b_0}{2}\\
\vdots \\
\tau \frac{b_{n-1} + b_{n-2}}{2}\\
\tau \frac{b_n + b_{n-1}}{2}
\end{bmatrix}
\end{equation}
Again we take the transpose of A and get the adjoint system
\begin{equation}
\begin{bmatrix}
\frac{1}{\tau}-\frac{A^*_0}{2} & -\frac{1}{\tau}-\frac{A^*_0}{2}\\
 & \frac{1}{\tau}-\frac{A^*_1}{2} & -\frac{1}{\tau}-\frac{A^*_1}{2}\\
& \ddots & \ddots\\
& & & \frac{1}{\tau}-\frac{A^*_{n-1}}{2} & -\frac{1}{\tau}-\frac{A^*_{n-1}}{2} \\
& & & & \frac{1}{\tau}-\frac{A^*_{n}}{2} \\
\end{bmatrix}
u^* = 
\begin{bmatrix}
\\
\\
\vdots \\
\\
\left(\frac{1}{\tau}-\frac{A^*_{n}}{2}\right)u^*_n
\end{bmatrix}
\end{equation}
\begin{eqnarray}
 & = & \left(\frac{1}{\tau} - \frac{A^*_{n}}{2}\right) u^*_{n} +
\left(-\frac{1}{\tau} - \frac{A^*_{n}}{2}\right) u^*_{n+1}\\
& = & -\frac{u^*_{n+1} - u^*_n}{\tau} - \frac{A^*_{n}}{2} ( u^*_{n} + u^*_{n+1})
\end{eqnarray}
\subsection{Heun's Method}

\end{document}
