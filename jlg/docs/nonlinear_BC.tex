\title{Nonlinear Bondary Conditions}
\author{Jared Geer}
\date{\today}

\documentclass[12pt]{article}
\usepackage{url}
\usepackage{amsmath}
\usepackage{bm}
\begin{document}
\maketitle
\section{Neumann}
The Neumann BC is:
\begin{eqnarray}
\nabla u(0) & = & Left \\
\nabla u(N) & = & Right
\end{eqnarray}
In a linear problem of the form $A u = b$, the b vector is modified as follows.
\begin{eqnarray}
\tilde{b}_0 & = & b_0 + Left\\
\tilde{b}_N & = & b_N + Right
\end{eqnarray}
In the nonlinear problem the jacobian is untouched and the residual is modified as
\begin{eqnarray}
\tilde{F}_0 & = & F_0 - Left\\ 
\tilde{F}_N & = & F_N - Right 
\end{eqnarray}
\section{Robin}
In a linear problem of the form $A u = b$, the A matrix and b vector are modified as follows.
\begin{eqnarray}
\tilde{b}_0 & = & b_0 + 2\times Left\\
\tilde{b}_N & = & b_N + 2\times Right
\end{eqnarray}
\begin{eqnarray}
\tilde{A}_{0,0} & = & A_{0,0} + \frac{1}{2}\\
\tilde{A}_{N,N} & = & A_{N,N} + \frac{1}{2}
\end{eqnarray}
In the nonlinear problem the jacobian is modified as
\begin{eqnarray}
\tilde{J}_{0,0} & = & J_{0,0} + \frac{1}{2}\\
\tilde{J}_{N,N} & = & J_{N,N} + \frac{1}{2}
\end{eqnarray}
and the residual
\begin{eqnarray}
\tilde{F}_{0} & = & \frac{1}{2} u_0 - 2\times Left\\
\tilde{F}_{N} & = & \frac{1}{2} u_N - 2\times Right
\end{eqnarray}
\section{Dirichlet}
The Dirichlet BC is:
\begin{eqnarray}
u_0 & = & Left \\
u_N & = & Right
\end{eqnarray}
In a linear problem of the form $A u = b$, the b vector is modified as follows for all $i$ indices.
\begin{eqnarray}
\tilde{b}_i & = & b_i + Left\times A_{i,N} - Right\times A_{i,0}\\
\end{eqnarray}
In the nonlinear problem the jacobian is modified as for a left condition
\begin{eqnarray}
\tilde{J}_{i,0} & = & 0\\
\tilde{J}_{0,i} & = & 0\\
\tilde{J}_{0,0} & = & 1
\end{eqnarray}
as as follows for a right condition
\begin{eqnarray}
\tilde{J}_{i,N} & = & 0\\
\tilde{J}_{N,i} & = & 0\\
\tilde{J}_{N,N} & = & 1
\end{eqnarray}
and the residual is modified as
\begin{eqnarray}
\tilde{F}_0 & = & u_0 - Left\\ 
\tilde{F}_N & = & u_N - Right 
\end{eqnarray}

\end{document}
